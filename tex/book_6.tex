\section[高帝求賢詔\quad{\small 西漢文}]{{\normalsize 西漢文}\quad \ProperName{高帝}求賢詔}
蓋聞王者莫高於\ProperName{周文},伯者莫高於\ProperName{齊桓},皆待賢人而成名。今天下賢者智能,豈特古之人乎?患在人主不交故也,士奚由進!今吾以天之靈,賢士大夫定有天下,以爲一家,欲其長久,世世奉宗廟亡絕也。賢人已與我共平之矣,而不與吾共安利之,可乎?賢士大夫有肯從我游者,吾能尊顯之。布告天下,使明知朕意。御史大夫\ProperName{昌}下相國,相國\ProperName{酇侯}下諸侯王,御史中執法下郡守,其有意稱明德者,必身勸,爲之駕,遣詣相國府,署行、義、年。有而弗言,覺,免。年老癃病,勿遣。 

\section[文帝議佐百姓詔\quad{\small 西漢文}]{{\normalsize 西漢文}\quad \ProperName{文帝}議佐百姓詔}
間者數年比不登,又有水旱疾疫之災,朕甚憂之。愚而不明,未達其咎。意者朕之政有所失而行有過與?乃天道有不順,地利或不得,人事多失和,鬼神廢不享與?何以致此?將百官之奉養或費,無用之事或多與?何其民食之寡乏也!夫度田非益寡,而計民未加益,以口量地,其於古猶有餘,而食之甚不足者,其咎安在?無乃百姓之從事於末以害農者蕃,爲酒醪以靡穀者多,六畜之食焉者衆與?細大之義,吾未能得其中。其與丞相列侯吏二千石博士議之,有可以佐百姓者,率意遠思,無有所隱。

\section[景帝令二千石修職詔\quad{\small 西漢文}]{{\normalsize 西漢文}\quad \ProperName{景帝}令二千石修職詔}
雕文刻鏤,傷農事者也;錦繡纂組,害女紅者也。農事傷則飢之本也,女紅害則寒之原也。夫飢寒並至,而能亡\endnote{\BookTitle{觀止}作「無」,據\BookTitle{漢書}改。}爲非者寡矣。朕親耕,后親桑,以奉宗廟粢盛祭服,爲天下先;不受獻,減太官,省繇賦,欲天下務農蠶,素有畜積,以備災害。彊毋攘弱,眾毋暴寡,老耆以壽終,幼孤得遂長。今歲或不登,民食頗寡,其咎安在?或詐僞爲吏,吏以貨賂爲市,漁奪百姓,侵牟萬民。縣丞,長吏也,奸法與盜盜,甚無謂也。其令二千石各修其職;不事官職耗亂者,丞相以聞,請其罪。布告天下,使明知朕意。

\theendnotes

\section[武帝求茂材異等詔\quad{\small 西漢文}]{{\normalsize 西漢文}\quad \ProperName{武帝}求茂材異等詔}
盖有非常之功,必待非常之人,故馬或奔踶而致千里,士或有負俗之累而立功名。夫泛駕之馬,跅弛之士,亦在御之而已。其令州郡察吏民有茂材異等可爲將相及使絕國者。 

\section[賈誼過秦論上\quad{\small 西漢文}]{{\normalsize 西漢文\ 賈誼}\quad 過\ProperName{秦}論上}
\ProperName{秦孝公}據\ProperName{殽}\ProperName{函}之固,擁\ProperName{雍州}之地,君臣固守,以窺\ProperName{周室},有席卷天下,包舉宇內,囊括四海之意,幷吞八荒之心。當是時也,\ProperName{商君}佐之,內立法度,務耕織,修守戰之具,外連衡而鬬諸侯,於是\ProperName{秦}人拱手而取\ProperName{西河}之外。

\ProperName{孝公}既沒,\ProperName{惠文}、\ProperName{武}、\ProperName{昭}蒙故業,因遺策,南取\ProperName{漢中},西舉\ProperName{巴}、\ProperName{蜀},東割膏腴之地,收要害之郡。諸侯恐懼,會盟而謀弱\ProperName{秦},不愛珍器重寶肥饒之地,以致天下之士,合從締交,相與爲一。當此之時,\ProperName{齊}有\ProperName{孟嘗},\ProperName{趙}有\ProperName{平原},\ProperName{楚}有\ProperName{春申},\ProperName{魏}有\ProperName{信陵}。此四君者,皆明智而忠信,寬厚而愛人,尊賢而重士,約從離衡,兼\ProperName{韓}、\ProperName{魏}、\ProperName{燕}、\ProperName{楚}、\ProperName{齊}、\ProperName{趙}、\ProperName{宋}、\ProperName{衛}、\ProperName{中山}之眾。於是六國之士有\ProperName{寧越}、\ProperName{徐尚}、\ProperName{蘇秦}、\ProperName{杜赫}之屬爲之謀,\ProperName{齊明}、\ProperName{周最}、\ProperName{陳軫}、\ProperName{召滑}、\ProperName{樓緩}、\ProperName{翟景}、\ProperName{蘇厲}、\ProperName{樂毅}之徒通其意,\ProperName{吳起}、\ProperName{孫臏}、\ProperName{帶佗}、\ProperName{兒良}、\ProperName{王廖}、\ProperName{田忌}、\ProperName{廉頗}、\ProperName{趙奢}之倫制其兵。嘗以什倍之地,百萬之眾,叩關而攻\ProperName{秦}。\ProperName{秦}人開關而延敵,九國之師遁逃而不敢進。\ProperName{秦}無亡矢遺鏃之費,而天下諸侯已困矣。於是從散約解,爭割地而賂\ProperName{秦}。\ProperName{秦}有餘力而制其弊,追亡逐北,伏尸百萬,流血漂櫓。因利乘便,宰割天下,分裂河山,彊國請服,弱國入朝。

施及\ProperName{孝文王}、\ProperName{莊襄王},享國日淺,國家無事。及至\ProperName{始皇},奮六世之餘烈,振長策而御宇內,吞\ProperName{二周}而亡諸侯,履至尊而制六合,執敲扑以鞭笞天下,威振四海。南取\ProperName{百越}之地,以爲\ProperName{桂林}、\ProperName{象郡},\ProperName{百越}之君俛首係頸,委命下吏。乃使\ProperName{蒙恬}北築長城而守藩籬,卻\ProperName{匈奴}七百餘里,\ProperName{胡}人不敢南下而牧馬,士不敢彎弓而報怨。

於是廢先王之道,燔百家之言,以愚黔首。隳名城,殺豪俊,收天下之兵聚之\ProperName{咸陽},銷鋒鍉鑄以爲金人十二,以弱天下之民。然後踐\ProperName{華}爲城,因\ProperName{河}爲池,據億丈之城,臨不測之谿以爲固。良將勁弩守要害之處,信臣精卒陳利兵而誰何?天下以定,\ProperName{始皇}之心,自以爲\ProperName{關中}之固,金城千里,子孫帝王萬世之業也。

\ProperName{秦王}既沒,餘威震於殊俗。然而\ProperName{陳涉},罋牖繩樞之子,氓隸之人,而遷徙之徒也,材能不及中庸,非有\ProperName{仲尼}、\ProperName{墨翟}之賢,\ProperName{陶朱}、\ProperName{猗頓}之富,躡足行伍之間,俛起阡陌之中,率罷弊之卒,將數百之眾,而轉攻\ProperName{秦}。斬木爲兵,揭竿爲旗,天下雲集而響應,贏糧而景從,\ProperName{山東}豪俊遂並起而亡\ProperName{秦}族矣。

且夫天下非小弱也,\ProperName{雍州}之地,\ProperName{殽}\ProperName{函}之固自若也。\ProperName{陳涉}之位,不尊於\ProperName{齊}、\ProperName{楚}、\ProperName{燕}、\ProperName{趙}、\ProperName{韓}、\ProperName{魏}、\ProperName{宋}、\ProperName{衛}、\ProperName{中山}之君也;鋤櫌棘矜,不銛於句戟長鎩也;謫戍之眾,非抗於九國之師也;深謀遠慮,行軍用兵之道,非及曩時之士也。然而成敗異變,功業相反。試使\ProperName{山東}之國與\ProperName{陳涉}度長絜大,比權量力,則不可同年而語矣。然\ProperName{秦}以區區之地,致萬乘之權,招八州而朝同列,百有餘年矣。然後以六合爲家,\ProperName{殽}\ProperName{函}爲宮,一夫作難而七廟隳,身死人手,爲天下笑者,何也?仁義不施而攻守之勢異也。

\section[賈誼治安策一\quad{\small 西漢文}]{{\normalsize 西漢文\ 賈誼}\quad 治安策一}
夫樹國固必相疑之勢,下數被其殃,上數爽其憂,甚非所以安上而全下也。今或親弟謀爲東帝,親兄之子西鄉而擊,今\ProperName{吳}又見告矣。天子春秋鼎盛,行義未過,德澤有加焉,猶尚如是,況莫大諸侯,權力且十此者虖!

然而天下少安,何也?大國之王幼弱未壯,\ProperName{漢}之所置傅相方握其事。數年之後,諸侯之王大抵皆冠,血氣方剛,\ProperName{漢}之傅相稱病而賜罷,彼自丞尉以上徧置私人,如此,有異\ProperName{淮南}、\ProperName{濟北}之爲邪!此時而欲爲治安,雖\ProperName{堯}\ProperName{舜}不治。

\ProperName{黃帝}曰:「日中必熭,操刀必割。」今令此道順而全安,甚易,不肯早爲,已乃墮骨肉之屬而抗剄之,豈有異\ProperName{秦}之季世虖!夫以天子之位,乘今之時,因天之助,尚憚以危爲安,以亂爲治,假設陛下居\ProperName{齊桓}之處,將不合諸侯而匡天下乎?臣又以知陛下有所必不能矣。假設天下如曩時,\ProperName{淮陰侯}尚王\ProperName{楚},\ProperName{黥布}王\ProperName{淮南},\ProperName{彭越}王\ProperName{梁},\ProperName{韓信}王\ProperName{韓},\ProperName{張敖}王\ProperName{趙},\ProperName{貫高}爲相,\ProperName{盧綰}王\ProperName{燕},\ProperName{陳豨}在\ProperName{代},令此六七公者皆亡恙,當是時而陛下即天子位,能自安乎?臣有以知陛下之不能也。天下殽亂,\ProperName{高皇帝}與諸公併起,非有仄室之勢以豫席之也。諸公幸者,乃爲中涓,其次廑得舍人,材之不逮至遠也。\ProperName{高皇帝}以明聖威武即天子位,割膏腴之地以王諸公,多者百餘城,少者乃三四十縣,惪至渥也,然其後十\endnote{\BookTitle{觀止}作「七」,據\BookTitle{漢書}改。}年之間,反者九起。陛下之與諸公,非親角材而臣之也,又非身封王之也,自\ProperName{高皇帝}不能以是一歲爲安,故臣知陛下之不能也。

然尚有可諉者,曰疏,臣請試言其親者。假令\ProperName{悼惠王}王\ProperName{齊},\ProperName{元王}王\ProperName{楚},中子王\ProperName{趙},\ProperName{幽王}王\ProperName{淮陽},\ProperName{共王}王\ProperName{梁},\ProperName{靈王}王\ProperName{燕},\ProperName{厲王}王\ProperName{淮南},六七貴人皆亡恙,當是時陛下即位,能爲治虖?臣又知陛下之不能也。若此諸王,雖名爲臣,實皆有布衣昆弟之心,慮亡不帝制而天子自爲者。擅爵人,赦死辠,甚者或戴黃屋,\ProperName{漢}法令非行也。雖行不軌如\ProperName{厲王}者,令之不肯聽,召之安可致乎!幸而來至,法安可得加!動一親戚,天下圜視而起,陛下之臣雖有悍如\ProperName{馮敬}者,適啓其口,匕首已陷其匈\endnote{\BookTitle{觀止}作「胸」,據\BookTitle{漢書}改。}矣。陛下雖賢,誰與領此?故疏者必危,親者必亂,已然之效也。其異姓負彊而動者,\ProperName{漢}已幸勝之矣,又不易其所以然。同姓襲是跡而動,既有徵矣,其勢盡又復然。殃旤之變,未知所移,明帝處之尚不能以安,後世將如之何!

屠牛\ProperName{坦}一朝解十二牛,而芒刃不頓者,所排擊剝割,皆衆理解也。至於髖髀之所,非斤則斧。夫仁義恩厚,人主之芒刃也;權勢法制,人主之斤斧也。今諸侯王皆衆髖髀也,釋斤斧之用,而欲嬰以芒刃,臣以爲不缺則折。胡不用之\ProperName{淮南}、\ProperName{濟北}?勢不可也。

臣竊跡前事,大抵彊者先反。\ProperName{淮陰}王\ProperName{楚}最彊,則最先反;\ProperName{韓信}倚\ProperName{胡},則又反;\ProperName{貫高}因\ProperName{趙}資,則又反;\ProperName{陳豨}兵精,則又反;\ProperName{彭越}用\ProperName{梁},則又反;\ProperName{黥布}用\ProperName{淮南},則又反;\ProperName{盧綰}最弱,最後反。\ProperName{長沙}乃在二萬五千戶耳,功少而最完,勢疏而最忠,非獨性異人也,亦形勢然也。曩令\ProperName{樊}、\ProperName{酈}、\ProperName{絳}、\ProperName{灌}據數十城而王,今雖以殘亡可也;令\ProperName{信}、\ProperName{越}之倫列爲徹侯而居,雖至今存可也。

然則天下之大計可知已。欲諸王之皆忠附,則莫若令如\ProperName{長沙王};欲臣子之勿葅醢,則莫若令如\ProperName{樊}、\ProperName{酈}等;欲天下之治安,莫若衆建諸侯而少其力。力少則易使以義,國小則亡邪心。令海內之勢如身之使臂,臂之使指,莫不制從,諸侯之君不敢有異心,輻湊並進而歸命天子,雖在細民,且知其安,故天下咸知陛下之明。割地定制,令\ProperName{齊}、\ProperName{趙}、\ProperName{楚}各爲若干國,使\ProperName{悼惠王}、\ProperName{幽王}、\ProperName{元王}之子孫畢以次各受祖之分地,地盡而止,及\ProperName{燕}、\ProperName{梁}它\endnote{\BookTitle{觀止}作「他」,據\BookTitle{漢書}改。}國皆然。其分地衆而子孫少者,建以爲國,空而置之,須其子孫生者,舉使君之。諸侯之地其削頗入\ProperName{漢}者,爲徙其侯國及封其子孫也,所以數償之;一寸之地,一人之衆,天子亡所利焉,誠以定治而已,故天下咸知陛下之廉。地制壹定,宗室子孫莫慮不王,下無倍畔之心,上無誅伐之志,故天下咸知陛下之仁。法立而不犯,令行而不逆,\ProperName{貫高}、\ProperName{利幾}之謀不生,\ProperName{柴奇}、\ProperName{開章}之計不萌,細民鄉善,大臣致順,故天下咸知陛下之義。臥赤子天下之上而安,植遺腹,朝委裘,而天下不亂,當時大治,後世誦聖。壹動而五業附,陛下誰憚而久不爲此?

天下之勢方病大瘇。一脛之大幾如要,一指之大幾如股,平居不可屈信,一二指搐,身慮亡\endnote{\BookTitle{觀止}作「無」,據\BookTitle{漢書}改。}聊。失今不治,必爲錮疾,後雖有\ProperName{扁鵲},不能爲已。病非徒瘇也,又苦{\fontfamily{songext}\selectfont 𨂂}盭。\ProperName{元王}之子,帝之從弟也;今之王者,從弟之子也。\ProperName{惠王},親兄子也;今之王者,兄子之子也。親者或亡分地以安天下,疏者或制大權以偪天子,臣故曰非徒病瘇也,又苦{\fontfamily{songext}\selectfont 𨂂}盭。可痛哭者,此病是也。 

\theendnotes

\section[鼂錯論貴粟疏\quad{\small 西漢文}]{{\normalsize 西漢文\ 鼂錯}\quad 論貴粟疏}
聖王在上而民不凍饑者,非能耕而食之,織而衣之也,爲開其資財之道也。故\ProperName{堯}、\ProperName{禹}有九年之水,\ProperName{湯}有七年之旱,而國亡\endnote{\BookTitle{觀止}作「無」,據\BookTitle{漢書}改。下文「四方亡擇」、「亡日休息」、「出於口而亡窮」同。}捐瘠者,以畜積多而備先具也。今海內爲一,土地人民之眾不避\ProperName{湯}、\ProperName{禹}\endnote{\BookTitle{觀止}倒作「禹湯」,據\BookTitle{漢書}改。},加以亡天災數年之水旱,而畜積未及者,何也?地有遺\endnote{\BookTitle{觀止}涉下作「餘」,據\BookTitle{漢書}改。}利,民有餘力,生穀之土未盡墾,山澤之利未盡出也,游食之民未盡歸農也。民貧,則姦邪生。貧生於不足,不足生於不農,不農則不地著,不地著則離鄉輕家,民如鳥獸,雖有高城深池,嚴法重刑,猶不能禁也。

夫寒之於衣,不待輕煖;饑之於食,不待甘旨;饑寒至身,不顧廉恥。人情,一日不再食則饑,終歲不製衣則寒。夫腹饑不得食,膚寒不得衣,雖慈母不能保其子,君安能以有其民哉!明主知其然也,故務民於農桑,薄賦斂,廣畜積,以實倉廩,備水旱,故民可得而有也。

民者,在上所以牧之,趨利如水走下,四方亡擇也。夫珠玉金銀,饑不可食,寒不可衣,然而眾貴之者,以上用之故也。其爲物輕微易臧\endnote{\BookTitle{觀止}作「藏」,據\BookTitle{漢書}改。下文「秋穫冬臧」同。},在於把握,可以周海內而亡饑寒之患。此令臣輕背其主,而民易去其鄉,盜賊有所勸,亡逃者得輕資也。粟米布帛生於地,長於時,聚於力,非可一日成也;數石之重,中人弗勝,不爲姦邪所利,一日弗得而饑寒至。是故明君貴五穀而賤金玉。

今農夫五口之家,其服役者不下二人,其能耕者不過百畮,百畮之收不過百石。春耕夏耘,秋穫冬臧,伐薪樵,治官府,給繇役;春不得避風塵,夏不得避暑熱,秋不得避陰雨,冬不得避寒凍,四時之間亡日休息;又私自送往迎來,弔死問疾,養孤長幼在其中。勤苦如此,尚復被水旱之災,急政暴虐,賦斂不時,朝令而暮改。當具有者半賈而賣,亡者取倍稱之息,於是有賣田宅鬻子孫以償責者矣。而商賈大者積貯倍息,小者坐列販賣,操其奇贏,日游都市,乘上之急,所賣必倍。故其男不耕耘,女不蠶織,衣必文采,食必粱肉;亡農夫之苦,有仟伯\endnote{\BookTitle{觀止}作「阡陌」,據\BookTitle{漢書}改。}之得。因其富厚,交通王侯,力過吏勢,以利相傾;千里游敖,冠蓋相望,乘堅策肥,履絲曳縞。此商人所以兼幷農人,農人所以流亡者也。

今法律賤商人,商人已富貴矣;尊農夫,農夫已貧賤矣。故俗之所貴,主之所賤也;吏之所卑,法之所尊也。上下相反,好惡乖迕,而欲國富法立,不可得也。

方今之務,莫若使民務農而已矣。欲民務農,在於貴粟;貴粟之道,在於使民以粟爲賞罰。今募天下入粟縣官,得以拜爵,得以除罪。如此,富人有爵,農民有錢,粟有所渫。夫能入粟以受爵,皆有餘者也;取於有餘,以供上用,則貧民之賦可損,所謂損有餘補不足,令出而民利者也。順於民心,所補者三:一曰主用足,二曰民賦少,三曰勸農功。今令民有車騎馬一匹者,復卒三人。車騎者,天下武備也,故爲復卒。\ProperName{神農}之教曰:「有石城十仞,湯池百步,帶甲百萬,而亡粟,弗能守也。」以是觀之,粟者,王者大用,政之本務。令民入粟受爵至五大夫以上,乃復一人耳,此其與騎馬之功相去遠矣。爵者,上之所擅,出於口而亡窮;粟者,民之所種,生於地而不乏。夫得高爵與免罪,人之所甚欲也。使天下人入粟於邊,以受爵免罪,不過三歲,塞下之粟必多矣。

\theendnotes

\section[鄒陽獄中上梁王書\quad{\small 西漢文}]{{\normalsize 西漢文\ 鄒陽}\quad 獄中上\ProperName{梁王}書}
\ProperName{鄒陽}從\ProperName{梁孝王}游。\ProperName{陽}爲人有智略,忼慨不苟合,介於\ProperName{羊勝}、\ProperName{公孫詭}之間。\ProperName{勝}等疾\ProperName{陽},惡之\ProperName{孝王}。\ProperName{孝王}怒,下\ProperName{陽}吏,將殺之。\ProperName{陽}迺從獄中上書,曰\endnote{\BookTitle{漢書}原作:「是時,\ProperName{景帝}少弟\ProperName{梁孝王}貴盛,亦待士。於是\ProperName{鄒陽}、\ProperName{枚乘}、\ProperName{嚴忌}知\ProperName{吳}不可說,皆去之\ProperName{梁},從\ProperName{孝王}游。\ProperName{陽}爲人有智略,忼慨不苟合,介於\ProperName{羊勝}、\ProperName{公孫詭}之間。\ProperName{勝}等疾\ProperName{陽},惡之\ProperName{孝王}。\ProperName{孝王}怒,下\ProperName{陽}吏,將殺之。\ProperName{陽}客游以讒見禽,恐死而負絫,乃從獄中上書曰。」}:

\begin{quotation}
臣聞忠無不報,信不見疑,臣常以爲然,徒虛語耳。昔\ProperName{荊軻}慕\ProperName{燕丹}之義,白虹貫日,太子畏之。\ProperName{衛先生}爲\ProperName{秦}畫\ProperName{長平}之事,\ProperName{太白}食\ProperName{昴},\ProperName{昭王}疑之。夫精變天地,而信不諭兩主,豈不哀哉?今臣盡忠竭誠,畢議願知,左右不明,卒從吏訊,爲世所疑。是使\ProperName{荊軻}、\ProperName{衛先生}復起,而\ProperName{燕}、\ProperName{秦}不寤也。願大王孰察之。

昔玉人獻寶,\ProperName{楚王}誅之;\ProperName{李斯}竭忠,\ProperName{胡亥}極刑。是以\ProperName{箕子}陽狂,\ProperName{接輿}避世,恐遭此患也。願大王察玉人、\ProperName{李斯}之意,而後\ProperName{楚王}、\ProperName{胡亥}之聽,毋使臣爲\ProperName{箕子}、\ProperName{接輿}所笑。臣聞\ProperName{比干}剖心,\ProperName{子胥}鴟夷,臣始不信,乃今知之。願大王孰察,少加憐焉!

語曰「有白頭如新,傾蓋如故。」何則?知與不知也。故\ProperName{樊於期}逃\ProperName{秦}之\ProperName{燕},藉\ProperName{荊軻}首以奉\ProperName{丹}事;\ProperName{王奢}去\ProperName{齊}之\ProperName{魏},臨城自剄以卻\ProperName{齊}而存\ProperName{魏}。夫\ProperName{王奢}、\ProperName{樊於期}非新於\ProperName{齊}、\ProperName{秦},而故於\ProperName{燕}、\ProperName{魏}也,所以去二國,死兩君者,行合於志,慕義無窮也。是以\ProperName{蘇秦}不信於天下,爲\ProperName{燕}\ProperName{尾生};\ProperName{白圭}戰亡六城,爲\ProperName{魏}取\ProperName{中山}。何則?誠有以相知也。\ProperName{蘇秦}相\ProperName{燕},人惡之\ProperName{燕王},\ProperName{燕王}按劍而怒,食以駃騠;\ProperName{白圭}顯於\ProperName{中山},人惡之於\ProperName{魏文侯},\ProperName{文侯}賜以夜光之璧。何則?兩主二臣,剖心析肝相信,豈移於浮辭哉?

故女無美惡,入宮見妒;士無賢不肖,入朝見嫉。昔\ProperName{司馬喜}臏腳於\ProperName{宋},卒相\ProperName{中山};\ProperName{范睢}拉脅折齒於\ProperName{魏},卒爲\ProperName{應侯}。此二人者,皆信必然之畫,捐朋黨之私,挾孤獨之交,故不能自免於嫉妒之人也。是以\ProperName{申徒狄}蹈雍之\ProperName{河},\ProperName{徐衍}負石入海。不容於世,義不苟取比周於朝以移主上之心。故\ProperName{百里奚}乞食於道路,\ProperName{繆公}委之以政;\ProperName{甯戚}飯牛車下,\ProperName{桓公}任之以國。此二人者,豈素宦於朝,借譽於左右,然後二主用之哉?感於心,合於行,堅如膠桼,昆弟不能離,豈惑於衆口哉?

故偏聽生姦,獨任成亂。昔\ProperName{魯}聽\ProperName{季孫}之說逐\ProperName{孔子};\ProperName{宋}任\ProperName{子冉}之計囚\ProperName{墨翟}。夫以\ProperName{孔}、\ProperName{墨}之辯,不能自免於讒諛,而二國以危,何則?衆口鑠金,積毀銷骨也。\ProperName{秦}用\ProperName{戎}人\ProperName{由余},而伯中國;\ProperName{齊}用\ProperName{越}人\ProperName{子臧},而彊\ProperName{威}、\ProperName{宣}。此二國豈係於俗,牽於世,繫奇偏之浮辭哉?公聽並觀,垂明當世。故意合則\ProperName{胡}\endnote{\BookTitle{觀止}作「吳」,據\BookTitle{漢書}改。}\ProperName{越}爲兄弟,\ProperName{由余}、\ProperName{子臧}是矣;不合則骨肉爲讎敵,\ProperName{朱}、\ProperName{象}、\ProperName{管}、\ProperName{蔡}是矣。今人主誠能用\ProperName{齊}、\ProperName{秦}之明,後\ProperName{宋}、\ProperName{魯}之聽,則\ProperName{五伯}不足侔,而\ProperName{三王}易爲也。

是以聖王覺寤,捐\ProperName{子之}之心,而不說\ProperName{田常}之賢,封\ProperName{比干}之後,修孕婦之墓,故功業覆於天下。何則?欲善亡厭也。夫\ProperName{晉文}親其讎,彊伯諸侯;\ProperName{齊桓}用其仇,而一匡天下。何則?慈仁殷勤,誠加於心,不可以虛辭借也。

至夫\ProperName{秦}用\ProperName{商鞅}之法,東弱\ProperName{韓}、\ProperName{魏},立彊天下,卒車裂之。\ProperName{越}用\ProperName{大夫種}之謀,禽勁\ProperName{吳}而伯中國,遂誅其身。是以\ProperName{孫叔敖}三去相而不悔,\ProperName{於陵子仲}辭三公爲人灌園。今人主誠能去驕傲之心,懷可報之意,披心腹,見情素,墮肝膽,施德厚,終與之窮達,無愛於士。則\ProperName{桀}之犬可使吠\ProperName{堯},\ProperName{跖}之客可使刺\ProperName{由},何況因萬乘之權,假聖王之資乎?然則\ProperName{荊軻}湛七族,\ProperName{要離}燔妻子,豈足爲大王道哉?

臣聞明月之珠,夜光之璧,以闇投人於道,衆莫不按劍相眄者。何則?無因而至前也。蟠木根柢,輪囷離奇,而爲萬乘器者,以左右先爲之容也。故無因而至前,雖出\ProperName{隨}珠\ProperName{和}璧,秪怨結而不見德;有人先游,則枯木朽株,樹功而不忘。今夫天下布衣窮居之士,身在貧羸,雖蒙\ProperName{堯}、\ProperName{舜}之術,挾\ProperName{伊}、\ProperName{管}之辯;懷\ProperName{龍逢}、\ProperName{比干}之意,而素無根柢之容;雖竭\endnote{\BookTitle{觀止}作「極」,據\BookTitle{漢書}改。}精神,欲開忠於當世之君,則人主必襲按劍相眄之迹矣。是使布衣之士不得爲枯木朽株之資也。

是以聖王制世御俗,獨化於陶鈞之上,而不牽乎卑辭之語,不奪乎衆多之口,故\ProperName{秦皇帝}任中庶子\ProperName{蒙嘉}之言,以信\ProperName{荊軻},而匕首竊發;\ProperName{周文王}獵\ProperName{涇}\ProperName{渭},載\ProperName{呂尚}歸,以王天下。\ProperName{秦}信左右而亡,\ProperName{周}用烏集而王。何則?以其能越攣拘之語,馳域外之議,獨觀乎昭曠之道也。

今人主沈諂諛之辭,牽帷廧之制,使不羈之士與牛驥同皁,此\ProperName{鮑焦}所以憤於世也。

臣聞盛飾入朝者不以私汙義,底厲名號者不以利傷行。故里名\ProperName{勝母},\ProperName{曾子}不入;邑號\ProperName{朝歌},\ProperName{墨子}回車。今欲使天下寥廓之士,籠於威重之權,脅於位勢之貴,回面汙行,以事諂諛之人,而求親近於左右,則士有伏死堀穴巖藪之中耳,安有盡忠信而趨闕下者哉?
\end{quotation}
\vspace{-1em}
\theendnotes

\section[司馬相如上書諫獵\quad{\small 西漢文}]{{\normalsize 西漢文\ 司馬相如}\quad 上書諫獵}
\ProperName{相如}\endnote{\BookTitle{史記}、\BookTitle{漢書}無「相如」二字,\BookTitle{觀止}據意增。}從上至\ProperName{長楊}獵。是時天子方好自擊熊豕,馳逐野獸,\ProperName{相如}因上疏諫,曰\endnote{\BookTitle{史記}、\BookTitle{漢書}「曰」上有「其辭」。}:

\begin{quotation}
臣聞物有同類而殊能者,故力稱\ProperName{烏獲},捷言\ProperName{慶忌},勇期\ProperName{賁}\ProperName{育}。臣之愚,竊以爲人誠有之,獸亦宜然。今陛下好陵阻險,射猛獸,卒然遇逸材之獸,駭不存之地,犯屬車之清塵,輿不及還轅,人不暇施巧,雖有\ProperName{烏獲}、\ProperName{逢蒙}之技不能用,枯木朽株盡爲難矣。是\ProperName{胡}\ProperName{越}起於轂下,而\ProperName{羌}\ProperName{夷}接軫也,豈不殆哉!雖萬全而無患,然本非天子之所宜近也。

且夫清道而後行,中路而馳,猶時有銜橛之變。況乎涉豐草,騁丘\endnote{\BookTitle{觀止}避諱作「邱」,改回。}虛,前有利獸之樂,而內無存變之意,其爲害也不亦難矣!夫輕萬乘之重不以爲安,樂出萬有一危之塗以爲娛,臣竊爲陛下不取。

蓋明者遠見於未萌,而知者避危於無形,旤固多藏於隱微而發於人之所忽者也。故鄙諺曰:「家絫千金,坐不垂堂。」此言雖小,可以喻大。臣願陛下留意幸察。
\end{quotation}
\vspace{-1em}
\theendnotes

\section[李陵答蘇武書\quad{\small 西漢文}]{{\normalsize 西漢文\ 李陵}\quad 答\ProperName{蘇武}書}
\ProperName{子卿}足下:勤宣令德,策名清時,榮問休暢。幸甚幸甚!遠託異國,昔人所悲,望風懷想,能不依依!昔者不遺,遠辱還答,慰誨勤勤,有踰骨肉。\ProperName{陵}雖不敏,能不慨然!

自從初降,以至今日,身之窮困,獨坐愁苦,終日無覩,但見異類,韋韝毳幕,以禦風雨,羶肉酪漿,以充飢渴。舉目言笑,誰與爲歡?\ProperName{胡}地玄冰,邊土慘裂,但聞悲風蕭條之聲。涼秋九月,塞外草衰,夜不能寐,側耳遠聽,胡笳互動,牧馬悲鳴,吟嘯成羣,邊聲四起。晨坐聽之,不覺淚下。嗟乎\ProperName{子卿}!\ProperName{陵}獨何心,能不悲哉!

與子別後,益復無聊。上念老母,臨年被戮;妻子無辜,並爲鯨鯢。身負國恩,爲世所悲,子歸受榮;我留受辱,命也如何!身出禮義之鄉,而入無知之俗,違棄君親之恩,長爲蠻夷之域,傷已!令先君之嗣,更成戎狄之族,又自悲矣!功大罪小,不蒙明察,孤負\ProperName{陵}心區區之意,每一念至,忽然忘生。\ProperName{陵}不難刺心以自明,刎頸以見志;顧國家於我已矣,殺身無益,適足增羞,故每攘臂忍辱,輒復苟活。左右之人,見\ProperName{陵}如此,以爲不入耳之歡,來相勸勉。異方之樂,秖\endnote{\BookTitle{觀止}及各集作「秪」,據\BookTitle{文選}改。}令人悲,增忉怛耳!

嗟乎\ProperName{子卿}!人之相知,貴相知心。前書倉卒,未盡所懷,故復略而言之:昔先帝授\ProperName{陵}步卒五千,出征絕域,五將失道,\ProperName{陵}獨遇戰。而裹萬里之糧,帥徒步之師,出天\ProperName{漢}之外,入彊\ProperName{胡}之域。以五千之衆,對十萬之軍,策疲乏之兵,當新羈之馬。然猶斬將搴旗,追奔逐北,滅跡掃塵,斬其梟帥。使三軍之士,視死如歸。\ProperName{陵}也不才,希當大任,意謂此時,功難堪矣。\ProperName{匈奴}既敗,舉國興師,更練精兵,彊踰十萬。單于臨陣,親自合圍。客主之形,既不相如;步馬之勢,又甚懸絕。疲兵再戰,一以當千,然猶扶乘創痛,決命爭首,死傷積野,餘不滿百,而皆扶病,不任干戈。然\ProperName{陵}振臂一呼,創病皆起,舉刃指虜,\ProperName{胡}馬奔走。兵盡矢窮,人無尺鐵,猶復徒首奮呼,爭爲先登。當此時也,天地爲\ProperName{陵}震怒,戰士爲\ProperName{陵}飲血。單于謂\ProperName{陵}不可復得,便欲引還。而賊臣教之,遂使復戰。故\ProperName{陵}不免耳。

昔\ProperName{高皇帝}以三十萬衆,困於\ProperName{平城}。當此之時,猛將如雲,謀臣如雨,然猶七日不食,僅乃得免。況當\ProperName{陵}者,豈易爲力哉?而執事者云云,苟怨\ProperName{陵}以不死。然\ProperName{陵}不死,罪也;\ProperName{子卿}視\ProperName{陵},豈偷生之士,而惜死之人哉?寧有背君親,捐妻子,而反爲利者乎?然\ProperName{陵}不死,有所爲也,故欲如前書之言,報恩於國主耳。誠以虛死不如立節,滅名不如報德也。昔\ProperName{范蠡}不殉\ProperName{會稽}之恥,\ProperName{曹沬}不死三敗之辱,卒復\ProperName{句踐}之讎,報\ProperName{魯國}之羞。區區之心,切慕此耳。何圖志未立而怨已成,計未從而骨肉受刑,此\ProperName{陵}所以仰天椎心而泣血也!

足下又云:「\ProperName{漢}與功臣不薄。」子爲\ProperName{漢}臣,安得不云爾乎?昔\ProperName{蕭}、\ProperName{樊}囚縶,\ProperName{韓}、\ProperName{彭}葅醢,\ProperName{鼂錯}受戮,\ProperName{周}、\ProperName{魏}見辜,其餘佐命立功之士,\ProperName{賈誼}、\ProperName{亞夫}之徒,皆信命世之才,抱將相之具,而受小人之讒,並受禍敗之辱,卒使懷才受謗,能不得展。彼二子之遐舉,誰不爲之痛心哉?\ProperName{陵}先將軍,功略蓋天地,義勇冠三軍,徒失貴臣之意,剄身絕域之表。此功臣義士所以負戟而長歎者也!何謂不薄哉?

且足下昔以單車之使,適萬乘之虜,遭時不遇,至於伏劍不顧,流離辛苦,幾死朔北之野。丁年奉使,皓首而歸。老母終堂,生妻去帷。此天下所希聞,古今所未有也。蠻貊之人,尚猶嘉子之節,況爲天下之主乎?\ProperName{陵}謂足下,當享茅土之薦,受千乘之賞。聞子之歸,賜不過二百萬,位不過典屬國,無尺土之封,加子之勤。而妨功害能之臣,盡爲萬戶侯,親戚貪佞之類,悉爲廊廟宰。子尚如此,\ProperName{陵}復何望哉?

且\ProperName{漢}厚誅\ProperName{陵}以不死,薄賞子以守節,欲使遠聽之臣,望風馳命,此實難矣。所以每顧而不悔者也。\ProperName{陵}雖孤恩,\ProperName{漢}亦負德。昔人有言:「雖忠不烈,視死如歸。」\ProperName{陵}誠能安,而主豈復能眷眷乎?男兒生以不成名,死則葬蠻夷中,誰復能屈身稽顙,還向北闕,使刀筆之吏,弄其文墨邪?願足下勿復望\ProperName{陵}!

嗟乎\ProperName{子卿}!夫復何言!相去萬里,人絕路殊。生爲別世之人,死爲異域之鬼,長與足下,生死辭矣!幸謝故人,勉事聖君。足下胤子無恙,勿以爲念,努力自愛!時因北風,復惠德音。\ProperName{李陵}頓首。 

\theendnotes

\section[路溫舒尚德緩刑書\quad{\small 西漢文}]{{\normalsize 西漢文\ 路溫舒}\quad 尚德緩刑書書}
\ProperName{昭帝}崩\endnote{\BookTitle{漢書}「昭帝」上有「會」字,\BookTitle{觀止}刪。},\ProperName{昌邑王賀}廢,\ProperName{宣帝}初即位,\ProperName{路溫舒}\endnote{\BookTitle{漢書}「溫舒」上無「路」字,\BookTitle{觀止}增。}上書,言宜尚德緩刑。其辭曰:

\begin{quotation}
臣聞\ProperName{齊}有\ProperName{無知}之禍,而\ProperName{桓公}以興;\ProperName{晉}有\ProperName{驪姬}之難,而\ProperName{文公}用伯。近世\ProperName{趙王}不終,諸\ProperName{呂}作難,而\ProperName{孝文}爲\ProperName{大宗}。繇是觀之,禍亂之作,將以開聖人也。故\ProperName{桓}\ProperName{文}扶微興壞,尊\ProperName{文}\ProperName{武}之業,澤加百姓,功潤諸侯,雖不及\ProperName{三王},天下歸仁焉。\ProperName{文帝}永思至德,以承天心,崇仁義,省刑罰,通關梁,一遠近,敬賢如大賓,愛民如赤子,內恕情之所安,而施之於海內,是以囹圄空虛,天下太平。夫繼變化之後,必有異舊之恩,此賢聖所以昭天命也。

往者,\ProperName{昭帝}即世而無嗣,大臣憂戚,焦心合謀,皆以\ProperName{昌邑}尊親,援而立之。然天不授命,淫亂其心,遂以自亡。深察禍變之故,乃皇天之所以開至聖也。故大將軍受命\ProperName{武帝},股肱\ProperName{漢國},披肝膽,決大計,黜亡義,立有德,輔天而行,然後宗廟以安,天下咸寧。

臣聞\BookTitle{春秋}正即位,大一統而慎始也。陛下初登至尊,與天合符,宜改前世之失,正始受命之統,滌煩文,除民疾,存亡繼絕,以應天意。

臣聞\ProperName{秦}有十失,其一尚存,治獄之吏是也。\ProperName{秦}之時,羞文學,好武勇,賤仁義之士,貴治獄之吏;正言者謂之誹謗,遏過者謂之妖言。故盛服先生不用於世,忠良切言皆鬱於胸,譽諛之聲日滿於耳;虛美熏心,實禍蔽塞。此乃\ProperName{秦}之所以亡天下也。方今天下賴陛下恩厚,亡金革之危,飢寒之患,父子夫妻戮力安家,然太平未洽者,獄亂之也。

夫獄者,天下之大命也,死者不可復生,{\fontfamily{songext}\selectfont 𢇍}者不可復屬。\BookTitle{書}曰:「與其殺不辜,寧失不經。」今治獄吏則不然,上下相敺,以刻爲明;深者獲公名,平者多後患。故治獄之吏皆欲人死,非憎人也,自安之道在人之死。是以死人之血流離於市,被刑之徒比肩而立,大辟之計歲以萬數,此仁聖之所以傷也。太平之未洽,凡以此也。夫人情安則樂生,痛則思死。棰楚之下,何求而不得?故囚人不勝痛,則飾辭以視之;吏治者利其然,則指道以明之;上奏畏卻,則鍛練而周內之。蓋奏當之成,雖\ProperName{咎繇}聽之,猶以爲死有餘辜。何則?成練者眾,文致之罪明也。是以獄吏專爲深刻,殘賊而亡極,媮爲一切,不顧國患,此世之大賊也。故俗語曰:「畫地爲獄,議不入;刻木爲吏,期不對。」此皆疾吏之風,悲痛之辭也。故天下之患,莫深於獄;敗法亂正,離親塞道,莫甚乎治獄之吏。此所謂一尚存者也。

臣聞烏鳶之卵不毀,而後鳳凰集;誹謗之罪不誅,而後良言進。故古人有言:「山藪藏疾,川澤納汙,瑾瑜匿惡,國君含詬。」唯陛下除誹謗以招切言,開天下之口,廣箴諫之路,掃亡\ProperName{秦}之失,尊\ProperName{文}\ProperName{武}之悳,省法制,寬刑罰,以廢治獄,則太平之風可興於世,永履和樂,與天亡極,天下幸甚。
\end{quotation}

上善其言。

\theendnotes

\section[楊惲報孫會宗書\quad{\small 西漢文}]{{\normalsize 西漢文\ 楊惲}\quad 報\ProperName{孫會宗}書}
\ProperName{惲}既失爵位,家居治產業,起室宅,以財自娛。歲餘,其友人\ProperName{安定}太守\ProperName{西河}\ProperName{孫會宗},知略士也,與\ProperName{惲}書諫戒之,爲言大臣廢退,當闔門惶懼,爲可憐之意,不當治產業,通賓客,有稱譽。\ProperName{惲}宰相子,少顯朝廷,一朝晻昧語言見廢,內懷不服,報\ProperName{會宗}書曰:

\begin{quotation}
\ProperName{惲}材朽行穢,文質無所底,幸賴先人餘業得備宿衛,遭遇時變以獲爵位,終非其任,卒與禍會。足下哀其愚,蒙賜書,教督以所不及,殷勤甚厚。然竊恨足下不深惟\endnote{\BookTitle{觀止}作「推」,據\BookTitle{漢書}改。}其終始,而猥隨俗之毀譽也。言鄙陋之愚心,若逆指而文過,默而息乎,恐違\ProperName{孔氏}「各言爾志」之義,故敢略陳其愚,唯君子察焉!

\ProperName{惲}家方隆盛時,乘朱輪者十人,位在列卿,爵爲通侯,總領從官,與聞政事,曾不能以此時有所建明,以宣德化,又不能與羣僚同心幷力,陪輔朝廷之遺忘,已負竊位素餐之責久矣。懷祿貪勢,不能自退,遭遇變故,橫被口語,身幽北闕,妻子滿獄。當此之時,自以夷滅不足以塞責,豈意得全首領,復奉先人之丘\endnote{\BookTitle{觀止}避諱作「邱」,改回。}墓乎?伏惟聖主之恩,不可勝量。君子游道,樂以忘憂;小人全軀,說以忘罪。竊自思\endnote{\BookTitle{觀止}作「私」,據\BookTitle{漢書}改。}念,過已大矣,行已虧矣,長爲農夫以沒世矣。是故身率妻子,戮力耕桑,灌園治產,以給公上,不意當復用此爲譏議也。

夫人情所不能止者,聖人弗禁,故君父至尊親,送其終也,有時而既。臣之得罪,已三年矣。田家作苦,歲時伏臘,亨羊炰羔,斗酒自勞。家本\ProperName{秦}也,能爲\ProperName{秦}聲。婦,\ProperName{趙}女也,雅善鼓瑟。奴婢歌者數人,酒後耳熱,仰天拊缶而呼烏烏。其詩曰:「田彼南山,蕪穢不治,種一頃豆,落而爲萁。人生行樂耳,須富貴何時!」是日也,拂衣而喜,奮褎低卬,頓足起舞,誠淫荒無度,不知其不可也。\ProperName{惲}幸有餘祿,方糴賤販貴,逐什一之利,此賈豎之事,汙辱之處,\ProperName{惲}親行之。下流之人,眾毀所歸,不寒而栗。雖雅知\ProperName{惲}者,猶隨風而靡,尚何稱譽之有!\ProperName{董生}不云乎?「明明求仁義,常恐不能化民者,卿大夫意也;明明求財利,常恐困乏者,庶人之事也。」故「道不同,不相爲謀。」今子尚安得以卿大夫之制而責僕哉!

夫\ProperName{西河}\ProperName{魏}土,\ProperName{文侯}所興,有\ProperName{段干木}、\ProperName{田子方}之遺風,漂然皆有節槩,知去就之分。頃者,足下離舊土,臨\ProperName{安定},\ProperName{安定}山谷之間,\ProperName{昆戎}舊壤,子弟貪鄙,豈習俗之移人哉?於今乃睹子之志矣。方當盛漢之隆,願勉旃,毋多談。
\end{quotation}
\vspace{-1em}
\theendnotes

\section[光武帝臨淄勞耿弇\quad{\small 東漢文}]{{\normalsize 東漢文}\quad \ProperName{光武帝}\ProperName{臨淄}勞\ProperName{耿弇}}
車駕至\ProperName{臨淄}自勞軍,羣臣大會。帝謂\ProperName{弇}曰:「昔\ProperName{韓信}破\ProperName{歷下}以開基,今將軍攻\ProperName{祝阿}以發跡,此皆\ProperName{齊}之西界,功足相方。而\ProperName{韓信}襲擊已降,將軍獨拔勍敵,其功乃難於\ProperName{信}也。又\ProperName{田橫}亨\endnote{\BookTitle{觀止}作「烹」,據\BookTitle{後漢書}改。}\ProperName{酈生},及\ProperName{田橫}降,\ProperName{高帝}詔衛尉不聽爲仇。\ProperName{張步}前亦殺\ProperName{伏隆},若\ProperName{步}來歸命,吾當詔大司徒釋其怨,又事尤相類也。將軍前在\ProperName{南陽}建此大策,常以爲落落難合,有志者事竟成也!」

\theendnotes

\section[馬援誡兄子嚴敦書\quad{\small 東漢文}]{{\normalsize 東漢文\ 馬援}\quad 誡兄子\ProperName{嚴}\ProperName{敦}書}
\ProperName{援}兄子\ProperName{嚴}、\ProperName{敦}並喜譏議,而通輕俠客。\ProperName{援}前在\ProperName{交阯},還書誡之曰:「吾欲汝曹聞人過失,如聞父母之名,耳可得聞,口不可得言也。好論議人長短,妄是非正法,此吾所大惡也,寧死不願聞子孫有此行也。汝曹知吾惡之甚矣,所以復言者,施衿結褵,申父母之戒,欲使汝曹不忘之耳。\ProperName{龍伯高}敦厚周慎,口無擇言,謙約節儉,廉公有威,吾愛之重之,願汝曹效之。\ProperName{杜季良}豪俠好義,憂人之憂,樂人之樂,清濁無所失,父喪致客,數郡畢至,吾愛之重之,不願汝曹效也。效\ProperName{伯高}不得,猶爲謹敕之士,所謂刻鵠不成尚類鶩者也。效\ProperName{季良}不得,陷爲天下輕薄子,所謂畫虎不成反類狗者也。訖今\ProperName{季良}尚未可知,郡將下車輒切齒,州郡以爲言,吾常爲寒心,是以不願子孫效也。」

\section[諸葛亮前出師表\quad{\small 後漢文}]{{\normalsize 後漢文\ 諸葛亮}\quad 前出師表}
臣\ProperName{亮}言:先帝創業未半而中道崩殂,今天下三分,\ProperName{益州}疲弊\endnote{\BookTitle{觀止}作「敝」,據\BookTitle{蜀志}改。\BookTitle{文選}作「罷弊」,六臣本注:五臣作「疲敝」。},此誠危急存亡之秋也。然侍衞之臣不懈於內,忠志之士忘身於外者,蓋追先帝之殊遇,欲報之於陛下也。誠宜開張聖聽,以光先帝遺德,恢弘\endnote{\BookTitle{觀止}避諱作「宏」,據\BookTitle{蜀志}改。\BookTitle{文選}「恢」下無「弘」字。}志士之氣,不宜妄自菲薄,引喻失義,以塞忠諫之路也。宮中府中俱爲一體,陟罰臧否,不宜異同。若有作姦犯科及爲忠善者,宜付有司論其刑賞,以昭陛下平明之治,不宜偏私,使內外異法也。

侍中、侍郎\ProperName{郭攸之}、\ProperName{費禕}、\ProperName{董允}等,此皆良實,志慮忠純,是以先帝簡拔以遺陛下。愚以爲宮中之事,事無大小,悉以咨之,然後施行,必能裨補闕漏,有所廣益。將軍\ProperName{向寵},性行淑均,曉暢軍事,試用於昔日,先帝稱之曰能,是以衆議舉\ProperName{寵}爲督。愚以爲營中之事,悉以咨之\endnote{\BookTitle{觀止}「營中之事」下有「事無大小」,涉上衍,據\BookTitle{蜀志}、\BookTitle{文選}改。},必能使行陣和穆,優劣得所也。親賢臣,遠小人,此\ProperName{先漢}所以興隆也;親小人,遠賢臣,此\ProperName{後漢}所以傾頹也。先帝在時,每與臣論此事,未嘗不歎息痛恨於\ProperName{桓}、\ProperName{靈}也。侍中、尚書、長史、參軍,此悉貞亮死節之臣也,願陛下親之信之,則\ProperName{漢}室之隆,可計日而待也。

臣本布衣,躬耕於\ProperName{南陽},苟全性命於亂世,不求聞達於諸侯。先帝不以臣卑鄙,猥自枉屈,三顧臣於草廬之中,諮臣以當世之事,由是感激,遂許先帝以驅馳。後值傾覆,受任於敗軍之際,奉命於危難之間,爾來二十有一年矣。先帝知臣謹慎,故臨崩寄臣以大事也。受命以來,夙夜憂歎,恐託付不效,以傷先帝之明,故五月渡\ProperName{瀘},深入不毛。今南方已定,兵甲已足,當獎帥三軍,北定中原,庶竭駑鈍,攘除姦凶,興復\ProperName{漢}室,還于舊都。此臣所以報先帝,而忠陛下之職分也。

至於斟酌損益,進盡忠言,則\ProperName{攸之}、\ProperName{禕}、\ProperName{允}之任也。願陛下託臣以討賊興復之效;不效,則治臣之罪,以告先帝之靈。若無興德之言,則責\ProperName{攸之}、\ProperName{禕}、\ProperName{允}等之咎,以彰其慢。陛下亦宜自謀,以咨諏善道,察納雅\endnote{\BookTitle{觀止}作「人」,據\BookTitle{蜀志}、\BookTitle{文選}改。}言,深追先帝遺詔。臣不勝受恩感激,今當遠離,臨表涕零,不知所云。

\theendnotes

\section[諸葛亮後出師表\quad{\small 後漢文}]{{\normalsize 後漢文\ 諸葛亮\endnote{疑僞託。}}\quad 後出師表}
先帝慮\ProperName{漢}、賊不兩立,王業不偏安,故託臣以討賊也。以先帝之明,量臣之才,故知臣伐賊才弱敵彊也;然不伐賊,王業亦亡,惟坐而待亡,孰與伐之?是故託臣而弗疑也。臣受命之日,寢不安席,食不甘味,思惟北征,宜先入南,故五月渡\ProperName{瀘},深入不毛,幷日而食。臣非不自惜也,顧王業不得偏安於\ProperName{蜀都},故冒危難以奉先帝之遺意也\endnote{\BookTitle{觀止}下脱「也」字,據\ProperName{裴}注本補。},而議者謂爲非計。今賊適疲於西,又務於東,兵法乘勞,此進趨之時也。謹陳其事如左:

\ProperName{高帝}明並日月,謀臣淵深,然涉險被創,危然後安。今陛下未及\ProperName{高帝},謀臣不如\ProperName{良}、\ProperName{平},而欲以長計取勝,坐定天下,此臣之未解一也。

\ProperName{劉繇}、\ProperName{王朗}各據州郡,論安言計,動引聖人,羣疑滿腹,衆難塞胸,今歲不戰,明年不征,使\ProperName{孫策}坐大,遂幷\ProperName{江東},此臣之未解二也。

\ProperName{曹操}智計殊絕於人,其用兵也,髣髴\ProperName{孫}、\ProperName{吳},然困於\ProperName{南陽},險於\ProperName{烏巢},危於\ProperName{祁連},偪於\ProperName{黎陽},幾敗\ProperName{北山},殆死\ProperName{潼關},然後僞定一時耳,況臣才弱,而欲以不危而定之,此臣之未解三也。

\ProperName{曹操}五攻\ProperName{昌霸}不下,四越\ProperName{巢湖}不成,任用\ProperName{李服}而\ProperName{李服}圖之,委\ProperName{夏侯}而\ProperName{夏侯}敗亡,先帝每稱\ProperName{操}爲能,猶有此失,況臣駑下,何能必勝?此臣之未解四也。

自臣到\ProperName{漢中},中間朞年耳,然喪\ProperName{趙雲}、\ProperName{陽羣}、\ProperName{馬玉}、\ProperName{閻芝}、\ProperName{丁立}、\ProperName{白壽}、\ProperName{劉郃}、\ProperName{鄧銅}等及曲長屯將七十餘人,突將無前。\ProperName{賨}、\ProperName{叟}、\ProperName{青羌}散騎、武騎一千餘人,此皆數十年之內所糾合四方之精銳,非一州之所有,若復數年,則損三分之二也,當何以圖敵?此臣之未解五也。

今民窮兵疲,而事不可息,事不可息,則住與行勞費正等,而不及早圖之,欲以一州之地與賊持久,此臣之未解六也。

夫難平者,事也。昔先帝敗軍於\ProperName{楚},當此時,\ProperName{曹操}拊手,謂天下以定。然後先帝東連\ProperName{吳}、\ProperName{越},西取\ProperName{巴}、\ProperName{蜀},舉兵北征,\ProperName{夏侯}授首,此\ProperName{操}之失計而\ProperName{漢}事將成也。然後\ProperName{吳}更違盟,\ProperName{關羽}毀敗,\ProperName{秭歸}蹉跌,\ProperName{曹丕}稱帝。凡事如是,難可逆見\endnote{\BookTitle{觀止}作「料」,據\ProperName{裴}注本改。}。臣鞠躬盡力,死而後已,至於成敗利鈍,非臣之明所能逆覩也。

\theendnotes

% Proofed 16 July 2022
% Ref. 
% - 點校本漢書, 中華書局, 1962
% - 點校本後漢書, 中華書局, 1965
% - 點校本三國志, 中華書局, 1959
% - 文選, 上海古籍, 1982
% - 古文觀止, 中華書局, 1959
