\section[祭公諫征犬戎\quad{\small 國語 周語上}]{{\normalsize 國語\ 周語上}\quad \ProperName{祭公}諫征\ProperName{犬戎}}
\ProperName{穆王}將征\ProperName{犬戎},\ProperName{祭公}\ProperName{謀父}諫曰:

\begin{quotation}
不可。先王耀德不觀兵。夫兵戢而時動,動則威,觀則玩,玩則無震。是故\ProperName{周文公}之\BookTitle{頌}曰:「載戢干戈,載櫜弓矢。我求懿德,肆于時夏,允王保之。」先王之於民也,{茂}正其德而厚其性,阜其財求而利其器用,明利害之鄉,以文修之,使務利而避害,懷德而畏威,故能保世以滋大。

昔我先王世后稷,以服事\ProperName{虞}、\ProperName{夏}。及\ProperName{夏}之衰也,棄稷不務,我先王\ProperName{不窋}用失其官,而自竄於\ProperName{戎}、\ProperName{翟}之間,不敢怠業,時序其德,纂修其緒,修其訓典,朝夕恪勤,守以{惇}篤,奉以忠信,奕世載德,不忝前人。至于\ProperName{武王},昭前之光明而加之以慈和,事神保民,莫不欣喜。\ProperName{商王}\ProperName{帝辛},大惡於民。庶民弗忍,欣戴\ProperName{武王},以致戎于\ProperName{商牧}。是先王非務武也,勤恤民隱而除其害也。

夫先王之制:邦內甸服,邦外侯服,侯、衛賓服,\ProperName{蠻}、\ProperName{夷}要服,\ProperName{戎}、\ProperName{翟}荒服。甸服者祭,侯服者祀,賓服者享,要服者貢,荒服者王。日祭、月祀、時享、歲貢、終王,先王之訓也。有不祭則修意,有不祀則修言,有不享則修文,有不貢則修名,有不王則修德,序成而有不至則修刑。於是乎有刑不祭,伐不祀,征不享,讓不貢,告不王。於是乎有刑罰之辟,有攻伐之兵,有征討之備,有威讓之令,有文告之辭。布令陳辭而又不至,則又增修於德無勤民於遠,是以近無不聽,遠無不服。

今自\ProperName{大畢}、\ProperName{伯士}之終也,\ProperName{犬戎氏}以其職來王,天子曰:「予必以不享征之,且觀之兵。」其無乃廢先王之訓而王幾頓乎!吾聞夫\ProperName{犬戎}樹惇,能帥舊德而守終純固,其有以禦我矣!
\end{quotation}

王不聽,遂征之,得四白狼,四白鹿以歸。自是荒服者不至。

\section[召公諫厲王弭謗\quad{\small 國語 周語上}]{{\normalsize 國語\ 周語上}\quad \ProperName{召公}諫\ProperName{厲王}弭謗}
\ProperName{厲王}虐,國人謗王,\ProperName{召公}告曰:「民不堪命矣!」王怒,得\ProperName{衛}巫,使監謗者,以告,則殺之。國人莫敢言,道路以目。王喜,告\ProperName{召公}曰:「吾能弭謗矣,乃不敢言。」

\ProperName{召公}曰:「是障之也,防民之口,甚於防川。川壅而潰,傷人必多,民亦如之。是故爲川者決之使導,爲民者宣之使言。故天子聽政,使公卿至於列士獻詩,瞽獻{典},史獻書,師箴,瞍賦,矇誦,百工諫,庶人傳語,近臣盡規,親戚補察,瞽、史教誨,耆、艾修之,而後王斟酌焉,是以事行而不悖。民之有口,猶土之有山川也,財用於是乎出;猶其有原隰衍沃也,衣食於是乎生。口之宣言也,善敗於是乎興,行善而備敗,所以阜財用、衣食者也。夫民慮之於心而宣之於口,成而行之,胡可壅也?若壅其口,其與能幾何?」

王弗聽,於是國人莫敢出言。三年,乃流王於\ProperName{彘}。

\section[襄王不許請隧\quad{\small 國語 周語中}]{{\normalsize 國語\ 周語中}\quad \ProperName{襄王}不許請隧}
\ProperName{晉文公}既定\ProperName{襄王}於\ProperName{郟},王勞之以地,辭,請隧焉。王弗許,曰:「昔我先王之有天下也,規方千里以爲甸服,以供上帝山川百神之祀;以備百姓兆民之用,以待不庭不虞之患。其餘以均分公侯伯子男,使各有寧宇,以順及天地,無逢其災害。先王豈有賴焉。內官不過九御,外官不過九品,足以供給神祇而已,豈敢{猒}縱其耳目心腹以亂百度?亦唯是死生之服物采章,以臨長百姓而輕重布之,王何異之有?今天降禍災於\ProperName{周室},余一人僅亦守府,又不佞以勤叔父,而班先王之大物以賞私德。其叔父實應且憎,以非余一人,余一人豈敢有愛也?先民有言曰:『改玉改行。』叔父若能光裕大德,更姓改物,以創制天下,自顯庸也,而縮取備物以鎮撫百姓,余一人其流辟於裔土,何辭之與有?若由\endnote{\BookTitle{觀止}作「猶」,據\BookTitle{國語}各本改。\ProperName{徐元誥}\BookTitle{集解}:「由」與「猶」通用。}是\ProperName{姬}姓也,尚將列爲公侯,以復先王之職,大物其未可改也。叔父其茂昭明德,物將自至,余敢以私勞變前之大章,以忝天下。其若先王與百姓何?何政令之爲也?若不然,叔父有地而隧焉,余安能知之?」

\ProperName{文公}遂不敢請,受地而還。

\theendnotes

\section[單子知陳必亡\quad{\small 國語 周語中}]{{\normalsize 國語\ 周語中}\quad \ProperName{單子}知\ProperName{陳}必亡}
\ProperName{定王}使\ProperName{單襄公}聘於\ProperName{宋}。遂假道於\ProperName{陳},以聘於\ProperName{楚}。火朝覿矣,道茀不可行也,候不在疆,司空不視塗,澤不陂,川不梁,野有庾積,場功未畢,道無列樹,墾田若蓺,膳宰不致餼,司里不授館,國無寄寓,縣無施舍,民將築臺於\ProperName{夏氏}。及\ProperName{陳},\ProperName{陳靈公}與\ProperName{孔寧}、\ProperName{儀行父}南冠以如\ProperName{夏氏},留賓弗見。

\ProperName{單子}歸,告王曰:「\ProperName{陳侯}不有大咎,國必亡。」王曰:「何故?」對曰:

\begin{quotation}
夫辰角見而雨畢,天根見而水涸,本見而草木節解,駟見而隕霜,火見而清風戒寒。故先王之教曰:「雨畢而除道,水涸而成梁,草木節解而備藏,隕霜而冬裘具,清風至而修城郭宮室。」故\BookTitle{夏令}曰:「九月除道,十月成梁。」其時儆曰:「收而場功,偫而畚梮\endnote{\BookTitle{觀止}作「挶」,據\BookTitle{國語}校本改。\ProperName{汪遠孫}\BookTitle{考異}:\BookTitle{舊音}出「畚挶」,案\BookTitle{內傳}\BookTitle{襄九年}「陳畚挶」,\ProperName{唐}石經從木,\BookTitle{說文}「挶,{\fontfamily{songext}\selectfont 𢧢}持也。」非此義。},營室之中,土功其始。火之初見,期於司里。」此先王所以不用財賄,而廣施德於天下者也。今\ProperName{陳國}火朝覿矣,而道路若塞,野場若棄,澤不陂障,川無舟梁,是廢先王之教也。

\ProperName{周}制有之曰:「列樹以表道,立鄙食以守路。國有郊牧,畺有寓望,藪有圃草,囿有林池,所以禦災也。其餘無非穀土,民無懸耜,野無奧草。不奪民時,不蔑民功。有優無匱,有逸無罷。國有班事,縣有序民。」今\ProperName{陳國}道路不可知,田在草閒,功成而不收,民罷於逸樂,是棄先王之法制也。

\ProperName{周}之\BookTitle{秩官}有之曰:「敵國賓至,關尹以告,行理以節逆之,候人爲導,卿出郊勞,門尹除門,宗祝執祀,司里授館,司徒具徒,司空視塗,司寇詰姦,虞人入材,甸人積薪,火師監燎,水師監濯,膳宰致餐,廩人獻餼,司馬陳芻,工人展車,百官\endnote{\BookTitle{觀止}「百官」下有「各」字,據\BookTitle{國語}校本刪。}以物至,賓入如歸。是故小大莫不懷愛。其貴國之賓至,則以班加一等,益虔。至於王使,則皆官正蒞事,上卿監之。若王巡守,則君親監之。」今雖\ProperName{朝}也不才,有分族於\ProperName{周},承王命以爲過賓於\ProperName{陳},而司事莫至,是蔑先王之官也。

先王之令有之曰:「天道賞善而罰淫,故凡我造國,無從非\endnote{\BookTitle{觀止}作「匪」,從\BookTitle{國語}各本改。}彝,無即慆淫,各守爾典,以承天休。」今\ProperName{陳侯}不念胤續之常,棄其伉儷妃嬪,而帥其卿佐以淫於\ProperName{夏氏},不亦瀆姓矣乎?\ProperName{陳},我\ProperName{大姬}之後也。棄袞冕而南冠以出,不亦簡彝乎?是又犯先王之令也。昔先王之教,茂帥其德也,猶恐殞越。若廢其教而棄其制,蔑其官而犯其令,將何以守國?居大國之閒,而無此四者,其能久乎?
\end{quotation}

六年,\ProperName{單子}如\ProperName{楚}。八年,\ProperName{陳侯}殺於\ProperName{夏氏}。九年,\ProperName{楚子}入\ProperName{陳}。

\theendnotes

\section[展禽論祀爰居\quad{\small 國語 魯語上}]{{\normalsize 國語\ 魯語上}\quad \ProperName{展禽}論祀爰居}
海鳥曰「爰居」,止於\ProperName{魯}東門之外三日,\ProperName{臧文仲}使國人祭之。\ProperName{展禽}曰:

\begin{quotation}
越哉,\ProperName{臧孫}之爲政也!夫祀,國之大節也;而節,政之所成也。故慎制祀以爲國典。今無故而加典,非政之宜也。

夫聖王之制祀也,法施於民則祀之,以死勤事則祀之,以勞定國則祀之,能禦大災則祀之,能捍大患則祀之。非是族也,不在祀典。昔\ProperName{烈山氏}之有天下也,其子曰\ProperName{柱},能殖\endnote{\BookTitle{觀止}作「植」,據\BookTitle{國語}各本改。}百穀百蔬;\ProperName{夏}之興也,\ProperName{周棄}繼之,故祀以爲稷。\ProperName{共工氏}之伯九有也,其子曰\ProperName{后土},能平九土,故祀以爲社。\ProperName{黃帝}能成命百物,以明民共財,\ProperName{顓頊}能修之。\ProperName{帝嚳}能序三辰以固民,\ProperName{堯}能單均刑法以儀民,\ProperName{舜}勤民事而野死,\ProperName{鮌}鄣洪水而殛死,\ProperName{禹}能以德修\ProperName{鮌}之功,\ProperName{契}爲司徒而民輯,\ProperName{冥}勤其官而水死,\ProperName{湯}以寬治民而除其邪,\ProperName{稷}勤百穀而山死,\ProperName{文王}以文昭,\ProperName{武王}去民之穢。故\ProperName{有虞氏}禘\ProperName{黃帝}而祖\ProperName{顓頊},郊\ProperName{堯}而宗\ProperName{舜};\ProperName{夏后氏}禘\ProperName{黃帝}而祖\ProperName{顓頊},郊\ProperName{鮌}而宗\ProperName{禹};\ProperName{商}人禘\ProperName{舜}而祖\ProperName{契},郊\ProperName{冥}而宗\ProperName{湯};\ProperName{周}人禘\ProperName{嚳}而郊\ProperName{稷},祖\ProperName{文王}而宗\ProperName{武王};\ProperName{幕},能帥\ProperName{顓頊}者也,\ProperName{有虞氏}報焉;\ProperName{杼},能帥\ProperName{禹}者也,\ProperName{夏后氏}報焉;\ProperName{上甲微},能帥\ProperName{契}者也,\ProperName{商}人報焉;\ProperName{高圉}、\ProperName{大王},能帥\ProperName{稷}者也,\ProperName{周}人報焉。凡禘、郊、祖、宗、報,此五者國之典祀也。

加之以社稷山川之神,皆有功烈於民者也;及前哲令德之人,所以爲明\endnote{\BookTitle{觀止}作「民」,據\BookTitle{國語}各本改。}質也;及天之三辰,民所以瞻仰也;及地之五行,所以生殖也;及九州名山川澤,所以出財用也。非是,不在祀典。今海鳥至,己\endnote{\BookTitle{觀止}作「已」,據\BookTitle{國語}各本改。}不知而祀之,以爲國典,難以爲仁且知矣。夫仁者講功,而知者處物。無功而祀之,非仁也;不知而不問,非知也。今茲海其有災乎?夫廣川之鳥獸,恆知避其災也。
\end{quotation}

是歲也,海多大風,冬煗。\ProperName{文仲}聞\ProperName{柳下季}之言,曰:「信吾過也,\ProperName{季子}之言不可不法也。」使書以爲三筴。

\theendnotes

\section[里革斷罟匡君\quad{\small 國語 魯語上}]{{\normalsize 國語\ 魯語上}\quad \ProperName{里革}斷罟匡君}
\ProperName{宣公}夏濫於\ProperName{泗}淵,\ProperName{里革}斷其罟而棄之,曰:「古者大寒降,土蟄發,水虞於是乎講罛罶,取名魚,登川禽,而嘗之寢廟,行諸國人,助宣氣也。鳥獸孕,水蟲成,獸虞於是乎禁罝羅,矠魚鱉以爲夏槁,助生阜也。鳥獸成,水蟲孕,水虞於是乎禁罝䍡,設穽鄂,以實廟庖,畜功用也。且夫山不槎蘖,澤不伐夭,魚禁鯤鮞,獸長麑䴠,鳥翼鷇卵,蟲舍蚳蝝,蕃庶物也,古之訓也。今魚方別孕,不教魚長,又行網罟,貪無藝也。」

公聞之曰:「吾過而\ProperName{里革}匡我,不亦善乎!是良罟也,爲我得法。使有司藏之,使吾無忘諗。」\ProperName{師存}侍,曰:「藏罟不如寘\ProperName{里革}於側之不忘也。」

\section[敬姜論勞逸\quad{\small 國語 魯語下}]{{\normalsize 國語\ 魯語下}\quad \ProperName{敬姜}論勞逸}
\ProperName{公父文伯}退朝,朝其母,其母方績,\ProperName{文伯}曰:「以歜之家而主猶績,懼干\ProperName{季孫}之怒也。其以歜爲不能事主乎?」其母歎曰:

\begin{quotation}
\ProperName{魯}其亡乎!使僮子備官而未之聞耶?居,吾語女。昔聖王之處民也,擇瘠土而處之,勞其民而用之,故長王天下。夫民勞則思,思則善心生;逸則淫,淫則忘善,忘善則惡心生。沃土之民不材,淫也;瘠土之民莫不嚮義,勞也。是故天子大采朝日,與三公、九卿祖識地德,日中考政,與百官之政事、師尹、惟旅、牧、相宣序民事;少采夕月,與大史、司載糾虔天刑;日入監九御,使潔奉禘、郊之粢盛,而後即安。諸侯朝修天子之業命,晝考其國職,夕省其典刑,夜儆百工,使無慆淫,而後即安。卿大夫朝考其職,晝講其庶政,夕序其業,夜庀其家事,而後即安。士朝受業,晝而講貫,夕而習復,夜而計過無憾,而後即安。自庶人以下,明而動,晦而休,無日以怠。

王后親織玄紞,公侯之夫人加之紘、綖,卿之內子爲大帶,命婦成祭服,列士之妻加之以朝服,自庶士以下,皆衣其夫。社而賦事,烝而獻功,男女效績,愆則有辟,古之制也。君子勞心,小人勞力,先王之訓也。自上以下,誰敢淫心舍力?

今我,寡也,爾又在下位,朝夕處事,猶恐忘先人之業。況有怠惰,其何以避辟?吾冀而朝夕修我,曰:「必無廢先人。」爾今曰:「胡不自安?」以是承君之官,余懼\ProperName{穆伯}之絕祀也。
\end{quotation}

\ProperName{仲尼}聞之曰:「弟子志之,\ProperName{季氏}之婦不淫矣!」

\section[叔向賀貧\quad{\small 國語 晉語八}]{{\normalsize 國語\ 晉語八}\quad \ProperName{叔向}賀貧}
\ProperName{叔向}見\ProperName{韓宣子},\ProperName{宣子}憂貧,\ProperName{叔向}賀之。\ProperName{宣子}曰:「吾有卿之名,而無其實,無以從二三子,吾是以憂,子賀我何故?」

對曰:「昔\ProperName{欒武子}無一卒之田,其官不備其宗器,宣其德行,順其憲則,使越于諸侯,諸侯親之,\ProperName{戎}、\ProperName{狄}懷之,以正\ProperName{晉國},行刑不疚,以免於難。及\ProperName{桓子}驕泰奢侈,貪慾無藝,略則行志,假貸\endnote{\BookTitle{觀止}作「貨」,據\BookTitle{國語}各本改。}居賄,宜及於難,而賴\ProperName{武}之德,以沒其身。及\ProperName{懷子}改\ProperName{桓}之行,而修\ProperName{武}之德,可以免於難,而離\ProperName{桓}之罪,以亡於\ProperName{楚}。夫\ProperName{郤昭子},其富半公室,其家半三軍,恃其富寵,以泰于國,其身尸於朝,其宗滅於\ProperName{絳}。不然,夫八\ProperName{郤},五大夫三卿,其寵大矣,一朝而滅,莫之哀也,唯無德也。今吾子有\ProperName{欒武子}之貧,吾以爲能其德矣,是以賀。若不憂德之不建,而患貨之不足,將弔不暇,何賀之有?」

\ProperName{宣子}拜稽首焉,曰:「\ProperName{起}也將亡,賴子存之,非\ProperName{起}也敢專承之,其自\ProperName{桓叔}以下嘉吾子之賜。」

\theendnotes

\section[王孫圉論楚寶\quad{\small 國語 楚語下}]{{\normalsize 國語\ 楚語下}\quad \ProperName{王孫圉}論\ProperName{楚}寶}
\ProperName{王孫圉}聘於\ProperName{晉},\ProperName{定公}饗之,\ProperName{趙簡子}鳴玉以相,問於\ProperName{王孫圉}曰:「\ProperName{楚}之白珩猶在乎?」對曰:「然。」\ProperName{簡子}曰:「其爲寶也,幾何矣。」

曰:「未嘗爲寶。\ProperName{楚}之所寶者,曰\ProperName{觀射父},能作訓辭,以行事於諸侯,使無以寡君爲口實。又有\ProperName{左史倚相},能道訓典,以敘百物,以朝夕獻善敗於寡君,使寡君無忘先王之業;又能上下說乎鬼神,順道其欲惡,使神無有怨痛於\ProperName{楚國}。又有藪曰\ProperName{雲連徒洲},金木竹箭之所生也。龜、珠、角、齒、皮、革、羽、毛,所以備賦,以戒不虞者也。所以共幣帛,以賓享於諸侯者也。若諸侯之好幣具,而導之以訓辭,有不虞之備,而皇神相之,寡君其可以免罪於諸侯,而國民保焉。此\ProperName{楚國}之寶也。若夫白珩,先王之玩也,何寶焉?\ProperName{圉}聞國之寶六而已。聖能制議百物,以輔相國家,則寶之;玉足以庇廕嘉穀,使無水旱之災,則寶之;龜足以憲臧否,則寶之;珠足以禦火災,則寶之;金足以禦兵亂,則寶之;山林藪澤足以備財用,則寶之。若夫譁囂之美,\ProperName{楚}雖\ProperName{蠻}\ProperName{夷},不能寶也。」

\section[諸稽郢行成於吳\quad{\small 國語 吳語}]{{\normalsize 國語\ 吳語}\quad \ProperName{諸稽郢}行成於\ProperName{吳}}
\ProperName{吳王}\ProperName{夫差}起師伐\ProperName{越},\ProperName{越王}\ProperName{句踐}起師逆之江\endnote{\BookTitle{觀止}從\ProperName{公序}本「之」下有「江」字。\ProperName{汪遠孫}\BookTitle{考異}:\ProperName{公序}本下有「江」字,或涉注而衍。}。大夫\ProperName{種}乃獻謀曰:「夫\ProperName{吳}之與\ProperName{越},唯天所授,王其無庸戰。夫\ProperName{申胥}、\ProperName{華登}簡服\ProperName{吳國}之士於甲兵,而未嘗有所挫也。夫一人善射,百夫決拾,勝未可成也。夫謀必素見成事焉,而後履之,不可以授命。王不如設戎,約辭行成,以喜其民,以廣侈\ProperName{吳王}之心。吾以卜之於天,天若棄\ProperName{吳},必許吾成而不吾足也,將必寬然有伯諸侯之心焉。既罷弊其民,而天奪之食,安受其燼,乃無有命矣。」\ProperName{越王}許諾。

乃命\ProperName{諸稽郢}行成於\ProperName{吳},曰:

\begin{quotation}
寡君\ProperName{句踐}使下臣\ProperName{郢}不敢顯然布幣行禮,敢私告於下執事曰:昔者\ProperName{越國}見禍,得罪於天王。天王親趨玉趾,以心孤\ProperName{句踐},而又宥赦之。君王之於\ProperName{越}也,繄起死人而肉白骨也。孤不敢忘天災,其敢忘君王之大賜乎!今\ProperName{句踐}申禍無良,草鄙之人,敢忘天王之大德,而思邊垂之小怨,以重得罪於下執事?\ProperName{句踐}用帥二三之老,親委重罪,頓顙於邊。

今君王不察,盛怒屬兵,將殘伐\ProperName{越國}。\ProperName{越國}固貢獻之邑也,君王不以鞭箠使之,而辱軍士使寇令焉。\ProperName{句踐}請盟:一介嫡女,執箕帚以晐姓於王宮;一介嫡男,奉槃匜以隨諸御;春秋貢獻,不解於王府。天王豈辱裁之?亦征諸侯之禮也。

夫諺曰:「狐埋之而狐搰之,是以無成功。」今天王既封殖\ProperName{越國},以明聞於天下,而又刈亡之,是天王之無成勞也。雖四方之諸侯,則何實以事\ProperName{吳}?敢使下臣盡辭,唯天王秉利度義焉!
\end{quotation}
\vspace{-1em}
\theendnotes

\section[申胥諫許越成\quad{\small 國語 吳語}]{{\normalsize 國語\ 吳語}\quad \ProperName{申胥}諫許\ProperName{越}成}
\ProperName{吳王}\ProperName{夫差}乃告諸大夫曰:「孤將有大志於\ProperName{齊},吾將許\ProperName{越}成,而無拂吾慮。若\ProperName{越}既改,吾又何求?若其不改,反行,吾振旅焉。」

\ProperName{申胥}諫曰:「不可許也。夫\ProperName{越}非實忠心好\ProperName{吳}也,又非懾畏吾兵甲之彊也。大夫\ProperName{種}勇而善謀,將還玩\ProperName{吳國}於股掌之上,以得其志。夫固知君王之蓋威以好勝也,故婉約其辭,以從逸王志,使淫樂於諸夏之國,以自傷也。使吾甲兵鈍弊,民人離落,而日以憔悴,然後安受吾燼。夫\ProperName{越王}好信以愛民,四方歸之,年穀時熟,日長炎炎。及吾猶可以戰也,爲虺弗摧,爲蛇將若何?」

\ProperName{吳王}曰:「大夫奚隆於\ProperName{越},\ProperName{越}曾足以爲大虞乎?若無\ProperName{越},則吾何以春秋曜吾軍士?」乃許之成。

將盟,\ProperName{越王}又使\ProperName{諸稽郢}辭曰:「以盟爲有益乎?前盟口血未乾,足以結信矣。以盟爲無益乎?君王舍甲兵之威以臨使之,而胡重於鬼神而自輕也?」\ProperName{吳王}乃許之,荒成不盟。

\section[春王正月\quad{\small 公羊傳 隱公元年}]{{\normalsize 公羊傳 隱公元年}\quad 春王正月}
元年者何?君之始年也。春者何?歲之始也。王者孰謂,謂\ProperName{文王}也。曷爲先言王而後言正月?王正月也。何言乎王正月?大一統也。

公何以不言即位?成公意也。何成乎公之意?公將平國而反之\ProperName{桓}。曷爲反之\ProperName{桓}?\ProperName{桓}幼而貴,\ProperName{隱}長而卑。其爲尊卑也微,國人莫知。\ProperName{隱}長又賢,諸大夫扳\ProperName{隱}而立之。\ProperName{隱}於是焉而辭立,則未知\ProperName{桓}之將必得立也。且如\ProperName{桓}立,則恐諸大夫之不能相幼君也,故凡\ProperName{隱}之立爲\ProperName{桓}立也。\ProperName{隱}長又賢,何以不宜立?立適以長不以賢,立子以貴不以長。\ProperName{桓}何以貴?母貴也。母貴則子何以貴?子以母貴,母以子貴。

\section[宋人及楚人平\quad{\small 公羊傳 宣公十五年}]{{\normalsize 公羊傳\ 宣公十五年}\quad \ProperName{宋}人及\ProperName{楚}人平}
外平不書,此何以書?大其平乎己也。何大乎\endnote{\BookTitle{觀止}「大」下脱「乎」字,據\BookTitle{公羊傳}校本補。}其平乎己?\ProperName{莊王}圍\ProperName{宋},軍有七日之糧爾,盡此不勝,將去而歸爾。於是使\ProperName{司馬子反}乘堙而闚\ProperName{宋}城,\ProperName{宋華元}亦乘堙而出見之。

\ProperName{司馬子反}曰:「子之國如何?」\ProperName{華元}曰:「憊矣!」曰:「何如?」曰:「易子而食之,析骸而炊之。」\ProperName{司馬子反}曰:「嘻!甚矣憊!雖然,吾聞之也。圍者柑馬而秣之,使肥者應客,是何子之情也?」\ProperName{華元}曰:「吾聞之,君子見人之厄則矜之,小人見人之厄則幸之。吾見子之君子也,是以告情于子也。」\ProperName{司馬子反}曰:「諾,勉之矣!吾軍亦有七日之糧爾,盡此不勝,將去而歸爾。」揖而去之。

反于\ProperName{莊王}。\ProperName{莊王}曰:「何如?」\ProperName{司馬子反}曰;「憊矣!」曰:「何如?」曰:「易子而食之,析骸而炊之。」\ProperName{莊王}曰:「嘻!甚矣憊!雖然,吾今取此,然後而歸爾。」\ProperName{司馬子反}曰:「不可。臣已告之矣,軍有七日之糧爾。」\ProperName{莊王}怒曰:「吾使子往視之,子曷爲告之?」\ProperName{司馬子反}曰:「以區區之\ProperName{宋},猶有不欺人之臣,可以\ProperName{楚}而無乎?是以告之也。」\ProperName{莊王}曰:「諾,舍而止。雖然,吾猶取此然後歸爾。」\ProperName{司馬子反}曰:「然則君請處于此,臣請歸爾。」\ProperName{莊王}曰:「子去我而歸,吾孰與處于此?吾亦從子而歸爾。」引師而去之。故君子大其平乎己也。此皆大夫也。其稱人何?貶。曷爲貶?平者在下也。

\theendnotes

\section[吳子使季札來聘\quad{\small 公羊傳 襄公二十九年}]{{\normalsize 公羊傳\ 襄公二十九年}\quad \ProperName{吳子}使\ProperName{季札}來聘}
\ProperName{吳}無君無大夫,此何以有君有大夫?賢\ProperName{季子}也。何賢乎\ProperName{季子}?讓國也。其讓國奈何?\ProperName{謁}也、\ProperName{餘祭}也、\ProperName{夷昧}也與\ProperName{季子}同母者四。\ProperName{季子}弱而才,兄弟皆愛之,同欲立之以爲君。\ProperName{謁}曰:「今若是迮而與\ProperName{季子}國,\ProperName{季子}猶不受也。請無與子而與弟,弟兄迭爲君,而致國乎\ProperName{季子}。」皆曰:「諾。」故諸爲君者,皆輕死爲勇,飲食必祝,曰:「天苟有\ProperName{吳國},尚速有悔於予身。」故\ProperName{謁}也死,\ProperName{餘祭}也立。\ProperName{餘祭}也死,\ProperName{夷昧}也立。\ProperName{夷昧}也死,則國宜之\ProperName{季子}者也。\ProperName{季子}使而亡焉。

\ProperName{僚}者,長庶也,即之。\ProperName{季子}使而反,至而君之爾。\ProperName{闔廬}曰:「先君之所以不與子國,而與弟者,凡爲\ProperName{季子}故也。將從先君之命與?則國宜之\ProperName{季子}者也。如不從先君之命與?則我宜立者也。\ProperName{僚}惡得爲君乎?」於是使\ProperName{專諸}刺\ProperName{僚},而致國乎\ProperName{季子}。\ProperName{季子}不受,曰:「爾弒吾君,吾受爾國,是吾與爾爲篡也。爾殺吾兄,吾又殺爾,是父子兄弟相殺,終身無已也。」去之\ProperName{延陵},終身不入\ProperName{吳國}。故君子以其不受爲義,以其不殺爲仁。

賢\ProperName{季子}則\ProperName{吳}何以有君有大夫?以\ProperName{季子}爲臣,則宜有君者也。\ProperName{札}者何?\ProperName{吳季子}之名也。\BookTitle{春秋}賢者不名,此何以名?許夷狄者不壹而足也。\ProperName{季子}者所賢也,曷爲不足乎\ProperName{季子}?許人臣者必使臣,許人子者必使子也!

\section[鄭伯克段于鄢\quad{\small 穀梁傳 隱公元年}]{{\normalsize 穀梁傳\ 隱公元年}\quad \ProperName{鄭伯}克\ProperName{段}于\ProperName{鄢}}
克者何?能也。何能也?能殺也。何以不言殺?見\ProperName{段}之有徒眾也。\ProperName{段},\ProperName{鄭伯}弟也。何以知其爲弟也?殺世子母弟目君,以其目君,知其爲弟也。\ProperName{段},弟也,而弗謂弟;公子也,而弗謂公子;貶之也。\ProperName{段}失子弟之道矣,賤\ProperName{段}而甚\ProperName{鄭伯}也。何甚乎\ProperName{鄭伯}?甚\ProperName{鄭伯}之處心積慮,成於殺也。于\ProperName{鄢},遠也。猶曰取之其母之懷中而殺之云爾,甚之也。然則爲\ProperName{鄭伯}者宜奈何?緩追逸賊,親親之道也。

\section[虞師晉師滅夏陽\quad{\small 穀梁傳 僖公二年}]{{\normalsize 穀梁傳\ 僖公二年}\quad \ProperName{虞}師\ProperName{晉}師滅\ProperName{夏陽}}
非國而曰滅,重\ProperName{夏陽}也。\ProperName{虞}無師,其曰師,何也?以其先\ProperName{晉},不可以不言師也。其先\ProperName{晉},何也?爲主乎滅\ProperName{夏陽}也。\ProperName{夏陽}者,\ProperName{虞}、\ProperName{虢}之塞邑也,滅\ProperName{夏陽}而\ProperName{虞}、\ProperName{虢}舉矣。

\ProperName{虞}之爲主乎滅\ProperName{夏陽},何也?\ProperName{晉獻公}欲伐\ProperName{虢},\ProperName{荀息}曰:「君何不以\ProperName{屈}產之乘,\ProperName{垂棘}之璧而借道乎\ProperName{虞}也?」公曰:「此\ProperName{晉國}之寶也,如受吾幣而不借吾道則如之何?」\ProperName{荀息}曰:「此小國之所以事大國也,彼不借吾道必不敢受吾幣,如受吾幣而借吾道,則是我取之中府而藏之外府,取之中廄而置之外廄也。」公曰:「\ProperName{宮之奇}存焉,必不使受之也。」\ProperName{荀息}曰:「\ProperName{宮之奇}之爲人也,達心而懦,又少長於君。達心則其言略,懦則不能彊諫,少長於君則君輕之。且夫玩好在耳目之前,而患在一國之後,此中知以上乃能慮之。臣料\ProperName{虞}君中知以下也。」

公遂借道而伐\ProperName{虢}。\ProperName{宮之奇}諫曰:「\ProperName{晉國}之使者,其辭卑而幣重,必不便於\ProperName{虞}。」\ProperName{虞公}弗聽,遂受其幣而借之道。\ProperName{宮之奇}\endnote{\BookTitle{觀止}「宮之奇」下有「又」字,據\BookTitle{穀梁傳}校本改。}諫曰:「語曰『脣亡則齒寒』,其斯之謂與!」挈其妻子以奔\ProperName{曹}。

\ProperName{獻公}亡\ProperName{虢},五年而後舉\ProperName{虞}。\ProperName{荀息}牽馬操璧而前曰:「璧則猶是也,而馬齒加長矣。」

\theendnotes

\section[晉獻公殺世子申生\quad{\small 禮記 檀弓上}]{{\normalsize 禮記\ 檀弓上}\quad \ProperName{晉獻公}殺世子\ProperName{申生}}
\ProperName{晉獻公}將殺其世子\ProperName{申生}。\ProperName{公子重耳}謂之曰:「子蓋言子之志於公乎?」世子曰:「不可。君安\ProperName{驪姬},是我傷公之心也。」曰:「然則蓋行乎?」世子曰:「不可。君謂我欲弑君也。天下豈有無父之國哉!吾何行如之?」

使人辭於\ProperName{狐突}曰:「\ProperName{申生}有罪,不念\ProperName{伯氏}之言也,以至於死。\ProperName{申生}不敢愛其死。雖然,吾君老矣,子少,國家多難。\ProperName{伯氏}不出而圖吾君,\ProperName{伯氏}苟出而圖吾君,\ProperName{申生}受賜而死!」再拜稽首,乃卒。是以爲\ProperName{恭世子}也。

\section[曾子易簀\quad{\small 禮記 檀弓上}]{{\normalsize 禮記\ 檀弓上}\quad \ProperName{曾子}易簀}
\ProperName{曾子}寢疾,病。\ProperName{樂正子春}坐於牀下,\ProperName{曾元}、\ProperName{曾申}坐於足,童子隅坐而執燭。童子曰:「華而睆,大夫之簀與?」\ProperName{子春}曰:「止!」\ProperName{曾子}聞之,瞿然曰:「呼!」曰:「華而睆,大夫之簀與?」\ProperName{曾子}曰:「然。斯\ProperName{季孫}之賜也,我未之能易也。\ProperName{元},起易簀!」\ProperName{曾元}曰:「夫子之病革矣,不可以變,幸而至於旦,請敬易之。」\ProperName{曾子}曰:「爾之愛我也不如彼。君子之愛人也以德,細人之愛人也以姑息。吾何求哉?吾得正而斃焉,斯已矣。」舉扶而易之,反席未安而沒。

\section[有子之言似夫子\quad{\small 禮記 檀弓上}]{{\normalsize 禮記\ 檀弓上}\quad \ProperName{有子}之言似夫子}
\ProperName{有子}問於\ProperName{曾子}曰:「問喪於夫子乎?」曰:「聞之矣:『喪欲速貧,死欲速朽。』」\ProperName{有子}曰:「是非君子之言也!」\ProperName{曾子}曰:「\ProperName{參}也聞諸夫子也!」\ProperName{有子}又曰:「是非君子之言也!」\ProperName{曾子}曰:「\ProperName{參}也與\ProperName{子游}聞之。」\ProperName{有子}曰:「然,然則夫子有爲言之也。」

\ProperName{曾子}以斯言告於\ProperName{子游}。\ProperName{子游}曰:「甚哉!有子之言似夫子也。昔者夫子居於\ProperName{宋},見\ProperName{桓司馬}自爲石椁,三年而不成,夫子曰:『若是其靡也,死不如速朽之愈也。』死之欲速朽,爲\ProperName{桓司馬}言之也。\ProperName{南宮敬叔}反,必載寶而朝。夫子曰:『若是其貨也,喪不如速貧之愈也。』喪之欲速貧,爲\ProperName{敬叔}言之也。」

\ProperName{曾子}以\ProperName{子游}之言告於\ProperName{有子}。\ProperName{有子}曰:「然。吾固曰非夫子之言也。」\ProperName{曾子}曰:「子何以知之?」\ProperName{有子}曰:「夫子制於\ProperName{中都},四寸之棺,五寸之椁,以斯知不欲速朽也。昔者夫子失\ProperName{魯}司寇,將之\ProperName{荊},蓋先之以\ProperName{子夏},又申之以\ProperName{冉有},以斯知不欲速貧也。」

\section[公子重耳對秦客\quad{\small 禮記 檀弓下}]{{\normalsize 禮記\ 檀弓下}\quad \ProperName{公子重耳}對\ProperName{秦}客}
\ProperName{晉獻公}之喪,\ProperName{秦穆公}使人弔\ProperName{公子重耳},且曰:「寡人聞之,亡國恆於斯,得國恆於斯。雖吾子儼然在憂服之中,喪亦不可久也,時亦不可失也,孺子其圖之!」以告\ProperName{舅犯}。\ProperName{舅犯}曰:「孺子其辭焉。喪人無寶,仁親以爲寶。父死之謂何?又因以爲利,而天下其孰能說之?孺子其辭焉!」\ProperName{公子重耳}對客曰:「君惠弔亡臣\ProperName{重耳}。身喪父死,不得與於哭泣之哀,以爲君憂。父死之謂何?或敢有他志,以辱君義。」稽顙而不拜,哭而起,起而不私。

\ProperName{子顯}以致命於\ProperName{穆公}。\ProperName{穆公}曰:「仁夫\ProperName{公子重耳}!夫稽顙而不拜,則未爲後也,故不成拜。哭而起,則愛父也。起而不私,則遠利也。」

\section[杜蕢揚觶\quad{\small 禮記 檀弓下}]{{\normalsize 禮記\ 檀弓下}\quad \ProperName{杜蕢}揚觶}
\ProperName{知悼子}卒,未葬。\ProperName{平公}飲酒,\ProperName{師曠}、\ProperName{李調}侍鼓鐘。\ProperName{杜簣}自外來,聞鐘聲,曰:「安在?」曰:「在寢。」\ProperName{杜簣}入寢,歷階而升。酌曰:「\ProperName{曠}飲斯。」又酌曰:「\ProperName{調}飲斯。」又酌,堂上北面坐飲之。降,趨而出。

\ProperName{平公}呼而進之,曰:「\ProperName{蕢},曩者爾心或開予,是以不與爾言。爾飲\ProperName{曠}何也?」曰:「子卯不樂。\ProperName{知悼子}在堂,斯其爲子卯也大矣!\ProperName{曠}也太師也,不以詔。是以飲之也。」「爾飲\ProperName{調},何也?」曰:「\ProperName{調}也,君之褻臣也,爲一飲一食,忘君之疾,是以飲之也。」「爾飲何也?」曰:「\ProperName{簣}也,宰夫也,非刀匕是共,又敢與知防,是以飲之也。」

\ProperName{平公}曰:「寡人亦有過焉,酌而飲寡人。」\ProperName{杜簣}洗而揚觶。公謂侍者曰:「如我死,則必毋廢斯爵也。」至于今,既畢獻,斯揚觶,謂之「杜舉」。

\section[晉獻文子成室\quad{\small 禮記 檀弓下}]{{\normalsize 禮記\ 檀弓下}\quad \ProperName{晉}獻\ProperName{文子}成室}
\ProperName{晉}獻\ProperName{文子}成室,\ProperName{晉}大夫發焉。\ProperName{張老}曰:「美哉輪焉!美哉奐焉!歌於斯,哭於斯,聚國族於斯!」\ProperName{文子}曰:「\ProperName{武}也得歌於斯,哭於斯,聚國族於斯,是全要領以從先大夫於\ProperName{九京}也!」北面再拜稽首。君子謂之善頌、善禱。

% Proofed 9 July 2022
% Ref. 
% - 國語, 上海古籍, 1978
% - 國語集解, 中華書局 2002
% - 春秋公羊傳注疏, 北大, 2000
% - 春秋榖梁傳注疏, 北大, 2000
% - 禮記正義, 上海古籍, 2008

