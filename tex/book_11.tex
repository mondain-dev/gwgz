\section[上梅直講書\quad{\small 蘇軾}]{{\normalsize 蘇軾}\quad 上\ProperName{梅直講}書}
\ProperName{軾}每讀\BookTitle{詩}至\BookTitle{鴟鴞},讀\BookTitle{書}至\BookTitle{君奭},常竊悲\ProperName{周公}之不遇。及觀史,見\ProperName{孔子}厄於\ProperName{陳}、\ProperName{蔡}之閒,而弦歌之聲不絕,\ProperName{顏淵}、\ProperName{仲由}之徒,相與問答。夫子曰:「『匪兕匪虎,率彼曠野。』吾道非耶!吾何爲於此?」\ProperName{顏淵}曰:「夫子之道至大,故天下莫能容。雖然,不容何病?不容然後見君子。」夫子油然而笑曰:「\ProperName{回},使爾多財,吾爲爾宰。」夫天下雖不能容,而其徒自足以相樂如此。乃今知\ProperName{周公}之富貴,有不如夫子之貧賤。夫以\ProperName{召公}之賢,以\ProperName{管}、\ProperName{蔡}之親而不知其心,則\ProperName{周公}誰與樂其富貴。而夫子之所與共貧賤者,皆天下之賢才,則亦足以樂乎此矣。

\ProperName{軾}七八歲時,始知讀書,聞今天下有\ProperName{歐陽公}者,其爲人如古\ProperName{孟軻}、\ProperName{韓愈}之徒。而又有\ProperName{梅公}者從之遊,而與之上下其議論。其後益壯,始能讀其文詞,想見其爲人,意其飄然脫去世俗之樂而自樂其樂也。方學爲對偶聲律之文,求{斗升}\endnote{\BookTitle{觀止}作「升斗」,\BookTitle{文章軌範}同,據各本改。}之祿,自度無以進見於諸公之間。來京師逾年,未嘗窺其門。% 經進東坡文集事略原作「斗升」,文章軌範「升斗」

今年春,天下之士,羣至於禮部,執事與\ProperName{歐陽公}實親試之。{誠}\endnote{觀止作「軾」,據各本改。}不自意,獲在第二。既而聞之{人},執事愛其文,以爲有\ProperName{孟軻}之風。而\ProperName{歐陽公}亦以其能不爲世俗之文也而取{焉}\endnote{\BookTitle{觀止}脱「焉」字,據各本補。}。是以在此。非左右爲之先容,非親舊爲之請屬,而嚮之十餘年間,聞其名而不得見者,一朝爲知己。退而思之,人不可以苟富貴,亦不可以徒貧賤。有大賢焉而爲其徒,則亦足恃矣。苟其僥一時之幸,從車騎數十人,使閭巷小民聚觀而贊歎之,亦何以易此樂也?\BookTitle{傳}曰:「不怨天,不尤人。」蓋優哉游哉,可以卒歲。執事名滿天下,而位不過五品,其容色溫然而不怒,其文章寬厚敦朴而無怨言,此必有所樂乎斯道也,\ProperName{軾}願與聞焉。% 觀止作「軾不自意」; 觀止脱「人」字; 觀止脱「焉」字

\theendnotes

\section[喜雨亭記\quad{\small 蘇軾}]{{\normalsize 蘇軾}\quad \ProperName{喜雨亭}記}
亭以雨名,志喜也。古者有喜,則以名物,示不忘也。\ProperName{周公}得禾,以名其書;\ProperName{漢武}得鼎,以名其年;\ProperName{叔孫}勝\ProperName{{狄}}\endnote{\BookTitle{觀止}作「敵」,據各本改。},以名其子。其喜之大小不齊,其示不忘一也。% 觀止避諱作「敵」,另據古籍庫CADAL文章辨體亦作「敵」。

予至\ProperName{扶風}之明年,始治官舍,爲亭於堂之北,而鑿池其南,引流種樹,以爲休息之所。是歲之春,雨麥於\ProperName{岐山}之陽,其占爲有年。既而彌月不雨,民方以爲憂。越三月乙卯,乃雨,甲子又雨,民以爲未足,丁卯,大雨,三日乃止。官吏相與慶於庭,商賈相與歌於市,農夫相與{抃}\endnote{\BookTitle{觀止}作「忭」,\BookTitle{崇古文訣}、\BookTitle{古文集成}、\BookTitle{續文章正宗}、\BookTitle{文章辨體彙選}同,據各本改。}於野,憂者以樂,病者以愈,而吾亭適成。% 經進東坡文集事略原作「抃」,崇古文訣作「忭」

於是舉酒於亭上以屬客,而告之曰:「五日不雨,可乎?」曰:「五日不雨,則無麥。」「十日不雨,可乎?」曰:「十日不雨,則無禾。」無麥無禾,歲且{荐}饑,獄訟繁興,而盜賊滋熾,則吾與二三子,雖欲優游以樂於此亭,其可得耶!今天不遺斯民,始旱而賜之以雨,使吾與二三子,得相與優游而樂於此亭者,皆雨之賜也。其又可忘耶!% 觀止作「薦」下注「同荐」。

既以名亭,又從而歌之,曰:「使天而雨珠,寒者不得以爲襦。使天而雨玉,饑者不得以爲粟。一雨三日,{繄}誰之力。民曰太守,太守不有。歸之天子,天子曰不然。歸之造物,造物不自以爲功。歸之太空,太空冥冥。不可得而名,吾以名吾亭。」% 觀止作「伊」

\theendnotes

\section[凌虛臺記\quad{\small 蘇軾}]{{\normalsize 蘇軾}\quad \ProperName{凌虛臺}記}
國於\ProperName{南山}之下,宜若起居飲食與山接也。四方之山,莫高於\ProperName{終南},而都邑之\ProperName{麗山}者,莫近於\ProperName{扶風}。以至近求最高,其勢必得。而太守之居,未嘗知有山焉。雖非事之所以損益,而物理有不當然者,此\ProperName{凌虛}之所爲築也。

方其未築也,太守\ProperName{陳公}杖屨\endnote{\BookTitle{觀止}作「履」,\BookTitle{事類備要}、\BookTitle{續文章正宗}、\BookTitle{事文類聚}、\BookTitle{唐宋八大家文鈔}同,據各本改。}逍遙於其下,見山之出於林木之上者,纍纍如人之旅行於牆外而見其髻也,曰:「是必有異。」使工鑿其前爲方池,以其土築臺,高出於屋之簷而止。然後人之至於其上者,怳然不知臺之高,而以爲山之踴躍奮迅而出也。公曰:「是宜名『凌虛』。」以告其從事\ProperName{蘇軾},而求文以爲記。% 經進東坡文集事略原作「屨」,宋真德秀

\ProperName{軾}復於公曰:「物之廢興成毀,不可得而知也。昔者荒草野田,霜露之所蒙翳,狐虺之所竄伏,方是時,豈知有\ProperName{凌虛臺}耶?廢興成毀相尋於無窮,則臺之復爲荒草野田,皆不可知也。嘗試與公登臺而望,其東則\ProperName{秦穆}之\ProperName{祈年}、\ProperName{橐泉}也,其南則\ProperName{漢武}之\ProperName{長楊}、\ProperName{五柞},而其北則\ProperName{隋}之\ProperName{仁壽}、\ProperName{唐}之\ProperName{九成}也。計其一時之盛,宏傑詭麗,堅固而不可動者,豈特百倍於臺而已哉!然而數世之後,欲求其髣髴,而破瓦頹垣無復存者,既已化爲禾黍荊棘丘墟隴畝矣,而況於此臺歟?夫臺猶不足恃以長久,而況於人事之得喪,忽往而忽來者歟?而或者欲以夸世而自足,則過矣。蓋世有足恃者,而不在乎臺之存亡也。」既已言於公,退而爲之記。

\theendnotes

\section[超然臺記\quad{\small 蘇軾}]{{\normalsize 蘇軾}\quad \ProperName{超然臺}記}
凡物皆有可觀。苟有可觀,皆有可樂,非必怪奇偉麗者也。餔糟啜{漓}\endnote{\BookTitle{觀止}作「醨」,\BookTitle{續文章正宗}、\BookTitle{文章辨體彙選}、\BookTitle{唐宋八大家文鈔}同,據各本改。}皆可以醉,果蔬草木皆可以飽。推此類也,吾安往而不樂?夫所爲求福而辭禍者,以福可喜而禍可悲也。人之所欲無窮,而物之可以足吾欲者有盡。美惡之辨戰{乎}\endnote{\BookTitle{觀止}作「於」,據各本改。}中,而去取之擇交乎前,則可樂者常少,而可悲者常多。是謂求禍而辭福。夫求禍而辭福,豈人之情也哉?物有以蓋之矣。彼遊於物之內,而不遊於物之外。物非有大小也,自其內而觀之,未有不高且大者也。彼挾其高大以臨我,則我常眩亂反覆,如隙中之觀鬬,又烏知勝負之所在?是以美惡橫生,而憂樂出焉。可不大哀乎!% 經進東坡文集事略原作「漓」,續文章正宗作「醨」;觀止作「於」

予自\ProperName{錢塘}移守\ProperName{膠西},釋舟楫之安,而服車馬之勞,去雕牆之美,而庇采椽之居,背湖山之觀,而行桑麻之野。始至之日,歲比不登,盜賊滿野,獄訟充斥,而齋廚索然,日食杞菊。人固疑予之不樂也。處之朞年,而貌加豐,髮之白者,日以反黑。予既樂其風俗之淳,而其吏民亦安予之拙也,於是治其園圃,潔其庭宇,伐\ProperName{安丘}、\ProperName{高密}之木以修補破敗,爲苟完之計。而園之北,因城以爲臺者舊矣,稍葺而新之。時相與登覽,放意肆志焉。南望\ProperName{馬耳}、\ProperName{常山},出沒隱見,若近若遠,庶幾有隱君子乎?而其東則\ProperName{盧山},\ProperName{秦}人\ProperName{盧敖}之所從遁也。西望\ProperName{穆陵},隱然如城郭,\ProperName{師尚父}、\ProperName{齊{桓}公}\endnote{\BookTitle{觀止}作「齊威公」,從\ProperName{宋}諱。}之遺烈,猶有存者。北俯\ProperName{濰水},慨然{太}息,思\ProperName{淮陰}之功,而弔其不終。臺高而安,深而明,夏涼而冬溫。雨雪之朝,風月之夕,予未嘗不在,客未嘗不從。擷園蔬,取池魚,釀秫酒,瀹脫粟而食之,曰:樂哉遊乎!% 觀止作「齊威公」,避宋諱 ; 慨然「太」息:觀止作「大」

方是時\endnote{「方是時」:\BookTitle{觀止}脱,據各本補。},予弟\ProperName{子由}適在\ProperName{濟南},聞而賦之,且名其臺曰「超然」。以見予之無所往而不樂者,蓋遊於物之外也。% 「方是時」:觀止脱

\theendnotes

\section[放鶴亭記\quad{\small 蘇軾}]{{\normalsize 蘇軾}\quad \ProperName{放鶴亭}記}
\ProperName{熙寧}十年秋,\ProperName{彭城}大水,\ProperName{雲龍山人}\ProperName{張君}之草堂,水及其半扉。明年春,水落,遷於故居之東,東山之麓。升高而望,得異境焉,作亭於其上。\ProperName{彭城}之山,岡嶺四合,隱然如大環,獨缺其西{十二}\endnote{\BookTitle{觀止}作「西一面」,\BookTitle{唐宋八大家文鈔}同,據各本改。},而山人之亭適當其缺。春夏之交,草木際天。秋冬雪月,千里一色。風雨晦明之間,俯仰百變。山人有二鶴,甚馴而善飛。旦則望西山之缺而放焉,縱其所如,或立於陂田,或翔於雲表,暮則傃東山而歸。故名之曰「放鶴亭」。% 原作「十二」,明茅坤唐宋八大家文鈔作「西一面」

郡守\ProperName{蘇軾},時從賓{客}\endnote{\BookTitle{觀止}作「佐」,據各本改。}僚吏往見山人,飲酒於斯亭而樂之,挹山人而告之曰:「子知隱居之樂乎?雖南面之君,未可與易也。\BookTitle{易}曰:『鳴鶴在陰,其子和之。』\BookTitle{詩}曰:『鶴鳴于九皋,聲聞于天。』蓋其爲物,清遠閒放,超然於塵垢之外,故\BookTitle{易}、\BookTitle{詩}人以比賢人君子隱德之士。狎而玩之,宜若有益而無損者,然\ProperName{衛懿公}好鶴則亡其國。\ProperName{周公}作\BookTitle{酒誥},\ProperName{衛武公}作\BookTitle{抑戒},以爲荒惑敗亂無若酒者,而\ProperName{劉伶}、\ProperName{阮籍}之徒以此全其真而名後世。嗟夫,南面之君,雖清遠閒放如鶴者猶不得好,好之則亡其國,而山林{遁}世之士,雖荒惑敗亂如酒者猶不能爲害,而況於鶴乎?由此觀之,其爲樂未可以同日而語也。」山人欣然而笑曰:「有是哉。」乃作放鶴招鶴之歌曰: % 觀止作「賓佐僚吏」; 觀止作「遯」

\begin{quote}
    鶴飛去兮,西山之缺。高翔而下覽兮,擇所適。翻然斂翼,婉將集兮,忽何所見,矯然而復擊。獨終日於澗谷之間兮,啄蒼苔而履白石。鶴歸來兮,東山之陰。其下有人兮,黃冠草履葛衣而鼓琴。躬耕而食兮,其餘以汝飽。歸來歸來兮,西山不可以久留。
\end{quote}
% 元豐元年十一月初八日記
\vspace{-1em}
\theendnotes

\section[石鐘山記\quad{\small 蘇軾}]{{\normalsize 蘇軾}\quad \ProperName{石鐘山}記}
\BookTitle{水經}云:「\ProperName{彭蠡}之口,有\ProperName{石鐘山}焉。」\ProperName{酈元}以爲下臨深潭,微風鼓浪,水石相搏,聲如洪鐘。是說也,人常疑之。今以鐘磬置水中,雖大風浪,不能鳴也,而況石乎!至\ProperName{唐}\ProperName{李渤}始訪其遺蹤,得雙石於潭上,扣而聆之,南聲函胡,北音清越,枹止響騰,餘韻徐歇,自以爲得之矣。然是說也,余尤疑之。石之鏗然有聲者,所在皆是也,而此獨以鐘名,何哉?

\ProperName{元豐}七年六月丁丑,余自\ProperName{齊安}舟行適\ProperName{臨汝},而長子\ProperName{邁}將赴\ProperName{饒}之\ProperName{德興}尉,送之至\ProperName{湖口},因得觀所謂石鐘者。寺僧使小童持斧,於亂石間擇其一二扣之,硿硿{焉}\endnote{\BookTitle{觀止}作「然」,據各本改。},余固笑而不信也。至其夜月明,獨與\ProperName{邁}乘小舟至絕壁下,大石側立千尺,如猛獸奇鬼,森然欲搏人。而山上栖鶻,聞人聲亦驚起,磔磔雲霄間。又有若老人欬且笑於山谷中者,或曰:「此鸛鶴也。」余方心動欲還,而大聲發於水上,噌吰如鐘鼓不絕,舟人大恐。徐而察之,則山下皆石穴罅,不知其淺深,微波入焉,涵澹澎湃而爲此也。舟迴至兩山間,將入港口,有大石當中流,可坐百人,空中而多竅,與風水相吞吐,有窾坎鏜鞳之聲,與向之噌吰者相應,如樂作焉。% 硿硿「焉」:觀止作「然」

因笑謂\ProperName{邁}曰:「汝識之乎?噌吰者,\ProperName{周景王}之\ProperName{無射}也。窾坎鏜鞳者,\ProperName{魏莊子}之歌鐘也。古之人不余欺也。事不目見耳聞,而臆斷其有無,可乎?」\ProperName{酈元}之所見聞,殆與余同,而言之不詳。士大夫終不肯以小舟夜泊絕壁之下,故莫能知。而漁工水師,雖知而不能言,此世所以不傳也。而陋者乃以斧斤考擊而求之,自以爲得其實。余是以記之,蓋歎\ProperName{酈元}之簡,而笑\ProperName{李渤}之陋也。

\section[潮州韓文公廟碑\quad{\small 蘇軾}]{{\normalsize 蘇軾}\quad \ProperName{潮州}\ProperName{韓文公廟}碑}
匹夫而爲百世師,一言而爲天下法。是皆有以參天地之化,關盛衰之運。其生也有自來,其逝也有所爲。故\ProperName{申}、\ProperName{呂}自嶽降,傅說爲列星,古今所傳,不可誣也。\ProperName{孟子}曰:「我善養吾浩然之氣。」是氣也,寓於尋常之中,而塞乎天地之間。卒然遇之,則王公失其貴,\ProperName{晉}、\ProperName{楚}失其富,\ProperName{良}、\ProperName{平}失其智,\ProperName{賁}、\ProperName{育}失其勇,\ProperName{儀}、\ProperName{秦}失其{辯},是孰使之然哉?其必有不依形而立,不恃力而存,不恃生而存,不隨死而亡者矣。故在天爲星辰,在地爲河嶽。幽則爲鬼神,而明則復爲人。此理之常,無足怪者。% 觀止作「辨」

自\ProperName{東漢}以來,道喪文弊,異端並起,歷\ProperName{唐}\ProperName{貞觀}、\ProperName{開元}之盛,輔以\ProperName{房}、\ProperName{杜}、\ProperName{姚}、\ProperName{宋}而不能救。獨\ProperName{韓文公}起布衣,談笑而麾之,天下靡然從公,復歸於正,蓋三百年於此矣。文起八代之衰,而道濟天下之溺,忠犯人主之怒,而勇奪三軍之帥。豈非參天地、關盛衰、浩然而獨存者乎!蓋嘗論天人之辨,以謂人無所不至,惟天不容僞。智可以欺王公,不可以欺豚魚;力可以得天下,不可以得匹夫匹婦之心。故公之精誠,能開\ProperName{衡山}之雲,而不能回\ProperName{憲宗}之惑;能馴鱷魚之暴,而不能弭\ProperName{皇甫鎛}、\ProperName{李逢吉}之謗;能信於\ProperName{南海}之民,廟食百世,而不能使其身一日安於朝廷之上。蓋公之所能者,天也。所不能者,人也。

始,\ProperName{潮}人未知學,公命進士\ProperName{趙德}爲之師。自是\ProperName{潮}之士,皆篤於文行,延及齊民,至於今,號稱易治。信乎\ProperName{孔子}之言:「君子學道則愛人,小人學道則易使也。」\ProperName{潮}人之事公也,飲食必祭,水旱疾疫,凡有求必禱焉。而廟在刺史公堂之後,民以出入爲艱。前守欲請諸朝作新廟,不果。\ProperName{元祐}五年,朝散郎\ProperName{王}君\ProperName{滌}來守是邦,凡所以養士治民者,一以公爲師。民既悅服,則出令曰:「願新公廟者聽。」民{讙}\endnote{\BookTitle{觀止}作「懽」,據各本改。}趨之。卜地於州城之南七里,朞年而廟成。% 觀止作「懽」

或曰:「公去國萬里,而謫于\ProperName{潮},不能一歲而歸,沒而有知,其不眷戀於\ProperName{潮}也審矣。」\ProperName{軾}曰:「不然。公之神在天下者,如水之在地中,無所往而不在也。而\ProperName{潮}人獨信之深,思之至,焄蒿悽愴,若或見之。譬如鑿井得泉,而曰水專在是,豈理也哉!」\ProperName{元豐}{七}\endnote{\BookTitle{觀止}作「元」,據各本改。}年,詔封公\ProperName{昌黎伯},故榜曰「\ProperName{昌黎伯}\ProperName{韓文公}之廟」。\ProperName{潮}人請書其事於石,因作詩以遺之,使歌以祀公。其詞曰:% 觀止作「元」

\begin{center}
    \begin{tabular}{lll}
        公昔騎龍白雲鄉,&手抉雲漢分天章,&\ProperName{天孫}爲織雲錦裳。\\
        飄然乘風來帝旁,&下與濁世掃秕糠,&西{遊}\ProperName{咸池}略\ProperName{扶桑}。\\
        草木衣被昭回光,&追逐\ProperName{李}\ProperName{杜}參翱翔,&汗流\ProperName{籍}\ProperName{湜}走且僵。\\
        減沒倒{景}不可望,&作書詆佛譏君王,&要觀\ProperName{南海}窺\ProperName{衡}\ProperName{湘}。\\
        歷\ProperName{舜}\ProperName{九{疑}}弔\ProperName{英}\ProperName{皇}, &\ProperName{祝融}先驅\ProperName{海若}藏, &約束蛟鱷如驅羊。\\
        鈞天無人帝悲傷, &謳吟下招遣\ProperName{巫陽}, &犦牲雞卜羞我觴。\\
        於{粲}\endnote{\BookTitle{觀止}作「餐」,據各本改。}荔丹與蕉黃, &公不少留我涕滂,& 翩然被髮下\ProperName{大荒}。
    \end{tabular}
\end{center}
% 原作「遊」,文章軌範作「游」; 觀止作「影」; 觀止作「嶷」; 觀止作「餐」
\theendnotes

\section[乞校正陸贄奏議進御劄子\quad{\small 蘇軾}]{{\normalsize 蘇軾}\quad 乞校正\ProperName{陸贄}奏議進御劄子}
% \ProperName{元祐}八年五月七日,\ProperName{端明殿}學士兼翰林侍讀學士左朝奉郎守禮部尚書\ProperName{蘇軾},同\ProperName{呂希哲}、\ProperName{吳安詩}、\ProperName{豐稷}、\ProperName{趙彥若}、\ProperName{范祖禹}、\ProperName{顧臨}劄子奏。
臣等猥以空疎,備員講讀,聖明天縱,學問日新,臣等才有限而道無窮,心欲言而口不逮,以此自愧,莫知所爲。竊謂人臣之納忠,譬如醫者之用藥,藥雖進於醫手,方多傳於古人。若已經效於世間,不必皆從於己出。伏見\ProperName{唐}宰相\ProperName{陸贄},才本王佐,學爲帝師。論深切於事情,言不離於道德。智如\ProperName{子房},而文則過,辯如\ProperName{賈誼},而術不疎。上以格君心之非,下以通天下之志。但其不幸,仕不遇時,\ProperName{德宗}以苛刻爲能,而\ProperName{贄}諫之以忠厚。\ProperName{德宗}以猜{疑}\endnote{\BookTitle{觀止}作「忌」,\BookTitle{唐宋八大家文鈔}、\BookTitle{文編}同,據各本改。}爲術,而\ProperName{贄}勸之以推誠。\ProperName{德宗}好用兵,而\ProperName{贄}以消兵爲先。\ProperName{德宗}好聚財,而\ProperName{贄}以散財爲急。至於用人聽言之法,治邊{馭}\endnote{\BookTitle{觀止}作「御」,據各本改。}將之方,罪己以收人心,改過以應天道,去小人以除民患,惜名器以待有功,如此之流,未易悉數。可謂進苦口之藥石,鍼害身之膏肓。使\ProperName{德宗}盡用其言,則\ProperName{貞觀}可得而復。臣等每退自西閣,即私相告言,以陛下聖明,必喜\ProperName{贄}議論,但使聖賢之相契,即如臣主之同時。昔\ProperName{馮唐}論\ProperName{頗}、\ProperName{牧}之賢,則\ProperName{漢文}爲之太息;\ProperName{魏相}條\ProperName{晁}、\ProperName{董}之對,則\ProperName{孝宣}以致中興。若陛下能自得師,莫若近取諸\ProperName{贄}。夫六經三史、諸子百家,非無可觀,皆足爲治。但聖言幽遠,末學支離,譬如山海之崇深,難以一二而推擇。如\ProperName{贄}之論,開卷了然。聚古今之精英,實治亂之龜鑑。臣等欲取其奏議,稍加校正,繕寫進呈。願陛下置之坐隅,如見\ProperName{贄}面,反覆熟讀,如與\ProperName{贄}言。必能發聖性之高明,成治功於歲月。臣等不勝區區之意。取進止。% 原作「疑」; 觀止作「御」

\theendnotes

\section[前赤壁賦\quad{\small 蘇軾}]{{\normalsize 蘇軾}\quad 前\ProperName{赤壁}賦}
壬戌之秋,七月既望,\ProperName{蘇子}与客泛舟,遊於\ProperName{赤壁}之下。清風徐来,水波不興,舉酒屬客,誦\BookTitle{明月}之詩,歌\BookTitle{窈窕}之章。少焉,月出於東山之上,徘徊於斗牛之間,白露橫江,水光接天。縱一葦之所如,凌萬頃之茫然。浩浩乎如馮虛御風,而不知其所止;飄飄乎如遺世獨立,羽化而登仙。

於是飲酒樂甚,扣舷而歌之。歌曰:「桂棹兮蘭槳,擊空眀兮泝流光。渺渺兮予懷,望美人兮天一方。」客有吹洞簫者,{倚}\endnote{\BookTitle{觀止}作「依」,據各本改。}歌而和之,其聲嗚嗚然,如怨如慕,如泣如訴,餘音嫋嫋,不絕如縷。舞幽壑之潛蛟,泣孤舟之嫠婦。% 觀止作「依」

\ProperName{蘇子}愀然,正襟危坐,而問客曰:「何爲其然也?」客曰:「『月眀星稀,烏鵲南飛』,此非\ProperName{曹孟德}之詩乎?西望\ProperName{夏口},東望\ProperName{武昌},山川相繆,鬱乎蒼蒼,此非\ProperName{孟德}之困於\ProperName{周郎}者乎?方其破\ProperName{荆州},下\ProperName{江陵},順流而東也,舳艫千里,旌旗蔽空,釃酒臨江,橫槊賦詩,固一世之雄也,而今安在哉?況吾与子漁樵於江渚之上,侶魚蝦而友麋鹿。駕一葉之扁舟,舉匏樽以相屬。寄蜉蝣於天地,渺滄海之一粟。哀吾生之須臾,羨\ProperName{長江}之無窮。挾飛仙以遨遊,抱眀月而長終。知不可乎驟得,託遺響於悲風。」

\ProperName{蘇子}曰:「客亦知夫水與月乎?逝者如斯,而未嘗往也。盈虛者如彼,而卒莫消長也,蓋将自其變者而觀之,則天地曾不能以一瞬。自其不變者而觀之,則物與我皆無盡也,而又何羨乎?且夫天地之間,物各有主,苟非吾之所有,雖一毫而莫取。惟江上之清風,與山間之眀月,耳得之而爲聲,目遇之而成色,取之無禁,用之不竭,是造物者之無盡藏也,而吾與子之所共適。」客喜而笑,洗盞更酌。肴核既盡,杯盤狼籍,相與枕藉乎舟中,不知東方之既白。

\theendnotes

\section[後赤壁賦\quad{\small 蘇軾}]{{\normalsize 蘇軾}\quad 後\ProperName{赤壁}賦}
是歲十月之望,步自\ProperName{雪堂},將歸於\ProperName{臨皋}。二客從予,過\ProperName{黃泥}之\ProperName{阪}。霜露既降,木葉盡脫。人影在地,仰見明月。顧而樂之,行歌相答。已而歎曰:「有客無酒,有酒無肴,月白風清,如此良夜何?」客曰:「今者薄暮,舉網得魚,巨口細鱗,狀如\ProperName{松江}之鱸,顧安所得酒乎?」歸而謀諸婦。婦曰:「我有斗酒,藏之久矣,以待子不時之須。」於是攜酒與魚,復遊於\ProperName{赤壁}之下。江流有聲,斷岸千尺。山高月小,水落石出。曾日月之幾何,而江山不可復識矣。予乃攝衣而上,履巉巖,披蒙茸,踞虎豹,登虬龍,攀栖鶻之危巢,俯\ProperName{馮夷}之幽宮。蓋二客不能從焉。劃然長嘯,草木震動,山鳴谷應,風起水{涌}。予亦悄然而悲,肅然而恐,凜乎其不可留也。反而登舟,放乎中流,聽其所止而休焉。時夜將半,四顧寂寥,適有孤鶴,橫江東來,翅如車輪,玄裳縞衣,戞然長鳴,掠予舟而西也。須臾客去,予亦就睡,夢一道士羽衣翩躚,過\ProperName{臨皋}之下,揖予而言曰:「\ProperName{赤壁}之遊樂乎?」問其姓名,俛而不答。嗚呼噫嘻,我知之矣!疇昔之夜,飛鳴而過我者,非子也耶?道士顧笑,予亦驚{悟}\endnote{\BookTitle{觀止}作「寤」,\BookTitle{事類備要}同,據各本改。}。開戶視之,不見其處。% 觀止作「湧」; 觀止作「寤」

\theendnotes

\section[三槐堂銘\quad{\small 蘇軾}]{{\normalsize 蘇軾}\quad \ProperName{三槐堂}銘}
天可必乎?賢者不必貴,仁者不必壽。天不可必乎?仁者必有後。二者將安取衷哉!吾聞之\ProperName{申包胥}曰:「人{眾}\endnote{\BookTitle{觀止}作「定」,\BookTitle{事文類聚}同,據各本改。}者勝天,天定亦能勝人。」世之論天者,皆不待其定而求之,故以天爲茫茫。善者以怠,惡者以肆,\ProperName{盜跖}之壽,\ProperName{孔}\ProperName{顏}之厄,此皆天之未定者也。松柏生於山林,其始也困於蓬蒿,厄於牛羊,而其終也,貫四時閱千歲而不改者,其天定也。善惡之報,至於子孫,而其定也久矣。吾以所見所聞所傳聞\endnote{\BookTitle{觀止}「所見所聞」下無「所傳聞」,\BookTitle{崇古文訣}、\BookTitle{文章軌範}同,據各本補。}考之,而其可必也審矣。國之將興,必有世德之臣,厚施而不食其報,然後其子孫能與守文太平之主共天下之福。故兵部侍郎\ProperName{晉國王公}顯於\ProperName{漢}、\ProperName{周}之際,歷事\ProperName{太祖}、\ProperName{太宗},文武忠孝,天下望以爲相,而公卒以直道不容於時。蓋嘗手植三槐於庭曰:「吾子孫必有爲三公者。」已而其子\ProperName{魏國文正公}相\ProperName{真宗皇帝}於\ProperName{景德}、\ProperName{祥符}之間朝廷清明天下無事之時,享其福祿榮名者十有八年。今夫寓物於人,明日而取之,有得有否。而\ProperName{晉公}修德於身,責報於天,取必於數十年之後,如持左契,交手相付。吾是以知天之果可必也。吾不及見\ProperName{魏公},而見其子\ProperName{懿敏公},以直諫事\ProperName{仁宗皇帝},出入侍從將帥三十餘年,位不滿其德。天將復興\ProperName{王氏}也歟?何其子孫之多賢也。世有以\ProperName{晉公}比\ProperName{李栖筠}者,其雄才直氣,真不相上下。而\ProperName{栖筠}之子\ProperName{吉甫},其孫\ProperName{德裕},功名富貴,略與\ProperName{王氏}等,而忠{信}\endnote{\BookTitle{觀止}作「忠恕」,\BookTitle{文章軌範}同,據各本改。}仁厚,不及\ProperName{魏公}父子。由此觀之,\ProperName{王氏}之福蓋未艾也。\ProperName{懿敏公}之子\ProperName{鞏}與吾遊,好德而文,以世其家。吾是以錄之。銘曰:% 觀止作「定」,亦見於宋祝穆事文類聚; 原文「所聞」下有「所傳聞」,觀止脱,同宋楼昉崇古文訣、文章軌範等 ; 原作「忠信」,觀止同文章軌範作「忠恕」

\begin{quote}
    嗚呼休哉!\ProperName{魏公}之業,與槐俱萌。封植之勤,必世乃成。既相\ProperName{真宗},四方砥平。歸視其家,槐陰滿庭。吾儕小人,朝不及夕。相時射利,皇䘏闕德。庶幾僥倖,不種而獲。不有君子,其何能國。王城之東,\ProperName{晉公}所廬。鬱鬱三槐,惟德之符。嗚呼休哉!
\end{quote}
\vspace{-1em}
\theendnotes

\section[方山子傳\quad{\small 蘇軾}]{{\normalsize 蘇軾}\quad \ProperName{方山子}傳}
\ProperName{方山子},\ProperName{光}、\ProperName{黃}間隱人也。少時慕\ProperName{朱家}、\ProperName{郭解}爲人,閭里之俠皆宗之。稍壯,折節讀書,欲以此馳騁當世。然終不遇。晚乃遯於\ProperName{光}、\ProperName{黃}間曰\ProperName{岐亭}。庵居蔬食,不與世相聞。棄車馬,毀冠服,徒步往來,山中人莫識也。見其所著帽,方聳而高,曰:「此豈古方山冠之遺像乎?」因謂之\ProperName{方山子}。

余謫居於\ProperName{黃},過\ProperName{岐亭},適見焉。曰:「嗚呼,此吾故人\ProperName{陳慥}\ProperName{季常}也,何爲而在此?」\ProperName{方山子}亦矍然問余所以至此者。余告之故,俯而不答,仰而笑,呼余宿其家。環堵蕭然,而妻子奴婢皆有自得之意。余既聳然異之。

獨念\ProperName{方山子}少時使酒好劍,用財如糞土。前十{有}\endnote{\BookTitle{觀止}脱「有」字,據各本補。}九年,余在\ProperName{岐下},見\ProperName{方山子}從兩騎,挾二矢,遊西山。鵲起於前,使騎逐而射之,不獲。\ProperName{方山子}怒馬獨出,一發得之。因與余馬上論用兵及古今成敗,自謂一{世}\endnote{\BookTitle{觀止}作「時」,據各本改。}豪士。今幾日耳,精悍之色,猶見於眉間,而豈山中之人哉!% 觀止脱「有」; 觀止作「岐山」,亦見於唐宋八大家文鈔; 原作「游」; 觀止作「時」

然\ProperName{方山子}世有勳閥,當得官,使從事於其間,今已顯聞。而其家在\ProperName{洛陽},園宅壯麗與公侯等。\ProperName{河}北有田,歲得帛千匹,亦足以富樂。皆棄不取,獨來窮山中,此豈無得而然哉?

余聞\ProperName{光}、\ProperName{黃}間多異人,往往{陽}狂\endnote{\BookTitle{觀止}作「佯狂」,\BookTitle{文章辨體彙選}同,據各本改。}垢汙,不可得而見,\ProperName{方山子}儻見之歟?% 原作「陽狂」,觀止作「佯狂」亦見於文章辨體彙選

\theendnotes

\section[六國論\quad{\small 蘇轍}]{{\normalsize 蘇轍}\quad 六國論}
嘗讀六國世家,竊怪天下之諸侯,以五倍之地,十倍之眾,發憤西向,以攻山西千里之\ProperName{秦},而不免於滅亡。常爲之深思遠慮,以爲必有可以自安之計,蓋未嘗不咎其當時之士慮患之疎,而見利之淺、且不知天下之勢也。

夫\ProperName{秦}之所與諸侯爭天下者,不在\ProperName{齊}、\ProperName{楚}、\ProperName{燕}、\ProperName{趙}也,而在\ProperName{韓}、\ProperName{魏}之郊;諸侯之所與\ProperName{秦}爭天下者,不在\ProperName{齊}、\ProperName{楚}、\ProperName{燕}、\ProperName{趙}也,而在\ProperName{韓}、\ProperName{魏}之野。\ProperName{秦}之有\ProperName{韓}、\ProperName{魏},譬如人之有腹心之疾也。\ProperName{韓}、\ProperName{魏}塞\ProperName{秦}之衝,而蔽山東之諸侯,故夫天下之所重者,莫如\ProperName{韓}、\ProperName{魏}也。昔者\ProperName{範睢}用於\ProperName{秦}而收\ProperName{韓},\ProperName{商鞅}用於\ProperName{秦}而收\ProperName{魏};\ProperName{昭王}未得\ProperName{韓}、\ProperName{魏}之心,而出兵以攻\ProperName{齊}之\ProperName{剛壽},而\ProperName{範睢}以爲憂。然則\ProperName{秦}之所忌者可{以}\endnote{\BookTitle{觀止}脱「以」字,據各本補。}見矣。\ProperName{秦}之用兵於\ProperName{燕}、\ProperName{趙},\ProperName{秦}之危事也。越\ProperName{韓}過\ProperName{魏}而攻人之國都,\ProperName{燕}、\ProperName{趙}拒之於前,而\ProperName{韓}、\ProperName{魏}乘之於後,此危道也。而\ProperName{秦}之攻\ProperName{燕}、\ProperName{趙},未嘗有\ProperName{韓}、\ProperName{魏}之憂,則\ProperName{韓}、\ProperName{魏}之附\ProperName{秦}故也。夫\ProperName{韓}、\ProperName{魏},諸侯之障,而使\ProperName{秦}人得出入於其間,此豈知天下之勢耶?委區區之\ProperName{韓}、\ProperName{魏}以當強虎狼之\ProperName{秦},彼安得不折而入於\ProperName{秦}哉?\ProperName{韓}、\ProperName{魏}折而入於\ProperName{秦},然後\ProperName{秦}人得通其兵於東諸侯,而使天下{遍}受其禍。% 觀止脱「以」;欒城集原作「遍」

夫\ProperName{韓}、\ProperName{魏}不能獨當\ProperName{秦},而天下之諸侯藉之以蔽其西,故莫如厚\ProperName{韓}親\ProperName{魏}以擯\ProperName{秦}。\ProperName{秦}人不敢逾\ProperName{韓}、\ProperName{魏}以窺\ProperName{齊}、\ProperName{楚}、\ProperName{燕}、\ProperName{趙}之國,而\ProperName{齊}、\ProperName{楚}、\ProperName{燕}、\ProperName{趙}之國因得以自完於其間矣。以四無事之國,佐當寇之\ProperName{韓}、\ProperName{魏},使\ProperName{韓}、\ProperName{魏}無東顧之憂,而爲天下出身以當\ProperName{秦}兵。以二國委\ProperName{秦},而四國休息於內,以陰助其急。若此,可以應夫無窮,彼\ProperName{秦}者將何爲哉?不知出此,而乃貪疆埸尺寸之利,背盟敗約,以自相屠滅,\ProperName{秦}兵未出,而天下諸侯已自困矣,至{使}\endnote{\BookTitle{觀止}作「至於」,據各本改。}\ProperName{秦}人得伺其隙以取其國。可不悲哉!% 觀止作「至於」

\theendnotes

\section[上樞密韓太尉書\quad{\small 蘇轍}]{{\normalsize 蘇轍}\quad 上樞密\ProperName{韓太尉}書}
太尉執事:\ProperName{轍}生好爲文,思之至深,以爲文者氣之所形。然文不可以學而能,氣可以養而致。\ProperName{孟子}曰:「我善養吾浩然之氣。」今觀其文章,寬厚宏博,充乎天地之間,稱其氣之小大。\ProperName{太史公}行天下,周覽四海名山大川,與\ProperName{燕}、\ProperName{趙}間豪俊交{遊},故其文疎蕩,頗有奇氣。此二子者,豈嘗執筆學爲如此之文哉?其氣充乎其中而溢乎其貌,動乎其言而見乎其文,而不自知也。% 欒城集原作「游」

\ProperName{轍}生十有九{年}\endnote{\BookTitle{觀止}作「轍生年十有九」,據各本改。}矣,其居家所與游者,不過其鄰里鄉黨之人,所見不過數百里之間,無高山大野可登覽以自廣。百氏之書雖無所不讀,然皆古人之陳跡,不足以激發其志氣。恐遂汩沒,故決然捨去,求天下奇聞壯觀,以知天地之廣大。過\ProperName{秦}、\ProperName{漢}之故都,恣觀\ProperName{終南}、\ProperName{嵩}、\ProperName{華}之高,北顧\ProperName{黃河}之奔流,慨然想見古之豪傑。至京師仰觀天子宮闕之壯,與倉廩府庫城池苑囿之富且大也,而後知天下之巨麗。見翰林\ProperName{歐陽公},聽其議論之宏辯,觀其容貌之秀偉,與其門人賢士大夫遊,而後知天下之文章聚乎此也。% 觀止作「轍生年十有九」,倒。

太尉以才略冠天下,天下之所恃以無憂,四夷之所憚以不敢發,入則\ProperName{周公}、\ProperName{召公},出則\ProperName{方叔}、\ProperName{召虎}。而\ProperName{轍}也,未之見焉。且夫人之學也,不志其大,雖多而何爲?\ProperName{轍}之來也,於山見\ProperName{終南}、\ProperName{嵩}、\ProperName{華}之高,於水見\ProperName{黃河}之大且深,於人見\ProperName{歐陽公}。而猶以爲未見太尉也。故願得觀賢人之光耀,聞一言以自壯,然後可以盡天下之大觀而無憾者矣。

\ProperName{轍}年少,未能通習吏事。嚮之來非有取於斗升之祿,偶然得之,非其所樂。然幸得賜歸待選,使得優游數年之間,將{歸}\endnote{\BookTitle{觀止}作「以」,\BookTitle{崇古文訣}同,據各本改。}益治其文,且學爲政。太尉茍以爲可教而辱教之,又幸矣。% 欒城集原作「歸」,觀止作「以」同崇古文訣

\theendnotes

\section[黃州快哉亭記\quad{\small 蘇轍}]{{\normalsize 蘇轍}\quad \ProperName{黃州}\ProperName{快哉亭}記}
\ProperName{江}出\ProperName{西陵},始得平地。其流奔放肆大,南合\ProperName{湘}、\ProperName{沅},北合\ProperName{漢}、\ProperName{沔},其勢益張。至於\ProperName{赤壁}之下,波流浸灌,與海相若。\ProperName{清河}\ProperName{張}君\ProperName{夢得}謫居\ProperName{齊安},即其廬之西南爲亭,以覽觀\ProperName{江}流之勝,而余兄\ProperName{子瞻}名之曰「快哉」。蓋亭之所見,南北百里,東西一舍。濤瀾洶湧,風雲開闔。晝則舟楫出沒於其前,夜則魚龍悲嘯於其下,變化倏忽,動心駭目,不可久視。今乃得玩之几席之上,舉目而足。西望\ProperName{武昌}諸山,岡陵起伏,草木行列,煙消日出,漁夫樵父之舍,皆可指數,此其所以爲「快哉」者也。至於長洲之濱,故城之墟,\ProperName{曹孟德}、\ProperName{孫仲謀}之所睥睨,\ProperName{周瑜}、\ProperName{陸遜}之所騁騖,其{流風}\endnote{\BookTitle{觀止}倒作「風流」,據各本改。}遺跡,亦足以稱快世俗。% 觀止倒作「風流」

昔\ProperName{楚襄王}從\ProperName{宋玉}、\ProperName{景差}於\ProperName{蘭臺}之宮,有風颯然至者,王披襟當之曰:「快哉此風!寡人所與庶人共者耶?」\ProperName{宋玉}曰:「此獨大王之雄風耳,庶人安得共之!」\ProperName{玉}之言蓋有諷焉。夫風無雌雄之異,而人有遇不遇之變。\ProperName{楚王}之所以爲樂,與庶人之所以爲憂,此則人之變也,而風何與焉?士生於世,使其中不自得,將何往而非病?使其中坦然,不以物傷性,將何適而非快?今\ProperName{張君}不以謫爲患,{竊會計之餘功}\endnote{\BookTitle{觀止}作「收會稽之餘」,\BookTitle{續文章正宗}、\BookTitle{文章辨體彙選}、\BookTitle{文編}作「收會計之餘功」,\BookTitle{欒城集}、\BookTitle{宋文鍳}作「竊會計之餘功」,據改。},而自放山水之間,此其中宜有以過人者。將蓬戶甕牖無所不快,而況乎濯\ProperName{長江}之清流,{揖}\endnote{\BookTitle{觀止}作「挹」,\BookTitle{續文章正宗}、\BookTitle{文章辨體彙選}、\BookTitle{唐宋八大家文鈔}同,\BookTitle{欒城集}、\BookTitle{宋文鍳}作「揖」,據改。}\ProperName{西山}之白雲,窮耳目之勝以自適也哉!不然,連山絕壑,長林古木,振之以清風,照之以明月,此皆騷人思士之所以悲傷憔悴而不能勝者,烏睹其爲快也{哉}\endnote{\BookTitle{觀止}脱「哉」,據各本補。}? % 觀止作「收會稽之餘」,續文章正宗等作「收會計之餘功」,今從欒城集原作「竊會計之餘功」; 欒城集原作「揖」續文章正宗等作「挹」; 觀止脱「哉」

\theendnotes

\section[寄歐陽舍人書\quad{\small 曾鞏}]{{\normalsize 曾鞏}\quad 寄\ProperName{歐陽舍人}書}
% \ProperName{鞏}頓首載拜舍人先生:
去秋人還,蒙賜書及所撰先大父墓碑銘,反覆觀誦,感與慚幷。

夫銘誌之著於世,義近於史,而亦有與史異者。蓋史之於善惡無所不書,而銘者,蓋古之人有功德材行志義之美者,懼後世之不知,則必銘而見之。或納於廟,或存於墓,一也。茍其人之惡,則於銘乎何有?此其所以與史異也。其辭之作,所以使死者無有所憾,生者得致其嚴。而善人喜於見傳,則勇於自立;惡人無有所紀,則以媿而懼。至於通材達識,義烈節士,嘉言善狀,皆見於篇,則足爲後法警勸之道。非近乎史,其將安近?

及世之衰,人之子孫者,一欲褒揚其親而不本乎理。故雖惡人,皆務勒銘以誇後世。立言者既莫之拒而不爲,又以其子孫之所請也,書其惡焉,則人情之所不得,於是乎銘始不實。後之作銘者,常觀其人。茍託之非人,則書之非公與是,則不足以行世而傳後。故千百年來,公卿大夫至於里巷之士,莫不有銘,而傳者蓋少,其故非他,託之非人,書之非公與是故也。

然則孰爲其人而能盡公與是歟?非畜道德而能文章者無以爲也。蓋有道德者之於惡人,則不受而銘之,於眾人則能辨焉。而人之行,有情善而跡非,有意奸而外淑,有善惡相懸而不可以實指,有實大於名,有名侈於實。猶之用人,非畜道德者惡能辨之不惑,議之不徇?不惑不徇,則公且是矣。而其辭之不工,則世猶不傳,於是又在其文章兼勝焉。故曰非畜道德而能文章者無以爲也。豈非然哉?

然畜道德而能文章者,雖或並世而有,亦或數十年或一二百年而有之。其傳之難如此,其遇之難又如此。若先生之道德文章,固所謂數百年而有者也。先祖之言行卓卓,幸遇而得銘其公與是,其傳世行後無疑也。而世之學者,每觀傳記所書古人之事,至{其}\endnote{\BookTitle{觀止}作「至於」,據各本改。}所可感,則往往衋然不知涕之流落也,況其子孫也哉?況\ProperName{鞏}也哉?其追晞祖德而思所以傳之之{繇},則知先生推一賜於\ProperName{鞏}而及其三世,其感與報,宜若何而圖之?% 觀止作「至於」; 觀止作「由」

抑又思若\ProperName{鞏}之淺薄滯拙,而先生進之;先祖之屯蹶否塞以死,而先生顯之。則世之魁閎豪傑不世出之士,其誰不願進於門?潛遁幽抑之士,其誰不有望於世?善誰不爲?而惡誰不媿以懼?爲人之父祖者,孰不欲教其子孫?爲人之子孫者,孰不欲寵榮其父祖?此數美者,一歸於先生。既拜賜之辱,且敢進其所以然。所諭世族之次,敢不承教而加詳焉。愧甚。不宣。

\theendnotes

\section[贈黎安二生序\quad{\small 曾鞏}]{{\normalsize 曾鞏}\quad 贈\ProperName{黎}\ProperName{安}二生序}
\ProperName{趙郡}\ProperName{蘇軾},余之同年友也,自\ProperName{蜀}以書至京師遺余,稱\ProperName{蜀}之士曰\ProperName{黎生}、\ProperName{安生}者。既而\ProperName{黎生}攜其文數十萬言,\ProperName{安生}攜其文亦數千言,辱以顧余。讀其文,誠閎壯雋偉,善反{復}馳騁,窮盡事理,而其材力之放縱,若不可極者也。二生固可謂魁奇特起之士,而\ProperName{蘇君}固可謂善知人者也。% 元豐類稿原作「復」

頃之,\ProperName{黎生}補\ProperName{江陵府}司法參軍,將行,請予言以爲贈。余曰:「余之知生,既得之於心矣,乃將以言相求於外邪?」\ProperName{黎生}曰:「生與\ProperName{安生}之學於斯文,里之人皆笑以爲迂闊。今求子之言,蓋將解惑於里人。」余聞之,自顧而笑。夫世之迂闊,孰有甚於予乎?知信乎古而不知合乎世,知志乎道而不知同乎俗,此余所以困於今而不自知也。世之迂闊,孰有甚於予乎?今生之迂,特以文不近俗,迂之小者耳,患爲笑於里之人。若余之迂大矣,使生持吾言而歸,且重得罪,庸詎止於笑乎?然則若余之於生,將何言哉?謂余之迂爲善,則其患若此;謂爲不善,則有以合乎世,必違乎古,有以同乎俗,必離乎道矣。生其無急於解里人之惑,則於是焉,必能擇而取之。遂書以贈二生,幷示\ProperName{蘇君},以爲何如也。

\section[讀孟嘗君傳\quad{\small 王安石}]{{\normalsize 王安石}\quad 讀\BookTitle{孟嘗君傳}}
世皆稱\ProperName{孟嘗君}能得士,士以故歸之,而卒頼其力以脫於虎豹之\ProperName{秦}。嗟乎!\ProperName{孟嘗君}特雞鳴狗盗之雄耳,豈足以言得士?不然,擅\ProperName{齊}之強,得一士焉,宜可以南面而制\ProperName{秦},尚何取雞鳴狗盜之力哉?{夫}\endnote{\BookTitle{觀止}脱「夫」,\BookTitle{文章軌範}同,據各本改。}雞鳴狗盜之出其門,此士之所以不至也。% 觀止脱「夫」同文章軌範

\theendnotes

\section[同學一首別子固\quad{\small 王安石}]{{\normalsize 王安石}\quad 同學一首別\ProperName{子固}}
\ProperName{江}之南有賢人焉,字\ProperName{子固},非今所謂賢人者,予慕而友之。\ProperName{淮}之南有賢人焉,字\ProperName{正之},非今所謂賢人者,予慕而友之。二賢人者,足未嘗相過也,口未嘗相語也,辭幣未嘗相接也。其師若友,豈盡同哉?予考其言行,其不相似者,何其少也!曰,學聖人而已矣。學聖人,則其師若友,必學聖人者。聖人之言行豈有二哉?其相似也適然。

予在\ProperName{淮南}爲\ProperName{正之}道\ProperName{子固}。\ProperName{正之}不予疑也。還\ProperName{江南}爲\ProperName{子固}道\ProperName{正之},\ProperName{子固}亦以爲然。予又知所謂賢人者,既相似,又相信不疑也。

\ProperName{子固}作\BookTitle{懷友}一首遺予,其大略欲相扳以至乎中庸而後已。\ProperName{正之}蓋亦常云爾。夫安驅徐行,轥中庸之庭,而造於其{堂}\endnote{\BookTitle{觀止}作「室」,據各本改。},舍二賢人者而誰哉?予昔非敢自必其有至也,亦願從事於左右焉爾。輔而進之,其可也。噫!官有守,私有繫,會合不可以常也,作\BookTitle{同學一首別子固}以相警且相慰云。% 觀止作「室」

\theendnotes

\section[遊褒禪山記\quad{\small 王安石}]{{\normalsize 王安石}\quad 遊\ProperName{褒禪山}記}
\ProperName{褒禪山}亦謂之\ProperName{華山},\ProperName{唐}浮圖\ProperName{慧褒}始舍於其址,而卒葬之,以故其後名之曰「褒禪」。今所謂\ProperName{慧空禪院}者,\ProperName{褒}之廬冢也。距其院東五里,所謂\ProperName{華{山}洞}者,以其在\ProperName{華山}之陽名之也。距洞百餘步,有碑仆道,其文漫滅,獨其爲文猶可識,曰「花山」。今言「華」如「華實」之「華」者,蓋音謬也。其下平曠,有泉側出,而記遊者甚眾,所謂「前洞」也。由山以上五六里,有穴窈然,入之甚寒,問其深,則其好遊者不能窮也,謂之「後洞」。余與四人擁火以入,入之愈深,其進愈難,而其見愈奇。有怠而欲出者,曰:「不出,火且盡。」遂與之俱出。蓋予所至,比好遊者尚不能十一,然視其左右,來而記之者已少。蓋其又深,則其至又加少矣。方是時,予之力尚足以入,火尚足以明也。既其出,則或咎其欲出者,而予亦悔其隨之,而不得極夫遊之樂也。

於是予有歎焉。古人之觀於天地、山川、草木、蟲魚、鳥獸,往往有得,以其求思之深,而無不在也。夫夷以近,則遊者眾;險以遠,則至者少。而世之奇偉瑰怪非常之觀,常在於險遠,而人之所罕至焉。故非有志者,不能至也。有志矣,不隨以止也,然力不足者,亦不能至也。有志與力,而又不隨以怠,至於幽暗昏惑,而無物以相之,亦不能至也。然力足以至焉,於人爲可譏,而在己爲有悔。盡吾志也而不能至者,可以無悔矣,其孰能譏之乎?此予之所得也。余於仆碑,又以悲夫古書之不存,後世之謬其傳而莫能名者,何可勝道也哉!此所以學者不可以不深思而慎取之也。

四人者:\ProperName{廬陵}\ProperName{蕭君圭}\ProperName{君玉},\ProperName{長樂}\ProperName{王回}\ProperName{深父},予弟\ProperName{安國}\ProperName{平父}、\ProperName{安上}\ProperName{純父}。% \ProperName{至和}元年七月某甲子,\ProperName{臨川}\ProperName{王}某記。

\section[泰州海陵縣主簿許君墓誌銘\quad{\small 王安石}]{{\normalsize 王安石}\quad \ProperName{泰州}\ProperName{海陵縣}主簿\ProperName{許君}墓誌銘}
君諱\ProperName{平},宇\ProperName{秉之},姓\ProperName{許氏}。余嘗譜其世家,所謂今\ProperName{泰州}\ProperName{海陵縣}主簿者也。

君既與兄\ProperName{元}相友愛稱天下,而自少卓犖不羈,善{辨}說,與其兄俱以智略爲當世大人所器。\ProperName{寶元}時,朝廷開方略之選,以招天下異能之士,而\ProperName{陝西}大帥\ProperName{范文正公}、\ProperName{鄭文肅公}爭以君所爲書以薦。於是得召試爲太廟齋郎,已而選\ProperName{泰州}\ProperName{海陵縣}主簿。貴人多薦君有大才,可試以事,不宜棄之州縣。君亦常慨然自許,欲有所爲,然終不得一用其智能以卒。噫!其可哀也已。% 觀止作「辯」

士固有離世異俗,獨行其意,罵譏、笑侮、困辱而不悔。彼皆無眾人之求,而有所待於後世者也,其齟齬固宜。若夫智謀功名之士,窺時俯仰,以赴勢{物}\endnote{\BookTitle{觀止}作「勢利」,據各本改。}之會,而輒不遇者,乃亦不可勝數。辯足以移萬物,而窮於用說之時;謀足以奪三軍,而辱於右武之國。此又何說哉?嗟乎!彼有所待而不悔者,其知之矣。% 觀止作「勢利」

君年五十九,以\ProperName{嘉祐}某年某月某甲子,葬\ProperName{真州}之\ProperName{楊子縣}\ProperName{甘露鄉}某所之原。夫人\ProperName{李氏}。子男\ProperName{瓌},不仕;\ProperName{璋},\ProperName{真州}司戶參軍;\ProperName{琦},太廟齋郎;\ProperName{琳},進士。女子五人,已嫁二人,進士\ProperName{周奉先},\ProperName{泰州}\ProperName{泰興}令\ProperName{陶舜元}。銘曰:

\begin{quote}
有拔而起之,莫擠而止之。嗚呼\ProperName{許君}!而已於斯,誰或使之?
\end{quote}
\vspace{-1em}
\theendnotes
% Proofed 29 June 2022
% Ref. 
% - 蘇軾文集, 中華書局(1986)
% - 欒城集, 上海古籍(1987)
% - 曾鞏集, 中華書局(1984)
% - 臨川先生文集, 中華書局(1959)
% - 古文觀止, 中華書局(1959)