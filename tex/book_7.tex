\section[陳情表\quad{\small 李密}]{{\normalsize 李密}\quad 陳情表}
臣\ProperName{密}言:臣以險釁,夙遭閔凶。生孩六月,慈父見背。行年四歲,舅奪母志。祖母\ProperName{劉},愍臣孤弱,躬親撫養。臣少多疾病,九歲不行,零丁孤苦,至于成立。既無叔伯,終鮮兄弟;門衰祚薄,晚有兒息。外無朞功強近之親,內無應門五尺之僮;煢煢孑立,形影相弔。而\ProperName{劉}夙嬰疾病,常在牀蓐;臣侍湯藥,未曾廢離。

逮奉聖朝,沐浴清化。前太守臣\ProperName{逵}察臣孝廉,後刺史臣\ProperName{榮}舉臣秀才。臣以供養無主,辭不赴命。詔書特下,拜臣郎中,尋蒙國恩,除臣洗馬。猥以微賤,當侍東宮,非臣隕首所能上報。臣具以表聞,辭不就職。詔書切峻,責臣逋慢。郡縣逼迫,催臣上道;州司臨門,急於星火。臣欲奉詔奔馳,則\ProperName{劉}病日篤;欲苟順私情,則告訴不許。臣之進退,實爲狼狽。

伏惟聖朝以孝治天下,凡在故老,猶蒙矜育,況臣孤苦,特爲尤甚。且臣少事僞朝,歷職郎署,本圖宦達,不矜名節。今臣亡國賤俘,至微至陋,過蒙拔擢,寵命優渥\endnote{\BookTitle{觀止}脱「寵命優渥」,據\BookTitle{文選}補。},豈敢盤桓,有所希冀!但以\ProperName{劉}日薄西山,氣息奄奄,人命危淺,朝不慮夕。臣無祖母,無以至今日;祖母無臣,無以終餘年。母孫二人,更相爲命。是以區區,不能廢遠。臣\ProperName{密}今年四十有四,祖母\ProperName{劉}今年九十有六,是臣盡節於陛下之日長,報養\ProperName{劉}之日短也。烏鳥私情,願乞終養!

臣之辛苦,非獨\ProperName{蜀}之人士及二州牧伯所見明知,皇天后土,實所共鑒。願陛下矜愍愚誠,聽臣微志,庶\ProperName{劉}僥倖,保卒餘年。臣生當隕首,死當結草。

臣不勝犬馬怖懼之情,謹拜表以聞。 

\theendnotes

\section[蘭亭集序\quad{\small 王羲之}]{{\normalsize 王羲之}\quad \ProperName{蘭亭}集序}
\ProperName{永和}九年,歲在癸丑,暮春之初,會于\ProperName{會稽}\ProperName{山陰}之\ProperName{蘭亭},脩稧事也。羣賢畢至,少長咸集。此地有崇山峻嶺,茂林脩竹,又有清流激湍,映帶左右,引以爲流觴曲水,列坐其次。雖無絲竹管弦之盛,一觴一詠,亦足以暢敘幽情。

是日也,天朗氣清,恵風和暢,仰觀宇宙之大,俯察品類之盛,所以遊目騁懷,足以極視聽之娛,信可樂也。

夫人之相與,俯仰一世,或取諸懷抱,悟\endnote{\BookTitle{觀止}作「晤」,據\BookTitle{晉書}等改。}言一室之内,或因寄所託,放浪形骸之外。雖趣舍萬殊,静躁不同,當其欣於所遇,暫得於己,快然自足,不知老之将至\endnote{\BookTitle{觀止}「不知」上有「曾」字,據\BookTitle{晉書}等刪。}。及其所之旣惓,情随事遷,感慨係之矣。向之所欣,俛仰之閒以爲陳迹,猶不能不以之興懷。況脩短随化,終期於盡。古人云:「死生亦大矣。」豈不痛哉!

每攬昔人興感之由,若合一契,未嘗不臨文嗟悼,不能喻之於懷。固知一死生爲虛誕,齊\ProperName{彭}殤爲妄作。後之視今,亦猶今之視昔,悲夫!故列敘時人,錄其所述,雖世殊事異,所以興懷,其致一也。後之覽者,亦将有感於斯文。

\theendnotes

\section[歸去來辭\quad{\small 陶淵明}]{{\normalsize 陶淵明}\quad 歸去來辭}
歸去來兮,田園將蕪胡不歸?旣自以心爲形役,奚惆悵而獨悲?悟已往之不諫,知來者之可追。實迷途其未遠,覺今是而昨非。舟遙遙以輕颺,風飄飄而吹衣。問征夫以前路,恨晨光之熹微。乃瞻衡宇,載欣載奔。僮僕歡迎,稚子候門。三徑就荒,松菊猶存。攜幼入室,有酒盈罇。引壺觴以自酌,眄庭柯以怡顏。倚南牕以寄傲,審容膝之易安。園日涉以成趣,門雖設而常關。策扶老以流憩,時矯首而遐觀。雲無心以出岫,鳥倦飛而知還。景翳翳以將入,撫孤松而盤桓。歸去來兮,請息交以絕游。世與我而相遺,復駕言兮焉求?悅親戚之情話,樂琴書以消憂。農人告余以春及,將有事于西疇。或命巾車,或棹孤舟。既窈窕以尋壑,亦崎嶇而經丘\endnote{\BookTitle{觀止}作避諱「邱」,改回。}。木欣欣以向榮,泉涓涓而始流。善萬物之得時,感吾生之行休。已矣乎!寓形宇內復幾時,曷不委心任去留?胡爲乎遑遑欲何之?富貴非吾願,帝鄉不可期。懷良辰以孤往,或植杖而耘耔。登東皋以舒嘯,臨清流而賦詩。聊乘化以歸盡,樂夫天命復奚疑? 

\theendnotes

\section[桃花源記\quad{\small 陶淵明}]{{\normalsize 陶淵明}\quad 桃花源記}
\ProperName{晉}\ProperName{太元}中,\ProperName{武陵}人捕魚爲業。緣溪行,忘路之遠近,忽逢桃花林,夾岸數百步,中無雜樹,芳草鮮美,落英繽紛。漁人甚異之。復前行,欲窮其林,林盡水源,便得一山,山有小口,髣髴若有光。便捨船從口入,初極狹,纔通人,復行數十步,豁然開㓪,土地平曠,屋舍儼然。有良田、美池、桑竹之屬。阡陌交通,雞犬相聞。其中往來種作,男女衣着,悉如外人。黃髪垂髫,並怡然自樂。見漁人乃大驚,問所從來,具答之。便要還家,設酒殺雞作食。村中聞有此人,咸來問訊。自云先世避秦時亂,率妻子邑人,來此絶境,不復出焉,遂與外人間隔。問今是何世,乃不知有\ProperName{漢},無論\ProperName{魏}\ProperName{晉}。此人一一爲具言所聞,皆歎惋。餘人各復延至其家,皆出酒食,停數日,辭去。此中人語云:「不足爲外人道也!」旣出,得其船,便扶向路,處處誌之,及郡下。詣太守說如此,太守即遣人隨其往,尋向所誌,遂迷不復得路。\ProperName{南陽}\ProperName{劉子驥},髙尚士也,聞之,欣然親往,未果,尋病終,後遂無問津者。

\section[五柳先生傳\quad{\small 陶淵明}]{{\normalsize 陶淵明}\quad 五柳先生傳}
先生不知何許人也,亦不詳其姓字。宅邊有五柳樹,因以爲號焉。閑靜少言,不慕榮利。好讀書,不求甚解,毎有會意,便欣然忘食。性嗜酒,家貧不能常得,親舊知其如此,或置酒而招之。造飲輒盡,期在必醉,旣醉而退,曾不吝情去留。環堵蕭然,不蔽風日,短褐穿結,簞瓢屢空,晏如也。常著文章自娛,頗示己志。忘懷得失,以此自終。贊曰:

\ProperName{黔婁}有言:「不戚戚於貧賤,不汲汲於富貴。」極其言茲若人之儔乎?酣\endnote{\BookTitle{觀止}作「銜」,據\BookTitle{陶淵明集}改。\ProperName{龔斌}\BookTitle{陶淵明集校箋}:\ProperName{曾}本云,一作「酒酣自得,賦詩樂志」;\ProperName{蘇}寫本、\ProperName{咸豐}本同;「酣」\ProperName{李}本作「酬」。}觴賦詩,以樂其志。\ProperName{無懷}氏之民歟?\ProperName{葛天}氏之民歟?

\theendnotes

\section[北山移文\quad{\small 孔稚珪}]{{\normalsize 孔稚珪}\quad 北山移文}
\ProperName{鍾山}之英,\ProperName{草堂}之靈,馳煙驛路,勒移山庭。夫以耿介拔俗之標,蕭灑出塵之想。度白雪以方絜,干青雲而直上。吾方知之矣。若其亭亭物表,皎皎霞外,芥千金而不盼,屣萬乘其如脫。聞鳳吹於\ProperName{洛}浦,值薪歌於\ProperName{延}瀨。固亦有焉。豈其終始參差,蒼黃翻覆,淚\ProperName{翟子}之悲,慟\ProperName{朱公}之哭。乍迴跡以心染,或先貞而後黷,何其謬哉!嗚呼,\ProperName{尚生}不存,\ProperName{仲氏}既往,山阿寂寥,千載誰賞?

世有\ProperName{周子},儁俗之士,旣文旣博,亦玄亦史。然而學遁\ProperName{東魯},習隱\ProperName{南郭}。竊吹\ProperName{草堂},濫巾北岳。誘我松桂,欺我雲壑。雖假容於江皋,乃纓情於好爵。其始至也,將欲排\ProperName{巢父},拉\ProperName{許由},傲百氏。蔑王侯。風情張日,霜氣橫秋。或歎幽人長往,或怨王孫不游。談空空於釋部,覈玄玄於道流。\ProperName{務光}何足比,\ProperName{涓子}不能儔。

及其鳴騶入谷,鶴書赴隴,形馳魄散,志變神動。爾乃眉軒席次,袂聳筵上。焚芰製而裂荷衣,抗塵容而走俗狀。風雲悽其帶憤,石泉咽而下愴。望林巒而有失,顧草木而如喪。至其鈕金章,綰墨綬。跨屬城之雄,冠百里之首。張英風於海甸,馳妙譽於\ProperName{浙右}。道帙長擯,法筵久埋。敲扑喧囂犯其慮,牒訴倥傯裝其懷。\BookTitle{琴歌}既斷,\BookTitle{酒賦}無續。常綢繆於結課,每紛綸於折獄。籠\ProperName{張}\ProperName{趙}於往圖,架\ProperName{卓}\ProperName{魯}於前籙。希蹤\ProperName{三輔}豪,馳聲\ProperName{九州}牧。使我\endnote{\ProperName{觀止}作「其」,據\BookTitle{文選}改。}高霞孤映,明月獨舉,青松落陰,白雲誰侶?礀戶摧絕無與歸,石逕荒涼徒延佇。至於還飈入幕,寫霧出楹,蕙帳空兮夜鶴怨,山人去兮曉猨驚。昔聞投簪逸海岸,今見解蘭縛塵纓。

於是南岳獻嘲,北隴騰笑。列壑爭譏,攢峯竦誚。慨遊子之我欺,悲無人以赴弔。故其林慙無盡,澗愧不歇。秋桂遣風,春蘿罷月。騁西山之逸議,馳東皋之素謁。今又促裝下邑,浪拽\endnote{\ProperName{觀止}作「栧」,據\BookTitle{文選}改。}上京。雖情投於魏闕,或假步於山扃。豈可使芳杜厚顏,薜荔蒙恥,碧嶺再辱,丹崖重滓,塵游躅於蕙路,汙淥池以洗耳?宜扃岫幌,掩雲關。斂輕霧,藏鳴湍。截來轅於谷口,杜妄轡於郊端。於是叢條瞋膽,疊穎怒魄,或飛柯以折輪,乍低枝而掃跡。請迴俗士駕,爲君謝逋客。

\theendnotes

\section[諫太宗十思疏\quad{\small 魏徵}]{{\normalsize 魏徵}\quad 諫\ProperName{太宗}十思疏}
臣聞求木之長者,必固其根本;欲流之遠者,必浚其泉源;思國之安者,必積其德義。源不深而望流之遠,根不固而求木之長?德不厚而思國之治\endnote{\BookTitle{觀止}作「安」,\BookTitle{冊府元龜}、\BookTitle{文章辨體}同,\BookTitle{舊唐書}及各集作「治」,據改,\BookTitle{貞觀政要}作「理」。},臣雖下愚,知其不可,而況於明哲乎!人君當神器之重,居域中之大\endnote{各本「居域中之大」下有「將崇極天之峻,永保無疆之休」句,\BookTitle{觀止}省文,\BookTitle{文章辨體}同。}。不念居安思危,戒奢以儉\endnote{各本下有「德不處其厚,情不勝其欲」,\BookTitle{觀止}省文,\BookTitle{文章辨體}同。},斯亦伐根以求木茂,塞源而欲流長者\endnote{\BookTitle{觀止}無「者」字,\BookTitle{文章辨體}同,據各本補。}也。

凡百\endnote{\BookTitle{觀止}作「昔」,\BookTitle{文章辨體}同,據各本改。}元首,承天景命\endnote{各本「承天景命」下有「莫不殷憂而道著,功成而德衰」句\BookTitle{觀止}省文,\BookTitle{文章辨體}同。}。有善始者實繁,能\endnote{\BookTitle{觀止}無「能」字,\BookTitle{文章辨體}同,據各本補。}克終者蓋寡,豈其取之易而\endnote{\BookTitle{觀止}無「而」字,\BookTitle{文章辨體}同,據各本補。}守之難乎?\endnote{各本「守之難乎」下有「昔取之而有餘,今守之而不足,何也?」\BookTitle{觀止}省文,\BookTitle{文章辨體}同。}蓋\endnote{\BookTitle{觀止}同\BookTitle{文章辨體},各本作「夫」。}在殷憂必竭誠以待下,既得志則縱情以傲物。竭誠則\ProperName{胡}\endnote{\BookTitle{觀止}作「吳」,據各本改。}\ProperName{越}爲一體,傲物則骨肉爲行路。雖董之以嚴刑,振之以威怒,終茍免而不懷仁,貌恭而不心服。怨不在大,可畏惟人。載舟覆舟,所宜深愼,奔車朽索,其可忽乎!\endnote{各本「其可忽乎」下有「奔車朽索,其可忽乎」,\BookTitle{觀止}省文,\BookTitle{文章辨體}同。}

誠\endnote{各本「誠」上有「君人者」,\BookTitle{觀止}省文,\BookTitle{文章辨體}同。}能見可欲則思知足以自戒,將有所作則思知止以安人,念高危則思謙沖而自牧,懼滿溢則思江海下百川,樂盤遊則思三驅以爲度,憂懈怠則思愼始而敬終,慮壅蔽則思虛心以納下,想\endnote{\BookTitle{觀止}作「懼」,\BookTitle{文章辨體}同,據各本改。}讒邪則思正身以黜惡,恩所加則思無因喜以謬賞,罰所及則思無因\endnote{\BookTitle{觀止}作「以」,\BookTitle{文章辨體}同,據各本改。}怒而濫刑。總此十思,弘\endnote{\BookTitle{觀止}避諱作「宏」,據各本改。}茲九德,簡能而任之,擇善而從之。則智者盡其謀,勇者竭其力,仁者播其惠,信者效其忠。文武並用,垂拱而治\endnote{「文武並用,垂拱而治」:\BookTitle{觀止}同\BookTitle{文章辨體},各本作:「文武爭馳,君臣無事,可以盡豫遊之樂,可以養\ProperName{松}\ProperName{喬}之壽,鳴琴垂拱,不言而化。」\ProperName{羅士琳}\BookTitle{舊唐書校勘記}:「君臣無事」\BookTitle{御覽}同\BookTitle{英華}作「在君」。},何必勞神苦思,代百司之職役哉!
\endnote{「代百司之職役哉」:\BookTitle{觀止}同\BookTitle{文章辨體},各本作:「代下司職,役聰明之耳目,虧無爲之大道哉!」}

\theendnotes

\section[爲徐敬業討武曌檄\quad{\small 駱賓王}]{{\normalsize 駱賓王}\quad 爲\ProperName{徐敬業}討\ProperName{武曌}檄}
偽臨朝\ProperName{武氏}者,性非和順,地實寒微。昔充\ProperName{太宗}下陳,曾以更衣入侍,洎乎晚節,穢亂春宮。潛隱先帝之私,陰圖後房之嬖。入門見嫉,蛾眉不肯讓人;掩袖工讒,狐媚偏能惑主。踐元后於翬翟,陷吾君於聚麀。加以虺蜴爲心,豺狼成性。近狎邪僻,殘害忠良,殺姊屠兄,弒君鴆母。人神之所同嫉,天地之所不容。猶復包藏禍心,窺竊神器。君之愛子,幽之於別宮;賊之宗盟,委之以重任。嗚呼!\ProperName{霍子孟}之不作,\ProperName{朱虛侯}之已亡。鷰啄皇孫,知\ProperName{漢}祚之將盡;龍漦帝后,識\ProperName{夏}庭之遽衰。

\ProperName{敬業},\ProperName{皇唐}舊臣,公侯冢子。奉先君之成業,荷本朝之厚恩。\ProperName{宋微子}之興悲,良有以也。\ProperName{袁君山}之流涕,豈徒然哉!是用氣憤風雲,志安社稷,因天下之失望,順宇內之推心,爰舉義旗,以清妖孽。南連\ProperName{百越},北盡\ProperName{三河}。鐵騎成羣,玉軸相接。\ProperName{海陵}紅粟,倉儲之積靡窮;\ProperName{江}浦黃旗,匡復之功何遠!班聲動而北風起,劍氣衝而南斗平。喑嗚則山岳崩頹,叱咤則風雲變色。以此制敵,何敵不摧?以此圖功,何功不克?

公等或居\ProperName{漢}地,或協\ProperName{周}親,或膺重寄於話言,或受顧命於\ProperName{宣室}。言猶在耳,忠豈忘心。一抔之土未乾,六尺之孤何託?倘能轉禍爲福,送往事居,共立勤王之勳,無廢大君之命,凡諸爵賞,同指山河。若其眷戀窮城,徘徊歧路,坐昧先機之兆,必貽後至之誅。請看今日之域中,竟是誰家之天下! 

\section[滕王閣序\quad{\small 王勃}]{{\normalsize 王勃}\quad \ProperName{滕王閣}序}
\ProperName{南昌}故郡,\ProperName{洪都}新府。星分翼軫,地接\ProperName{衡}\ProperName{廬}。襟三江而帶五湖,控\ProperName{蠻荊}而引\ProperName{甌越}。物華天寶,龍光射牛斗之墟;人傑地靈,\ProperName{徐孺}下\ProperName{陳蕃}之榻。雄州霧列,俊彩星馳。臺隍枕夷夏之交,賓主盡東南之美。都督\ProperName{閻公}之雅望,棨戟遙臨;\ProperName{宇文}新州之懿範,襜帷暫駐。十旬休暇,勝友如雲。千里逢迎,高朋滿座。騰蛟起鳳,\ProperName{孟學士}之詞宗;紫電青霜,\ProperName{王將軍}之武庫。家君作宰,路出名區。童子何知,躬逢勝餞。

時維九月,序屬三秋。潦水盡而寒潭清,煙光凝而暮山紫。儼驂騑於上路,訪風景於崇阿。臨帝子之長洲,得仙人之舊館。層巒聳翠,上出重霄;飛閣流丹,下臨無地。鶴汀鳧渚,窮島嶼之縈廻;桂殿蘭宮,列岡巒之體勢。

披繡闥,俯雕甍。山原曠其盈視,川澤盱其駭矚。閭閻撲地,鐘鳴鼎食之家;舸艦彌\endnote{\BookTitle{觀止}作「迷」,\BookTitle{事類備要}、\BookTitle{事文類聚}、\BookTitle{文章辨體彙選}同,\BookTitle{文苑英華}、\BookTitle{王子安集}、\BookTitle{全唐文}作「彌」,據改。}津,青雀黃龍之軸。虹銷雨霽,彩徹雲衢。落霞與孤鶩齊飛,秋水共長天一色。漁舟唱晚,響窮\ProperName{彭蠡}之濱;雁陣驚寒,聲斷\ProperName{衡陽}之浦。

遙吟俯暢,逸興遄飛。爽籟發而清風生,纖歌凝而白雲遏。\ProperName{睢園}綠竹,氣凌\ProperName{彭澤}之樽;\ProperName{鄴水}朱華,光照\ProperName{臨川}之筆。四美具,二難幷。窮睇盼\endnote{\BookTitle{觀止}作「眄」,各本作「盼」,據改。}於中天,極娛遊於暇日。天高地迥,覺宇宙之無窮;興盡悲來,識盈虛之有數。望\ProperName{長安}於日下,指\ProperName{吳會}於雲間。地勢極而\ProperName{南溟}深,天柱高而北辰遠。關山難越,誰悲失路之人;萍水相逢,盡是他鄉之客。懷帝閽而不見,奉\ProperName{宣室}以何年?

嗟乎!\endnote{\BookTitle{觀止}作「嗚呼」,\BookTitle{文章辨體彙選}同,據各本改。}時運不齊,命途多舛。\ProperName{馮唐}易老,\ProperName{李廣}難封。屈\ProperName{賈誼}於\ProperName{長沙},非無聖主;竄\ProperName{梁鴻}於海曲,豈乏明時?所賴君子安貧,達人知命。老當益壯,寧知白首之心;窮且益堅,不墜青雲之志。酌\ProperName{貪泉}而覺爽,處涸轍以猶懽。北海雖賒,扶搖可接;東隅已逝,桑榆非晚。\ProperName{孟嘗}高潔,空懷報國之情\endnote{\BookTitle{觀止}作「心」,\BookTitle{事類備要}、\BookTitle{文章辨體彙選}同,各本作「情」,據改,\BookTitle{事文類聚}作「思」。};\ProperName{阮籍}猖狂,豈效窮途之哭?

\ProperName{勃},三尺微命,一介書生,無路請纓,等\ProperName{終軍}之弱冠;有懷投筆,慕\ProperName{宗慤}之長風。捨簪笏於百齡,奉晨昏於萬里。非\ProperName{謝家}之寶樹,接\ProperName{孟氏}之芳鄰。他日趨庭,叨陪\ProperName{鯉}對;今晨捧袂,喜托龍門。\ProperName{楊意}不逢,撫凌雲而自惜;\ProperName{鍾期}既遇,奏\BookTitle{流水}以何慚?

嗚呼!勝地不常,盛筵難再。\ProperName{蘭亭}已矣,\ProperName{梓澤}丘\endnote{\BookTitle{觀止}作「坵」,各集作「丘」,據改,\BookTitle{全唐文}作「邱」。}墟。臨別贈言,幸承恩於偉餞;登高作賦,是所望於羣公。敢竭鄙誠,恭疏短引。一言均賦,四韻俱成。請灑潘江,各傾陸海云爾。\endnote{「請灑潘江各傾陸海云爾」\BookTitle{觀止}無,\BookTitle{事類備要}、\BookTitle{事文類聚}、\BookTitle{文章辨體彙選}同,據各集補。}
\begin{center}
\begin{tabular}{ll}
\ProperName{滕王}高閣臨江渚,&佩玉鳴鸞罷歌舞。\\
畫棟朝飛南浦雲,&珠簾暮捲西山雨。\\
閒雲潭影日悠悠,&物換星移幾度秋。\\
閣中帝子今何在?&檻外\ProperName{長江}空自流!
\end{tabular}
\end{center}
\theendnotes

\section[與韓荊州書\quad{\small 李白}]{{\normalsize 李白}\quad 與\ProperName{韓荊州}書}
\ProperName{白}聞天下談士相聚而言曰:生不用萬戶侯,但願一識\ProperName{韓荊州}。何令人之景慕一至於此耶?豈不以有\ProperName{周公}之風,躬吐握之事,使海內豪俊奔走而歸之,一登\ProperName{龍門},則聲譽十倍!所以龍盤鳳逸之士,皆欲收名定價於君侯。願君侯不以富貴而驕之,寒賤而忽之,則三千賓中有\ProperName{毛遂},使\ProperName{白}得脫穎而出,即其人焉。\ProperName{白}\ProperName{隴西}布衣,流落\ProperName{楚}、\ProperName{漢}。十五好劍術,徧干諸侯;三十成文章,歷抵卿相。雖長不滿七尺,而心雄萬夫。王公大臣許與氣義。此疇曩心跡,安敢不盡於君侯哉?

君侯制作侔神明,德行動天地,筆參造化,學究於天人。幸願開張心顏,不以長揖見拒。必若接之以高宴,縱之以清談,請日試萬言,倚馬可待。今天下以君侯爲文章之司命,人物之權衡,一經品題,便作佳士。而君侯何惜階前盈尺之地,不使\ProperName{白}揚眉吐氣,激昂青雲耶?

昔\ProperName{王子師}爲\ProperName{豫州},未下車即辟\ProperName{荀慈明};既下車又辟\ProperName{孔文舉}。\ProperName{山濤}作\ProperName{冀州},甄拔三十餘人,或爲侍中、尚書,先代所美。而君侯亦薦一\ProperName{嚴協律},入爲秘書郎。中間\ProperName{崔宗之}、\ProperName{房爲祖}、\ProperName{黎昕}、\ProperName{許瑩}之徒,或以才名見知,或以清白見賞。\ProperName{白}每觀其銜恩撫躬,忠義奮發,以此感激,知君侯推赤心於諸賢腹中,所以不歸他人而願委身國士。儻急難有用,敢效微驅。

且人非\ProperName{堯}、\ProperName{舜},誰能盡善?\ProperName{白}謨猷籌畫,安能自矜?至於制作,積成卷軸,則欲塵穢視聽,恐雕蟲小伎,不合大人。若賜觀芻蕘,請給{以}\endnote{\BookTitle{觀止}脱「以」字,據\BookTitle{李白集}校本補。}紙筆,兼之書人。然後退掃閒軒,繕寫呈上。庶青萍、結綠,長價於\ProperName{薛}、\ProperName{卞}之門。幸惟下流大開獎飾,惟君侯圖之。

\theendnotes 

\section[春夜宴桃李園序\quad{\small 李白}]{{\normalsize 李白}\quad 春夜宴桃李園序}
夫天地者,萬物之逆旅也;光陰者,百代之過客也。而浮生若夢,爲歡幾何?古人秉燭夜遊,良有以也。況陽春召我以煙景,大塊假我以文章。會桃花之芳園,序天倫之樂事。羣季俊秀,皆爲\ProperName{惠連};吾人詠歌,獨慚\ProperName{康樂}。幽賞未已,高談轉清。開瓊筵以坐花,飛羽觴而醉月。不有佳詠,何伸雅懷?如詩不成,罰依\ProperName{金谷}酒數。

\section[弔古戰場文\quad{\small 李華}]{{\normalsize 李華}\quad 弔古戰場文}
浩浩乎平沙無垠,敻不見人,河水縈帶,羣山糾紛。黯兮慘悴,風悲日曛。蓬斷草枯,凜若霜晨。鳥飛不下,獸鋌亡羣。亭長告余曰:「此古戰場也。常覆三軍,往往鬼哭,天陰則聞。」

傷心哉!\ProperName{秦}歟?\ProperName{漢}歟?將近代歟?吾聞夫\ProperName{齊}\ProperName{魏}徭戌,\ProperName{荊}\ProperName{韓}召募,萬里奔走,連年暴露。沙草晨牧,河冰夜渡;地闊天長,不知歸路。寄身鋒刃,腷臆誰訴?\ProperName{秦}\ProperName{漢}而還,多事四夷;中州耗斁,無世無之。古稱\ProperName{戎夏},不抗王師。文教失宣,武臣用奇;奇兵有異於仁義,王道迂闊而莫爲。

嗚呼噫嘻!吾想夫北風振漠,\ProperName{胡}兵伺便。主將驕敵,期門受戰。野豎旄旗,川迴組練。法重心駭,威尊命賤。利鏃穿骨,驚沙入面。主客相搏,山川震眩。聲{析}江河,勢崩雷電。

至若窮陰凝閉,凜冽海隅;積雪沒脛,堅冰在鬚。鷙鳥休巢,征馬踟躕,繒纊無溫,墮指裂膚。當此苦寒,天假強\ProperName{胡},憑陵殺氣,以相剪屠。徑截輜重,橫攻士卒;都尉新降,將軍{復}\endnote{\BookTitle{觀止}作「覆」,據各集改。}沒。屍填巨港之岸,血滿\ProperName{長城}之窟。無貴無賤,同爲枯骨,可勝言哉!鼓衰兮力盡,矢竭兮弦絕。白刃交兮寶刀折,兩軍蹙兮生死決。降矣哉,終身夷狄;戰矣哉,骨暴沙礫。鳥無聲兮山寂寂,夜正長兮風淅淅。魂魄結兮天沈沈,鬼神聚兮雲冪冪。日光寒兮草短,月色苦兮霜白。傷心慘目,有如是耶?

吾聞之:\ProperName{牧}用\ProperName{趙}卒,大破\ProperName{林胡},開地千里,遁逃\ProperName{匈奴}。\ProperName{漢}傾天下,財殫力痡。任人而已,其在多乎?\ProperName{周}逐\ProperName{獫狁},北至\ProperName{太原},既城朔方,全師而還。飲至策勳,和樂且閑。穆穆棣棣,君臣之間。\ProperName{秦}起\ProperName{長城},竟海爲關,荼毒生靈,萬里朱殷。\ProperName{漢}擊\ProperName{匈奴},雖得\ProperName{陰山},枕骸徧野,功不補患。

蒼蒼蒸民,誰無父母?提攜捧負,畏其不壽。誰無兄弟,如手如足?誰無夫婦,如賓如友?生也何恩,殺之何咎?其存其沒,家莫聞知。人或有言,將信將疑。悁悁心目,寢寐見之。布奠傾觴,哭望天涯。天地爲愁,草木淒悲。弔祭不至,精魂何依?必有兇年,人其流離。鳴呼噫嘻!時耶命耶?從古如斯。爲之奈何,守在四夷。

\theendnotes

\section[陋室銘\quad{\small 劉禹錫}]{{\normalsize 劉禹錫\endnote{疑委託。}}\quad 陋室銘}
山不在高,有仙則名。水不在深,有龍則靈。斯是陋室,惟吾德馨。苔痕上堦綠、草色入簾青。談笑有鴻儒、往來無白丁。可以調素琴,閱金經。無絲竹之亂耳,無案牘之勞形。\ProperName{南陽}\ProperName{諸葛廬}、\ProperName{西蜀}\ProperName{子雲亭},孔子云:何陋之有。

\theendnotes

\section[阿房宮賦\quad{\small 杜牧}]{{\normalsize 杜牧}\quad \ProperName{阿房宮}賦}
六王畢,四海一。\ProperName{蜀}山兀,\ProperName{阿房}出。覆壓三百餘里,隔離天日。\ProperName{驪山}北構而西折,直走\ProperName{咸陽}。二川溶溶,流入宮牆。五步一樓,十步一閣。廊腰縵迴,簷牙高啄。各抱地勢,鉤心鬬角。盤盤焉,囷囷焉,蜂房水渦,矗不知乎幾千萬落。長橋臥波,未雲何龍?複道行空,不霽何虹?高低冥迷,不知西東。歌臺暖響,春光融融;舞殿冷袖,風雨淒淒。一日之內,一宮之間,而氣候不齊。

妃嬪媵嬙,王子皇孫,辭樓下殿,輦來於\ProperName{秦},朝歌夜絃,爲\ProperName{秦}宮人。明星熒熒,開妝鏡也;綠雲擾擾,梳曉鬟也;\ProperName{渭}流漲膩,棄脂水也;煙斜霧橫,焚椒蘭也;雷霆乍驚,宮車過也;轆轆遠聽,杳不知其所之也。一肌一容,盡態極妍。縵立遠視,而望幸焉。有不得見者,三十六年。

\ProperName{燕}、\ProperName{趙}之收藏,\ProperName{韓}、\ProperName{魏}之經營,\ProperName{齊}、\ProperName{楚}之精英,幾世幾年,取掠其人,倚疊如山。一旦不能有,輸來其間。鼎鐺玉石,金塊珠礫,棄擲邐迤,\ProperName{秦}人視之,亦不甚惜。嗟乎!一人之心,千萬人之心也。\ProperName{秦}愛紛奢,人亦念其家。奈何取之盡錙銖,用之如泥沙。使負棟之柱,多於南畝之農夫;架梁之椽,多於機上之工女;釘頭磷磷,多於在庾之粟粒;瓦縫參差,多於周身之帛縷;直欄橫檻,多於九土之城郭;管絃嘔啞,多於市人之言語。使天下之人,不敢言而敢怒,獨夫之心,日益驕固。戌卒叫,\ProperName{函谷}舉,\ProperName{楚}人一炬,可憐焦土。

嗚呼!滅六國者,六國也,非\ProperName{秦}也。族\ProperName{秦}者,\ProperName{秦}也,非天下也。嗟夫!使六國各愛其人,則足以拒\ProperName{秦}。\ProperName{秦}復愛六國之人,則遞三世可至萬世而爲君,誰得而族滅也?\ProperName{秦}人不暇自哀而後人哀之。後人哀之而不鑑之,亦使後人而復哀後人也。

\section[原道\quad{\small 韓愈}]{{\normalsize 韓愈}\quad 原道}
博愛之謂仁,行而宜之之謂義,由是而之焉之謂道,足乎己,無待於外之謂德。仁與義,爲定名;道與德,爲虛位:故道有君子小人,而德有凶有吉。\ProperName{老子}之小仁義,非毀之也,其見者小也。坐井而觀天,曰天小者,非天小也;彼以煦煦爲仁,孑孑爲義,其小之也則宜。其所謂道,道其所道,非吾所謂道也;其所謂德,德其所德,非吾所謂德也。凡吾所謂道德云者,合仁與義言之也,天下之公言也;\ProperName{老子}之所謂道德云者,去仁與義言之也,一人之私言也。\ProperName{周}道衰,\ProperName{孔子}沒,火于\ProperName{秦},\ProperName{黃}\ProperName{老}于\ProperName{漢},佛于\ProperName{晉}、\ProperName{魏}、\ProperName{梁}、\ProperName{隋}之間,其言道德仁義者,不入于\ProperName{楊},則入于\ProperName{墨},不入于\ProperName{老},則入于佛。入于彼,必出于此。入者主之,出者奴之;入者附之,出者汙之。噫!後之人其欲聞仁義道德之說,孰從而聽之?\ProperName{老}者曰:\ProperName{孔子},吾師之弟子也。佛者曰:\ProperName{孔子},吾師之弟子也。爲\ProperName{孔子}者習聞其說,樂其誕而自小也,亦曰:吾師亦嘗師之云爾。不惟舉之於其口,而又筆之於其書。噫!後之人雖欲聞仁義道德之說,其孰從而求之?甚矣,人之好怪也!不求其端,不訊其末,惟怪之欲聞。

古之爲民者四,今之爲民者六;古之教者處其一,今之教者處其三。農之家一,而食粟之家六;工之家一,而用器之家六;賈之家一,而資焉之家六;奈之何民不窮且盜也!古之時人之害多矣。有聖人者立,然後教之以相生養之道。爲之君,爲之師,驅其蟲蛇禽獸而處之中土。寒,然後爲之衣,飢,然後爲之食;木處而顛,土處而病也,然後爲之宮室。爲之工,以贍其器用;爲之賈,以通其有無;爲之醫藥,以濟其夭死,爲之葬埋祭祀,以長其恩愛;爲之禮,以次其先後,爲之樂,以宣其湮鬱;爲之政,以率其怠勌,爲之刑;以鋤其強梗。相欺也,爲之符璽、斗斛、權衡以信之;相奪也,爲之城郭、甲兵以守之。害至而爲之備,患生而爲之防。今其言曰:「聖人不死,大盜不止;剖斗折衡,而民不爭。」嗚呼!其亦不思而已矣!如古之無聖人,人之類滅久矣。何也?無羽毛鱗介以居寒熱也,無爪牙以爭食也。

是故:君者,出令者也;臣者,行君之令而致之民者也;民者,出粟米麻絲,作器皿、通貨財,以事其上者也。君不出令,則失其所以爲君;臣不行君之令而致之民,則失其所以爲臣;民不出粟米麻絲,作器皿、通貨財,以事其上,則誅。今其法曰:必棄而君臣,去而父子,禁而相生養之道,以求其所謂清淨寂滅者;嗚呼!其亦幸而出於三代之後,不見黜於\ProperName{禹}、\ProperName{湯}、\ProperName{文}、\ProperName{武}、\ProperName{周公}、\ProperName{孔子}也;其亦不幸而不出於三代之前,不見正於\ProperName{禹}、\ProperName{湯}、\ProperName{文}、\ProperName{武}、\ProperName{周公}、\ProperName{孔子}也。

帝之與王,其號雖殊,其所以爲聖一也。夏葛而冬裘,渴飲而饑食,其事殊,其所以爲智一也。今其言曰:曷不爲太古之無事?是亦責冬之裘者曰:曷不爲葛之之易也?責饑之食者曰:曷不爲飲之之易也?

傳曰:「古之欲明明德於天下者,先治其國;欲治其國者,先齊其家;欲齊其家者,先修其身;欲修其身者,先正其心;欲正其心者,先誠其意。」然則古之所謂正心而誠意者,將以有爲也。今也欲治其心,而外天下國家,滅其天常;子焉而不父其父,臣焉而不君其君,民焉而不事其事。\ProperName{孔子}之作\BookTitle{春秋}也,諸侯用夷禮,則夷之,進於中國,則中國之。經曰:「夷狄之有君,不如諸夏之亡也。」\BookTitle{詩}曰:「戎狄是膺,\ProperName{荊}\ProperName{舒}是懲。」今也,舉夷狄之法,而加之先王之教之上,幾何其不胥而爲夷也!

夫所謂先王之教者,何也?博愛之謂仁;行而宜之之謂義;由是而之焉之謂道,足乎己,無待於外之謂德。其文:\BookTitle{詩}、\BookTitle{書}、\BookTitle{易}、\BookTitle{春秋};其法:禮、樂、刑、政;其民:士、農、工、賈;其位:君臣、父子、師友、賓主、昆弟、夫婦;其服:麻、絲;其居宮室;其食:粟米、果蔬、魚肉:其爲道易明,而其爲教易行也。是故以之爲己,則順而祥;以之爲人,則愛而公;以之爲心,則和而平;以之爲天下國家,無所處而不當。是故生則得其情,死則盡其常,郊焉而天神假,廟焉而人鬼饗。曰:斯道也,何道也?曰:斯吾所謂道也,非向所謂\ProperName{老}與佛之道也。\ProperName{堯}以是傳之\ProperName{舜},\ProperName{舜}以是傳之\ProperName{禹},\ProperName{禹}以是傳之\ProperName{湯},\ProperName{湯}以是傳之\ProperName{文}、\ProperName{武}、\ProperName{周公},\ProperName{文}、\ProperName{武}、\ProperName{周公}傳之\ProperName{孔子},\ProperName{孔子}傳之\ProperName{孟軻}。\ProperName{軻}之死,不得其傳焉。\ProperName{荀}與\ProperName{楊}也,擇焉而不精,語焉而不詳。由\ProperName{周公}而上,上而爲君,故其事行;由\ProperName{周公}而下,下而爲臣,故其說長。

然則如之何而可也?曰:不塞不流,不止不行。人其人,火其書,廬其居,明先王之道以道之,鰥寡孤獨廢疾者有養也,其亦庶乎其可也。

\section[原毀\quad{\small 韓愈}]{{\normalsize 韓愈}\quad 原毀}
古之君子,其責己也重以周;其待人也輕以約。重以周,故不怠;輕以約,故人樂爲善。聞古之人有\ProperName{舜}者,其爲人也,仁義人也。求其所以爲\ProperName{舜}者,責於己曰:「彼人也,予人也;彼能是,而我乃不能是?」早夜以思,去其不如\ProperName{舜}者,就其如\ProperName{舜}者。聞古之人有\ProperName{周公}者,其爲人也,多才與藝人也。求其所以爲\ProperName{周公}者,責於己曰:「彼人也,予人也;彼能是,而我乃不能是?」早夜以思,去其不如\ProperName{周公}者,就其如\ProperName{周公}者。\ProperName{舜},大聖人也,後世無及焉;\ProperName{周公},大聖人也,後世無及焉。是人也,乃曰:「不如\ProperName{舜},不如\ProperName{周公},吾之病也。」是不亦責於己者重以周乎!其於人也,曰:「彼人也,能有是,是足爲良人矣;能善是,是足爲藝人矣!」取其一,不責其二;即其新,不究其舊;恐恐然惟懼其人之不得爲善之利。一善易修也,一藝易能也,其於人也,乃曰:「能有是,是亦足矣。」曰:「能善是,是亦足矣。」不亦待於人者輕以約乎?

今之君子則不然。其責人也詳,其待己也廉。詳,故人難於爲善,廉,故自取也少。己未有善,曰:「我善是,是亦足矣。」己未有能,曰:「我能是,是亦足矣。」外以欺於人,內以欺於心,未少有得而止矣,不亦待其身者已廉乎?其於人也,曰:「彼雖能是,其人不足稱也;彼雖善是,其用不足稱也。」舉其一,不計其十;究其舊,不圖其新。恐恐然惟懼其人之有聞也。是不亦責于人者已詳乎!夫是之謂不以衆人待其身,而以聖人望於人,吾未見其尊己也。

雖然,爲是者有本有原。怠與忌之謂也。怠者不能修,而忌者畏人修。吾嘗試之矣,嘗試語於衆曰:「某良士,某良士。」其應者,必其人之與也;不然,則其所疏遠不與同其利者也;不然,則其畏也。不若是,強者必怒於言,懦者必怒於色矣。又嘗語於眾曰:「某非良士,某非良士。」其不應者,必其人之與也;不然,則其所疏遠不與同其利者也;不然,則其畏也。不若是,強者必說於言,懦者必說於色矣。是故事修而謗興,德高而毀來。嗚呼!士之處此世,而望名譽之光、道德之行,難已!

將有作於上者,得吾說而存之,其國家可幾而理歟!

\section[獲麟解\quad{\small 韓愈}]{{\normalsize 韓愈}\quad 獲麟解}
麟之爲靈昭昭也。詠於\BookTitle{詩},書於\BookTitle{春秋},雜出於傳記百家之書,雖婦人小子皆知其爲祥也。

然麟之爲物,不畜於家,不恆有於天下。其爲形也不類,非若馬牛犬豕豺狼麋鹿然。然則雖有麟,不可知其爲麟也。

角者吾知其爲牛,鬛者吾知其爲馬,犬豕豺狼麋鹿,吾知其爲犬豕豺狼麋鹿,惟麟也不可知。不可知,則其謂之不祥也亦宜。雖然,麟之出,必有聖人在乎位。麟爲聖人出也。聖人者必知麟,麟之果不爲不祥也?

又曰:麟之所以爲麟者,以德不以形。若麟之出不待聖人,則謂之不祥也亦宜。

\section[雜說一\quad{\small 韓愈}]{{\normalsize 韓愈}\quad 雜說一}
龍噓氣成雲,雲固弗靈於龍也。然龍乘是氣,茫洋窮乎玄間,薄日月,伏光景,感震電,神變化,水下土,汩陵谷。雲亦靈怪矣哉!

雲,龍之所能使爲靈也,若龍之靈,則非雲之所能使爲靈也。然龍弗得雲,無以神其靈矣:失其所憑依,信不可歟?異哉!其所憑依,乃其所自爲也。

\BookTitle{易}曰:「雲從龍。」既曰龍,雲從之矣。

\section[雜說四\quad{\small 韓愈}]{{\normalsize 韓愈}\quad 雜說四}
世有\ProperName{伯樂}然後有千里馬。千里馬常有,而\ProperName{伯樂}不常有。故雖有名馬,祗辱於奴隸人之手,駢死於槽櫪之間,不以千里稱也。

馬之千里者,一食或盡粟一石。食馬者不知其能千里而食也。是馬也,雖有千里之能,食不飽,力不足,才美不外見,且欲與常馬等,不可得,安求其能千里也?

策之不以其道,食之不能盡其材,鳴之而不能通其意,執策而臨之曰:「天下無馬!」嗚呼!其真無馬邪?其真不知馬也。

% Proofed 18 July 2022
% Ref. 
%  - 文選, 上海古籍, 1982
%  - 點校本舊唐書, 中華書局, 1975
%  - 龔斌, 陶淵明集校箋, 上海古籍, 1996
%  - 全唐文, 中華書局影本
%  - 瞿蜕園, 李白集校注, 上海古籍, 1980
%  - 陳允吉, 樊川文集, 上海古籍, 1978
%  - 馬其昶, 馬茂元, 韓昌黎文集校注, 上海古籍, 1986
%  - 古文觀止, 中華書局, 1959