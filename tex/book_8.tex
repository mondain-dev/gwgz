
\section[師說\quad{\small 韓愈}]{{\normalsize 韓愈}\quad 師說}
古之學者必有師。師者,所以傳道授業解惑也。人非生而知之者,孰能無惑?惑而不從師,其爲惑也終不解矣。生乎吾前,其聞道也固先乎吾,吾從而師之;生乎吾後,其聞道也亦先乎吾,吾從而師之。吾師道也,夫庸知其年之先後生於吾乎?是故無貴無賤、無長無少,道之所存,師之所存也。嗟乎,師道之不傳也久矣,欲人之無惑也難矣!古之聖人,其出人也遠矣,猶且從師而問焉;今之衆人,其下聖人也亦遠矣,而恥學於師。是故聖益聖,愚益愚,聖人之所以爲聖,愚人之所以爲愚,其皆出於此乎?

愛其子,擇師而教之;於其身也,則恥師焉;惑矣!彼童子之師,授之書而習其句讀者也。非吾所謂傳其道解其惑者也。句讀之不知,惑之不解,或師焉,或不焉,小學而大遺,吾未見其明也。

巫醫樂師百工之人,不恥相師。士大夫之族,曰師、曰弟子云者,則羣聚而笑之。問之,則曰:彼與彼年相若也,道相似也。位卑則足羞,官盛則近諛。嗚呼!師道之不復可知矣!巫醫樂師百工之人,君子不齒,今其智乃反不能及,其可怪也歟!

聖人無常師,\ProperName{孔子}師\ProperName{郯子}、\ProperName{萇弘}、\ProperName{師襄}、\ProperName{老聃}。\ProperName{郯子}之徒,其賢不及\ProperName{孔子},\ProperName{孔子}曰:「三人行,則必有我師。」是故弟子不必不如師,師不必賢於弟子。聞道有先後,術業有專攻,如是而已。

\ProperName{李氏}子\ProperName{蟠},年十七,好古文,六藝經傳皆通習之,不拘於時,學於余。余嘉其能行古道,作\BookTitle{師說}以貽之。

\section[進學解\quad{\small 韓愈}]{{\normalsize 韓愈}\quad 進學解}
國子先生晨入太學,招諸生立館下,誨之曰:「業精於勤荒於嬉,行成於思毀於隨。方今聖賢相逢,治具畢張,拔去兇邪,登崇俊良。占小善者率以錄,名一藝者無不庸,爬羅剔抉,刮垢磨光。蓋有幸而獲選,孰云多而不揚。諸生業患不能精,無患有司之不明,行患不能成,無患有司之不公。」

言未既,有笑於列者曰:「先生欺余哉!弟子事先生于茲有年矣。先生口不絕吟於六藝之文,手不停披於百家之編;記事者必提其要,纂言者必鉤其玄;貪多務得,細大不捐,焚膏油以繼晷,恆兀兀以窮年:先生之業,可謂勤矣。觝排異端,攘斥佛\ProperName{老};補苴罅漏,張皇幽眇;尋墜緒之茫茫,獨旁搜而遠紹,障百川而東之,迴狂瀾於既倒。先生之於儒,可謂有勞矣。沈浸醲郁,含英咀華,作爲文章,其書滿家。上規\ProperName{姚}\ProperName{姒},渾渾無涯,\BookTitle{周誥}\BookTitle{殷盤},佶屈聱牙。\BookTitle{春秋}謹嚴,\BookTitle{左氏}浮誇。\BookTitle{易}奇而法,\BookTitle{詩}正而葩;下逮\BookTitle{莊}\BookTitle{騷},\ProperName{太史}所錄,\ProperName{子雲}\ProperName{相如},同工異曲;先生之於文,可謂閎其中而肆其外矣。少始知學,勇於敢爲;長通於方,左右具宜。先生之於爲人,可謂成矣。」

「然而公不見信於人,私不見助於友,跋前疐後,動輒得咎。暫爲御史,遂竄南夷;三年博士,冗不見治。命與仇謀,取敗幾時。冬暖而兒號寒,年豐而妻啼飢。頭童齒豁,竟死何裨。不知慮此,而反教人爲?」

先生曰:「吁!子來前!夫大木爲杗,細木爲桷,欂櫨侏儒,椳闑扂楔,各得其宜,施以成室者,匠氏之工也;玉札丹砂,赤箭青芝,牛溲馬勃,敗鼓之皮,俱收並蓄,待用無遺者,醫師之良也;登明選公,雜進巧拙,紆餘爲妍,卓犖爲傑,校短量長,惟器是適者,宰相之方也。昔者\ProperName{孟軻}好辯,\ProperName{孔}道以明,轍環天下,卒老於行;\ProperName{荀卿}守正,大論是弘,逃讒於\ProperName{楚},廢死\ProperName{蘭陵}。是二儒者,吐辭爲經,舉足爲法,絕類離倫,優入聖域。其遇於世何如也?」

「今先生學雖勤而不{繇}其統,言雖多而不要其中,文雖奇而不濟於用,行雖修而不顯於衆,猶且月費俸錢,歲靡廩粟,子不知耕,婦不知織,乘馬從徒,安坐而食,踵常途之促促,窺陳編以盜竊;然而聖主不加誅,宰臣不見斥:非其幸歟?動而得謗,名亦隨之,投閒置散,乃分之宜。」% 觀止作不由其統

「若夫商財賄之有亡,計班資之崇庳,忘己量之所稱,指前人之瑕疵:是所謂詰匠氏之不以杙爲楹,而訾醫師以昌陽引年,欲進其豨苓也。」

\section[圬者王承福傳\quad{\small 韓愈}]{{\normalsize 韓愈}\quad 圬者\ProperName{王承福}傳}
圬之爲技,賤且勞者也。有業之其色若自得者。聽其言,約而盡;問之:\ProperName{王}其姓,\ProperName{承福}其名,世爲\ProperName{京兆}\ProperName{長安}農夫。\ProperName{天寶}之亂,發人爲兵,持弓矢十三年,有官勳,棄之來歸,喪其土田,手鏝衣食,餘三十年。舍於市之主人,而歸其屋食之當焉。視時屋食之貴賤,而上下其圬之傭以償之,有餘,則以與道路之廢疾餓者焉。

又曰:粟,稼而生者也;若布與帛,必蠶織而後成者也;其他所以養生之具,皆待人力而後完也:吾皆賴之。然人不可徧爲,宜乎各致其能以相生也。故君者,理我所以生者也;而百官者,承君之化者也。任有小大,惟其所能,若器皿焉。食焉而怠其事,必有天殃,故吾不敢一日舍鏝以嬉。夫鏝,易能可力焉,又誠有功,取其直,雖勞無愧,吾心安焉。夫力,易強而有功也;心,難強而有智也:用力者使於人,用心者使人,亦其宜也;吾特擇其易爲而無愧者取焉。嘻!吾操鏝以入富貴之家有年矣,有一至者焉,又往過之,則爲墟矣;有再至三至者焉,而往過之,則爲墟矣。問之其鄰,或曰:噫!刑戮也。或曰:身既死,而其子孫不能有也。或曰:死而歸之官也。吾以是觀之,非所謂食焉怠其事,而得天殃者邪!非強心以智而不足,不擇其才之稱否而冒之者邪!非多行可愧,知其不可而強爲之者邪!將貴富難守,薄功而厚饗之者邪!抑豐悴有時,一去一來而不可常者邪!吾之心憫焉,是故擇其力之可能者行焉。樂富貴而悲貧賤,我豈異於人哉!

又曰:功大者,其所以自奉也博,妻與子,皆養於我者也,吾能薄而功小,不有之可也。又吾所謂勞力者,若立吾家而力不足,則心又勞也,一身而二任焉,雖聖者不可{能}\endnote{\BookTitle{觀止}作「爲」,據校本改。}也。

\ProperName{愈}始聞而惑之,又從而思之,蓋賢者也!蓋所謂獨善其身者也。然吾有譏焉,謂其自爲也過多,其爲人也過少,其學\ProperName{楊朱}之道者耶?\ProperName{楊}之道,不肯拔我一毛而利天下,而夫人以有家爲勞心,不肯一動其心以畜其妻子,其肯勞其心以爲人乎哉!雖然,其賢於世之患不得之而患失之者,以濟其生之欲、貪邪而亡道以喪其身者,其亦遠矣!又其言有可以警余者,故余爲之傳而自鑒焉。

\theendnotes

\section[諱辯\quad{\small 韓愈}]{{\normalsize 韓愈}\quad 諱辯}
\ProperName{愈}與\ProperName{李賀}書,勸\ProperName{賀}舉進士。\ProperName{賀}舉進士有名,與\ProperName{賀}爭名者毀之,曰:「\ProperName{賀}父名\ProperName{晉肅},\ProperName{賀}不舉進士爲是,勸之舉者爲非。」聽者不察也,和而唱之,同然一辭。\ProperName{皇甫湜}曰:「若不明白,子與\ProperName{賀}且得罪。」

\ProperName{愈}曰:「然。」\BookTitle{律}曰:「二名不偏諱。」釋之者曰:謂若言「徵」不稱「在」,言「在」不稱「徵」是也。\BookTitle{律}曰:「不諱嫌名。」釋之者曰:謂若「禹」與「雨」、「丘」與「蓲」之類是也。今\ProperName{賀}父名\ProperName{晉肅},\ProperName{賀}舉進士,爲犯「二名律」乎?爲犯「嫌名律」乎?父名\ProperName{晉肅},子不得舉進士;若父名「仁」,子不得爲人乎?

夫諱始於何時?作法制以教天下者,非\ProperName{周公}、\ProperName{孔子}歟?\ProperName{周公}作詩不諱;\ProperName{孔子}不偏諱二名;\BookTitle{春秋}不譏不諱嫌名;\ProperName{康王}\ProperName{釗}之孫實爲\ProperName{昭王};\ProperName{曾參}之父名\ProperName{皙},\ProperName{曾子}不諱「昔」。\ProperName{周}之時有\ProperName{騏期},\ProperName{漢}之時有\ProperName{杜度},此其子宜如何諱?將諱其嫌,遂諱其姓乎?將不諱其嫌者乎?\ProperName{漢}諱武帝名\ProperName{徹}爲「通」,不聞又諱「車轍」之「轍」爲某字也;諱\ProperName{呂后}名\ProperName{雉}爲「野雞」,不聞又諱「治天下」之「治」爲某字也。今上章及詔,不聞諱「滸」、「勢」、「秉」、「機」也,惟宦官宮妾乃不敢言「諭」及「機」,以爲觸犯。士君子立言行事,宜何所法守也?今考之於經,質之於律,稽之以國家之典,\ProperName{賀}舉進士爲可邪?爲不可邪?

凡事父母得如\ProperName{曾參},可以無譏矣;作人得如\ProperName{周公}、\ProperName{孔子},亦可以止矣。今世之士不務行\ProperName{曾參}、\ProperName{周公}、\ProperName{孔子}之行,而諱親之名則務勝於\ProperName{曾參}、\ProperName{周公}、\ProperName{孔子},亦見其惑也!夫\ProperName{周公}、\ProperName{孔子}、\ProperName{曾參},卒不可勝,勝\ProperName{周公}、\ProperName{孔子}、\ProperName{曾參},乃比於宦官宮妾:則是宦官宮妾之孝於其親,賢於\ProperName{周公}、\ProperName{孔子}、\ProperName{曾參}者邪?

\section[爭臣論\quad{\small 韓愈}]{{\normalsize 韓愈}\quad 爭臣論}
或問諫議大夫\ProperName{陽城}於\ProperName{愈}:「可以爲有道之士乎哉?學廣而聞多,不求聞於人也。行古人之道,居於\ProperName{晉}之鄙,\ProperName{晉}之鄙人薰其德而善良者幾千人。大臣聞而薦之,天子以爲諫議大夫,人皆以爲華,\ProperName{陽子}不色喜。居於位五年矣,視其德如在野:彼豈以富貴移易其心哉?」\ProperName{愈}應之曰:「是\BookTitle{易}所謂『恆其德,貞,而夫子凶』者也,惡得爲有道之士乎哉?在\BookTitle{易}\BookTitle{蠱}之上九云:『不事王侯,高尚其事』;\BookTitle{蹇}之六二則曰:『王臣蹇蹇,匪躬之故』:夫亦以所居之時不一,而所蹈之德不同也?若\BookTitle{蠱}之上九,居無用之地,而致匪躬之節;以\BookTitle{蹇}之六二,在王臣之位,而高不事之心。則冒進之患生,曠官之刺興,志不可則,而尤不終無也。今\ProperName{陽子}在位不爲不久矣,聞天下之得失不爲不熟矣,天子待之不爲不加矣,而未嘗一言及於政。視政之得失若\ProperName{越}人視\ProperName{秦}人之肥瘠,忽焉不加喜戚於其心。問其官,則曰諫議也;問其祿,則曰下大夫之秩也;問其政,則曰我不知也:有道之士,固如是乎哉?且吾聞之,有官守者,不得其職則去;有言責者,不得其言則去。今\ProperName{陽子}以爲得其言乎哉?得其言而不言,與不得其言而不去,無一可者也。\ProperName{陽子}將爲祿仕乎?古之人有云:仕不爲貧,而有時乎爲貧,謂祿仕者也。宜乎辭尊而居卑,辭富而居貧,若抱關擊柝者可也。蓋\ProperName{孔子}嘗爲委吏矣,嘗爲乘田矣。亦不敢曠其職,必曰『會計當而已矣』,必曰『牛羊遂而已矣』。若\ProperName{陽子}之秩祿不爲卑且貧,章章明矣,而如此,其可乎哉?」

或曰:「否,非若此也。夫\ProperName{陽子}惡訕上者,惡爲人臣招其君之過而以爲名者。故雖諫且議,使人不得而知焉。\BookTitle{書}曰:『爾有嘉謀嘉猷,則入告爾后于內,爾乃順之于外』,曰:『斯謀斯猷,惟我后之德』。夫\ProperName{陽子}之用心,亦若此者!」\ProperName{愈}應之曰:「若\ProperName{陽子}之用心如此,滋所謂惑者矣!入則諫其君,出不使人知者,大臣宰相者之事,非\ProperName{陽子}之所宜行也。夫\ProperName{陽子}本以布衣隱於蓬蒿之下,主上嘉其行誼,擢在此位,官以諫爲名,誠宜有以奉其職,使四方後代知朝廷有直言骨鯁之臣,天子有不僭賞從諫如流之美。庶巖穴之士聞而慕之,束帶結髮,願進於闕下,而伸其辭說,致吾君於\ProperName{堯}\ProperName{舜},熙鴻號於無窮也。若\BookTitle{書}所謂,則大臣宰相之事,非\ProperName{陽子}之所宜行也。且\ProperName{陽子}之心,將使君人者惡聞其過乎?是啓之也!」

或曰:「\ProperName{陽子}之不求聞而人聞之,不求用而君用之,不得已而起,守其道而不變,何子過之深也?」\ProperName{愈}曰:「自古聖人賢士,皆非有求於聞用也,閔其時之不平,人之不乂,得其道,不敢獨善其身,而必以兼濟天下也,孜孜矻矻,死而後已。故\ProperName{禹}過家門不入,\ProperName{孔}席不暇暖,而\ProperName{墨}突不得黔:彼二聖一賢者,豈不知自安佚之爲樂哉?誠畏天命而悲人窮也。夫天授人以賢聖才能,豈使自有餘而已?誠欲以補其不足者也。耳目之於身也,耳司聞而目司見,聽其是非,視其險易,然後身得安焉。聖賢者,時人之耳目也;時人者,聖賢之身也。且\ProperName{陽子}之不賢,則將役於賢,以奉其上矣。若果賢,則固畏天命而閔人窮也:惡得以自暇逸乎哉?」

或曰:「吾聞君子不欲加諸人,而惡訐以爲直者。若吾子之論,直則直矣,無乃傷於德而費於辭乎?好盡言以招人過,\ProperName{國武子}之所以見殺於\ProperName{齊}也。吾子其亦聞乎!」\ProperName{愈}曰:「君子居其位,則思死其官;未得位,則思修其辭以明其道:我將以明道也,非以爲直而加人也。且\ProperName{國武子}不能得善人而好盡言於亂國,是以見殺。\BookTitle{傳}曰:『惟善人,能受盡言。』謂其聞而能改之也。子告我曰:\ProperName{陽子}可以爲有道之士也。今雖不能及己,\ProperName{陽子}將不得爲善人乎哉?」

\section[後十九日復上宰相書\quad{\small 韓愈}]{{\normalsize 韓愈}\quad 後十九日復上宰相書}
{\parindent=0pt 二月十六日,前鄉貢進士\ProperName{韓愈}謹再拜言相公閣下:}

向上書及所著文後,待命凡十有九日,不得命,恐懼不敢逃遁,不知所爲;乃復敢自納於不測之誅,以求畢其說而請命於左右。

\ProperName{愈}聞之:蹈水火者之求免於人也,不惟其父兄子弟之慈愛然後呼而望之也;將有介於其側者,雖其所憎怨,苟不至乎欲其死者,則將大其聲疾呼而望其仁之也。彼介於其側者,聞其聲而見其事,不惟其父兄子弟之慈愛然後往而全之也,雖有所憎怨,苟不至乎欲其死者,則將狂奔盡氣、濡手足、焦毛髮救之而不辭也。若是者何哉?其勢誠急,而其情誠可悲也。\ProperName{愈}之彊學力行有年矣,愚不惟道之險夷,行且不息,以蹈於窮餓之水火,其既危且亟矣,大其聲而疾呼矣。閣下其亦聞而見之矣,其將往而全之歟?抑將安而不救歟?有來言於閣下者曰:有觀溺於水而爇於火者,有可救之道而終莫之救也,閣下且以爲仁人乎哉?不然,若\ProperName{愈}者,亦君子之所宜動心者也。

或謂\ProperName{愈}:子言則然矣,宰相則知子矣,如時不可何?\ProperName{愈}竊謂之不知言者,誠其材能不足當吾賢相之舉耳;若所謂時者,固在上位者之爲耳,非天之所爲也。前五六年時,宰相薦聞尚有自布衣蒙抽擢者,與今豈異時哉?且今節度觀察使及防禦營田諸小使等,尚得自舉判官,無間於已仕未仕者,況在宰相,吾君所尊敬者,而曰不可乎?

古之進人者,或取於盜,或舉於管庫;今布衣雖賤,猶足以方於此。情隘辭蹙,不知所裁,亦惟少垂憐焉。\ProperName{愈}再拜。

\section[後廿九日復上宰相書\quad{\small 韓愈}]{{\normalsize 韓愈}\quad 後廿九日復上宰相書}
{\parindent=0pt 三月十六日,前鄉貢進士\ProperName{韓愈}謹再拜言相公閣下:}

\ProperName{愈}聞\ProperName{周公}之爲輔相,其急於見賢也,方一食三吐其哺,方一沐三握其髮。當是時,天下之賢才皆已舉用,姦邪讒佞欺負之徒皆已除去;四海皆已無虞;九夷八蠻之在荒服之外者,皆已賓貢;天災時變、昆蟲草木之妖,皆已銷息;天下之所謂禮樂刑政教化之具,皆已修理;風俗皆已敦厚;動植之物、風雨霜露之所霑被者,皆已得宜;休徵嘉瑞、麟鳳龜龍之屬,皆已備至:而\ProperName{周公}以聖人之才,憑叔父之親,其所輔理承化之功又盡章章如是,其所求進見之士豈復有賢於\ProperName{周公}者哉?不惟不賢於\ProperName{周公}而已,豈復有賢於時百執事者哉?豈復有所計議,能補於\ProperName{周公}之化者哉?然而\ProperName{周公}求之如此其急,惟恐耳目有所不聞見,思慮有所未及,以負\ProperName{成王}託\ProperName{周公}之意,不得於天下之心。如\ProperName{周公}之心,設使其時輔理承化之功未盡章章如是,而非聖人之才,而無叔父之親,則將不暇食與沐矣;豈特吐哺握髮爲勤而止哉!維其如是,故于今頌\ProperName{成王}之德而稱\ProperName{周公}之功不衰。

今閣下爲輔相亦近耳,天下之賢才豈盡舉用?姦邪讒佞欺負之徒豈盡除去?四海豈盡無虞?九夷八蠻之在荒服之外者,豈盡賓貢?天災時變,昆蟲草木之妖,豈盡銷息?天下之所謂禮樂刑政教化之具,豈盡修理?風俗豈盡敦厚?動植之物、風雨霜露之所霑被者,豈盡得宜?休徵嘉瑞、麟鳳龜龍之屬,豈盡備至?其所求進見之士,雖不足以希望盛德,至比於百執事,豈盡出其下哉?其所稱說,豈盡無所補哉?今雖不能如\ProperName{周公}吐哺握髮,亦宜引而進之,察其所以而去就之,不宜默默而已也。\ProperName{愈}之待命四十餘日矣,書再上,而志不得通;足三及門,而閽人辭焉:惟其昏愚不知逃遁,故復有\ProperName{周公}之說焉。閣下其亦察之。

古之士三月不仕則相弔,故出疆必載質。然所以重於自進者,以其於\ProperName{周}不可,則去之\ProperName{魯};於\ProperName{魯}不可,則去之\ProperName{齊};於\ProperName{齊}不可,則去之\ProperName{宋}之\ProperName{鄭}之\ProperName{秦}之\ProperName{楚}也。今天下一君,四海一國,舍乎此則夷狄矣,去父母之邦矣;故士之行道者,不得於朝,則山林而已矣。山林者,士之所獨善自養而不憂天下者之所能安也。如有憂天下之心,則不能矣。故\ProperName{愈}每自進而不知愧焉;書亟上,足數及門,而不知止焉。寧獨如此而已?惴惴焉惟不得出大賢之門下是懼,亦惟少垂察焉。瀆冒威尊,惶恐無已。\ProperName{愈}再拜。

\section[與于襄陽書\quad{\small 韓愈}]{{\normalsize 韓愈}\quad 與\ProperName{于襄陽}書}
{\parindent=0pt 七月三日,將仕郎守國子四門博士\ProperName{韓愈}謹奉書尚書閣下:}

士之能享大名顯當世者,莫不有先達之士負天下之望者爲之前焉;士之能垂休光照後世者,亦莫不有後進之士負天下之望者爲之後焉。莫爲之前,雖美而不彰;莫爲之後,雖盛而不傳。是二人者,未始不相須也,然而千百載乃一相遇焉;豈上之人無可援,下之人無可推歟?何其相須之殷而相遇之疎也?其故在下之人負其能不肯諂其上,上之人負其位不肯顧其下;故高材多戚戚之窮,盛位無赫赫之光:是二人者之所爲皆過也。未嘗干之,不可謂上無其人;未嘗求之,不可謂下無其人:\ProperName{愈}之誦此言久矣,未嘗敢以聞於人。側聞閣下抱不世之才,特立而獨行,道方而事實,卷舒不隨乎時,文武唯其所用,豈\ProperName{愈}所謂其人哉?抑未聞後進之士有遇知於左右,獲禮於門下者,豈求之而未得邪?將志存乎立功,而事專乎報主,雖遇其人,未暇禮邪?何其宜聞而久不聞也!\ProperName{愈}雖不才,其自處不敢後於恆人,閣下將求之而未得歟?古人有言:「請自\ProperName{隗}始。」

\ProperName{愈}今者惟朝夕芻米僕賃之資是急,不過費閣下一朝之享而足也。如曰:吾志存乎立功,而事專乎報主,雖遇其人,未暇禮焉;則非\ProperName{愈}之所敢知也。世之齪齪者既不足以語之,磊落奇偉之人又不能聽焉,則信乎命之窮也。

謹獻舊所爲文一十八首,如賜覽觀,亦足知其志之所存。\ProperName{愈}恐懼再拜。

\section[與陳給事書\quad{\small 韓愈}]{{\normalsize 韓愈}\quad 與\ProperName{陳給事}書}
\ProperName{愈}再拜:\ProperName{愈}之獲見於閣下有年矣。始者亦嘗辱一言之譽。貧賤也,衣食於奔走,不得朝夕繼見,其後閣下位益尊,伺候於門牆者日益進。夫位益尊,則賤者日隔。伺候於門牆者日益進,則愛博而情不專。\ProperName{愈}也道不加修而文日益有名。夫道不加修,則賢者不與;文日益有名,則同進者忌。始之以日隔之疏,加之以不專之望,以不與者之心,而聽忌者之說:由是閣下之庭無\ProperName{愈}之跡矣。

去年春,亦嘗一進謁於左右矣,溫乎其容若加其新也,屬乎其言若閔其窮也,退而喜也以告於人。其後如\ProperName{東京}取妻子,又不得朝夕繼見。及其還也,亦嘗一進謁於左右矣,邈乎其容若不察其愚也,悄乎其言若不接其情也,退而懼也不敢復進。今則釋然悟,翻然悔曰:其邈也,乃所以怒其來之不繼也;其悄也,乃所以示其意也。不敏之誅無所逃避,不敢遂進,輒自疏其所以,幷獻近所爲\BookTitle{復志賦}以下十首爲一卷,卷有標軸;\BookTitle{送孟郊序}一首生紙寫,不加裝飾,皆有揩字注字處,急於自解而謝,不能竢更寫,閣下取其意而略其禮可也。\ProperName{愈}恐懼再拜。

\section[應科目時與人書\quad{\small 韓愈}]{{\normalsize 韓愈}\quad 應科目時與人書}
月日\ProperName{愈}再拜:天池之濱,\ProperName{大江}之濆,曰有怪物焉;蓋非常鱗凡介之品彙匹儔也!其得水,變化風雨上下於天不難也;其不及水,蓋尋常尺寸之間耳。無高山大陵曠途絕險爲之關隔也;然其窮涸不能自致乎水,爲獱獺之笑者,蓋十八九矣。如有力者哀其窮而運轉之,蓋一舉手一投足之勞也。

然是物也,負其異於衆也,且曰:爛死於沙泥,吾寧樂之;若俛首帖耳搖尾而乞憐者,非我之志也。是以有力者遇之,熟視之若無覩也。其死其生,固不可知也。今又有有力者當其前矣,聊試仰首一鳴號焉,庸詎知有力者不哀其窮,而忘一舉手一投足之勞而轉之清波乎?

其哀之,命也;其不哀之,命也;知其在命而且鳴號之者,亦命也:\ProperName{愈}今者實有類於是。是以忘其疏愚之罪,而有是說焉。閣下其亦憐察之!

\section[送孟東野序\quad{\small 韓愈}]{{\normalsize 韓愈}\quad 送\ProperName{孟東野}序}
大凡物不得其平則鳴:草木之無聲,風撓之鳴;水之無聲,風蕩之鳴。其躍也或激之,其趨也或梗之,其沸也或炙之;金石之無聲,或擊之鳴。人之於言也亦然:有不得已者而後言,其謌也有思,其哭也有懷,凡出乎口而爲聲者,其皆有弗平者乎!樂也者,鬱於中而泄於外者也;擇其善鳴者而假之鳴:金、石、絲、竹、匏、土、革、木八者,物之善鳴者也。維天之於時也亦然,擇其善鳴者而假之鳴。是故以鳥鳴春,以雷鳴夏,以蟲鳴秋,以風鳴冬,四時之相推敓,其必有不得其平者乎!

其於人也亦然:人聲之精者爲言,文辭之於言,又其精也,尤擇其善鳴者而假之鳴。其在\ProperName{唐虞},\ProperName{咎陶}、\ProperName{禹}其善鳴者也,而假以鳴;\ProperName{夔}弗能以文辭鳴,又自假於\BookTitle{韶}以鳴;\ProperName{夏}之時,五子以其歌鳴;\ProperName{伊尹}鳴\ProperName{殷};\ProperName{周公}鳴\ProperName{周}。凡載於\BookTitle{詩}\BookTitle{書}六藝,皆鳴之善者也。\ProperName{周}之衰,\ProperName{孔子}之徒鳴之,其聲大而遠。\BookTitle{傳}曰:「天將以夫子爲木鐸。」其弗信矣乎!其末也,\ProperName{莊周}以其荒唐之辭鳴。\ProperName{楚}大國也,其亡也,以\ProperName{屈原}鳴。\ProperName{臧孫辰}、\ProperName{孟軻}、\ProperName{荀卿}以道鳴者也,\ProperName{楊朱}、\ProperName{墨翟}、\ProperName{管夷吾}、\ProperName{晏嬰}、\ProperName{老聃}、\ProperName{申不害}、\ProperName{韓非}、\ProperName{眘到}、\ProperName{田駢}、\ProperName{鄒衍}、\ProperName{尸佼}、\ProperName{孫武}、\ProperName{張儀}、\ProperName{蘇秦}之屬,皆以其術鳴。\ProperName{秦}之興,\ProperName{李斯}鳴之。\ProperName{漢}之時,\ProperName{司馬遷}、\ProperName{相如}、\ProperName{揚雄},最其善鳴者也。其下\ProperName{魏}、\ProperName{晉}氏,鳴者不及於古,然亦未嘗絕也;就其善者,其聲清以浮,其節數以急,其辭淫以哀,其志弛以肆,其爲言也,亂雜而無章。將天醜其德莫之顧邪?何爲乎不鳴其善鳴者也?

\ProperName{唐}之有天下,\ProperName{陳子昂}、\ProperName{蘇源明}、\ProperName{元結}、\ProperName{李白}、\ProperName{杜甫}、\ProperName{李觀}皆以其所能鳴。其存而在下者,\ProperName{孟郊}\ProperName{東野},始以其詩鳴;其高出\ProperName{魏}\ProperName{晉},不懈而及於古,其他浸淫乎\ProperName{漢}氏矣。從吾遊者,\ProperName{李翱}、\ProperName{張籍}其尤也,三子者之鳴信善矣,抑不知天將和其聲,而使鳴國家之盛邪?抑將窮餓其身,思愁其心腸,而使自鳴其不幸邪?三子者之命,則懸乎天矣。其在上也奚以喜,其在下也奚以悲!

\ProperName{東野}之役於\ProperName{江南}也,有若不釋然者,故吾道其命於天者以解之。

\section[送李愿歸盤谷序\quad{\small 韓愈}]{{\normalsize 韓愈}\quad 送\ProperName{李愿}歸\ProperName{盤谷}序}

\ProperName{太行}之陽有\ProperName{盤谷}。\ProperName{盤谷}之間,泉甘而土肥,草木藂茂,居民鮮少。或曰:謂其環兩山之間,故曰「盤」。或曰:是谷也,宅幽而勢阻,隱者之所盤旋。友人\ProperName{李愿}居之。

\ProperName{愿}之言曰:人之稱大丈夫者,我知之矣:利澤施於人,名聲昭於時,坐於廟朝,進退百官,而佐天子出令。其在外,則樹旗旄,羅弓矢,武夫前呵,從者塞途,供給之人,各執其物,夾道而疾馳。喜有賞,怒有刑。才畯滿前,道古今而譽盛德,入耳而不煩。曲眉豐頰,清聲而便體,秀外而惠中。飄輕裾,翳長袖,粉白黛綠者,列屋而閒居,妒寵而負恃,爭妍而取憐。大丈夫之遇知於天子,用力於當世者之所爲也。吾非惡此而逃之,是有命焉,不可幸而致也。窮居而野處,升高而望遠,坐茂樹以終日,濯清泉以自潔。採於山,美可茹;釣於水,鮮可食。起居無時,惟適之安。與其有譽於前,孰若無毀於其後;與其有樂於身,孰若無憂於其心。車服不維,刀鋸不加,理亂不知,黜陟不聞。大丈夫不遇於時者之所爲也,我則行之。伺候於公卿之門,奔走於形勢之途,足將進而趦趄,口將言而囁嚅,處汙穢而不羞,觸刑辟而誅戮,徼倖於萬一,老死而後止者,其於爲人賢不肖何如也?

\ProperName{昌黎}\ProperName{韓愈}聞其言而壯之,與之酒而爲之歌曰:\ProperName{盤}之中,維子之宮。\ProperName{盤}之土,可以稼。\ProperName{盤}之泉,可濯可沿。\ProperName{盤}之阻,誰爭子所。窈而深,廓其有容。繚而曲,如往而復。嗟\ProperName{盤}之樂兮,樂且無央;虎豹遠跡兮,蛟龍遁藏;鬼神守護兮,呵禁不祥。飲且食兮壽而康,無不足兮奚所望;膏吾車兮秣吾馬,從子於\ProperName{盤}兮,終吾生以徜徉。

\section[送董邵南序\quad{\small 韓愈}]{{\normalsize 韓愈}\quad 送\ProperName{董邵南}序}
\ProperName{燕}、\ProperName{趙}古稱多感慨悲歌之士。\ProperName{董生}舉進士,連不得志於有司,懷抱利器,鬱鬱適茲土,吾知其必有合也。\ProperName{董生}勉乎哉!夫以子之不遇時,苟慕義彊仁者皆愛惜焉。矧\ProperName{燕}、\ProperName{趙}之士出乎其性者哉!

然吾嘗聞風俗與化移易,吾惡知其今不異於古所云邪?聊以吾子之行卜之也。\ProperName{董生}勉乎哉!

吾因子有所感矣,爲我弔\ProperName{望諸君}之墓,而觀於其市復有昔時屠狗者乎?爲我謝曰:明天子在上,可以出而仕矣!

\section[送楊少尹序\quad{\small 韓愈}]{{\normalsize 韓愈}\quad 送\ProperName{楊少尹}序}
昔\ProperName{疏廣}、\ProperName{受}二子以年老一朝辭位而去,於時公卿設供張,祖道都門外,車數百兩,道路觀者多歎息泣下,共言其賢。\ProperName{漢}史既傳其事,而後世工畫者又圖其跡,至今照人耳目,赫赫若前日事。國子司業\ProperName{楊}君\ProperName{巨源}方以能詩訓後進,一旦以年滿七十,亦白丞相去歸其鄉。世常說古今人不相及,今\ProperName{楊}與二\ProperName{疏},其意豈異也?

予忝在公卿後,遇病不能出,不知\ProperName{楊侯}去時,城門外送者幾人?車幾兩?馬幾匹?道邊觀者亦有歎息知其爲賢以否?而太史氏又能張大其事爲傳繼二\ProperName{疏}蹤跡否?不落莫否?見今世無工畫者,而畫與不畫固不論也。然吾聞\ProperName{楊侯}之去,丞相有愛而惜之者,白以爲其都少尹,不絕其祿,又爲歌詩以勸之,京師之長於詩者亦屬而和之;又不知當時二\ProperName{疏}之去有是事否?古今人同不同,未可知也。

中世士大夫以官爲家,罷則無所於歸。\ProperName{楊侯}始冠舉於其鄉,歌\BookTitle{鹿鳴}而來也;今之歸,指其樹曰:「某樹吾先人之所種也,某水某丘吾童子時所釣遊也。」鄉人莫不加敬,誡子孫以\ProperName{楊侯}不去其鄉爲法。古之所謂「鄉先生沒而可祭於社」者,其在斯人歟!其在斯人歟!

\section[送石處士序\quad{\small 韓愈}]{{\normalsize 韓愈}\quad 送\ProperName{石處士}序}
\ProperName{河陽軍}節度御史大夫\ProperName{烏公}爲節度之三月,求士於從事之賢者,有薦\ProperName{石先生}者。公曰:「先生何如?」曰:「先生居\ProperName{嵩}、\ProperName{邙}、\ProperName{瀍}、\ProperName{穀}之間,冬一裘,夏一葛,食朝夕飯一盂、蔬一盤。人與之錢則辭,請與出遊,未嘗以事免,勸之仕,不應。坐一室,左右圖書。與之語道理,辨古今事當否,論人高下,事後當成敗,若河決下流而東注,若駟馬駕輕車就熟路,而\ProperName{王良}、\ProperName{造父}爲之先後也,若燭照、數計而龜卜也。」大夫曰:「先生有以自老,無求於人,其肯爲某來邪?」從事曰:「大夫文武忠孝,求士爲國,不私於家。方今寇聚於\ProperName{恆},師環其疆,農不耕收,財粟殫亡。吾所處地,歸輸之塗,治法征謀,宜有所出。先生仁且勇,若以義請而彊委重焉,其何說之辭!」於是譔書詞,具馬幣,卜日以{授}\endnote{\BookTitle{觀止}作「受」,據校本改。}使者,求先生之廬而請焉。先生不告於妻子,不謀於朋友,冠帶出見客,拜受書禮於門內,宵則沐浴戒行李,載書冊,問道所由,告行於常所來往;晨則畢至,張\ProperName{上東門}外。

酒三行,且起,有執爵而言者曰:「大夫真能以義取人,先生真能以道自任,決去就,爲先生別。」又酌而祝曰:「凡去就出處何常,惟義之歸。遂以爲先生壽。」又酌而祝曰:「使大夫恆無變其初,無務富其家而飢其師,無甘受佞人而外敬正士,無{味}\endnote{\BookTitle{觀止}作「昧」,據校本改。}於諂言,惟先生是聽,以能有成功,保天子之寵命。」又祝曰:「使先生無圖利於大夫而私便其身圖。」先生起拜祝辭曰:「敢不敬蚤夜以求從祝規。」

於是\ProperName{東都}之人士咸知大夫與先生果能相與以有成也,遂各爲歌詩六韻。遣\ProperName{愈}爲之序云。

\theendnotes

\section[送溫處士赴河陽軍序\quad{\small 韓愈}]{{\normalsize 韓愈}\quad 送\ProperName{溫處士}赴\ProperName{河陽軍}序}
\ProperName{伯樂}一過\ProperName{冀北}之野,而馬羣遂空。夫\ProperName{冀北}馬多天下,\ProperName{伯樂}雖善知馬,安能空其羣邪?解之者曰:吾所謂空,非無馬也;無良馬也。\ProperName{伯樂}知馬,遇其良,輒取之,羣無留良焉。苟無良,雖謂無馬,不爲虛語矣。

\ProperName{東都}固士大夫之\ProperName{冀北}也。恃才能,深藏而不市者,\ProperName{洛}之北涯曰\ProperName{石生},其南涯曰\ProperName{溫生}。大夫\ProperName{烏公},以鈇鉞鎮\ProperName{河陽}之三月,以\ProperName{石生}爲才,以禮爲羅,羅而致之幕下。未數月也,以\ProperName{溫生}爲才,於是以\ProperName{石生}爲媒,以禮爲羅,又羅而致之幕下。\ProperName{東都}雖信多才士,朝取一人焉,拔其尤;暮取一人焉,拔其尤:自居守、\ProperName{河南}尹以及百司之執事,與吾輩二縣之大夫,政有所不通,事有所可疑,奚所諮而處焉?士大夫之去位而巷處者,誰與嬉遊?小子後生於何考德而問業焉?縉紳之東西行過是都者,無所禮於其廬。若是而稱曰:大夫\ProperName{烏公}一鎮\ProperName{河陽},而\ProperName{東都}處士之廬無人焉,豈不可也?

夫南面而聽天下,其所託重而恃力者,惟相與將耳。相爲天子得人於朝廷,將爲天子得文武士於幕下:求內外無治,不可得也。

\ProperName{愈}縻於茲不能自引去,資二生以待老;今皆爲有力者奪之,其何能無介然於懷邪?生既至,拜公於軍門,其爲吾以前所稱爲天下賀,以後所稱爲吾致私怨於盡取也。留守相公首爲四韻詩歌其事,\ProperName{愈}因推其意而序之。

\section[祭十二郎文\quad{\small 韓愈}]{{\normalsize 韓愈}\quad 祭\ProperName{十二郎}文}
{\parindent=0pt 年月日,季父\ProperName{愈}聞汝喪之七日,乃能銜哀致誠,使\ProperName{建中}遠具時羞之奠,告汝\ProperName{十二郎}之靈:}

嗚呼!吾少孤,及長不省所怙,惟兄嫂是依。中年兄歿南方,吾與汝俱幼,從嫂歸葬\ProperName{河陽},既又與汝就食\ProperName{江南},零丁孤苦,未嘗一日相離也。吾上有三兄,皆不幸早世。承先人後者,在孫惟汝,在子惟吾;兩世一身,形單影隻。嫂嘗撫汝指吾而言曰:「\ProperName{韓}氏兩世,惟此而已!」汝時尤小,當不復記憶;吾時雖能記憶,亦未知其言之悲也!

吾年十九,始來京城,其後四年,而歸視汝。又四年,吾往\ProperName{河陽}省墳墓,遇汝從嫂喪來葬。又二年,吾佐\ProperName{董丞相}於\ProperName{汴州},汝來省吾,止一歲,請歸取其孥;明年丞相薨,吾去\ProperName{汴州},汝不果來。是年,吾佐戎\ProperName{徐州},使取汝者始行,吾又罷去,汝又不果來。吾念汝從於東,東亦客也,不可以久;圖久遠者,莫如西歸,將成家而致汝。嗚呼!孰謂汝遽去吾而歿乎!吾與汝俱少年,以爲雖暫相別,終當久相與處;故捨汝而旅食京師,以求斗斛之祿;誠知其如此,雖萬乘之公相,吾不以一日輟汝而就也!

去年\ProperName{孟東野}往,吾書與汝曰:「吾年未四十,而視茫茫,而髮蒼蒼,而齒牙動搖。念諸父與諸兄,皆康彊而早世,如吾之衰者,其能久存乎!吾不可去,汝不肯來,恐旦暮死,而汝抱無涯之戚也!」孰謂少者歿而長者存,彊者夭而病者全乎!嗚呼,其信然邪?其夢邪?其傳之非其真邪?信也,吾兄之盛德而夭其嗣乎?汝之純明而不克蒙其澤乎?少者彊者而夭歿,長者衰者而全存乎?未可以爲信也,夢也,傳之非其真也;\ProperName{東野}之書,\ProperName{耿蘭}之報,何爲而在吾側也?嗚呼!其信然矣,吾兄之盛德而夭其嗣矣!汝之純明宜業其家者不克蒙其澤矣。所謂天者誠難測,而神者誠難明矣!所謂理者不可推,而壽者不可知矣!雖然,我自今年來,蒼蒼者或化而爲白矣,動搖者或脫而落矣,毛血日益衰,志氣日益微,幾何不從汝而死也!死而有知,其幾何離;其無知,悲不幾時,而不悲者無窮期矣!汝之子始十歲,吾之子始五歲,少而彊者不可保,如此孩提者又可冀其成立邪?嗚呼哀哉,嗚呼哀哉!

汝去年書云:比得軟腳病,往往而劇。吾曰:是疾也,\ProperName{江南}之人常常有之。未始以爲憂也。嗚呼!其竟以此而殞其生乎?抑別有疾而至斯乎?汝之書六月十七日也,\ProperName{東野}云:汝歿以六月二日,\ProperName{耿蘭}之報無月日:蓋\ProperName{東野}之使者不知問家人以月日,如\ProperName{耿蘭}之報不知當言月日,\ProperName{東野}與吾書,乃問使者,使者妄稱以應之耳。其然乎?其不然乎?

今吾使\ProperName{建中}祭汝,弔汝之孤與汝之乳母。彼有食可守以待終喪,則待終喪而取以來;如不能守以終喪,則遂取以來。其餘奴婢,並令守汝喪。吾力能改葬,終葬汝於先人之兆,然後惟其所願。嗚呼!汝病吾不知時,汝歿吾不知日;生不能相養於共居,歿不能撫汝以盡哀,斂不憑其棺,窆不臨其穴;吾行負神明而使汝夭,不孝不慈,而不得與汝相養以生,相守以死;一在天之涯,一在地之角,生而影不與吾形相依,死而魂不與吾夢相接:吾實爲之,其又何尤?彼蒼者天,曷其有極!

自今已往,吾其無意於人世矣。當求數頃之田於\ProperName{伊}\ProperName{潁}之上,以待餘年,教吾子與汝子幸其成,長吾女與汝女待其嫁:如此而已。嗚呼!言有窮而情不可終,汝其知也邪?其不知也邪?嗚呼哀哉,尚饗。

\section[祭鱷魚文\quad{\small 韓愈}]{{\normalsize 韓愈}\quad 祭鱷魚文}
{\parindent=0pt 維年月日,\ProperName{潮州}刺史\ProperName{韓愈},使軍事衙推\ProperName{秦濟},以羊一豬一投\ProperName{惡谿}之潭水,以與鱷魚食,而告之曰:}

昔先王既有天下,列山澤,罔繩擉刃,以除蟲蛇惡物爲民害者,驅而出之四海之外。及後王德薄,不能遠有,則\ProperName{江}\ProperName{漢}之間,尚皆棄之以與\ProperName{蠻}夷\ProperName{楚}\ProperName{越},況\ProperName{潮}嶺海之間,去京師萬里哉?鱷魚之涵淹卵育於此,亦固其所。今天子嗣\ProperName{唐}位,神聖慈武,四海之外,六合之內,皆撫而有之,況\ProperName{禹}跡所揜,\ProperName{揚州}之近地,刺史、縣令之所治,出貢賦以供天地宗廟百神之祀之壤者哉?鱷魚其不可與刺史雜處此土也!

刺史受天子命,守此土,治此民,而鱷魚睅然不安谿潭,據處食民畜熊豕鹿麞,以肥其身,以種其子孫,與刺史亢拒,爭爲長雄。刺史雖駑弱,亦安肯爲鱷魚低首下心,伈伈睍睍,爲民吏羞,以偷活於此邪!且承天子命以來爲吏,固其勢不得不與鱷魚辨。鱷魚有知,其聽刺史言:

\ProperName{潮}之州,大海在其南。鯨鵬之大,蝦蟹之細,無不容歸,以生以食,鱷魚朝發而夕至也。今與鱷魚約:盡三日,其率醜類南徙於海,以避天子之命吏。三日不能至五日,五日不能至七日,七日不能,是終不肯徙也,是不有刺史,聽從其言也;不然,則是鱷魚冥頑不靈,刺史雖有言,不聞不知也。夫傲天子之命吏,不聽其言,不徙以避之;與冥頑不靈而爲民物害者,皆可殺。刺史則選材技吏民,操強弓毒矢,以與鱷魚從事,必盡殺乃止。其無悔!

\section[柳子厚墓誌銘\quad{\small 韓愈}]{{\normalsize 韓愈}\quad \ProperName{柳子厚}墓誌銘}
\ProperName{子厚}諱\ProperName{宗元}。七世祖\ProperName{慶},爲\ProperName{拓跋魏}侍中,封\ProperName{濟陰公}。曾伯祖\ProperName{奭}爲\ProperName{唐}宰相,與\ProperName{褚遂良}、\ProperName{韓瑗}俱得罪\ProperName{武后},死\ProperName{高宗}朝。皇考諱\ProperName{鎮},以事母棄太常博士,求爲縣令\ProperName{江南}。其後以不能媚權貴失御史。權貴人死,乃復拜侍御史,號爲剛直。所與游皆當世名人。

\ProperName{子厚}少精敏,無不通達。逮其父時,雖少年已自成人,能取進士第,嶄然見頭角;衆謂\ProperName{柳氏}有子矣。其後以博學宏詞授集賢殿正字。儁傑廉悍,議論證據今古,出入經史百子,踔厲風發,率常屈其座人;名聲大振,一時皆慕與之交,諸公要人爭欲令出我門下,交口薦譽之。

\ProperName{貞元}十九年,由\ProperName{藍田}尉拜監察御史。\ProperName{順宗}即位,拜禮部員外郎。遇用事者得罪,例出爲刺史;未至,又例貶州司馬。居閒益自刻苦,務記覽,爲詞章汎濫停蓄,爲深博無涯涘,而自肆於山水間。\ProperName{元和}中,嘗例召至京師,又偕出爲刺史,而\ProperName{子厚}得\ProperName{柳州}。既至,歎曰:「是豈不足爲政邪!」因其土俗,爲設教禁,州人順賴。其俗以男女質錢,約不時贖,子本相侔,則沒爲奴婢。\ProperName{子厚}與設方計,悉令贖歸;其尤貧力不能者,令書其傭,足相當,則使歸其質。觀察使下其法於他州,比一歲,免而歸者且千人。\ProperName{衡}、\ProperName{湘}以南爲進士者,皆以\ProperName{子厚}爲師,其經承\ProperName{子厚}口講指畫爲文詞者,悉有法度可觀。

其召至京師而復爲刺史也,\ProperName{中山}\ProperName{劉夢得}\ProperName{禹錫},亦在遣中,當詣\ProperName{播州}。\ProperName{子厚}泣曰:「\ProperName{播州}非人所居,而\ProperName{夢得}親在堂,吾不忍\ProperName{夢得}之窮,無辭以白其大人;且萬無母子俱往理。」請於朝,將拜疏,願以\ProperName{柳}易\ProperName{播},雖重得罪,死不恨。遇有以\ProperName{夢得}事白上者,\ProperName{夢得}於是改刺\ProperName{連州}。嗚呼!士窮乃見節義。今夫平居里巷相慕悅。酒食游戲相徵逐,詡詡強笑語以相取下,握手出肺肝相示,指天日涕泣,誓生死不相背負,真若可信;一旦臨小利害,僅如毛髮比,反眼若不相識;落陷穽,不一引手救,反擠之又下石焉者,皆是也。此宜禽獸夷狄所不忍爲,而其人自視以爲得計,聞\ProperName{子厚}之風,亦可以少媿矣!

\ProperName{子厚}前時少年,勇於爲人,不自貴重顧藉,謂功業可立就,故坐廢退;既退,又無相知有氣力得位者推挽,故卒死於窮裔,材不爲世用,道不行於時也。使\ProperName{子厚}在臺省時,自持其身已能如司馬、刺史時,亦自不斥;斥時有人力能舉之,且必復用不窮。然\ProperName{子厚}斥不久,窮不極,雖有出於人,其文學辭章,必不能自力以致必傳於後如今,無疑也。雖使\ProperName{子厚}得所願,爲將相於一時;以彼易此,孰得孰失,必有能辨之者。

\ProperName{子厚}以\ProperName{元和}十四年十一月八日卒,年四十七。以十五年七月十日歸葬\ProperName{萬年}先人墓側。\ProperName{子厚}有子男二人:長曰\ProperName{周六},始四歲;季曰\ProperName{周七},\ProperName{子厚}卒乃生。女子二人,皆幼。其得歸葬也,費皆出觀察使\ProperName{河東}\ProperName{裴君}\ProperName{行立}。\ProperName{行立}有節概,重然諾,與\ProperName{子厚}結交,\ProperName{子厚}亦爲之盡,竟賴其力。葬\ProperName{子厚}於\ProperName{萬年}之墓者,舅弟\ProperName{盧遵}。\ProperName{遵},\ProperName{涿}人,性謹慎,學問不厭。自\ProperName{子厚}之斥,\ProperName{遵}從而家焉,逮其死不去;既往葬\ProperName{子厚},又將經紀其家,庶幾有始終者。銘曰:

是惟\ProperName{子厚}之室,既固既安,以利其嗣人。

% Proofed 19 June 2022
% Ref:
%  - 馬其昶, 馬茂元, 韓昌黎文集校注, 上海古籍(1986)
%  - 古文觀止, 中華書局(1959)
