\section[駁復讎議\quad{\small 柳宗元}]{{\normalsize 柳宗元}\quad 駁復讎議}
臣伏見\ProperName{天后}時,有\ProperName{同州}\ProperName{下邽}人\ProperName{徐元慶}者,父\ProperName{爽},爲縣尉\ProperName{趙師韞}所殺,卒能手刃父讎,束身歸罪。當時諫臣\ProperName{陳子昂}建議誅之,而旌其閭,且請編之於令,永爲國典。臣竊獨過之。

臣聞禮之大本,以防亂也,若曰無爲賊虐,凡爲子者殺無赦;刑之大本,亦以防亂也,若曰無爲賊虐,凡爲治者殺無赦。其本則合,其用則異,旌與誅莫得而並焉。誅其可旌,茲謂濫,黷刑甚矣;旌其可誅,茲謂僭,壞禮甚矣。果以是示於天下,傳於後代,趨義者不知所向,違害者不知所立,以是爲典可乎?

蓋聖人之制,窮理以定賞罰,本情以正褒貶,統於一而已矣。嚮使刺讞其誠僞,考正其曲直,原始而求其端,則刑禮之用,判然離矣。何者?若\ProperName{元慶}之父不陷於公罪,\ProperName{師韞}之誅,獨以其私怨,奮其吏氣,虐於非辜,州牧不知罪,刑官不知問,上下蒙冒,籲號不聞,而\ProperName{元慶}能以戴天爲大恥,枕戈爲得禮,處心積慮,以衝讎人之胸,介然自克,即死無憾,是守禮而行義也。執事者宜有慚色,將謝之不暇,而又何誅焉?其或\ProperName{元慶}之父不免於罪,\ProperName{師韞}之誅不愆於法,是非死於吏也,是死於法也。法其可讎乎?讎天子之法,而戕奉法之吏,是悖驁而凌上也。執而誅之,所以正邦典,而又何旌焉?

且其議曰:「人必有子,子必有親。親親相讎,其亂誰救?」是惑於禮也甚矣。禮之所謂讎者,蓋其冤抑沉痛而號無告也;非謂抵罪觸法,陷於大戮。而曰「彼殺之,我乃殺之」,不議曲直,暴寡脅弱而已。其非經背聖,不亦甚哉!\BookTitle{周禮}:「調人掌司萬人之讎。凡殺人而義者,令勿讎,讎之則死。」「有反殺者,邦國交讎之。」又安得親親相仇也?\BookTitle{春秋公羊傳}曰:「父不受誅,子復讎可也。父受誅,子復讎,此推刃之道。復讎不除害。」今若取此以斷兩下相殺,則合於禮矣。且夫不忘讎,孝也;不愛死,義也。\ProperName{元慶}能不越於禮,服孝死義,是必達理而聞道者也。夫達理聞道之人,豈其以王法爲敵讎者哉?議者反以爲戮,黷刑壞禮,其不可以爲典,明矣。

請下臣議,附於令,有斷斯獄者,不宜以前議從事。謹議。

\section[桐葉封弟辯\quad{\small 柳宗元}]{{\normalsize 柳宗元}\quad 桐葉封弟辯}
古之傳者有言,\ProperName{成王}以桐葉與小弱弟,戲曰:「以封汝。」\ProperName{周公}入賀。王曰:「戲也。」周公曰:「天子不可戲。」乃封小弱弟於\ProperName{唐}。

吾意不然。王之弟當封邪?\ProperName{周公}宜以時言於王,不待其戲而賀以成之也;不當封邪?\ProperName{周公}乃成其不中之戲,以地以人與小弱者爲之主,其得爲聖乎?且\ProperName{周公}以王之言,不可苟焉而已,必從而成之邪?設有不幸,王以桐葉戲婦寺,亦將舉而從之乎?凡王者之德,在行之何若。設未得其當,雖十易之不爲病;要於其當,不可使易也,而況以其戲乎?若戲而必行之,是\ProperName{周公}教王遂過也。

吾意\ProperName{周公}輔\ProperName{成王},宜以道,從容優樂,要歸之大中而已,必不逢其失而爲之辭。又不當束縛之,馳驟之,使若牛馬然,急則敗矣。且家人父子尚不能以此自克,況號爲君臣者耶?是直小丈夫{\fontfamily{songext}\selectfont 𡙇𡙇}者之事,非\ProperName{周公}所宜用,故不可信。

或曰:封\ProperName{唐叔},\ProperName{史佚}成之。

\section[箕子碑\quad{\small 柳宗元}]{{\normalsize 柳宗元}\quad \ProperName{箕子}碑}
凡大人之道有三:一曰正蒙難,二曰法授聖,三曰化及民。\ProperName{殷}有仁人曰\ProperName{箕子},實具茲道,以立於世。故\ProperName{孔子}述六經之旨,尤殷勤焉。

當\ProperName{紂}之時,大道悖亂,天威之動不能戒,聖人之言無所用。進死以倂命,誠仁矣,無益吾祀故不爲;委身以存祀,誠仁矣,與亡吾國,故不忍。具是二道,有行之者矣。是用保其明哲,與之俯仰,晦是謨範,辱於囚奴,昏而無邪,隤而不息。故在\BookTitle{易}曰「\ProperName{箕子}之明夷」,正蒙難也。及天命既改,生人以正,乃出大法,用爲聖師,\ProperName{周}人得以序彝倫而立大典。故在\BookTitle{書}曰「以\ProperName{箕子}歸,作\BookTitle{洪範}」,法授聖也。及封\ProperName{朝鮮},推道訓俗,惟德無陋,惟人無遠,用廣\ProperName{殷}祀,俾夷爲華,化及民也。率是大道,藂於厥躬,天地變化,我得其正,其大人歟?

於虖!當其\ProperName{周}時未至,\ProperName{殷}祀未殄,\ProperName{比干}已死,\ProperName{微子}已去,向使\ProperName{紂}惡未稔而自斃,\ProperName{武庚}念亂以圖存,國無其人,誰與興理?是固人事之或然者也。然則先生隱忍而爲此,其有志於斯乎?\ProperName{唐}某年作廟\ProperName{汲郡},歲時致祀。嘉先生獨列於\BookTitle{易}象,作是頌云。

% 蒙難以正,授聖以謨。宗祀用繁,夷民其蘇。憲憲大人,顯晦不渝。聖人之仁,道合隆污。明哲在躬,不陋爲奴。沖讓居禮,不盈稱孤。高而無危,卑不可踰。非死非去,有懷故都。時詘而伸,卒爲世模。\BookTitle{易}象是列,\ProperName{文王}爲徒。大明宣昭,崇祀式孚。古闕頌辭,繼在後儒。

\section[捕蛇者說\quad{\small 柳宗元}]{{\normalsize 柳宗元}\quad 捕蛇者說}
\ProperName{永州}之野產異蛇,黑質而白章,觸草木盡死,以齧人,無禦之者。然得而腊之以爲餌,可以已大風、攣踠、瘻、癘,去死肌,殺三蟲。其始,太醫以王命聚之,歲賦其二,募有能捕之者,當其租入,\ProperName{永}之人爭奔走焉。

有\ProperName{蔣氏}者,專其利三世矣。問之,則曰:「吾祖死於是,吾父死於是,今吾嗣爲之十二年,幾死者數矣。」言之,貌若甚{慼}者。余悲之,且曰:「若毒之乎?余將告於蒞事者,更若役,復若賦,則如何?」% 古文觀止作“戚”,據全唐文等改。

\ProperName{蔣氏}大{戚},汪然出涕曰:「君將哀而生之乎?則吾斯役之不幸,未若復吾賦不幸之甚也。嚮吾不爲斯役,則久已病矣。自吾氏三世居是鄉,積於今六十歲矣,而鄉鄰之生日蹙。殫其地之出,竭其廬之入,號呼而轉徙,飢渴而頓踣,觸風雨,犯寒暑,呼噓毒癘,往往而死者相籍也。曩與吾祖居者,今其室十無一焉;與吾父居者,今其室十無二三焉;與吾居十二年者,今其室十無四五焉,非死則徙爾。而吾以捕蛇獨存。悍吏之來吾鄉,叫囂乎東西,隳突乎南北,嘩然而駭者,雖雞狗不得寧焉。吾恂恂而起,視其缶,而吾蛇尚存,則弛然而臥。謹食之,時而獻焉。退而甘食其土之有,以盡吾齒。蓋一歲之犯死者二焉,其餘則熙熙而樂,豈若吾鄉鄰之旦旦有是哉!今雖死乎此,比吾鄉鄰之死則已後矣,又安敢毒{邪}?」

余聞而愈悲。\ProperName{孔子}曰:「苛政猛於虎也。」吾嘗疑乎是,今以\ProperName{蔣氏}觀之,猶信。嗚呼!孰知賦斂之毒,有甚是蛇者乎!故爲之說,以俟夫觀人風者得焉。

\section[種樹郭橐駝傳\quad{\small 柳宗元}]{{\normalsize 柳宗元}\quad 種樹\ProperName{郭橐駝}傳}
\ProperName{郭橐駝},不知始自何名。病僂,隆然伏行,有類橐駝者,故鄉人號之「駝」。\ProperName{駝}聞之曰:「甚善,名我固當。」因捨其名,亦自謂\ProperName{橐駝}云。其鄉曰\ProperName{豐樂鄉},在\ProperName{長安}西。\ProperName{駝}業種樹,凡\ProperName{長安}豪家富人爲觀遊及賣果者,皆爭迎取養。視\ProperName{駝}所種樹,或{移}\endnote{\BookTitle{觀止}作「遷」,據今校本\BookTitle{柳宗元集}等改。}徙,無不活,且碩茂蚤實以蕃。他植者雖窺伺傚慕,莫能如也。

有問之,對曰:「\ProperName{橐駝}非能使木之壽且孳也,能順木之天,以致其性焉耳。凡植木之性,其本欲舒,其培欲平,其土欲故,其築欲密。既然已,勿動勿慮,去不復顧。其蒔也若子,其置也若棄,則其天者全而其性得矣。故吾不害其長而已,非有能碩而茂之也;不抑耗其實而已,非有能蚤而蕃之也。他植者則不然,根拳而土易,其培之也,若不過焉則不及。苟有能反是者,則又愛之太殷,憂之太勤,旦視而暮撫,已去而復顧。甚者爪其膚以驗其生枯,搖其本以觀其疎密,而木之性日以離矣。雖曰愛之,其實害之;雖曰憂之,其實讎之,故不我若也。吾又何能爲哉?」

問者曰:「以子之道,移之官理可乎?」\ProperName{駝}曰:「我知種樹而已,官理非吾業也。然吾居鄉,見長人者好煩其令,若甚憐焉,而卒以禍。旦暮吏來而呼曰:『官命促爾耕,勗爾植,督爾獲。蚤繅而緒,蚤織而縷,字而幼孩,遂而雞豚。』鳴鼓而聚之,擊木而召之。吾小人輟飧饔以勞吏者,且不得暇,又何以蕃吾生而安吾性邪?故病且怠。若是,則與吾業者其亦有類乎?」    

問者曰:「嘻,不亦善夫!吾問養樹,得養人術。」傳其事以爲官戒也。

\theendnotes

\section[梓人傳\quad{\small 柳宗元}]{{\normalsize 柳宗元}\quad 梓人傳}
\ProperName{裴封叔}之第在\ProperName{光德里}。有梓人款其門,願傭隟宇而處焉。所職尋引、規矩、繩墨,家不居礱斲之器。問其能,曰:「吾善度材,視棟宇之制,高深、圓方、短長之宜,吾指使而羣工役焉。捨我,眾莫能就一宇。故食於官府,吾受祿三倍;作於私家,吾收其直大半焉。」他日,入其室,其牀闕足而不能理,曰:「將求他工。」余甚笑之,謂其無能而貪祿嗜貨者。

其後\ProperName{京兆}尹將飾官署,余往過焉。委羣材,會眾工。或執斧斤,或執刀鋸,皆環立嚮之。梓人左持引右執杖而中處焉。量棟宇之任,視木之能,舉揮其杖曰:「斧!」彼執斧者奔而右;顧而指曰:「鋸!」彼執鋸者趨而左。俄而斤者斲、刀者削,皆視其色,俟其言,莫敢自斷者。其不勝任者,怒而退之,亦莫敢慍焉。畫宮於堵,盈尺而曲盡其制,計其毫釐而構大廈,無進退焉。既成,書於上棟,曰「某年某月某日某建」,則其姓字也。凡執用之工不在列。余圜視大駭,然後知其術之工大矣。

繼而歎曰:「彼將捨其手藝,專其心智,而能知體要者歟?吾聞勞心者役人,勞力者役於人,彼其勞心者歟?能者用而智者謀,彼其智者歟?是足爲佐天子、相天下法矣!物莫近乎此也。彼爲天下者本於人。其執役者,爲徒隸,爲鄉師、里胥;其上爲下士;又其上爲中士、爲上士;又其上爲大夫、爲卿、爲公。離而爲六職,判而爲百役。外薄四海,有方伯、連率。郡有守,邑有宰,皆有佐政。其下有胥吏,又其下皆有嗇夫、版尹,以就役焉,猶眾工之各有執伎以食力也。彼佐天子相天下者,舉而加焉,指而使焉,條其綱紀而盈縮焉,齊其法制而整頓焉,猶梓人之有規矩、繩墨以定制也。擇天下之士,使稱其職;居天下之人,使安其業。視都知野,視野知國,視國知天下,其遠邇細大,可手據其圖而究焉,猶梓人畫宮於堵而績於成也。能者進而由之,使無所德;不能者退而休之,亦莫敢慍。不衒能,不矜名,不親小勞,不侵眾官,日與天下之英才討論其大經,猶梓人之善運眾工而不伐藝也。夫然後相道得而萬國理矣。相道既得,萬國既理,天下舉首而望曰:「吾相之功也。」後之人循跡而慕曰:「彼相之才也。」士或談\ProperName{殷}、\ProperName{周}之理者,曰\ProperName{伊}、\ProperName{傅}、\ProperName{周}、\ProperName{召},其百執事之勤勞而不得紀焉,猶梓人自名其功而執用者不列也。大哉相乎!通是道者,所謂相而已矣。其不知體要者反此:以恪勤爲功,以簿書爲尊,衒能矜名,親小勞,侵眾官,竊取六職百役之事,听听於府庭,而遺其大者遠者焉,所謂不通是道者也。猶梓人而不知繩墨之曲直、規矩之方圓、尋引之短長,姑奪眾工之斧斤刀鋸以佐其藝,又不能備其工,以至敗績用而無所成也。不亦謬歟?

或曰:「彼主爲室者,儻或發其私智,牽制梓人之慮,奪其世守而道謀是用,雖不能成功,豈其罪邪?亦在任之而已。」余曰:不然。夫繩墨誠陳,規矩誠設,高者不可抑而下也,狹者不可張而廣也。由我則固,不由我則圮。彼將樂去固而就圮也,則卷其術,默其智,悠爾而去,不屈吾道,是誠良梓人耳。其或嗜其貨利,忍而不能捨也,喪其制量,屈而不能守也,棟橈屋壞,則曰「非我罪也」,可乎哉,可乎哉?

余謂梓人之道類於相,故書而藏之。梓人,蓋古之審曲面勢者,今謂之都料匠云。余所遇者,\ProperName{楊氏},\ProperName{潛}其名。

\section[愚溪詩序\quad{\small 柳宗元}]{{\normalsize 柳宗元}\quad \ProperName{愚溪}詩序}
\ProperName{灌水}之陽有溪焉,東流入於\ProperName{瀟水}。或曰:\ProperName{冉氏}嘗居也,故姓是溪爲\ProperName{冉溪}。或曰:可以染也,名之以其能,故謂之\ProperName{染溪}。余以愚觸罪,謫\ProperName{瀟水}上,愛是溪,入二三里,得其尤絕者家焉。古有\ProperName{愚公谷},今余家是溪,而名莫能定,土之居者猶齗齗然,不可以不更也,故更之爲\ProperName{愚溪}。

\ProperName{愚溪}之上,買小丘爲\ProperName{愚丘}。自\ProperName{愚丘}東北行六十步,得泉焉,又買居之爲\ProperName{愚泉}。\ProperName{愚泉}凡六穴,皆出山下平地,蓋上出也。合流屈曲而南,爲\ProperName{愚溝},遂負土累石,塞其隘爲\ProperName{愚池}。\ProperName{愚池}之東爲\ProperName{愚堂}。其南爲\ProperName{愚亭}。池之中爲\ProperName{愚島}。嘉木異石錯置,皆山水之奇者,以余故,咸以愚辱焉。

夫水,智者樂也。今是溪獨見辱於愚,何哉?蓋其流甚下,不可以灌溉;又峻急,多坻石,大舟不可入也;幽邃淺狹,蛟龍不屑,不能興雲雨。無以利世,而適類於余,然則雖辱而愚之可也。\ProperName{甯武子}「邦無道則愚」,智而爲愚者也;\ProperName{顏子}「終日不違如愚」,睿而爲愚者也,皆不得爲真愚。今余遭有道,而違於理,悖於事,故凡爲愚者莫我若也。夫然,則天下莫能爭是溪,余得專而名焉。

溪雖莫利於世,而善鑒萬類,清瑩秀澈,鏘鳴金石,能使愚者喜笑眷慕,樂而不能去也。余雖不合於俗,亦頗以文墨自慰,漱滌萬物,牢籠百態,而無所避之。以愚辭歌\ProperName{愚溪},則茫然而不違,昏然而同歸,超鴻蒙,混希夷,寂寥而莫我知也。於是作\BookTitle{八愚詩},紀於溪石上。

\section[永州韋使君新堂記\quad{\small 柳宗元}]{{\normalsize 柳宗元}\quad \ProperName{永州}\ProperName{韋使君}新堂記}
將爲穹谷嵁巖淵池於郊邑之中,則必輦山石,溝澗壑,{凌}\endnote{\BookTitle{觀止}作「陵」,據今校本\BookTitle{柳宗元集}改。}絕險阻,疲極人力,乃可以有爲也。然而求天作地生之狀,咸無得焉。逸其人,因其地,全其天,昔之所難,今於是乎在。

\ProperName{永州}實惟\ProperName{九疑}之麓,其始度土者,環山爲城。有石焉,翳於奧草;有泉焉,伏於土塗。{虵}\endnote{\BookTitle{觀止}作「蛇」,據今校本改。}虺之所蟠,{狸}\endnote{\BookTitle{觀止}作「貍」,據今校本改。}鼠之所游,茂樹惡木,嘉葩毒卉,亂雜而爭植,號爲穢墟。\ProperName{韋公}之來既逾月,理甚無事,望其地,且異之。始命芟其蕪,行其塗,積之丘如,蠲之瀏如。既焚既釃,奇勢迭出,清濁辨質,美惡異位。視其植,則清秀敷舒;視其蓄,則溶漾紆餘。怪石森然,周於四隅,或列或跪,或立或仆,竅穴逶邃,堆阜突怒。乃作棟宇,以爲觀游。凡其物類,無不合形輔勢,效伎於堂廡之下。外之連山高原,林麓之崖,間廁隱顯。邇延野綠,遠混天碧,咸會於譙門之內。

已乃延客入觀,繼以宴娛。或贊且賀曰:「見公之作,知公之志。公之因土而得勝,豈不欲因俗以成化?公之擇惡而取美,豈不欲除殘而佑仁?公之蠲濁而流清,豈不欲廢貪而立廉?公之居高以望遠,豈不欲家撫而戶曉?夫然,則是堂也,豈獨草木土石水泉之適歟?山原林麓之觀歟?將使繼公之理者,視其細,知其大也」。\ProperName{宗元}請志諸石,措諸壁,编以爲二千石楷法。

\theendnotes

\section[鈷鉧潭西小丘記\quad{\small 柳宗元}]{{\normalsize 柳宗元}\quad \ProperName{鈷鉧潭}西小丘記}
得\ProperName{西山}後八日,尋山口西北道二百步,又得\ProperName{鈷鉧潭}。西二十五步,當湍而浚者爲魚梁。梁之上有丘焉,生竹樹。其石之突怒偃蹇,負土而出,爭爲奇狀者,殆不可數。其嶔然相累而下者,若牛馬之飲於溪;其衝然角列而上者,若熊羆之登於山。丘之小不能一畝,可以籠而有之。問其主,曰:「\ProperName{唐氏}之棄地,貨而不售。」問其價,曰:「止四百。」余憐而售之。\ProperName{李深源}、\ProperName{元克己}時同遊,皆大喜,出自意外。即更取器用,鏟刈穢草,伐去惡木,烈火而焚之。嘉木立,美竹露,奇石顯。由其中以望,則山之高,雲之浮,溪之流,鳥獸之邀遊,舉熙熙然迴巧獻技,以效茲丘之下。枕席而臥,則清泠之狀與目謀,瀯瀯之聲與耳謀,悠然而虛者與神謀,淵然而靜者與心謀。不匝旬而得異地者二,雖古好事之士,或未能至焉。

噫!以茲丘之勝,致之\ProperName{澧}、\ProperName{鎬}、\ProperName{鄠}、\ProperName{杜}則貴游之士爭買者,日增千金而愈不可得。今棄是州也,農夫漁父過而陋之,價四百,連歲不能售。而我與\ProperName{深源}、\ProperName{克己}獨喜得之,是其果有遭乎!書於石,所以賀茲丘之遭也。

\section[小石城山記\quad{\small 柳宗元}]{{\normalsize 柳宗元}\quad \ProperName{小石城山}記}
自\ProperName{西山}道口徑北,踰\ProperName{黃茅嶺}而下,有二道:其一西出,尋之無所得;其一少北而東,不過四十丈,土斷而川分,有積石橫當其垠。其上爲睥睨梁欐之形,其旁出堡塢,有若門焉。窺之正黑,投以小石,洞然有水聲,其響之激越,良久乃已。環之可上,望甚遠,無土壤,而生嘉樹美箭,益奇而堅,其疏數偃仰,類智者所施設也。

噫!吾疑造物者之有無久矣。及是,愈以爲誠有。又怪其不爲之於中州,而列是夷狄,更千百年不得一售其伎,是固勞而無用,神者儻不宜如是,則其果無乎?或曰:「以慰夫賢而辱於此者。」或曰:「其氣之靈不爲偉人,而獨爲是物,故\ProperName{楚}之南少人而多石。」是二者,余未信之。

\section[賀進士王參元失火書\quad{\small 柳宗元}]{{\normalsize 柳宗元}\quad 賀進士\ProperName{王參元}失火書}
得\ProperName{楊八}書,知足下遇火災,家無餘儲。僕始聞而駭,中而疑,終乃大喜,蓋將弔而更以賀也。道遠言略,猶未能究知其狀,果若蕩焉泯焉而悉無有,乃吾所以尤賀者也。

足下勤奉養,樂朝夕,惟恬安無事是望也。今乃有焚煬赫烈之虞,以震駭左右,而脂膏滫瀡之具,或以不給,吾是以始而駭也。凡人之言,皆曰盈虛倚伏,去來之不可常。或將大有爲也,乃始厄困震悸,於是有水火之孽,有羣小之慍,勞苦變動,而後能光明,古之人皆然。斯道遼闊誕漫,雖聖人不能以是必信,是故中而疑也。以足下讀古人書,爲文章,善小學,其爲多能若是,而進不能出羣士之上以取顯貴者,蓋無他焉。京城人多言足下家有積貨,士之好廉名者,皆畏忌不敢道足下之善,獨自得之,心蓄之,銜忍而不出諸口,以公道之難明,而世之多嫌也。一出口,則嗤嗤者以爲得重賂。僕自\ProperName{貞元}十五年見足下之文章,蓄之者蓋六七年未嘗言。是僕私一身而負公道久矣,非特負足下也。及爲御史尚書郎,自以幸爲天子近臣,得奮其舌,思以發明足下之鬱塞。然時稱道於行列,猶有顧視而竊笑者,僕良恨修己之不亮,素譽之不立,而爲世嫌之所加,常與\ProperName{孟幾道}言而痛之。乃今幸爲天火之所滌盪,凡眾之疑慮,舉爲灰埃。黔其廬,赭其垣,以示其無有,而足下之才能乃可顯白而不污。其實出矣,是\ProperName{祝融}、\ProperName{回祿}之相吾子也。則僕與\ProperName{幾道}十年之相知,不若茲火一夕之爲足下譽也。宥而彰之,使夫蓄於心者,咸得開其喙,發策決科者,授子而不慄,雖欲如嚮之蓄縮受侮,其可得乎?於茲吾有望於子!是以終乃大喜也。古者列國有災,同位皆相弔,\ProperName{許}不弔災,君子惡之。今吾之所陳若是,有以異乎古,故將弔而更以賀也。\ProperName{顏}、\ProperName{曾}之養,其爲樂也大矣,又何闕焉?

\section[待漏院記\quad{\small 王禹偁}]{{\normalsize 王禹偁}\quad 待漏院記}
天道不言,而品物亨,嵗功成者,何謂也?四時之吏,五行之佐,宣其氣矣。聖人不言,而百姓親、萬邦寧者,何謂也?三公論道,六卿分職,張其教矣。是知君逸於上,臣勞於下,法乎天也。古之善相天下者,自\ProperName{咎}、\ProperName{䕫}至\ProperName{房}、\ProperName{魏},可數也。是不獨有其德,亦皆務於勤耳。況夙興夜寐,以事一人,卿大夫猶然,況宰相乎!

朝廷自國初,因舊制,設宰相待漏院於\ProperName{丹鳯門}之右,示勤政也。至若北闕向曙,東方未明,相君啟行煌煌火城。相君至止,噦噦鸞聲。金門未闢,玉漏猶滴。撤蓋下車,於焉以息。

待漏之際,相君其有思乎:其或兆民未安,思所泰之;四夷未附,思所來之;兵革未息,何以弭之;田疇多蕪,何以闢之;賢人在野,我将進之;佞人立朝,我將斥之;六氣不和,災眚薦至,願避位以禳之;五刑未措,欺詐日生,請修德以釐之。憂心忡忡,待旦而入。九門既啓,四聰甚邇。相君言焉,時君納焉。皇風於是乎清夷,蒼生以之而富庶。若然,則總百官,食萬錢,非幸也,宜也。

其或私讎未復,思所逐之;舊恩未報,思所榮之;子女玉帛,何以致之;車馬{玩器},何以取之;姦人附勢,我將陟之;直士抗言,我將黜之;三時告災,上有憂色,構巧詞以悅之;羣吏弄法,君聞怨言,進諂容以媚之。私心慆慆,假寐而坐。九門既開,重瞳屢回。相君言焉,時君惑焉。政柄於是乎隳哉,帝位以之而危矣。若然,則死下獄,投遠方,非不幸也,亦宜也。% 諸本多作「器玩」,古文觀止從文章辨體

是知一國之政,萬人之命,懸於宰相,可不慎歟!復有無毀無譽,旅進旅退,竊位而苟祿,備員而全身者,亦無所取焉。棘寺小吏\ProperName{王禹偁}爲文請誌院壁,用規於執政者。

\section[黃岡竹樓記\quad{\small 王禹偁}]{{\normalsize 王禹偁}\quad \ProperName{黃岡}竹樓記}
\ProperName{黃岡}之地多竹,大者如椽。竹工破之,刳去其節,用代陶瓦,比屋皆然,以其價亷而工省也。

予城西北隅,雉堞圮毀,蓁莽荒穢,因作小樓二間,與\ProperName{月波樓}通。遠吞山光,平挹江瀨,幽闃遼夐,不可具狀。夏宜急雨,有瀑布聲;冬宜密雪,有碎玉聲。宜鼓琴,琴調和暢;宜詠詩,詩韻清絶;宜圍棊,子聲丁丁然;宜投壺,矢聲錚錚然:皆竹樓之所助也。

公退之暇,披鶴氅衣,戴華陽巾,手執\BookTitle{周易}一卷,焚香默坐,消遣世慮。江山之外,第見風帆沙鳥、煙雲竹樹而已。待其酒力醒,茶煙歇,送夕陽,迎素月,亦謫居之勝概也。

彼\ProperName{齊雲}、\ProperName{落星},高則高矣!\ProperName{井幹}、\ProperName{麗譙},華則華矣!止於貯妓女、藏歌舞,非騷人之事,吾所不取。吾聞竹工云:「竹之爲瓦僅十稔;若重覆之得二十稔。」噫!吾以\ProperName{至道}乙未嵗自翰林出\ProperName{滁}上,丙申移\ProperName{廣陵},丁酉又入西掖,戊戌嵗除日,有\ProperName{齊安}之命,已亥閏三月到郡。四年之間,奔走不暇;未知明年又在何處。豈懼竹樓之易朽乎!後之人與我同志嗣而葺之,庶斯樓之不朽也。%\ProperName{咸平}二年八月十五日記。

\section[書洛陽名園記後\quad{\small 李格非}]{{\normalsize 李格非}\quad 書\BookTitle{洛陽名園記}後}
\ProperName{洛陽}處天下之中,挾\ProperName{殽}、\ProperName{黽}之阻,當\ProperName{秦}、\ProperName{隴}之襟喉,而\ProperName{趙}、\ProperName{魏}之走集。蓋四方必爭之地也。天下常無事則已,有事則\ProperName{洛陽}必先受兵。予故嘗曰:「\ProperName{洛陽}之盛衰,天下治亂之候也。」

{方}\ProperName{唐}\ProperName{貞觀}、\ProperName{開元}之間,公卿貴戚開館列第於\ProperName{東都}者,號千有餘邸。及其亂離,繼以\ProperName{五季}之酷,其池塘竹樹,兵車蹂蹴,廢而爲丘墟;髙亭大榭,煙火焚燎,化而爲灰燼,與\ProperName{唐}共滅而俱亡,無餘處矣。予故嘗曰:「園囿之興廢,\ProperName{洛陽}盛衰之候也。」且天下之治亂候於\ProperName{洛陽}之盛衰而知,\ProperName{洛陽}之盛衰候於園囿之興廢而得,則\BookTitle{名園記}之作,予豈徒然哉。% 古文觀止脱

嗚呼!公卿大夫方進於朝,放乎以一己之私自爲之,而忘天下之治忽,欲退享此得乎?\ProperName{唐}之末路是已。

\section[嚴先生祠堂記\quad{\small 范仲淹}]{{\normalsize 范仲淹}\quad \ProperName{嚴先生}祠堂記}
先生,\ProperName{光武}之故人也。相尚以道,及帝握赤符,乘六龍,得聖人之時,臣妾億兆,天下孰加焉,惟先生以節高之;既而動星象,歸江湖,得聖人之清,泥塗軒冕,天下孰加焉,惟\ProperName{光武}以禮下之。在\BookTitle{蠱}之上九,衆方有爲,而獨「不事王侯,高尚其事」,先生以之;在\BookTitle{屯}之初九,陽德方亨,而能「以貴下賤,大得民也」,光武以之。蓋先生之心出乎日月之上,\ProperName{光武}之量包乎天地之外。微先生不能成\ProperName{光武}之大,微\ProperName{光武}豈能遂先生之高哉!而使貪夫廉,懦夫立,是有大功於名教也。\ProperName{仲淹}來守是邦,始構堂而奠焉。乃復爲其後者四家,以奉祠事。又從而歌曰:雲山蒼蒼,江水泱泱。先生之風,山高水長。

\section[岳陽樓記\quad{\small 范仲淹}]{{\normalsize 范仲淹}\quad \ProperName{岳陽樓}記}
\ProperName{慶歷}四年春,\ProperName{滕子京}謫守\ProperName{巴陵郡}。越明年,政通人和,百廢具興,乃重修\ProperName{岳陽樓},增其舊制,刻\ProperName{唐}賢、今人詩賦于其上,屬予作文以記之。

予觀夫\ProperName{巴陵}勝狀,在\ProperName{洞庭}一湖。銜遠山,吞\ProperName{長江},浩浩湯湯,橫無際涯,朝暉夕陰,氣象萬千,此則\ProperName{岳陽樓}之大觀也,前人之述備矣。然則北通\ProperName{巫峽},南極\ProperName{瀟}\ProperName{湘},遷客騷人,多會於此,覽物之情,得無異乎?

若夫霪雨霏霏,連日不開,陰風怒號,濁浪排空,日星隱耀,山岳潛形,商旅不行,檣傾楫摧,薄暮冥冥,虎嘯猿啼,登斯樓也,則有去國懷鄉,憂讒畏譏,滿目蕭然,感極而悲者矣。

至若春和景明,波瀾不驚,上下天光,一碧萬頃,沙鷗翔集,錦鱗游泳,岸芷汀蘭,郁郁青青;而或長煙一空,皓月千里,浮光曜金,靜影沉璧,漁歌互答,此樂何極,登斯樓也,則有心曠神怡,寵辱皆忘,把酒臨風,其喜洋洋者矣。

嗟夫!予嘗求古仁人之心,或異二者之爲,何哉?不以物喜,不以已悲。居廟堂之高,則憂其民;處江湖之遠,則憂其君。是進亦憂,退亦憂。然則何時而樂耶?其必曰:先天下之憂而憂,後天下之樂而樂歟!噫!微斯人,吾誰與歸!% 時六年九月十五日。

\section[諫院題名記\quad{\small 司馬光}]{{\normalsize 司馬光}\quad 諫院題名記}
古者諫無官,自公卿大夫至于工商,無不得諫者。\ProperName{漢}興以来,始置官。夫以天下之政,四海之衆,得失利病,萃于一官使言之,其爲任亦重矣。居是官者,當志其大,{捨}其細。先其急,後其緩,專利國家而不爲身謀。彼汲汲於名者,猶汲汲於利也。其間相去何遠哉!\ProperName{天禧}初,\ProperName{真宗}詔置諫官六員,責以職事;\ProperName{慶歷}中,\ProperName{錢君}始書其名於版,光恐久而漫滅;\ProperName{嘉祐}八年,刻著于石。後之人將歴指其名而議之曰:某也忠,某也詐,某也直,某也曲。嗚呼!可不懼哉?

\section[義田記\quad{\small 錢公輔}]{{\normalsize 錢公輔}\quad 義田記}
\ProperName{范文正公},\ProperName{蘇}人也,平生好施與,擇其親而貧,疎而賢者,咸施之。

方貴顯時,置負郭常稔之田千畝,號曰「義田」,以養濟羣族之人。日有食,歲有衣,嫁娶凶葬皆有贍。擇族之長而賢者主其計,而時{其}\endnote{\BookTitle{觀止}作「共」,據\BookTitle{宋文鑑}改。}出納焉。日食,人一升。歲衣,人一縑,嫁女者五十千,再嫁者三十千,娶婦者三十千,再娶者十五千,葬者如再嫁之數,葬幼者十千。族之聚者九十口,歲入給稻八百斛;以其所入,給其所聚,沛然有餘而無窮。屏而家居俟代者與焉;仕而居官者罷{其}\endnote{\BookTitle{觀止}作「莫」,據\BookTitle{宋文鑑}改。}給。此其大較也。

初,公之未貴顯也,嘗有志於是矣,而力未逮者二十年。既而爲西帥,及參大政,於是始有祿賜之入,而終其志。公既歿,後世子孫修其業,承其志,如公之存也。公雖位充祿厚,而貧終其身。歿之日,身無以爲斂,子無以爲喪,惟以施貧活族之義,遺其子而已。

昔\ProperName{晏平仲}敝車羸馬,\ProperName{桓子}曰:「是隱君之賜也。」\ProperName{晏子}日:「自臣之貴,父之族,無不乘車者;母之族,無不足於衣食者;妻之族,無凍餒者;\ProperName{齊國}之士,待臣而舉火者,三百餘人。如此而爲隱君之賜乎?彰君之賜乎?」於是\ProperName{齊侯}以\ProperName{晏子}之觴而觴\ProperName{桓子}。予嘗愛\ProperName{晏子}好仁,\ProperName{齊侯}知賢,而\ProperName{桓子}服義也。又愛\ProperName{晏子}之仁有等級,而言有次第也:先父族,次母族,次妻族,而後及其疏遠之賢。\ProperName{孟子}曰:「親親而仁民,仁民而愛物。」\ProperName{晏子}爲近之。今觀\ProperName{文正公}之義,賢於\ProperName{平仲},其規模遠舉又疑過之。

嗚呼!世之都三公位,享萬鍾祿,其邸第之雄,車輿之飾,聲色之多,妻孥之富,止乎一己而已;而族之人不得其門{而入}者\endnote{\BookTitle{觀止}作「不得其門者」,脫「而入」二字,據\BookTitle{宋文鑑}改。},豈少也哉!況於施賢乎!其下爲卿、爲大夫、爲士,廩稍之充,奉養之厚,止乎一己而已;而族之人操壺瓢爲溝中瘠者,又豈少哉?況於{他}人乎!是皆公之罪人也。公之忠義滿朝廷,事業滿邊隅,功名滿天下,後必有史官書之者,予可{略}\endnote{\BookTitle{觀止}作「無錄」,據\BookTitle{宋文鑑}改。}也。獨高其義,因以遺於世云。 % 「他」古文觀止作「它」,同明賀復徵文章辨體彙選
% 宋文鑑·卷八十

\theendnotes

\section[袁州州學記\quad{\small 李覯}]{{\normalsize 李覯}\quad \ProperName{袁州}州學記}
皇帝二十有三年,制詔州縣立學。惟時守令,有哲有愚。有屈力殫慮,祗順德意;有假官借師,苟具文書。或連數城,亡誦弦聲。倡而不和,教尼不行。

三十有二年,\ProperName{范陽}\ProperName{祖}君\ProperName{無{擇}}\endnote{\BookTitle{觀止}訛作「無澤」。\ProperName{祖無擇}(1011——1084),字\ProperName{擇之},\ProperName{宋}\ProperName{蔡州}\ProperName{上蔡}(今屬\ProperName{河南})人,事蹟具\BookTitle{宋史}卷三三一本傳。}知\ProperName{袁州}。始至,進諸生,知學宮闕狀。大懼人材放失,儒效闊疎,無以稱上旨。通判\ProperName{穎川}\ProperName{陳}君\ProperName{侁},聞而是之,議以克合。相舊\ProperName{夫子廟},陿隘不足改爲,乃營治之東。厥土燥剛,厥位面陽,厥材孔良,殿堂門廡,黝堊丹漆,舉以法,故生師有舍,庖廩有次,百爾器備,並手偕作。工善吏勤,晨夜展力,越明年成,舍菜且有日。\ProperName{盱江}\ProperName{李覯}諗於衆曰:「惟四代之學,考諸經可見已。\ProperName{秦}以\ProperName{山西}鏖六國,欲帝萬世,\ProperName{劉氏}一呼,而關門不守,武夫健將,賣降恐後,何邪?\BookTitle{詩}、\BookTitle{書}之道廢,人惟見利,而不聞義焉耳。\ProperName{孝武}乘豐富,\ProperName{世祖}出戎行,皆孳孳學術,俗化之厚,延於\ProperName{靈}、\ProperName{獻}。草茅危言者,折首而不悔;功烈震主者,聞命而釋兵;羣雄相視,不敢去臣位,尚數十年。教道之結人心如此。% 古文觀止作 無澤

今代遭聖神,爾\ProperName{袁}得聖君,俾爾由庠序,踐古人之迹。天下治,則譚禮樂以陶吾民;一有不幸,尤當仗大節,爲臣死忠,爲子死孝。使人有所賴,且有所法。是惟朝家教學之意。若其弄筆墨以徼利達而已,豈徒二三子之羞,抑亦爲國者之憂。」

% 四部叢刊·直講李先生文集·卷二三
\theendnotes

\section[朋黨論\quad{\small 歐陽脩}]{{\normalsize 歐陽脩}\quad 朋黨論}
臣聞朋黨之說自古有之,惟幸人君辨其君子小人而已。大凡君子與君子以同道爲朋,小人與小人以同利爲朋,此自然之理也。然臣謂小人無朋,惟君子則有之。其故何哉?小人所好者利祿也,所貪者貨財也。當其同利之時,暫相黨引以爲朋者,僞也。及其見利而爭先,或利盡而交疏,則反相賊害,雖其兄弟親戚不能相保。故臣謂小人無朋,其暫爲朋者,僞也。君子則不然。所守者道義,所行者忠信,所惜者名節。以之修身,則同道而相益,以之事國,則同心而共濟,終始如一,此君子之朋也。故爲人君者,但當退小人之僞朋,用君子之真朋,則天下治矣。

\ProperName{堯}之時,小人\ProperName{共工}、\ProperName{讙兜}等四人爲一朋,君子\ProperName{八元}、\ProperName{八愷}十六人爲一朋。\ProperName{舜}佐\ProperName{堯},退四凶小人之朋,而進\ProperName{元}\ProperName{愷}君子之朋,\ProperName{堯}之天下大治。及\ProperName{舜}自爲天子,而\ProperName{臯}、\ProperName{夔}、\ProperName{稷}、\ProperName{契}等二十二人並{列}\endnote{\BookTitle{觀止}作「立」,據校本改。}於朝,更相稱美,更相推讓,凡二十二人爲一朋,而\ProperName{舜}皆用之,天下亦大治。\BookTitle{書}曰:「\ProperName{紂}有臣億萬,惟億萬心;\ProperName{周}有臣三千,惟一心。」\ProperName{紂}之時,億萬人各異心,可謂不爲朋矣,然\ProperName{紂}以亡國。\ProperName{周武王}之臣,三千人爲一大朋,而\ProperName{周}用以興。\ProperName{後漢}\ProperName{獻帝}時,盡取天下名士囚禁之,目爲黨人。及\ProperName{黃巾}賊起,\ProperName{漢室}大亂,後方悔悟,盡解黨人而釋之,然已無救矣。\ProperName{唐}之晚年,漸起朋黨之論。及\ProperName{昭宗}時,盡殺朝之名士,或投之\ProperName{黃河},曰此輩清流,可投濁流。而\ProperName{唐}遂亡矣。

夫前世之主,能使人人異心不爲朋,莫如\ProperName{紂};能禁絕善人爲朋,莫如\ProperName{漢獻帝};能誅戮清流之朋,莫如\ProperName{唐昭宗}之世。然皆亂亡其國。更相稱美推讓而不自疑,莫如\ProperName{舜}之二十二臣,\ProperName{舜}亦不疑而皆用之。然而後世不誚\ProperName{舜}爲二十二人朋黨所欺,而稱\ProperName{舜}爲聰明之聖者,以能辨君子與小人也。\ProperName{周武}之世,舉其國之臣三千人共爲一朋,自古爲朋之多且大莫如\ProperName{周}。然\ProperName{周}用此以興者,善人雖多而不厭也。嗟{夫}\endnote{\BookTitle{觀止}作「嗟乎」,據\BookTitle{宋文選}改,今校本及各集多作「夫」無「嗟」字,\BookTitle{文章辨體}作「嗟乎夫」。}!興亡治亂\endnote{\BookTitle{觀止}作「治亂興亡」倒置,據校本改。}之迹,爲人君者可以鑒矣。

\theendnotes

\section[縱囚論\quad{\small 歐陽脩}]{{\normalsize 歐陽脩}\quad 縱囚論}
信義行於君子,而刑戮施於小人。刑入於死者,乃罪大惡極,此又小人之尤甚者也。寧以義死,不苟幸生,而視死如歸,此又君子之尤難者也。方\ProperName{唐太宗}之六年,錄大辟囚三百餘人,縱使還家,約其自歸以就死,是以君子之難能,期小人之尤者以必能也。其囚及期而卒自歸無後者,是君子之所難而小人之所易也。此豈近於人情?

或曰:罪大惡極誠小人矣,及施恩德以臨之,可使變而爲君子。蓋恩德入人之深而移人之速,有如是者矣。曰:「\ProperName{太宗}之爲此,所以求此名也。然安知夫縱之去也,不意其必來以冀免,所以縱之乎?又安知夫被縱而去也,不意其自歸而必獲免,所以復來乎?夫意其必來而縱之,是上賊下之情也;意其必免而復來,是下賊上之心也。吾見上下交相賊以成此名也,烏有所謂施恩德與夫知信義者哉!不然,\ProperName{太宗}施德於天下,於茲六年矣,不能使小人不爲極惡大罪,而一日之恩,能使視死如歸而存信義,此又不通之論也。」

「然則何爲而可?」曰:「縱而來歸,殺之無赦,而又縱之,而又來,則可知爲恩德之致爾。然此必無之事也。若夫縱而來歸而赦之,可偶一爲之耳,若屢爲之,則殺人者皆不死,是可爲天下之常法乎?不可爲常者,其聖人之法乎?是以\ProperName{堯}、\ProperName{舜}、\ProperName{三王}之治,必本於人情,不立異以爲高,不逆情以干譽。」

\section[釋祕演詩集序\quad{\small 歐陽脩}]{{\normalsize 歐陽脩}\quad \BookTitle{釋祕演詩集}序}
予少以進士游京師,因得盡交當世之賢豪。然猶以謂國家臣一四海,休兵革,養息天下,以無事者四十年,而智謀雄偉非常之士無所用其能者,往往伏而不出,山林屠販必有老死而世莫見者,欲從而求之不可得。其後得吾亡友\ProperName{石曼卿}。\ProperName{曼卿}爲人廓然有大志,時人不能用其材,\ProperName{曼卿}亦不屈以求合。無所放其意,則往往從布衣野老酣嬉淋漓,顛倒而不厭。予疑所謂伏而不見者,庶幾狎而得之,故嘗喜從\ProperName{曼卿}游,欲因以陰求天下奇士。

浮屠\ProperName{祕演}者,與\ProperName{曼卿}交最久,亦能遺外世俗,以氣節{相}高\endnote{\BookTitle{觀止}作「自高」,據校本改。},二人懽然無所間。\ProperName{曼卿}隱於酒,\ProperName{祕演}隱於浮屠,皆奇男子也。然喜爲歌詩以自娛。當其極飲大醉,歌吟笑呼,以適天下之樂,何其壯也!一時賢士皆願從其游,予亦時至其室。十年之間,\ProperName{祕演}北渡\ProperName{河},東之\ProperName{濟}、\ProperName{鄆},無所合,困而歸。\ProperName{曼卿}已死,\ProperName{祕演}亦老病。嗟夫!二人者,予乃見其盛衰,則{余}亦將老矣。% 古文觀止作「予」

夫\ProperName{曼卿}詩辭清絕,尤稱\ProperName{祕演}之作,以爲雅健有詩人之意。\ProperName{祕演}狀貌雄傑,其胸中浩然,既習於佛,無所用,獨其詩可行於世,而懶不自惜。已老,胠其橐,尚得三四百篇,皆可喜者。\ProperName{曼卿}死,\ProperName{祕演}漠然無所向,聞東南多山水,其巔崖崛峍,江濤洶湧,甚可壯也,遂欲往游焉。足以知其老而志在也。於其將行,爲敘其詩,因道其盛時以悲其衰。%\ProperName{慶歷}二年十二月二十八日,\ProperName{廬陵}\ProperName{歐陽修}序。

\theendnotes

% Proofed 22 June 2022
% Ref. 
% - 柳宗元集, 中華書局(1979)
% - 宋文鑑·卷七七: 待漏院記、竹樓記; 卷八十: 義田記; 卷一三一: 書洛陽名園記後
% - 四部叢刊·范文正公集
% - 四部叢刊·溫國司馬文正公集
% - 四部叢刊·直講李先生文集·卷二三
% - 歐陽脩全集, 中華書局(2001)
% - 古文觀止, 中華書局(1959)