\section[五帝本紀贊\quad{\small 史記}]{{\normalsize 史記}\quad \BookTitle{五帝本紀}贊}
\ProperName{太史公}曰:學者多稱五帝,尚矣。然\BookTitle{尚書}獨載\ProperName{堯}以來;而百家言\ProperName{黄帝},其文不雅馴,薦紳先生難言之。\ProperName{孔子}所傳\BookTitle{宰予問五帝德}及\BookTitle{帝繫姓},儒者或不傳。余嘗西至\ProperName{空峒},北過\ProperName{涿鹿},東漸於海,南浮\ProperName{江}\ProperName{淮}矣。至長老皆各往往稱\ProperName{黄帝}、\ProperName{堯}、\ProperName{舜}之處,風教固殊焉。總之不離古文者近是。予觀\BookTitle{春秋}、\BookTitle{國語},其發明\BookTitle{五帝德}、\BookTitle{帝繫姓}章矣。顧弟弗深考,其所表見皆不虛。\BookTitle{書}缺有間矣,其軼乃時時見於他說。非好學深思,心知其意,固難爲淺見寡聞道也。余幷論次,擇其言尤雅者,故著爲本紀書首。

\section[項羽本紀贊\quad{\small 史記}]{{\normalsize 史記}\quad\BookTitle{項羽本紀}贊}
\ProperName{太史公}曰:吾聞之\ProperName{周生}曰:「\ProperName{舜}目蓋重瞳子」,又聞\ProperName{項羽}亦重瞳子。\ProperName{羽}豈其苗裔邪?何興之暴也!夫\ProperName{秦}失其政,\ProperName{陳涉}首難,豪傑蠭起,相與並爭,不可勝數。然\ProperName{羽}非有尺寸,乘勢起隴畝之中,三年,遂將五諸侯滅\ProperName{秦},分裂天下,而封王侯,政由\ProperName{羽}出,號爲「霸王」,位雖不終,近古以來未嘗有也。及\ProperName{羽}背關懷\ProperName{楚},放逐\ProperName{義帝}而自立,怨王侯叛己,難矣。自矜功伐,奮其私智而不師古,謂霸王之業,欲以力征經營天下,五年卒亡其國,身死\ProperName{東城},尚不覺寤而不自責,過矣。乃引「天亡我,非用兵之罪也」,豈不謬哉!

\section[秦楚之際月表\quad{\small 史記}]{{\normalsize 史記}\quad\ProperName{秦}\ProperName{楚}之際月表}
\ProperName{太史公}讀\ProperName{秦}\ProperName{楚}之際,曰:初作難,發於\ProperName{陳涉};虐戾滅\ProperName{秦},自\ProperName{項氏};撥亂誅暴,平定海內,卒踐帝祚,成於\ProperName{漢家}。五年之間,號令三嬗,自生民以來未始有受命若斯之亟也。

昔\ProperName{虞}、\ProperName{夏}之興,積善累功數十年,德洽百姓,攝行政事,考之於天、然後在位。\ProperName{湯}、\ProperName{武}之王,乃由\ProperName{契}、\ProperName{后稷}脩仁行義十餘世,不期而會\ProperName{孟津}八百諸侯,猶以爲未可,其後乃放弒。\ProperName{秦}起\ProperName{襄公},章於\ProperName{文}、\ProperName{繆},\ProperName{獻}、\ProperName{孝}之後,稍以蠶食六國,百有餘載,至\ProperName{始皇}乃能幷冠帶之倫。以德若彼,用力如此,蓋一統若斯之難也。

\ProperName{秦}既稱帝,患兵革不休,以有諸侯也,於是無尺土之封,墮壞名城,銷鋒鏑,鉏豪傑,維萬世之安。然王跡之興,起於閭巷,合從討伐,軼於\ProperName{三代},鄉\ProperName{秦}之禁,適足以資賢者爲驅除難耳。故憤發其所爲天下雄,安在無土不王?此乃傳之所謂大聖乎?豈非天哉、豈非天哉!非大聖孰能當此受命而帝者乎?

\section[高祖功臣侯年表\quad{\small 史記}]{{\normalsize 史記}\quad\ProperName{高祖}功臣侯年表}
\ProperName{太史公}曰:古者人臣功有五品,以德立宗廟定社稷曰勳,以言曰勞,用力曰功,明其等曰伐,積日曰閱。封爵之誓曰:「使\ProperName{河}如帶,\ProperName{泰山}若厲。國以永寧,爰及苗裔。」始未嘗不欲固其根本,而枝葉稍陵夷衰微也。

余讀\ProperName{高祖}侯功臣,察其首封,所以失之者,曰:異哉所聞!\BookTitle{書}曰「協和萬國」,遷于\ProperName{夏}\ProperName{商},或數千歲。蓋\ProperName{周}封八百,\ProperName{幽}\ProperName{厲}之後,見於\BookTitle{春秋}。\BookTitle{尚書}有\ProperName{唐}\ProperName{虞}之侯伯,歷\ProperName{三代}千有餘載,自全以蕃衞天子,豈非篤于仁義,奉上法哉?\ProperName{漢}興,功臣受封者百有餘人。天下初定,故大城名都散亡戶口,可得而數者十二三,是以大侯不過萬家,小者五六百戶。後數世,民咸歸鄉里,戶益息,\ProperName{蕭}、\ProperName{曹}、\ProperName{絳}、\ProperName{灌}之屬,或至四萬,小侯自倍,富厚如之。子孫驕溢,忘其先,淫嬖。至\ProperName{太初}百年之間,見侯五,餘皆坐法隕命亡國,耗矣。罔亦少密焉,然皆身無兢兢於當世之禁云。

居今之世,志古之道,所以自鏡也,未必盡同。帝王者各殊禮而異務,要以成功爲統紀,豈可緄乎?觀所以得尊寵及所以廢辱,亦當世得失之林也,何必舊聞?於是謹其終始,表見其文,頗有所不盡本末;著其明,疑者闕之。後有君子,欲推而列之,得以覽焉。

\section[孔子世家贊\quad{\small 史記}]{{\normalsize 史記}\quad\BookTitle{孔子世家}贊}
\ProperName{太史公}曰:\BookTitle{詩}有之:「高山仰止。景行行止。」雖不能至,然心鄉往之。余讀\ProperName{孔氏}書,想見其爲人。適\ProperName{魯}觀\ProperName{仲尼}廟堂,車服禮器,諸生以時習禮其家,余低回留之不能去云。天下君王至於賢人衆矣,當時則榮,沒則已焉。\ProperName{孔子}布衣,傳十餘世,學者宗之。自天子王侯,中國言\BookTitle{六藝}者折中於夫子,可謂至聖矣!

\section[外戚世家序\quad{\small 史記}]{{\normalsize 史記}\quad \BookTitle{外戚世家}序}
自古受命帝王及繼體守文之君,非獨內德茂也,蓋亦有外戚之助焉。\ProperName{夏}之興也以\ProperName{塗山},而\ProperName{桀}之放也以\ProperName{末喜}\endnote{\BookTitle{觀止}作「妹喜」,據\BookTitle{史記}各本改。}。\ProperName{殷}之興也以\ProperName{有娀},\ProperName{紂}之殺也嬖\ProperName{妲己}。\ProperName{周}之興也以\ProperName{姜原}及\ProperName{大任},而\ProperName{幽王}之禽也淫於\ProperName{褒姒}。故\BookTitle{易}基\BookTitle{乾}\BookTitle{坤},\BookTitle{詩}始\BookTitle{關雎}、\BookTitle{書}美釐降,\BookTitle{春秋}譏不親迎。夫婦之際,人道之大倫也。禮之用,唯婚姻爲兢兢。夫樂調而四時和,陰陽之變,萬物之統也。可不慎與?人能弘道,無如命何。甚哉,妃匹之愛,君不能得之於臣,父不能得之於子,況卑下乎!既驩合矣,或不能成子姓;能成子姓矣,或不能要其終:豈非命也哉?\ProperName{孔子}罕稱命,蓋難言之也。非通幽明之變,惡能識乎性命哉?

\theendnotes

\section[伯夷列傳\quad{\small 史記}]{{\normalsize 史記}\quad\ProperName{伯夷}列傳}
夫學者載籍極博,猶考信於六蓺。\BookTitle{詩}\BookTitle{書}雖缺,然\ProperName{虞}\ProperName{夏}之文可知也。\ProperName{堯}將遜位,讓於\ProperName{虞舜},\ProperName{舜}\ProperName{禹}之間,岳牧咸薦,乃試之於位,典職數十年,功用既興,然後授政。示天下重器,王者大統,傳天下若斯之難也。而說者曰\ProperName{堯}讓天下於\ProperName{許由},\ProperName{許由}不受,恥之,逃隱。及\ProperName{夏}之時,有\ProperName{卞隨}、\ProperName{務光}者。此何以稱焉?\ProperName{太史公}曰:余登\ProperName{箕山},其上蓋有\ProperName{許由}冢云。\ProperName{孔子}序列古之仁聖賢人,如\ProperName{吳太伯}、\ProperName{伯夷}之倫詳矣。余以所聞\ProperName{由}、\ProperName{光}義至高,其文辭不少概見,何哉?

\ProperName{孔子}曰:「\ProperName{伯夷}、\ProperName{叔齊},不念舊惡,怨是用希。」「求仁得仁,又何怨乎?」余悲\ProperName{伯夷}之意,睹軼詩可異焉。其傳曰:
\begin{quotation}
\ProperName{伯夷}、\ProperName{叔齊},\ProperName{孤竹君}之二子也。父欲立\ProperName{叔齊},及父卒,\ProperName{叔齊}讓\ProperName{伯夷}。\ProperName{伯夷}曰:「父命也。」遂逃去。\ProperName{叔齊}亦不肯立而逃之。國人立其中子。於是\ProperName{伯夷}、\ProperName{叔齊}聞\ProperName{西伯昌}善養老,盍往歸焉。及至,\ProperName{西伯}卒,\ProperName{武王}載木主,號爲\ProperName{文王}、東伐\ProperName{紂}。\ProperName{伯夷}、\ProperName{叔齊}叩馬而諫曰:「父死不葬,爰及干戈,可謂孝乎?以臣弒君,可謂仁乎?」左右欲兵之。\ProperName{太公}曰:「此義人也。」扶而去之。\ProperName{武王}已平\ProperName{殷}亂,天下宗\ProperName{周},而\ProperName{伯夷}、\ProperName{叔齊}恥之,義不食\ProperName{周}粟,隱於\ProperName{首陽山},采薇而食之。及餓且死,作歌。其辭曰:「登彼\ProperName{西山}兮,采其薇矣。以暴易暴兮,不知其非矣。\ProperName{神農}、\ProperName{虞}、\ProperName{夏}忽焉沒兮,我安適歸矣?于嗟徂兮,命之衰矣!」遂餓死於\ProperName{首陽山}。
\end{quotation}
由此觀之,怨邪非邪?

或曰:「天道無親,常與善人。」若\ProperName{伯夷}、\ProperName{叔齊},可謂善人者非邪?積仁絜行如此而餓死!且七十子之徒,\ProperName{仲尼}獨薦\ProperName{顏淵}爲好學。然\ProperName{回}也屢空,糟糠不厭,而卒蚤夭。天之報施善人,其何如哉?\ProperName{盜跖}日殺不辜,肝人之肉,暴戾恣睢,聚黨數千人橫行天下,竟以壽終。是遵何德哉?此其尤大彰明較著者也。若至近世,操行不軌,事犯忌諱,而終身逸樂,富厚累世不絕。或擇地而蹈之,時然後出言,行不由徑,非公正不發憤,而遇禍災者,不可勝數也。余甚惑焉,儻所謂天道,是邪非邪?

子曰「道不同不相爲謀」,亦各從其志也。故曰「富貴如可求,雖執鞭之士,吾亦爲之。如不可求,從吾所好。」「歲寒,然後知松柏之後凋。」舉世混濁,清士乃見。豈以其重若彼,其輕若此哉?

「君子疾沒世而名不稱焉。」\ProperName{賈子}曰:「貪夫徇財,烈士徇名,夸者死權,衆庶馮生。」「同明相照,同類相求。」「雲從龍,風從虎,聖人作而萬物覩。」\ProperName{伯夷}、\ProperName{叔齊}雖賢,得夫子而名益彰。\ProperName{顏淵}雖篤學,附驥尾而行益顯。巖穴之士,趣舍有時若此,類名堙滅而不稱,悲夫!閭巷之人,欲砥行立名者,非附青雲之士,惡能施於後世哉?

\section[管晏列傳\quad{\small 史記}]{{\normalsize 史記}\quad\ProperName{管}\ProperName{晏}列傳}
\ProperName{管仲}\ProperName{夷吾}者,\ProperName{潁上}人也。少時常與\ProperName{鮑叔牙}游,\ProperName{鮑叔}知其賢。\ProperName{管仲}貧困,常欺\ProperName{鮑叔},\ProperName{鮑叔}終善遇之,不以爲言。已而\ProperName{鮑叔}事\ProperName{齊}\ProperName{公子小白},\ProperName{管仲}事\ProperName{公子糾}。及\ProperName{小白}立爲\ProperName{桓公},\ProperName{公子糾}死,\ProperName{管仲}囚焉。\ProperName{鮑叔}遂進\ProperName{管仲}。\ProperName{管仲}既用,任政於\ProperName{齊},\ProperName{齊桓公}以霸,九合諸侯,一匡天下,\ProperName{管仲}之謀也。

\ProperName{管仲}曰:「吾始困時,嘗與\ProperName{鮑叔}賈,分財利多自與,\ProperName{鮑叔}不以我爲貪,知我貧也。吾嘗爲\ProperName{鮑叔}謀事而更窮困,\ProperName{鮑叔}不以我爲愚,知時有利不利也。吾嘗三仕三見逐於君,\ProperName{鮑叔}不以我爲不肖,知我不遭時也。吾嘗三戰三走,\ProperName{鮑叔}不以我怯,知我有老母也。\ProperName{公子糾}敗,召忽死之,吾幽囚受辱,\ProperName{鮑叔}不以我爲無恥,知我不羞小節而恥功名不顯于天下也。生我者父母,知我者\ProperName{鮑子}也。」

\ProperName{鮑叔}既進\ProperName{管仲},以身下之。子孫世祿於\ProperName{齊},有封邑者十餘世,常爲名大夫。天下不多\ProperName{管仲}之賢而多\ProperName{鮑叔}能知人也。

\ProperName{管仲}既任政相\ProperName{齊},以區區之\ProperName{齊}在海濱,通貨積財,富國彊兵,與俗同好惡。故其稱曰:「倉廩實而知禮節,衣食足而知榮辱,上服度則六親固。四維不張,國乃滅亡。下令如流水之原,令順民心。」故論卑而易行。俗之所欲,因而予之;俗之所否,因而去之。

其爲政也,善因禍而爲福,轉敗而爲功。貴輕重,慎權衡。\ProperName{桓公}實怒\ProperName{少姬},南襲\ProperName{蔡},\ProperName{管仲}因而伐\ProperName{楚},責包茅不入貢於\ProperName{周室}。\ProperName{桓公}實北征\ProperName{山戎},而\ProperName{管仲}因而令\ProperName{燕}修\ProperName{召公}之政。於\ProperName{柯}之會,\ProperName{桓公}欲背\ProperName{曹沫}之約,\ProperName{管仲}因而信之,諸侯由是歸\ProperName{齊}。故曰:「知與之爲取,政之寶也。」

\ProperName{管仲}富擬於公室,有三歸、反坫,\ProperName{齊}人不以爲侈。\ProperName{管仲}卒,\ProperName{齊國}遵其政,常彊於諸侯。後百餘年而有\ProperName{晏子}焉。

\ProperName{晏平仲}\ProperName{嬰}者,\ProperName{萊}之\ProperName{夷維}人也。事\ProperName{齊靈公}、\ProperName{莊公}、\ProperName{景公},以節儉力行重於\ProperName{齊}。既相\ProperName{齊},食不重肉,妾不衣帛。其在朝,君語及之,即危言;語不及之,即危行。國有道,即順命;無道,即衡命。以此三世顯名於諸侯。

\ProperName{越石父}賢,在縲紲中。\ProperName{晏子}出,遭之塗,解左驂贖之,載歸。弗謝,入閨,久之。\ProperName{越石父}請絕。\ProperName{晏子}懼然,攝衣冠謝曰:「\ProperName{嬰}雖不仁,免子於戹,何子求絕之速也?」\ProperName{石父}曰:「不然。吾聞君子詘於不知己而信於知己者。方吾在縲紲中,彼不知我也。夫子既已感寤而贖我,是知己;知己而無禮,固不如在縲紲之中。」\ProperName{晏子}於是延入爲上客。

\ProperName{晏子}爲\ProperName{齊}相,出,其御之妻從門閒而闚其夫。其夫爲相御,擁大蓋,策駟馬,意氣揚揚甚自得也。既而歸,其妻請去。夫問其故。妻曰:「\ProperName{晏子}長不滿六尺,身相\ProperName{齊國},名顯諸侯。今者妾觀其出,志念深矣,常有以自下者。今子長八尺,乃爲人僕御,然子之意自以爲足,妾是以求去也。」其後夫自抑損。\ProperName{晏子}怪而問之,御以實對。\ProperName{晏子}薦以爲大夫。

\ProperName{太史公}曰:吾讀\ProperName{管氏}\BookTitle{牧民}、\BookTitle{山高}、\BookTitle{乘馬}、\BookTitle{輕重}、\BookTitle{九府},及\BookTitle{晏子春秋},詳哉其言之也。既見其著書,欲觀其行事,故次其傳。至其書,世多有之,是以不論,論其軼事。

\ProperName{管仲}世所謂賢臣,然\ProperName{孔子}小之。豈以爲\ProperName{周}道衰微,\ProperName{桓公}既賢,而不勉之至王,乃稱霸哉?語曰「將順其美,匡救其惡,故上下能相親也」。豈\ProperName{管仲}之謂乎?

方\ProperName{晏子}伏\ProperName{莊公}尸哭之,成禮然後去,豈所謂「見義不爲無勇」者邪?至其諫說,犯君之顏,此所謂「進思盡忠,退思補過」者哉!假令\ProperName{晏子}而在,余雖爲之執鞭,所忻慕焉。

\section[屈原列傳\quad{\small 史記}]{{\normalsize 史記}\quad\ProperName{屈原}列傳}
\ProperName{屈原}者,名\ProperName{平},\ProperName{楚}之同姓也。爲\ProperName{楚懷王}左徒。博聞彊志,明於治亂,嫺于辭令。入則與王圖議國事,以出號令;出則接遇賓客,應對諸侯。王甚任之。

\ProperName{上官大夫}與之同列,爭寵而心害其能。\ProperName{懷王}使\ProperName{屈原}造爲憲令,\ProperName{屈平}屬草稾未定。\ProperName{上\allowbreak 官\allowbreak 大\allowbreak 夫}見而欲奪之,\ProperName{屈平}不與,因讒之曰:「王使\ProperName{屈平}爲令,衆莫不知,每一令出,\ProperName{平}伐其功,曰以爲『非我莫能爲』也。」王怒而疏\ProperName{屈平}。

\ProperName{屈平}疾王聽之不聰也,讒諂之蔽明也,邪曲之害公也,方正之不容也,故憂愁幽思而作\BookTitle{離騷}。離騷者,猶離憂也。夫天者,人之始也;父母者,人之本也。人窮則反本,故勞苦倦極,未嘗不呼天也;疾痛慘怛,未嘗不呼父母也。\ProperName{屈平}正道直行,竭忠盡智以事其君,讒人間之,可謂窮矣。信而見疑,忠而被謗,能無怨乎?\ProperName{屈平}之作\BookTitle{離騷},蓋自怨生也。\BookTitle{國風}好色而不淫,\BookTitle{小雅}怨誹而不亂。若\BookTitle{離騷}者,可謂兼之矣。上稱\ProperName{帝嚳},下道\ProperName{齊桓},中述\ProperName{湯}\ProperName{武},以刺世事。明道德之廣崇,治亂之條貫,靡不畢見。其文約,其辭微,其志絜,其行廉,其稱文小而其指極大,舉類邇而見義遠。其志絜,故其稱物芳。其行廉,故死而不容。自疏濯淖汙泥之中,蟬蛻於濁穢,以浮游塵埃之外,不獲世之滋垢,皭然泥而不滓者也。推此志也,雖與日月爭光可也。

\ProperName{屈平}既絀,其後\ProperName{秦}欲伐\ProperName{齊},\ProperName{齊}與\ProperName{楚}從親,\ProperName{惠王}患之,乃令\ProperName{張儀}詳去\ProperName{秦},厚幣委質事\ProperName{楚},曰:「\ProperName{秦}甚憎\ProperName{齊},\ProperName{齊}與\ProperName{楚}從親,\ProperName{楚}誠能絕\ProperName{齊},\ProperName{秦}願獻\ProperName{商}、\ProperName{於}之地六百里。」\ProperName{楚懷王}貪而信\ProperName{張儀},遂絕\ProperName{齊},使使如\ProperName{秦}受地。\ProperName{張儀}詐之曰:「\ProperName{儀}與王約六里,不聞六百里。」\ProperName{楚}使怒去,歸告\ProperName{懷王}。\ProperName{懷王}怒,大興師伐\ProperName{秦}。\ProperName{秦}發兵擊之,大破\ProperName{楚}師於\ProperName{丹}、\ProperName{淅}\endnote{\BookTitle{觀止}作「浙」,據\BookTitle{史記}點校本改。\ProperName{張文虎}\BookTitle{札記}:\BookTitle{索隱}本、\ProperName{凌}、\ProperName{毛}本並譌「浙」,\BookTitle{注}同,依\BookTitle{攷異}改。},斬首八萬,虜\ProperName{楚}將\ProperName{屈匄},遂取\ProperName{楚}之\ProperName{漢中}地。\ProperName{懷王}乃悉發國中兵以深入擊\ProperName{秦},戰於\ProperName{藍田}。\ProperName{魏}聞之,襲\ProperName{楚}至\ProperName{鄧}。\ProperName{楚}兵懼,自\ProperName{秦}歸。而\ProperName{齊}竟怒不救\ProperName{楚},\ProperName{楚}大困。

明年,\ProperName{秦}割\ProperName{漢中}地與\ProperName{楚}以和。\ProperName{楚王}曰:「不願得地,願得\ProperName{張儀}而甘心焉。」\ProperName{張儀}聞,乃曰:「以一\ProperName{儀}而當\ProperName{漢中}地,臣請往如\ProperName{楚}。」如\ProperName{楚},又因厚幣用事者臣\ProperName{靳尚},而設詭辯於\ProperName{懷王}之寵姬\ProperName{鄭袖}。\ProperName{懷王}竟聽\ProperName{鄭袖},復釋去\ProperName{張儀}。是時\ProperName{屈平}既疏,不復在位,使於\ProperName{齊},顧反,諫\ProperName{懷王}曰:「何不殺\ProperName{張儀}?」\ProperName{懷王}悔,追\ProperName{張儀}不及。

其後諸侯共擊\ProperName{楚},大破之,殺其將\ProperName{唐眜}\endnote{\BookTitle{觀止}作「昧」,據\BookTitle{史記}點校本改。\ProperName{張文虎}\BookTitle{札記}:各本譌「昧」依\BookTitle{志疑}改。}。

時\ProperName{秦昭王}與\ProperName{楚}婚,欲與\ProperName{懷王}會。\ProperName{懷王}欲行,\ProperName{屈平}曰:「\ProperName{秦}虎狼之國,不可信,不如無行。」\ProperName{懷王}稚子\ProperName{子蘭}勸王行:「柰何絕\ProperName{秦}歡!」\ProperName{懷王}卒行。入\ProperName{武關},\ProperName{秦}伏兵絕其後,因留\ProperName{懷王},以求割地。\ProperName{懷王}怒,不聽。亡走\ProperName{趙},\ProperName{趙}不內。復之\ProperName{秦},竟死於\ProperName{秦}而歸葬。

長子\ProperName{頃襄王}立,以其弟\ProperName{子蘭}爲令尹。\ProperName{楚}人既咎\ProperName{子蘭}以勸\ProperName{懷王}入\ProperName{秦}而不反也。

\ProperName{屈平}既嫉之,雖放流,睠顧\ProperName{楚國},繫心\ProperName{懷王},不忘欲反,冀幸君之一悟,俗之一改也。其存君興國而欲反覆之,一篇之中三致志\endnote{\BookTitle{觀止}作「意」,據\BookTitle{史記}各本改。}焉。然終無可柰何,故不可以反,卒以此見\ProperName{懷王}之終不悟也。人君無愚智賢不肖,莫不欲求忠以自爲,舉賢以自佐,然亡國破家相隨屬,而聖君治國累世而不見者,其所謂忠者不忠,而所謂賢者不賢也。\ProperName{懷王}以不知忠臣之分,故內惑於\ProperName{鄭袖},外欺於\ProperName{張儀},疏\ProperName{屈平}而信\ProperName{上官大夫}、\ProperName{令尹子蘭}。兵挫地削,亡其六郡,身客死於\ProperName{秦},爲天下笑。此不知人之禍也。\BookTitle{易}曰:「井泄\endnote{\BookTitle{觀止}作「渫」,據\BookTitle{史記}點校本改。\BookTitle{易}作「渫」。}不食,爲我心惻,可以汲。王明,並受其福。」王之不明,豈足福哉!

\ProperName{令尹子蘭}聞之大怒,卒使\ProperName{上官大夫}短\ProperName{屈原}於\ProperName{頃襄王},\ProperName{頃襄王}怒而遷之。

\ProperName{屈原}至於江濱,被髪行吟澤畔。顏色憔悴,形容枯槁。漁父見而問之曰:「子非三閭大夫歟?何故而至此?」\ProperName{屈原}曰:「舉世混濁而我獨清,衆人皆醉而我獨醒,是以見放。」漁父曰:「夫聖人者,不凝滯於物而能與世推移。舉世混濁,何不隨其流而揚其波?衆人皆醉,何不餔其糟而啜其醨?何故懷瑾握瑜而自令見放爲?」\ProperName{屈原}曰:「吾聞之,新沐者必彈冠,新浴者必振衣,人又誰能以身之察察,受物之汶汶者乎!寧赴常流而葬乎江魚腹中耳,又安能以皓皓之白而蒙世俗之溫蠖乎!」

乃作\BookTitle{懷沙}之賦。\endnote{\BookTitle{觀止}注「\BookTitle{懷沙賦}刪去。」}於是懷石遂自投\ProperName{汨羅}以死。

\ProperName{屈原}既死之後,\ProperName{楚}有\ProperName{宋玉}、\ProperName{唐勒}、\ProperName{景差}之徒者,皆好辭而以賦見稱;然皆祖\ProperName{屈原}之從容辭令,終莫敢直諫。其後\ProperName{楚}日以削,數十年竟爲\ProperName{秦}所滅。

自\ProperName{屈原}沉\ProperName{汨羅}後百有餘年,\ProperName{漢}有\ProperName{賈生},爲\ProperName{長沙王}太傅,過湘水,投書以弔\ProperName{屈原}。\endnote{\BookTitle{史記}下有\BookTitle{賈生列傳},略。}

\ProperName{太史公}曰:余讀\BookTitle{離騷}、\BookTitle{天問}、\BookTitle{招魂}、\BookTitle{哀郢},悲其志。適\ProperName{長沙},觀\endnote{\BookTitle{觀止}作「過」,據\BookTitle{史記}各本改。}\ProperName{屈原}所自沉淵,未嘗不垂涕,想見其爲人。及見\ProperName{賈生}弔之,又怪\ProperName{屈原}以彼其材,游諸侯,何國不容,而自令若是。讀\BookTitle{服烏賦},同死生,輕去就,又爽然自失矣。

\theendnotes

\section[酷吏列傳序\quad{\small 史記}]{{\normalsize 史記}\quad \BookTitle{酷吏列傳}序}
\ProperName{孔子}曰:「導\endnote{\BookTitle{觀止}作「道」,同\BookTitle{論語}所引,據\BookTitle{史記}改。}之以政,齊之以刑,民免而無恥。導之以德,齊之以禮,有恥且格。」\ProperName{老氏}稱:「上德不德,是以有德;下德不失德,是以無德。法令滋章,盜賊多有。」\ProperName{太史公}曰:信哉是言也!法令者治之具,而非制治清濁之源也。昔天下之網嘗密矣,然姦僞萌起,其極也,上下相遁,至於不振。當是之時,吏治若救火揚沸,非武健嚴酷,惡能勝其任而愉快乎!言道德者,溺其職矣。故曰「聽訟,吾猶人也,必也使無訟乎」。「下士聞道大笑之」。非虛言也。\ProperName{漢}興,破觚而爲圜,斲雕而爲朴,網漏於吞舟之魚,而吏治烝烝,不至於姦,黎民艾安。由是觀之,在彼不在此。

\theendnotes

\section[游俠列傳序\quad{\small 史記}]{{\normalsize 史記}\quad \BookTitle{游俠列傳}序}
\ProperName{韓子}曰:「儒以文亂法,而俠以武犯禁。」二者皆譏,而學士多稱於世云。至如以術取宰相卿大夫,輔翼其世主,功名俱著於春秋,固無可言者。及若\ProperName{季次}、\ProperName{原憲},閭巷人也,讀書懷獨行君子之德,義不苟合當世,當世亦笑之。故\ProperName{季次}、\ProperName{原憲}終身空室蓬戶,褐衣疏食不厭。死而已四百餘年,而弟子志之不倦。今游俠,其行雖不軌於正義,然其言必信,其行必果,已諾必誠,不愛其軀,赴士之阸困,既已存亡死生矣,而不矜其能,羞伐其德,蓋亦有足多者焉。

且緩急,人之所時有也。\ProperName{太史公}曰:昔者\ProperName{虞舜}窘於井廩,\ProperName{伊尹}負於鼎俎,\ProperName{傅說}匿於\ProperName{傅險},\ProperName{呂尚}困於\ProperName{棘津},\ProperName{夷吾}桎梏,\ProperName{百里}飯牛,\ProperName{仲尼}畏\ProperName{匡},菜色\ProperName{陳}、\ProperName{蔡}。此皆學士所謂有道仁人也,猶然遭此菑,況以中材而涉亂世之末流乎?其遇害何可勝道哉!

鄙人有言曰:「何知仁義,已饗其利者爲有德。」故\ProperName{伯夷}醜\ProperName{周},餓死\ProperName{首陽山},而\ProperName{文}\ProperName{武}不以其故貶王;\ProperName{跖}、\ProperName{蹻}暴戾,其徒誦義無窮。由此觀之,「竊鉤者誅,竊國者侯,侯之門,仁義存」,非虛言也。

今拘學或抱咫尺之義,久孤於世,豈若卑論儕俗,與世沈浮\endnote{\BookTitle{觀止}倒作「浮沉」,據\BookTitle{史記}改。}而取榮名哉!而布衣之徒,設取予然諾,千里誦義,爲死不顧世,此亦有所長,非苟而已也。故士窮窘而得委命,此豈非人之所謂賢豪閒者邪?誠使鄉曲之俠,予\ProperName{季次}、\ProperName{原憲}比權量力,效功於當世,不同日而論矣。要以功見言信,俠客之義又曷可少哉!

古布衣之俠,靡得而聞已。近世\ProperName{延陵}、\ProperName{孟嘗}、\ProperName{春申}、\ProperName{平原}、\ProperName{信陵}之徒,皆因王者親屬,藉於有土卿相之富厚,招天下賢者,顯名諸侯,不可謂不賢者矣。比如順風而呼,聲非加疾,其埶激也。至如閭巷之俠,脩行砥名,聲施於天下,莫不稱賢,是爲難耳。然儒、\ProperName{墨}皆排擯不載。自\ProperName{秦}以前,匹夫之俠,湮滅不見,余甚恨之。以余所聞,\ProperName{漢}興有\ProperName{朱家}、\ProperName{田仲}、\ProperName{王公}、\ProperName{劇孟}、\ProperName{郭解}之徒,雖時扞當世之文罔,然其私義廉絜退讓,有足稱者。名不虛立,士不虛附。至如朋黨宗彊比周,設財役貧,豪暴侵淩孤弱,恣欲自快,游俠亦醜之。余悲世俗不察其意,而猥以\ProperName{朱家}、\ProperName{郭解}等令與暴豪之徒同類而共笑之也。

\theendnotes

\section[滑稽列傳\quad{\small 史記}]{{\normalsize 史記}\quad 滑稽列傳}
\ProperName{孔子}曰:「六蓺於治一也。\BookTitle{禮}以節人,\BookTitle{樂}以發和,\BookTitle{書}以道\endnote{\BookTitle{觀止}作「導」,據\BookTitle{史記}各本改。}事,\BookTitle{詩}以達意,\BookTitle{易}以神化,\BookTitle{春秋}以道義。」\ProperName{太史公}曰:天道恢恢,豈不大哉!談言微中,亦可以解紛。

\ProperName{淳于髡}者,\ProperName{齊}之贅壻也。長不滿七尺,滑稽多辯,數使諸侯,未嘗屈辱。\ProperName{齊威王}之時喜隱,好爲淫樂長夜之飲,沈湎不治,委政卿大夫。百官荒亂,諸侯並侵,國且危亡,在於旦暮,左右莫敢諫。\ProperName{淳于髡}說之以隱曰:「國中有大鳥,止王之庭,三年不蜚又不鳴,王知此鳥何也?」王曰:「此鳥不飛則已,一飛沖天\endnote{\BookTitle{觀止}「飛」作「蜚」,據\BookTitle{史記}改。};不鳴則已,一鳴驚人。」於是乃朝諸縣令長七十二人,賞一人,誅一人,奮兵而出。諸侯振驚,皆還\ProperName{齊}侵地。威行三十六年。語在\BookTitle{田完世家}中。

\ProperName{威王}八年,\ProperName{楚}大發兵加\ProperName{齊}。\ProperName{齊王}使\ProperName{淳于髡}之\ProperName{趙}請救兵,齎金百斤,車馬十駟。\ProperName{淳于髡}仰天大笑,冠纓索絕。王曰:「先生少之乎?」\ProperName{髡}曰:「何敢!」王曰:「笑豈有說乎?」\ProperName{髡}曰:「今者臣從東方來,見道傍有禳\endnote{\BookTitle{觀止}作「穰」,據\BookTitle{史記}點校本改。\ProperName{張文虎}\BookTitle{札記}:索隱本、舊刻、毛本「禳」,各本譌「穰」。}田者,操一豚蹄,酒一盂,而祝曰:『甌窶滿篝,汙邪滿車,五穀蕃熟,穰穰滿家。』臣見其所持者狹而所欲者奢,故笑之。」於是\ProperName{齊威王}乃益齎黃金千鎰,白璧十雙,車馬百駟。\ProperName{髡}辭而行,至\ProperName{趙}。\ProperName{趙王}與之精兵十萬,革車千乘。\ProperName{楚}聞之,夜引兵而去。

\ProperName{威王}大說,置酒後宮,召\ProperName{髡}賜之酒。問曰:「先生能飲幾何而醉?」對曰:「臣飲一斗亦醉,一石亦醉。」\ProperName{威王}曰:「先生飲一斗而醉,惡能飲一石哉!其說可得聞乎?」\ProperName{髡}曰:「賜酒大王之前,執法在傍,御史在後,\ProperName{髡}恐懼俯伏而飲,不過一斗徑醉矣。若親有嚴客,\ProperName{髡}帣韝鞠{\fontfamily{songext}\selectfont 𦜕},待酒於前,時賜餘瀝,奉觴上壽,數起,飲不過二斗徑醉矣。若朋友交遊,久不相見,卒然相覩,歡然道故,私情相語,飲可五六斗徑醉矣。若乃州閭之會,男女雜坐,行酒稽留,六博投壺,相引爲曹,握手無罰,目眙不禁,前有墮珥,后有遺簪,\ProperName{髡}竊樂此,飲可八斗而醉二參。日暮酒闌,合尊促坐,男女同席,履舄交錯,杯盤狼藉,堂上燭滅,主人留\ProperName{髡}而送客,羅襦襟解,微聞薌澤,當此之時,\ProperName{髡}心最歡,能飲一石。故曰酒極則亂,樂極則悲;萬事盡然。」言不可極,極之而衰。以諷諫焉。\ProperName{齊王}曰:「善。」乃罷長夜之飲,以\ProperName{髡}爲諸侯主客。宗室置酒,\ProperName{髡}嘗在側。

\theendnotes

\section[貨殖列傳序\quad{\small 史記}]{{\normalsize 史記}\quad \BookTitle{貨殖列傳}序}
\BookTitle{老子}曰:「至治之極,鄰國相望,雞狗之聲相聞,民各甘其食,美其服,安其俗,樂其業,至老死不相往來。」必用此爲務,輓近世塗民耳目,則幾無行矣。

\ProperName{太史公}曰:夫\ProperName{神農}以前,吾不知已。至若\BookTitle{詩}\BookTitle{書}所述\ProperName{虞}\ProperName{夏}以來,耳目欲極聲色之好,口欲窮芻豢之味,身安逸樂,而心誇矜埶能之榮使。俗之漸民久矣,雖戶說以眇論,終不能化。故善者因之,其次利道之,其次教誨之,其次整齊之,最下者與之爭。

夫\ProperName{山}西饒材、竹、穀、纑、旄、玉石;\ProperName{山}東多魚、鹽、漆、絲、聲色;\ProperName{江}南出枏、梓、薑、桂、金、錫、連、丹沙、犀、瑇瑁、珠璣、齒革;\ProperName{龍門}、\ProperName{碣石}北多馬、牛、羊、旃裘、筋角;銅、鐵則千里往往山出棊置:此其大較也。皆中國人民所喜好,謠俗被服飲食奉生送死之具也。故待農而食之,虞而出之,工而成之,商而通之。此寧有政教發徵期會哉?人各任其能,竭其力,以得所欲。故物賤之徵貴,貴之徵賤,各勸其業,樂其事,若水之趨下,日夜無休時,不召而自來,不求而民出之。豈非道之所符,而自然之驗邪?

\BookTitle{周書}曰:「農不出則乏其食,工不出則乏其事,商不出則三寶絕,虞不出則財匱少。」財匱少而山澤不辟矣。此四者,民所衣食之原也。原大則饒,原小則鮮。上則富國,下則富家。貧富之道,莫之奪予,而巧者有餘,拙者不足。故\ProperName{太公望}封於\ProperName{營丘},地澙鹵,人民寡,於是\ProperName{太公}勸其女功,極技巧,通魚鹽,則人物歸之,繦至而輻湊。故\ProperName{齊}冠帶衣履天下,海\ProperName{岱}之閒斂袂而往朝焉。其後\ProperName{齊}中衰,\ProperName{管子}修之,設輕重九府,則\ProperName{桓公}以霸,九合諸侯,一匡天下;而\ProperName{管氏}亦有三歸,位在陪臣,富於列國之君。是以\ProperName{齊}富彊至於\ProperName{威}、\ProperName{宣}也。

故曰:「倉廩實而知禮節,衣食足而知榮辱。」禮生於有而廢於無。故君子富,好行其德;小人富,以適其力。淵深而魚生之,山深而獸往之,人富而仁義附焉。富者得埶益彰,失埶則客無所之,以而不樂。夷狄益甚。\endnote{「夷狄益甚」句,\BookTitle{觀止}脱,據\BookTitle{史記}補。}諺曰:「千金之子,不死於市。」此非空言也。故曰:「天下熙熙,皆爲利來;天下壤壤,皆爲利往。」夫千乘之王,萬家之侯,百室之君,尚猶患貧,而況匹夫編戶之民乎!

\theendnotes

\section[太史公自序\quad{\small 史記}]{{\normalsize 史記}\quad \ProperName{太史公}自序}
\ProperName{太史公}曰:「先人有言:『自\ProperName{周公}卒五百歲而有\endnote{\BookTitle{觀止}作「生」,據\BookTitle{史記}本改。}\ProperName{孔子}。\ProperName{孔子}卒後至於今五百歲,有能紹明世,正\BookTitle{易傳},繼\BookTitle{春秋},本\BookTitle{詩}\BookTitle{書}\BookTitle{禮}\BookTitle{樂}之際?』意在斯乎!意在斯乎!小子何敢讓焉。」

上大夫\ProperName{壺遂}曰:「昔\ProperName{孔子}何爲而作\BookTitle{春秋}哉?」\ProperName{太史公}曰:「余聞\ProperName{董生}曰:『\ProperName{周}道衰廢,\ProperName{孔子}爲\ProperName{魯}司寇,諸侯害之,大夫壅之。\ProperName{孔子}知言之不用,道之不行也,是非二百四十二年之中,以爲天下儀表,貶天子,退諸侯,討大夫,以達王事而已矣。』子曰:『我欲載之空言,不如見之於行事之深切著明也。』夫\BookTitle{春秋},上明\ProperName{三王}之道,下辨人事之紀,別嫌疑,明是非,定猶豫,善善惡惡,賢賢賤不肖,存亡國,繼絕世,補敝起廢,王道之大者也。\BookTitle{易}著天地陰陽四時五行,故長於變;\BookTitle{禮}經紀人倫,故長於行;\BookTitle{書}記先王之事,故長於政;\BookTitle{詩}記山川谿谷禽獸草木牝牡雌雄,故長於風;\BookTitle{樂}樂所以立,故長於和;\BookTitle{春秋}辯是非,故長於治人。是故\BookTitle{禮}以節人,\BookTitle{樂}以發和,\BookTitle{書}以道事,\BookTitle{詩}以達意,\BookTitle{易}以道化,\BookTitle{春秋}以道義。撥亂世反之正,莫近於\BookTitle{春秋}。\BookTitle{春秋}文成數萬,其指數千。萬物之散聚皆在\BookTitle{春秋}。\BookTitle{春秋}之中,弒君三十六,亡國五十二,諸侯奔走不得保其社稷者不可勝數。察其所以,皆失其本已。故\BookTitle{易}曰『失之豪釐,差以千里』。故曰『臣弒君,子弒父,非一旦一夕之故也,其漸久矣』。故有國者不可以不知\BookTitle{春秋},前有讒而弗見,後有賊而不知。爲人臣者不可以不知\BookTitle{春秋},守經事而不知其宜,遭變事而不知其權。爲人君父而不通於\BookTitle{春秋}之義者,必蒙首惡之名。爲人臣子而不通於\BookTitle{春秋}之義者,必陷篡弒之誅,死罪之名。其實皆以爲善,爲之不知其義,被之空言而不敢辭。夫不通禮義之旨,至於君不君,臣不臣,父不父,子不子。君不君則犯,臣不臣則誅,父不父則無道,子不子則不孝。此四行者,天下之大過也。以天下之大過予之,則受而弗敢辭。故\BookTitle{春秋}者,禮義之大宗也。夫禮禁未然之前,法施已然之後;法之所爲用者易見,而禮之所爲禁者難知。」

\ProperName{壺遂}曰:「\ProperName{孔子}之時,上無明君,下不得任用,故作\BookTitle{春秋},垂空文以斷禮義,當一王之法。今夫子上遇明天子,下得守職,萬事既具,咸各序其宜,夫子所論,欲以何明?」

\ProperName{太史公}曰:「唯唯,否否,不然。余聞之先人曰:『\ProperName{伏羲}至純厚,作\BookTitle{易}\BookTitle{八卦}。\ProperName{堯}\ProperName{舜}之盛,\BookTitle{尚書}載之,禮樂作焉。\ProperName{湯}\ProperName{武}之隆,詩人歌之。\BookTitle{春秋}采善貶惡,推\ProperName{三代}之德,襃\ProperName{周室},非獨刺譏而已也。』\ProperName{漢}興以來,至明天子,獲符瑞,建封禪,改正朔,易服色,受命於穆清,澤流罔極,海外殊俗,重譯款塞,請來獻見者,不可勝道。臣下百官力誦聖德,猶不能宣盡其意。且士賢能而不用,有國者之恥;主上明聖而德不布聞,有司之過也。且余嘗掌其官,廢明聖盛德不載,滅功臣世家賢大夫之業不述,墮先人所言,罪莫大焉。余所謂述故事,整齊其世傳,非所謂作也,而君比之於\BookTitle{春秋},謬矣。」

於是論次其文。七年而\ProperName{太史公}遭\ProperName{李陵}之禍,幽於縲紲。乃喟然而歎曰:「是余之罪也夫!是余之罪也夫!身毀不用矣。」退而深惟曰:「夫\BookTitle{詩}\BookTitle{書}隱約者,欲遂其志之思也。昔\ProperName{西伯}拘\ProperName{羑里},演\BookTitle{周易};\ProperName{孔子}戹\ProperName{陳}\ProperName{蔡},作\BookTitle{春秋};\ProperName{屈原}放逐,著\BookTitle{離騷};\ProperName{左丘}失明,厥有\BookTitle{國語};\ProperName{孫子}臏腳,而論兵法;\ProperName{不韋}遷\ProperName{蜀},世傳\BookTitle{呂覽};\ProperName{韓非}囚\BookTitle{秦},\BookTitle{說難}、\BookTitle{孤憤};\BookTitle{詩}三百篇,大抵賢聖發憤之所爲作也。此人皆意有所鬱結,不得通其道也,故述往事,思來者。」於是卒述\ProperName{陶唐}以來,至于麟止,自\ProperName{黃帝}始。

\theendnotes

\section[報任安書\quad{\small 司馬遷}]{{\normalsize 司馬遷}\quad 報\ProperName{任安}書}
\ProperName{太史公}牛馬走\ProperName{司馬遷}再拜言,\ProperName{少卿}足下:曩者辱賜書,教以慎於接物,推賢進士爲務。意氣懃懃懇懇,若望僕不相師,而用流俗人之言。僕非敢如此也。僕雖罷駑,亦嘗側聞長者之遺風矣。顧自以爲身殘處穢,動而見尤,欲益反損,是以抑鬱而無誰語\endnote{據\BookTitle{漢書}改,\BookTitle{觀止}作「是以獨抑鬱而誰與語」,\BookTitle{文選}作「是以獨鬱悒而與誰語」。\ProperName{梁\allowbreak 章\allowbreak 鉅}\BookTitle{文選\allowbreak 旁證}:六臣本無「以」字,「與誰」作「誰與」,\BookTitle{漢書}作「是以悒鬱而無誰語」。}。諺曰:「誰爲爲之?孰令聽之?」蓋\ProperName{鍾子期}死,\ProperName{伯牙}終身不復鼓琴。何則?士爲知己者用,女爲說己者容。若僕大質已虧缺矣,雖才懷\ProperName{隨}\ProperName{和},行若\ProperName{由}\ProperName{夷},終不可以爲榮,適足以發笑而自點耳。

書辭宜答,會東從上來,又迫賤事,相見日淺,卒卒無須臾之間,得竭志意。今\ProperName{少卿}抱不測之罪,涉旬月,迫季冬;僕又薄從上雍,恐卒然不可爲諱。是僕終已不得舒憤懣以曉左右,則長逝者魂魄私恨無窮。請略陳固陋,闕然久不報,幸勿爲過。

僕聞之:修身者,智之符也;愛施者,仁之端也;取予者,義之表也;恥辱者,勇之決也;立名者,行之極也。士有此五者,然後可以託於世,而列於君子之林矣。故禍莫憯於欲利,悲莫痛於傷心,行莫醜於辱先,詬莫大於宮刑。刑餘之人,無所比數,非一世也,所從來遠矣。昔\ProperName{衞靈公}與\ProperName{雍渠}同載,\ProperName{孔子}適\ProperName{陳};\ProperName{商鞅}因\ProperName{景監}見,\ProperName{趙良}寒心;同子參乘,\ProperName{袁絲}變色。自古而恥之。夫以中材之人,事有關於宦豎,莫不傷氣,而況於忼慨之士乎!如今朝廷雖乏人,柰何令刀鋸之餘,薦天下之豪俊哉!

僕賴先人緒業,得待罪輦轂下,二十餘年矣。所以自惟,上之,不能納忠效信,有奇策材力之譽,自結明主;次之,又不能拾遺補闕,招賢進能,顯巖穴之士;外之,不能備行伍,攻城野戰,有斬將搴旗之功;下之,不能積日累勞,取尊官厚祿,以爲宗族交遊光寵。四者無一遂,苟合取容,無所短長之效,可見於此矣。嚮者,僕亦嘗廁下大夫之列,陪奉外廷末議。不以此時引綱維,盡思慮。今已虧形爲掃除之隸,在闒茸之中,乃欲仰首伸眉,論列是非,不亦輕朝廷,羞當世之士邪!嗟乎!嗟乎!如僕,尚何言哉!尚何言哉!

且事本末未易明也。僕少負不羈之才,長無鄉曲之譽,主上幸以先人之故,使得奏薄伎,出入周衞之中。僕以爲戴盆何以望天,故絶賓客之知,亡室家之業,日夜思竭其不肖之才力,務一心營職,以求親媚於主上。而事乃有大謬不然者。

夫僕與\ProperName{李陵}俱居門下,素非能相善也。趨舍異路,未嘗銜盃酒,接殷勤之餘歡。然僕觀其爲人,自守奇士,事親孝,與士信,臨財廉,取與義,分別有讓,恭儉下人,常思奮不顧身以徇\endnote{\BookTitle{觀止}作「殉」,據\BookTitle{漢書}、\BookTitle{文選}改。}國家之急。其素所蓄積也,僕以爲有國士之風。夫人臣出萬死不顧一生之計,赴公家之難,斯已奇矣。今舉事一不當,而全軀保妻子之臣隨而媒蘖其短,僕誠私心痛之。且\ProperName{李陵}提步卒不滿五千,深踐戎馬之地,足歷王庭,垂餌虎口,橫挑彊\ProperName{胡},仰億萬之師,與單于連戰十有餘日,所殺過當。虜救死扶傷不給,旃裘之君長咸震怖,乃悉徵其左右賢王,舉引弓之民,一國共攻而圍之。轉鬭千里,矢盡道窮,救兵不至,士卒死傷如積。然\ProperName{陵}一呼勞軍,士無不起,躬自流涕,沬血飲泣,更張空弮\endnote{\BookTitle{漢書}、\BookTitle{文選}六臣本無「更」字,\BookTitle{文選}「弮」作「拳」。},冒白刃,北嚮爭死敵者。\ProperName{陵}未沒時,使有來報,\ProperName{漢}公卿王侯皆奉觴上壽。後數日,\ProperName{陵}敗書聞,主上爲之食不甘味,聽朝不怡。大臣憂懼,不知所出。僕竊不自料其卑賤,見主上慘愴怛悼,誠欲效其款款之愚,以爲\ProperName{李陵}素與士大夫絶甘分少,能得人之死力,雖古之名將,不能過也。身雖陷敗,彼觀其意,且欲得其當而報於\ProperName{漢}。事已無可奈何,其所摧敗,功亦足以暴於天下矣。僕懷欲陳之,而未有路,適會召問,即以此指推言\ProperName{陵}之功,欲以廣主上之意,塞睚眦之辭。未能盡明,明主不曉,以爲僕沮貳師,而爲\ProperName{李陵}遊說,遂下於理。拳拳之忠,終不能自列。因爲誣上,卒從吏議。家貧,貨賂不足以自贖,交游莫救;左右親近,不爲一言。身非木石,獨與法吏爲伍,深幽囹圄之中,誰可告愬者!此真\ProperName{少卿}所親見,僕行事豈不然乎?\ProperName{李陵}既生降,頹其家聲;而僕又佴之蠶室,重爲天下觀笑。悲夫!悲夫!事未易一二爲俗人言也。

僕之先人,非有剖符丹書之功,文史星曆\endnote{\BookTitle{觀止}訛作「歷」,據\BookTitle{漢書}、\BookTitle{文選}改。},近乎卜祝之間,固主上所戲弄,倡優所畜,流俗之所輕也。假令僕伏法受誅,若九牛亡一毛,與螻蟻何以異?而世俗又不能與死節者次比,特以爲智窮罪極,不能自免,卒就死耳。何也?素所自樹立使然也。人固有一死,死有重於\ProperName{泰山},或輕於鴻毛,用之所趣異也。太上不辱先,其次不辱身,其次不辱理色,其次不辱辭令,其次詘體受辱,其次易服受辱,其次關木索被箠楚受辱,其次剔毛髮嬰金鐵受辱,其次毀肌膚斷肢體受辱,最下腐刑,極矣。傳曰:「刑不上大夫。」此言士節不可不勉厲也。猛虎在深山,百獸震恐,及在檻穽之中,搖尾而求食,積威約之漸也。故士有畫地爲牢勢不可入,削木爲吏議不可對,定計於鮮也。今交手足,受木索,暴肌膚,受榜箠,幽於圜牆之中。當此之時,見獄吏則頭槍\endnote{\BookTitle{觀止}作「搶」,據\BookTitle{漢書}、\BookTitle{文選}改。}地,視徒隸則心惕息,何者?積威約之勢也。及以至是言不辱者,所謂彊顏耳,曷足貴乎!且\ProperName{西伯},伯也,拘於\ProperName{羑里};\ProperName{李斯},相也,具于五刑;\ProperName{淮陰},\ProperName{王}也,受械於\ProperName{陳};\ProperName{彭越}、\ProperName{張敖},南面稱孤,繫獄抵罪;\ProperName{絳侯}誅諸\ProperName{呂},權傾五伯,囚於請室;\ProperName{魏其},大將也,衣赭衣,關三木;\ProperName{季布}爲\ProperName{朱家}鉗奴;\ProperName{灌夫}受辱於居室。此人皆身至王侯將相,聲聞鄰國,及罪至罔加,不能引決自裁,在塵埃之中,古今一體,安在其不辱也?由此言之,勇怯,勢也;彊弱,形也。審矣!何足怪乎?夫人不能蚤自裁繩墨之外,以稍陵遲至於鞭箠之間,乃欲引節,斯不亦遠乎!古人所以重施刑於大夫者,殆爲此也。

夫人情莫不貪生惡死,念父母,顧妻子,至激於義理者不然,乃有所不得已也。今僕不幸,早失父母,無兄弟之親,獨身孤立,\ProperName{少卿}視僕於妻子何如哉?且勇者不必死節,怯夫慕義,何處不勉焉!僕雖怯懦欲苟活,亦頗識去就之分矣,何至自沈溺縲紲之辱哉?且夫臧獲婢妾,猶能引決,況僕之不得已乎?所以隱忍苟活,幽於糞土之中而不辭者,恨私心有所不盡,鄙陋沒世,而文彩不表於後世也。

古者富貴而名摩滅,不可勝記,唯倜儻非常之人稱焉。蓋\ProperName{文王}拘而演\BookTitle{周易};\ProperName{仲尼}戹而作\BookTitle{春秋};\ProperName{屈原}放逐,乃賦\BookTitle{離騷};\ProperName{左丘}失明,厥有\BookTitle{國語};\ProperName{孫子}臏腳,兵法修列;\ProperName{不韋}遷\ProperName{蜀},世傳\BookTitle{呂覽};\ProperName{韓非}囚\ProperName{秦},\BookTitle{說難}、\BookTitle{孤憤}。\BookTitle{詩}三百篇,大厎\endnote{\BookTitle{觀止}作「底」,據\BookTitle{文選}改。漢書作「氐」。}聖賢發憤之所爲作也。此人皆意有所鬱結,不得通其道,故述往事,思來者。乃如\ProperName{左丘}無目,\ProperName{孫子}斷足,終不可用,退而論書策,以舒其憤,思垂空文以自見。

僕竊不遜,近自託於無能之辭,網羅天下放失舊聞,略考其事,綜其終始,稽其成敗興壞之紀,上計\ProperName{軒轅},下至于茲,爲十\BookTitle{表},\BookTitle{本紀}十二,\BookTitle{書}八章,\BookTitle{世家}三十,\BookTitle{列傳}七十,凡百三十篇,亦欲以究天人之際,通古今之變,成一家之言。草創未就,會遭此禍,惜其不成,是以就極刑而無慍色。僕誠以著此書藏諸名山,傳之其人,通邑大都,則僕償前辱之責,雖萬被戮,豈有悔哉?然此可爲智者道,難爲俗人言也。

且負下未易居,下流多謗議,僕以口語遇遭此禍,重爲鄉黨所戮笑,以汙辱先人,亦何面目復上父母丘墓乎?雖累百世,垢彌甚耳!是以腸一日而九迴,居則忽忽若有所亡,出則不知所往。每念斯恥,汗未嘗不發背霑衣也。身直爲閨閤之臣,寧得自引深臧巖穴邪?故且從俗浮沈,與時俯仰,以通其狂惑。今\ProperName{少卿}乃教以推賢進士,無乃與僕私心剌謬乎!今雖欲自雕琢,曼辭以自飾,無益於俗不信,適足取辱耳。要之死日,然後是非乃定。書不能悉意,略陳固陋,謹再拜。 

\theendnotes

% Proofed 13 July 2022
% Ref.
% - 點校本二十四史·史記, 中華書局, 1959
% - 點校本二十四史修訂本·史記, 中華書局, 2014
% - 文選, 上海古籍, 1986
% - 六臣注文選, 四部叢刊
% - 古文觀止, 中華書局, 1959