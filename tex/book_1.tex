\section[鄭伯克段於鄢\quad{\small 左傳\ 隱公元年}]{{\normalsize 左傳\ 隱公元年}\quad \ProperName{鄭伯}克\ProperName{段}於\ProperName{鄢}}
初,\ProperName{鄭武公}娶于\ProperName{申},曰\ProperName{武姜},生\ProperName{莊公}及\ProperName{共叔段}。\ProperName{莊公}寤生,驚\ProperName{姜氏},故名曰\ProperName{寤生},遂惡之。愛\ProperName{共叔段},欲立之。亟請於\ProperName{武公},公弗許。

及\ProperName{莊公}即位,爲之請制。公曰:「\ProperName{制},巖邑也。\ProperName{虢叔}死焉,它邑唯命。」請\ProperName{京},使居之,謂之\ProperName{京城大叔}。

\ProperName{祭仲}曰:「都城過百雉,國之害也。先王之制,大都,不過參國之一;中,五之一;小,九之一。今\ProperName{京}不度,非制也,君將不堪。」公曰:「\ProperName{姜氏}欲之,焉辟害?」對曰:「\ProperName{姜氏}何厭之有?不如早爲之所,無使滋蔓。蔓,難圖也。蔓草猶不可除,況君之寵弟乎?」公曰:「多行不義必自斃,子姑待之。」

既而\ProperName{大叔}命\ProperName{西鄙}、\ProperName{北鄙}貳於己。\ProperName{公子呂}曰:「國不堪貳。君將若之何?欲與\ProperName{大叔},臣請事之。若弗與,則請除之,無生民心。」公曰:「無庸,將自及。」\ProperName{大叔}又收貳以爲己邑,至于\ProperName{廩延}。\ProperName{子封}曰:「可矣!厚將得眾。」公曰:「不義不暱,厚將崩。」

\ProperName{大叔}完聚,繕甲兵,具卒乘,將襲\ProperName{鄭};夫人將啟之。公聞其期曰:「可矣。」命\ProperName{子封}帥車二百乘以伐\ProperName{京},\ProperName{京}叛\ProperName{大叔段}。段入于\ProperName{鄢},公伐諸\ProperName{鄢}。五月辛丑,\ProperName{大叔}出奔\ProperName{共}。

\BookTitle{書}曰:「\ProperName{鄭伯}克\ProperName{段}于\ProperName{鄢}。」\ProperName{段}不弟,故不言弟。如二君,故曰克。稱\ProperName{鄭伯},譏失教也,謂之\ProperName{鄭}志。不言出奔,難之也。

遂寘\ProperName{姜氏}于\ProperName{城潁},而誓之曰:「不及黃泉,無相見也。」既而悔之。

\ProperName{潁考叔}爲\ProperName{潁谷}封人,聞之。有獻於公,公賜之食,食舍肉,公問之。對曰:「小人有母,皆嘗小人之食矣。未嘗君之羹,請以遺之。」公曰:「爾有母遺,繄我獨無。」\ProperName{潁考叔}曰:「敢問何謂也?」公語之故,且告之悔。對曰:「君何患焉?若闕地及泉,隧而相見,其誰曰不然?」公從之。

公入而賦:「大隧之中,其樂也融融。」\ProperName{姜}出而賦:「大隧之外,其樂也洩洩。」遂爲母子如初。

君子曰:「\ProperName{潁考叔},純孝也,愛其母,施及\ProperName{莊公}。\BookTitle{詩}曰:『孝子不匱,永錫爾類。』其是之謂乎!」

\section[周鄭交質\quad{\small 左傳\ 隱公三年}]{{\normalsize 左傳\ 隱公三年}\quad \ProperName{周}\ProperName{鄭}交質}
\ProperName{鄭武公}、\ProperName{莊公}爲\ProperName{平王}卿士,王貳于\ProperName{虢},\ProperName{鄭伯}怨王。王曰:「無之。」故\ProperName{周}\ProperName{鄭}交質。\ProperName{王子狐}爲質於\ProperName{鄭},\ProperName{鄭}\ProperName{公子忽}爲質於\ProperName{周}。

王崩,\ProperName{周}人將畀\ProperName{虢公}政。四月,\ProperName{鄭}\ProperName{祭足}帥師取\ProperName{溫}之麥;秋,又取\ProperName{成周}之禾。\ProperName{周}\ProperName{鄭}交惡。

君子曰:「信不由中,質無益也。明恕而行,要之以禮,雖無有質,誰能間之?苟有明信,澗溪沼沚之毛,蘋蘩薀藻之菜,筐筥錡釜之器,潢汙行潦之水,可薦於鬼神,可羞於王公。而況君子結二國之信,行之以禮,又焉用質?\BookTitle{風}有\BookTitle{采蘩}、\BookTitle{采蘋},\BookTitle{雅}有\BookTitle{行葦}、\BookTitle{泂酌},昭忠信也。」

\section[石碏諫寵州吁\quad{\small 左傳\ 隱公三年}]{{\normalsize 左傳\ 隱公三年}\quad \ProperName{石碏}諫寵\ProperName{州吁}}
\ProperName{衛莊公}娶于\ProperName{齊}東宮\ProperName{得臣}之妹,曰\ProperName{莊姜}。美而無子,\ProperName{衛}人所爲賦\BookTitle{碩人}也。又娶于\ProperName{陳},曰\ProperName{厲媯},生\ProperName{孝伯},早死。其娣\ProperName{戴媯},生\ProperName{桓公},\ProperName{莊姜}以爲己子。

\ProperName{公子州吁},嬖人之子也,有寵而好兵,公弗禁,\ProperName{莊姜}惡之。

\ProperName{石碏}諫曰;「臣聞愛子,教之以義方,弗納於邪。驕奢淫佚,所自邪也。四者之來,寵祿過也。將立\ProperName{州吁},乃定之矣;若猶未也,階之爲禍。夫寵而不驕,驕而能降,降而不憾,憾而能眕者,鮮矣。且夫賤妨貴,少陵長,遠間親,新間舊,小加大,淫破義,所謂六逆也。君義,臣行,父慈,子孝,兄愛,弟敬,所謂六順也。去順效逆,所以速禍也。君人者,將禍是務去,而速之,無乃不可乎。」弗聽。

其子\ProperName{厚}與\ProperName{州吁}遊,禁之,不可。\ProperName{桓公}立,乃老。

\section[臧僖伯諫觀魚\quad{\small 左傳\ 隱公五年}]{{\normalsize 左傳\ 隱公五年}\quad \ProperName{臧僖伯}諫觀魚}
春,公將如\ProperName{棠}觀魚者。\ProperName{臧僖伯}諫曰:「凡物不足以講大事,其材不足以備器用,則君不舉焉。君將納民於軌、物者也,故講事以度軌量謂之軌,取材以章物采謂之物。不軌不物,謂之亂政。亂政亟行,所以敗也。故春蒐、夏苗、秋獮、冬狩,皆於農隙以講事也。三年而治兵,入而振旅,歸而飲至,以數軍實、昭文章、明貴賤、辨等列、順少長、習威儀也。鳥獸之肉,不登於俎;皮革齒牙、骨角毛羽,不登於器,則{君}\endnote{今校本\BookTitle{左傳}作「公」。\ProperName{洪亮吉}\BookTitle{春秋左傳詁}:「公」當作「君」。\ProperName{阮元}\BookTitle{校勘記}:\ProperName{何焯}校本「公」改「君」,非。}不射,古之制也。若夫山林川澤之實,器用之資,皁隸之事,官司之守,非君所及也。」

公曰:「吾將略地焉。」遂往。陳魚而觀之,\ProperName{僖伯}稱疾不從。

\BookTitle{書}曰:「公矢魚于\ProperName{棠}。」非禮也,且言遠地也。

\theendnotes

\section[鄭莊公戒飭守臣\quad{\small 左傳\ 隱公十一年}]{{\normalsize 左傳\ 隱公十一年}\quad \ProperName{鄭莊公}戒飭守臣}
秋,七月,公會\ProperName{齊侯}、\ProperName{鄭伯}伐\ProperName{許}。庚辰,傅于\ProperName{許}。\ProperName{潁考叔}取\ProperName{鄭伯}之旗蝥弧以先登,\ProperName{子都}自下射之,顛。\ProperName{瑕叔盈}又以蝥弧登,周麾而呼曰:「君登矣!」\ProperName{鄭}師畢登。壬午,遂入\ProperName{許}。\ProperName{許莊公}奔\ProperName{衛}。\ProperName{齊侯}以\ProperName{許}讓公。公曰:「君謂\ProperName{許}不共,故從君討之。\ProperName{許}既伏其罪矣,雖君有命,寡人弗敢與聞。」乃與\ProperName{鄭}人。

\ProperName{鄭伯}使\ProperName{許}大夫\ProperName{百里}奉\ProperName{許叔}以居\ProperName{許}東偏。曰:「天禍\ProperName{許國},鬼神實不逞于\ProperName{許君},而假手于我寡人。寡人唯是一二父兄不能共億,其敢以\ProperName{許}自爲功乎?寡人有弟,不能和協,而使餬其口於四方,其況能久有\ProperName{許}乎?吾子其奉\ProperName{許叔}以撫柔此民也,吾將使\ProperName{獲}也佐吾子。若寡人得沒于地,天其以禮悔禍于\ProperName{許},無寧茲\ProperName{許公}復奉其社稷。唯我\ProperName{鄭國}之有請謁焉,如舊昏媾,其能降以相從也。無滋他族實偪處此,以與我\ProperName{鄭國}爭此土也。吾子孫其覆亡之不暇,而況能禋祀\ProperName{許}乎?寡人之使吾子處此,不唯\ProperName{許國}之爲,亦聊以固吾圉也。」

乃使\ProperName{公孫獲}處\ProperName{許}西偏,曰:「凡而器用財賄,無寘於許。我死,乃亟去之。吾先君新邑於此,王室而既卑矣,\ProperName{周}之子孫日失其序。夫\ProperName{許},\ProperName{大岳}之胤也。天而既厭\ProperName{周}德矣,吾其能與\ProperName{許}爭乎?」

君子謂\ProperName{鄭莊公}於是乎有禮。禮,經國家,定社稷,序民人\endnote{\BookTitle{觀止}作「人民」,倒,據今校本改。},利後嗣者也。\ProperName{許},無刑而伐之,服而舍之,度德而處之,量力而行之,相時而動,無累後人,可謂知禮矣。

\theendnotes

\section[臧哀伯諫納郜鼎\quad{\small 左傳\ 桓公二年}]{{\normalsize 左傳\ 桓公二年}\quad \ProperName{臧哀伯}諫納\ProperName{郜}鼎}
夏,四月,取\ProperName{郜}大鼎于\ProperName{宋}。納\endnote{「納」上原有「戊申」,原文月日\BookTitle{觀止}多省文。}于大廟。非禮也。

\ProperName{臧哀伯}諫曰:「君人者,將昭德塞違,以臨照百官,猶懼或失之,故昭令德以示子孫。

是以清廟茅屋,大路越席,大羹不致,粢食不鑿,昭其儉也。袞冕黻珽,帶裳幅舄,衡紞紘綖,昭其度也。藻率鞞鞛,鞶厲游纓,昭其數也。火龍黼黻,昭其文也。五色比象,昭其物也。鍚鸞和鈴,昭其聲也。三辰旂旗,昭其明也。夫德,儉而有度,登降有數,文物以紀之,聲明以發之,以臨照百官。百官於是乎戒懼,而不敢易紀律。今滅德立違,而寘其賂器於大廟,以明示百官。百官象之,其又何誅焉?國家之敗由官邪也。官之失德,寵賂章也。\ProperName{郜}鼎在廟,章孰甚焉?\ProperName{武王}克\ProperName{商},遷九鼎於\ProperName{雒邑},義士猶或非之,而況將昭違亂之賂器於大廟,其若之何?」公不聽。

\ProperName{周}內史聞之曰:「\ProperName{臧孫達}其有後於\ProperName{魯}乎!君違,不忘諫之以德。」

\theendnotes

\section[季梁諫追楚師\quad{\small 左傳\ 桓公六年}]{{\normalsize 左傳\ 桓公六年}\quad \ProperName{季梁}諫追\ProperName{楚}師}
\ProperName{楚武王}侵\ProperName{隨},使\ProperName{薳章}求成焉,軍於\ProperName{瑕}以待之。\ProperName{隨}人使少師董成。

\ProperName{鬬伯比}言于\ProperName{楚子}曰:「吾不得志於\ProperName{漢}東也,我則使然。我張吾三軍,而被吾甲兵,以武臨之,彼則懼而協以謀我,故難間也。\ProperName{漢}東之國,\ProperName{隨}爲大。\ProperName{隨}張,必棄小國。小國離,\ProperName{楚}之利也。少師侈,請羸師以張之。」\ProperName{熊率且比}曰:「\ProperName{季梁}在,何益?」\ProperName{鬬伯比}曰:「以爲後圖,少師得其君。」王毀軍而納少師。

少師歸,請追\ProperName{楚}師。\ProperName{隨侯}將許之。\ProperName{季梁}止之,曰:「天方授\ProperName{楚},\ProperName{楚}之羸,其誘我也!君何急焉?臣聞小之能敵大也,小道大淫。所謂道,忠於民而信於神也。上思利民,忠也;祝史正辭,信也。今民餒而君逞欲,祝史矯舉以祭,臣不知其可也。」

公曰:「吾牲牷肥腯,粢盛豐備,何則不信?」對曰:「夫民,神之主也。是以聖王先成民,而後致力於神。故奉牲以告曰『博碩肥腯』,謂民力之普存也,謂其畜之碩大蕃滋也,謂其不疾瘯蠡也,謂其備腯咸有也。奉盛以告曰『絜\endnote{\BookTitle{觀止}作「潔」。\ProperName{阮元}\BookTitle{校勘記}:\BookTitle{後漢書}\BookTitle{列女傳}注引傳文「絜」作「潔」。}粢豐盛』,謂其三時不害而民和年豐也。奉酒醴以告曰『嘉栗旨酒』,謂其上下皆有嘉德而無違心也。所謂馨香,無讒慝也。故務其三時,脩其五教,親其九族,以致其禋祀,於是乎民和而神降之福,故動則有成。今民各有心,而鬼神乏主,君雖獨豐,其何福之有?君姑脩政而親兄弟之國,庶免於難。」

\ProperName{隨侯}懼而脩政,\ProperName{楚}不敢伐。

\theendnotes

\section[曹劌論戰\quad{\small 左傳\ 莊公十年}]{{\normalsize 左傳\ 莊公十年}\quad \ProperName{曹劌}論戰}
春,\ProperName{齊}師伐我,公將戰。\ProperName{曹劌}請見。其鄉人曰:「肉食者謀之,又何間焉?」\ProperName{劌}曰:「肉食者鄙,未能遠謀。」乃\endnote{\BookTitle{觀止}作「遂」,據\BookTitle{左傳}改。}入見。

問何以戰?公曰:「衣食所安,弗敢專也,必以分人。」對曰:「小惠未徧,民弗從也。」公曰:「犧牲玉帛,弗敢加也,必以信。」對曰:「小信未孚,神弗福也。」公曰:「小大之獄,雖不能察,必以情。」對曰:「忠之屬也,可以一戰。戰,則請從。」

公與之乘,戰於\ProperName{長勺}。公將鼓之。\ProperName{劌}曰:「未可。」\ProperName{齊}人三鼓,\ProperName{劌}曰:「可矣!」\ProperName{齊}師敗績。公將馳之,\ProperName{劌}曰:「未可。」下視其轍,登軾而望之,曰:「可矣!」遂逐\ProperName{齊}師。

既克,公問其故,對曰:「夫戰,勇氣也。一鼓作氣,再而衰,三而竭。彼竭我盈,故克之。夫大國,難測也,懼有伏焉。吾視其轍亂,望其旗靡,故逐之。」

\theendnotes

\section[齊桓公伐楚盟屈完\quad{\small 左傳\ 僖公四年}]{{\normalsize 左傳\ 僖公四年}\quad \ProperName{齊桓公}伐\ProperName{楚}盟\ProperName{屈完}}
春,\ProperName{齊侯}以諸侯之師侵\ProperName{蔡},\ProperName{蔡}潰,遂伐\ProperName{楚}。

\ProperName{楚子}使與師言曰:「君處北海,寡人處南海,唯是風馬牛不相及也。不虞君之涉吾地也,何故?」\ProperName{管仲}對曰:「昔\ProperName{召康公}命我先君\ProperName{大公}曰:『五侯九伯,女實征之,以夾輔\ProperName{周}室。』賜我先君履,東至于海,西至于\ProperName{河},南至于\ProperName{穆陵},北至于\ProperName{無棣}。爾貢包茅不入,王祭不供,無以縮酒,寡人是徵。\ProperName{昭王}南征而不復,寡人是問。」對曰:「貢之不入,寡君之罪也,敢不供給?\ProperName{昭王}之不復,君其問諸水濱!」

師進,次於\ProperName{陘}。夏,\ProperName{楚子}使\ProperName{屈完}如師。師退,次於\ProperName{召陵}。

\ProperName{齊侯}陳諸侯之師,與\ProperName{屈完}乘而觀之。\ProperName{齊侯}曰:「豈不榖是爲?先君之好是繼,與不榖同好如何?」對曰:「君惠徼福於敝邑之社稷,辱收寡君,寡君之願也。」\ProperName{齊侯}曰:「以此眾戰,誰能禦之?以此攻城,何城不克?」對曰:「君若以德綏諸侯,誰敢不服?君若以力,\ProperName{楚國}\ProperName{方城}以爲城,\ProperName{漢水}以爲池,雖眾,無所用之。」\ProperName{屈完}及諸侯盟。

\section[宮之奇諫假道\quad{\small 左傳\ 僖公五年}]{{\normalsize 左傳\ 僖公五年}\quad \ProperName{宮之奇}諫假道}
\ProperName{晉侯}復假道於\ProperName{虞}以伐\ProperName{虢},\ProperName{宮之奇}諫曰:「\ProperName{虢},\ProperName{虞}之表也;\ProperName{虢}亡,\ProperName{虞}必從之。\ProperName{晉}不可啓,寇不可翫,一之謂\endnote{\BookTitle{觀止}作「爲」,據\BookTitle{左傳}校本改。\ProperName{阮元}\BookTitle{校勘記}:{纂圖}本、\ProperName{閩}本、{監本}、\BookTitle{毛}本「謂」作「爲」,誤。}甚,其可再乎?諺所謂『輔車相依,脣亡齒寒』者,其\ProperName{虞}\ProperName{虢}之謂也。」

公曰:「\ProperName{晉},吾宗也,豈害我哉?」對曰:「\ProperName{大伯}、\ProperName{虞仲},\ProperName{大王}之昭也。\ProperName{大伯}不從,是以不嗣。\ProperName{虢仲}、\ProperName{虢叔},\ProperName{王季}之穆也。爲\ProperName{文王}卿士,勳在王室,藏於盟府。將\ProperName{虢}是滅,何愛於\ProperName{虞}?且\ProperName{虞}能親於\ProperName{桓}、\ProperName{莊}乎?其愛之也,\ProperName{桓}、\ProperName{莊}之族何罪?而以爲戮,不唯偪乎?親以寵偪,猶尚害之,況以國乎?」

公曰:「吾享祀豐{絜},神必據我。」對曰:「臣聞之,鬼神非人實親,惟德是依。故\BookTitle{周書}曰:『皇天無親,惟德是輔。』又曰:『黍稷非馨,明德惟馨。』又曰:『民不易物,惟德繄物。』如是,則非德民不和、神不享矣。神所馮依,將在德矣。若\ProperName{晉}取\ProperName{虞},而明德以薦馨香,神其吐之乎?」

弗聽,許\ProperName{晉}使。\ProperName{宮之奇}以其族行,曰:「\ProperName{虞}不臘矣!在此行也,\ProperName{晉}不更舉矣。」冬,\ProperName{晉}滅\ProperName{虢}。師還,館於\ProperName{虞},遂襲\ProperName{虞},滅之,執\ProperName{虞公}。

\theendnotes

\section[齊桓下拜受胙\quad{\small 左傳\ 僖公九年}]{{\normalsize 左傳\ 僖公九年}\quad \ProperName{齊桓}下拜受胙}
會于\ProperName{葵丘},尋盟,且脩好,禮也。

王使\ProperName{宰孔}賜\ProperName{齊侯}胙,曰:「天子有事于\ProperName{文}、\ProperName{武},使\ProperName{孔}賜伯舅胙。」\ProperName{齊侯}將下拜。\ProperName{孔}曰:「且有後命。天子使\ProperName{孔}曰:『以伯舅耋老,加勞,賜一級,無下拜。』」對曰:「天威不違顏咫尺,\ProperName{小白},余敢貪天子之命,無下拜?恐隕越于下,以遺天子羞,敢不下拜?」

\section[陰飴甥對秦伯\quad{\small 左傳\ 僖公十五年}]{{\normalsize 左傳\ 僖公十五年}\quad \ProperName{陰飴甥}對\ProperName{秦伯}}
十月,\ProperName{晉}\ProperName{陰飴甥}會\ProperName{秦伯},盟于\ProperName{王城}。

\ProperName{秦伯}曰:「\ProperName{晉國}和乎?」對曰:「不和。小人恥失其君而悼喪其親,不憚征繕以立\ProperName{圉}也,曰:『必報讎,寧事戎狄。』君子愛其君而知其罪,不憚征繕以待\ProperName{秦}命,曰:『必報德,有死無二。』以此不和。」

\ProperName{秦伯}曰:「國謂君何?」對曰:「小人慼,謂之不免;君子恕,以爲必歸。小人曰:『我毒\ProperName{秦},\ProperName{秦}豈歸君?』君子曰:『我知罪矣,\ProperName{秦}必歸君。貳而執之,服而舍之,德莫厚焉,刑莫威焉。服者懷德,貳者畏刑。此一役也,\ProperName{秦}可以霸。納而不定,廢而不立,以德爲怨,\ProperName{秦}不其然。』」

\ProperName{秦伯}曰:「是吾心也。」改館\ProperName{晉侯},饋七牢焉。

\section[子魚論戰\quad{\small 左傳\ 僖公二十二年}]{{\normalsize 左傳\ 僖公二十二年}\quad \ProperName{子魚}論戰}
\ProperName{楚}人伐\ProperName{宋}以救\ProperName{鄭},\ProperName{宋公}將戰,大司馬固諫曰:「天之棄\ProperName{商}久矣!君將興之,弗可赦也已。」弗聽。

及\endnote{「及楚人」上原有「冬十一月己巳朔\ProperName{宋公}」。}\ProperName{楚}人戰于\ProperName{泓}。\ProperName{宋}人既成列,\ProperName{楚}人未既濟。司馬曰:「彼眾我寡,及其未既濟也,請擊之。」公曰:「不可。」既濟而未成列,又以告。公曰:「未可。」既陳而後擊之,\ProperName{宋}師敗績。公傷股,門官殲焉。

國人皆咎公。公曰:「君子不重傷,不禽二毛。古之爲軍也,不以阻隘也。寡人雖亡國之餘,不鼓不成列。」\ProperName{子魚}曰:「君未知戰。勍敵之人,隘而不列,天贊我也。阻而鼓之,不亦可乎?猶有懼焉。且今之勍者,皆吾敵也。雖及胡耇,獲則取之,何有於二毛?明恥教戰,求殺敵也。傷未及死,如何勿重?若愛重傷,則如勿傷;愛其二毛,則如服焉。三軍以利用也,金鼓以聲氣也。利而用之,阻隘可也;聲盛致志,鼓儳可也。」

\theendnotes

\section[寺人披見文公\quad{\small 左傳 僖公二十四年}]{{\normalsize 左傳 僖公二十四年}\quad \ProperName{寺人披}見\ProperName{文公}}
\ProperName{呂}、\ProperName{郤}畏偪,將焚公宮而弒\ProperName{晉侯}。

\ProperName{寺人披}請見。公使讓之,且辭焉,曰:「\ProperName{蒲城}之役,君命一宿,女即至。其後余從\ProperName{狄君}以田\ProperName{渭濱},女爲\ProperName{惠公}來求殺余,命女三宿,女中宿至。雖有君命,何其速也?夫袪猶在,女其行乎!」

對曰:「臣謂君之入也,其知之矣。若猶未也,又將及難。君命無二,古之制也。除君之惡,唯力是視。\ProperName{蒲}人、\ProperName{狄}人,余何有焉?今君即位,其無\ProperName{蒲}、\ProperName{狄}乎?\ProperName{齊桓公}置射鉤而使\ProperName{管仲}相,君若易之,何辱命焉?行者甚眾,豈唯刑臣?」

公見之,以難告。三月,\ProperName{晉侯}潛會\ProperName{秦伯}于\ProperName{王城}。己丑晦,公宮火。\ProperName{瑕甥}、\ProperName{郤芮}不獲公。乃如\ProperName{河}上,\ProperName{秦伯}誘而殺之。

\section[介之推不言祿\quad{\small 左傳\ 僖公二十四年}]{{\normalsize 左傳\ 僖公二十四年}\quad \ProperName{介之推}不言祿}
\ProperName{晉侯}賞從亡者,\ProperName{介之推}不言祿,祿亦弗及。

\ProperName{推}曰:「\ProperName{獻公}之子九人,唯君在矣。\ProperName{惠}、\ProperName{懷}無親,外內棄之。天未絕\ProperName{晉},必將有主。主\ProperName{晉}祀者,非君而誰?天實置之,而二三子以爲己力,不亦誣乎?竊人之財,猶謂之盜,況貪天之功以爲己力乎?下義其罪,上賞其奸,上下相蒙,難與處矣。」

其母曰:「盍亦求之?以死,誰懟?」對曰:「尤而效之,罪又甚焉!且出怨言,不食其食。」其母曰:「亦使知之,若何?」對曰:「言,身之文也。身將隱,焉用文之?是求顯也。」其母曰:「能如是乎?與女偕隱。」遂隱而死。

\ProperName{晉侯}求之不獲,以\ProperName{緜上}爲之田。曰:「以志吾過,且旌善人。」

\section[展喜犒師\quad{\small 左傳\ 僖公二十六年}]{{\normalsize 左傳\ 僖公二十六年}\quad \ProperName{展喜}犒師}
\ProperName{齊孝公}伐我北鄙。\endnote{「齊孝公伐我北鄙」下原有「\ProperName{衛}人伐\ProperName{齊},\ProperName{洮}之盟故也。」}公使\ProperName{展喜}犒師,使受命於\ProperName{展禽}。

\ProperName{齊侯}未入竟,\ProperName{展喜}從之,曰:「寡君聞君親舉玉趾,將辱於敝邑,使下臣犒執事。」\ProperName{齊侯}曰:「\ProperName{魯}人恐乎?」對曰:「小人恐矣,君子則否。」\ProperName{齊侯}曰:「室如縣罄,野無青草,何恃而不恐?」對曰:「恃先王之命。昔\ProperName{周公}、\ProperName{大公}股肱\ProperName{周室},夾輔\ProperName{成王}。\ProperName{成王}勞之,而賜之盟,曰:『世世子孫無相害也。』載在盟府,大師職之。\ProperName{桓公}是以糾合諸侯,而謀其不協,彌縫其闕,而匡救其災,昭舊職也。及君即位,諸侯之望曰:『其率\ProperName{桓}之功。』我敝邑用不敢保聚,曰:『豈其嗣世九年,而棄命廢職?其若先君何?君必不然。』恃此以不恐。」\ProperName{齊侯}乃還。

\theendnotes

\section[燭之武退秦師\quad{\small 左傳 僖公三十年}]{{\normalsize 左傳 僖公三十年}\quad \ProperName{燭之武}退\ProperName{秦}師}
\ProperName{晉侯}、\ProperName{泰伯}圍\ProperName{鄭},以其無禮於\ProperName{晉},且貳於\ProperName{楚}也。\ProperName{晉}軍\ProperName{函陵},\ProperName{秦}軍\ProperName{氾}南。

\ProperName{佚之狐}言於\ProperName{鄭伯}曰:「國危矣!若使\ProperName{燭之武}見\ProperName{秦君},師必退。」公從之。辭曰:「臣之壯也,猶不如人;今老矣,無能爲也已。」公曰:「吾不能早用子,今急而求子,是寡人之過也。然\ProperName{鄭}亡,子亦有不利焉。」許之。

夜,縋而出。見\ProperName{秦伯}曰:「\ProperName{秦}、\ProperName{晉}圍\ProperName{鄭},\ProperName{鄭}既知亡矣。若亡\ProperName{鄭}而有益於君,敢以煩執事。越國以鄙遠,君知其難也。焉用亡\ProperName{鄭}以陪鄰?鄰之厚,君之薄也。若舍\ProperName{鄭}以爲東道主,行李之往來,共其乏困,君亦無所害。且君嘗爲\ProperName{晉君}賜矣,許君\ProperName{焦}、\ProperName{瑕},朝濟而夕設版焉,君之所知也。夫\ProperName{晉},何厭之有?既東封\ProperName{鄭},又欲肆其西封,若不闕\ProperName{秦},將焉取之?闕\ProperName{秦}以利\ProperName{晉},唯君圖之。」

\ProperName{秦伯}說,與\ProperName{鄭}人盟。使\ProperName{杞子}、\ProperName{逢孫}、\ProperName{楊孫}戍之,乃還。

\ProperName{子犯}請擊之,公曰:「不可,微夫人力不及此。因人之力而敝之,不仁;失其所與,不知;以亂易整,不武。吾其還也。」亦去之。

\section[蹇叔哭師\quad{\small 左傳\ 僖公三十二年}]{{\normalsize 左傳\ 僖公三十二年}\quad \ProperName{蹇叔}哭師}
\ProperName{杞子}自\ProperName{鄭}使告於\ProperName{秦}曰:「\ProperName{鄭}人使我掌其北門之管,若潛師以來,國可得也。」

\ProperName{穆公}訪諸\ProperName{蹇叔},\ProperName{蹇叔}曰:「勞師以襲遠,非所聞也。師勞力竭,遠主備之,無乃不可乎?師之所爲,\ProperName{鄭}必知之;勤而無所,必有悖心。且行千里,其誰不知?」

公辭焉。召\ProperName{孟明}、\ProperName{西乞}、\ProperName{白乙},使出師於東門之外。\ProperName{蹇叔}哭之曰:「\ProperName{孟子},吾見師之出,而不見其入也。」公使謂之曰:「爾何知?中壽,爾墓之木拱矣!」

\ProperName{蹇叔}之子與師,哭而送之,曰:「\ProperName{晉}人禦師必於\ProperName{殽},\ProperName{殽}有二陵焉。其南陵,\ProperName{夏后皋}之墓也;其北陵,\ProperName{文王}之所辟風雨也。必死是間,余收爾骨焉。」

\ProperName{秦}師遂東。

% Proofed 4 July 2022
% Ref.
% 楊伯峻, 春秋左傳註
% 十三经注疏整理本·春秋左傳正義
