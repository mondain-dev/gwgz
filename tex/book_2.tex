\section[鄭子家告趙宣子\quad{\small 左傳 文公十七年}]{{\normalsize 左傳\ 文公十七年}\quad \ProperName{鄭子家}告\ProperName{趙宣子}}
\ProperName{晉侯}復合諸侯于\ProperName{扈}\endnote{\BookTitle{左傳}原作「\ProperName{晉侯}蒐于\ProperName{黃父},遂復合諸侯于\ProperName{扈}」,\BookTitle{觀止}省文。},平\ProperName{宋}也\endnote{「平\ProperName{宋}也」下原有「公不與會,\ProperName{齊}難故也。書曰『諸侯』,無功也。」}。於是\ProperName{晉侯}不見\ProperName{鄭伯},以爲貳於\ProperName{楚}也。\ProperName{鄭子家}使執訊而與之書,以告\ProperName{趙宣子},曰:

\begin{quotation}寡君即位三年,召\ProperName{蔡侯}而與之事君。九月,\ProperName{蔡侯}入於敝邑以行。敝邑以\ProperName{侯宣多}之難,寡君是以不得與\ProperName{蔡侯}偕。十一月,克減\ProperName{侯宣多},而隨\ProperName{蔡侯}以朝於執事。十二年六月,\ProperName{歸生}佐寡君之嫡\ProperName{夷},以請\ProperName{陳侯}於\ProperName{楚},而朝諸君。十四年七月,寡君又朝以蕆\ProperName{陳}事。十五年五月,\ProperName{陳侯}自敝邑往朝於君。往年正月,\ProperName{燭之武}往,朝\ProperName{夷}也。八月,寡君又往朝。以\ProperName{陳}、\ProperName{蔡}之密邇於\ProperName{楚},而不敢貳焉,則敝邑之故也。雖敝邑之事君,何以不免?在位之中,一朝於\ProperName{襄},而再見於君。\ProperName{夷}與孤之二三臣相及於\ProperName{絳}。雖我小國,則蔑以過之矣。今大國曰:「爾未逞吾志。」敝邑有亡,無以加焉。
    
古人有言曰:「畏首畏尾,身其餘幾?」又曰:「鹿死不擇音。」小國之事大國也,德,則其人也;不德,則其鹿也。鋌而走險,急何能擇?命之罔極,亦知亡矣。將悉敝賦以待於\ProperName{鯈},唯執事命之。

\ProperName{文公}二年\endnote{「\ProperName{文公}二年」下原有「六月壬申」。},朝於\ProperName{齊}。四年\endnote{「四年」下原有「二月壬戌」。},爲\ProperName{齊}侵\ProperName{蔡},亦獲成於\ProperName{楚}。居大國之間,而從於強令,豈其罪也?大國若弗圖,無所逃命。
\end{quotation}

\ProperName{晉}\ProperName{鞏朔}行成於\ProperName{鄭},\ProperName{趙穿}、\ProperName{公壻池}爲質焉。

\theendnotes

\section[王孫滿對楚子\quad{\small 左傳\ 宣公三年}]{{\normalsize 左傳\ 宣公三年}\quad \ProperName{王孫滿}對\ProperName{楚子}}
\ProperName{楚子}伐\ProperName{陸渾之戎},遂至於\ProperName{雒},觀兵于\ProperName{周}疆。\ProperName{定王}使\ProperName{王孫滿}勞\ProperName{楚子}。\ProperName{楚子}問鼎之大小、輕重焉。對曰:「在德不在鼎。昔\ProperName{夏}之方有德也,遠方圖物,貢金九牧,鑄鼎象物,百物而爲之備,使民知神、姦。故民入川澤山林,不逢不若。螭魅罔兩,莫能逢之。用能協于上下,以承天休。\ProperName{桀}有昏德,鼎遷于\ProperName{商},載祀六百。\ProperName{商紂}暴虐,鼎遷于\ProperName{周}。德之休明,雖小,重也;其姦回昏亂,雖大,輕也。天祚明德,有所厎\endnote{\BookTitle{觀止}作「底」,據\BookTitle{左傳}校本改。\ProperName{阮元}\BookTitle{校勘記}:補刊石經缺,纂圖本、\ProperName{閩}、監、\ProperName{毛}本作「底」。}止。\ProperName{成王}定鼎于\ProperName{郟鄏},卜世三十,卜年七百,天所命也。\ProperName{周}德雖衰,天命未改。鼎之輕重,未可問也。」

\theendnotes 

\section[齊國佐不辱命\quad{\small 左傳 成公二年}]{{\normalsize 左傳 成公二年}\quad \ProperName{齊國佐}不辱命}
\ProperName{晉}師從\ProperName{齊}師,入自\ProperName{丘輿},擊\ProperName{馬陘}。\ProperName{齊侯}使\ProperName{賓媚人}賂以\ProperName{紀}甗、玉磬與地。「不可,則聽客之所爲。」

\ProperName{賓媚人}致賂,\ProperName{晉}人不可,曰:「必以\ProperName{蕭同叔子}爲質,而使\ProperName{齊}之封內盡東其畝。」對曰:「\ProperName{蕭同叔子}非他,寡君之母也。若以匹敵,則亦\ProperName{晉}君之母也。吾子布大命於諸侯,而曰必質其母以爲信,其若王命何?且是以不孝令也。\BookTitle{詩}曰:『孝子不匱,永錫爾類。』若以不孝令於諸侯,其無乃非德類也乎?先王疆理天下,物土之宜,而布其利。故\BookTitle{詩}曰:『我疆我理,南東其畝。』今吾子疆理諸侯,而曰『盡東其畝』而已,唯吾子戎車是利,無顧土宜,其無乃非先王之命也乎?反先王則不義,何以爲盟主?其\BookTitle{晉}實有闕。四王之王也,樹德而濟同欲焉;五伯之霸也,勤而撫之,以役王命。今吾子求合諸侯,以逞無疆之欲。\BookTitle{詩}曰:『布政\endnote{\BookTitle{觀止}作「敷政」,據\BookTitle{左傳}校本改。\ProperName{阮元}\BookTitle{校勘記}:\BookTitle{詩}作「敷政」,\ProperName{鄭氏}\BookTitle{儀禮}\BookTitle{聘禮}注云:「今文『布』作『敷』」。}優優,百祿是遒。』子實不優,而棄百祿,諸侯何害焉?不然,寡君之命使臣,則有辭矣!曰:『子以君師辱於敝邑,不腆敝賦,以犒從者。畏君之震,師徒橈敗。吾子惠徼\ProperName{齊}國之福,不泯其社稷,使繼舊好,唯是先君之敝器、土地不敢愛。子又不許,請收合餘燼,背城借一。敝邑之幸,亦云從也;況其不幸,敢不唯命是聽?』」

\theendnotes 

\section[楚歸晉知罃\quad{\small 左傳 成公三年}]{{\normalsize 左傳\ 成公三年}\quad \ProperName{楚}歸\ProperName{晉}\ProperName{知罃}}
\ProperName{晉}人歸\ProperName{楚}\ProperName{公子穀臣}與\ProperName{連尹襄老}之尸于\ProperName{楚},以求\ProperName{知罃}。於是\ProperName{荀首}佐中軍矣,故\ProperName{楚}人許之。

王送\ProperName{知罃},曰:「子其怨我乎?」對曰:「二國治戎,臣不才,不勝其任,以爲俘馘。執事不以釁鼓,使歸即戮,君之惠也。臣實不才,又誰敢怨?」王曰:「然則德我乎?」對曰:「二國圖其社稷而求紓其民,各懲其忿,以相宥也,兩釋纍囚,以成其好。二國有好,臣不與及,其誰敢德?」王曰:「子歸,何以報我?」對曰:「臣不任受怨,君亦不任受德,無怨無德,不知所報。」王曰:「雖然,必告不穀。」對曰:「以君之靈,纍臣得歸骨於晉,寡君之以爲戮,死且不朽。若從君之惠而免之,以賜君之外臣\ProperName{首};\ProperName{首}其請於寡君,而以戮於宗,亦死且不朽。若不獲命,而使嗣宗職,次及於事,而帥偏師,以脩封疆。雖遇執事,其弗敢違。其竭力致死,無有二心,以盡臣禮。所以報也。」王曰:「\ProperName{晉}未可與爭。」重爲之禮而歸之。

\section[呂相絕秦\quad{\small 左傳 成公十三年}]{{\normalsize 左傳\ 成公十三年}\quad \ProperName{呂相}絕\ProperName{秦}}
% 夏四月戊午
\ProperName{晉侯}使\ProperName{呂相}絕\ProperName{秦},曰:

\begin{quotation}昔逮我\ProperName{獻公}及\ProperName{穆公}相好,戮力同心,申之以盟誓,重之以昏姻。天禍\ProperName{晉國},\ProperName{文公}如\ProperName{齊},\ProperName{惠公}如\ProperName{秦}。無祿,\ProperName{獻公}即世。\ProperName{穆公}不忘舊德,俾我\ProperName{惠公}用能奉祀於于\ProperName{晉}。又不能成大勳,而爲\ProperName{韓}之師。亦悔于厥心,用集我\ProperName{文公},是\ProperName{穆}之成也。

\ProperName{文公}躬擐甲冑,跋履山川,踰越險阻,征東之諸侯,\ProperName{虞}、\ProperName{夏}、\ProperName{商}、\ProperName{周}之胤而朝諸\ProperName{秦},則亦既報舊德矣。\ProperName{鄭}人怒君之疆埸,我\ProperName{文公}帥諸侯及\ProperName{秦}圍\ProperName{鄭}。\ProperName{秦}大夫不詢于我寡君,擅及\ProperName{鄭}盟,諸侯疾之,將致命于\ProperName{秦}。\ProperName{文公}恐懼,綏靖諸侯,\ProperName{秦}師克還無害,則是我有大造于西也。

無祿,\ProperName{文公}即世,\ProperName{穆}爲不弔,蔑死我君,寡我\ProperName{襄公},迭我\ProperName{殽}地,奸絕我好,伐我保城,殄滅我\ProperName{費滑},散離我兄弟,撓亂我同盟,傾覆我國家。我\ProperName{襄公}未忘君之舊勳,而懼社稷之隕,是以有\ProperName{殽}之師。猶願赦罪于\ProperName{穆公},\ProperName{穆公}弗聽,而即\ProperName{楚}謀我。天誘其衷,\ProperName{成王}隕命,\ProperName{穆公}是以不克逞志于我。

\ProperName{穆}、\ProperName{襄}即世,\ProperName{康}、\ProperName{靈}即位。\ProperName{康公},我之自出,又欲闕翦我公室,傾覆我社稷,帥我蝥賊,以來蕩搖我邊疆,我是以有\ProperName{令狐}之役。\ProperName{康}猶不悛,入我\ProperName{河曲},伐我\ProperName{涑川},俘我\ProperName{王官},翦我\ProperName{羈馬},我是以有\ProperName{河曲}之戰。東道之不通,則是\ProperName{康公}絕我好也。

及君之嗣也,我君\ProperName{景公}引領西望,曰:「庶撫我乎!」君亦不惠稱盟,利吾有\ProperName{狄}難,入我\ProperName{河縣},焚我\ProperName{箕}、\ProperName{郜},芟夷我農功,虔劉我邊陲,我是以有\ProperName{輔氏}之聚。

君亦悔禍之延,而欲徼福于先君\ProperName{獻}、\ProperName{穆},使\ProperName{伯車}來命我\ProperName{景公}曰:「吾與女同好棄惡,復脩舊德,以追念前勳。」言誓未就,\ProperName{景公}即世,我寡君是以有\ProperName{令狐}之會。君又不祥,背棄盟誓。\ProperName{白狄}及君同州,君之仇讎,而我之昏姻也。君來賜命曰:「吾與女伐\ProperName{狄}。」寡君不敢顧昏姻,畏君之威,而受命于吏\endnote{\BookTitle{觀止}作「使」,據\BookTitle{左傳}改。}。君有二心於\ProperName{狄},曰:「\ProperName{晉}將伐女。」\ProperName{狄}應且憎,是用告我。\ProperName{楚}人惡君之二三其德也,亦來告我曰:「\ProperName{秦}背\ProperName{令狐}之盟,而來求盟于我,昭告昊天上帝、\ProperName{秦}三公、\ProperName{楚}三王,曰:『余雖與\ProperName{晉}出入,余唯利是視。』不穀惡其無成德,是用宣之,以懲不壹\endnote{\BookTitle{觀止}作「一」,據\BookTitle{左傳}改。}。」

諸侯備聞此言,斯是用痛心疾首,暱就寡人。寡人帥以聽命,唯好是求。君若惠顧諸侯,矜哀寡人,而賜之盟,則寡人之願也,其承寧諸侯以退,豈敢徼亂?君若不施大惠,寡人不佞,其不能以諸侯退矣。敢盡布之執事,俾執事實圖利之。
\end{quotation}

\theendnotes

\section[駒支不屈于晉\quad{\small 左傳 襄公十四年}]{{\normalsize 左傳\ 襄公十四年}\quad \ProperName{駒支}不屈于\ProperName{晉}}
會于\ProperName{向}\endnote{原作「十四年春,\ProperName{吳}告敗于{晉}。會于\ProperName{向},爲\ProperName{吳}謀\ProperName{楚}故也。\ProperName{范宣子}數\ProperName{吳}之不德也,以退\ProperName{吳}人。執\ProperName{莒}\ProperName{公子務婁},以其通\ProperName{楚}使也。」},將執\ProperName{戎子}\ProperName{駒支},\ProperName{范宣子}親數諸朝,曰:「來,\ProperName{姜戎氏}!昔\ProperName{秦}人迫逐乃祖\ProperName{吾離}于\ProperName{瓜州},乃祖\ProperName{吾離}被苫蓋,蒙荊棘以來歸我先君。我先君\ProperName{惠公}有不腆之田,與女剖分而食之。今諸侯之事我寡君不如昔者,蓋言語漏洩,則職女之由。詰朝之事,爾無與焉。與,將執女。」

對曰:「昔\ProperName{秦}人負恃其眾,貪于土地,逐我諸戎。\ProperName{惠公}蠲其大德,謂我諸戎,是\ProperName{四嶽}之裔冑也,毋是翦棄。賜我南鄙之田,狐貍\endnote{\BookTitle{觀止}作「狸」,據\BookTitle{左傳}校本改。\ProperName{阮元}\BookTitle{校勘記}:\ProperName{岳}本依\BookTitle{釋文}作「狸」,案\BookTitle{說文}無「狸」字,\ProperName{陸氏}云「本又作貍」。}所居,豺狼所嗥。我諸戎除翦其荊棘,驅其狐貍豺狼,以爲先君不侵不叛之臣,至于今不貳。昔\ProperName{文公}與\ProperName{秦}伐\ProperName{鄭},\ProperName{秦}人竊與\ProperName{鄭}盟而舍戍焉,於是乎有\ProperName{殽}之師。\ProperName{晉}禦其上,戎亢其下,\ProperName{秦}師不復,我諸戎實然。譬如捕鹿,\ProperName{晉}人角之,諸戎掎之,與\ProperName{晉}踣之。戎何以不免?自是以來,\ProperName{晉}之百役,與我諸戎相繼于時,以從執政,猶\ProperName{殽}志也。豈敢離逷?今官之師旅無乃實有所闕,以攜諸侯,而罪我諸戎!我諸戎飲食衣服不與華同,贄幣不通,言語不達,何惡之能爲?不與於會,亦無瞢焉。」賦\BookTitle{青蠅}而退。\ProperName{宣子}辭焉,使即事於會,成愷悌也。

\theendnotes

\section[祁奚請免叔向\quad{\small 左傳\ 襄公二十一年}]{{\normalsize 左傳\ 襄公二十一年}\quad \ProperName{祁奚}請免\ProperName{叔向}}
% 秋
\ProperName{欒盈}出奔\ProperName{楚},\ProperName{宣子}殺\ProperName{羊舌虎},囚\ProperName{叔向}\endnote{原作:「\ProperName{宣子}殺\ProperName{箕遺}、\ProperName{黃淵}、\ProperName{嘉父}、\ProperName{司空靖}、\ProperName{邴豫}、\ProperName{董叔}、\ProperName{邴師}、\ProperName{申書}、\ProperName{羊舌虎}、\ProperName{叔羆},囚\ProperName{伯華}、\ProperName{叔向}、\ProperName{籍偃}。」}。人謂\ProperName{叔向}曰:「子離於罪,其爲不知乎?」\ProperName{叔向}曰:「與其死亡若何?\BookTitle{詩}曰:『優哉游哉,聊以卒歲。』知也。」

\ProperName{樂王鮒}見\ProperName{叔向}曰:「吾爲子請。」\ProperName{叔向}弗應。出,不拜。其人皆咎\ProperName{叔向}。\ProperName{叔向}曰:「必\ProperName{祁大夫}。」室老聞之曰:「\ProperName{樂王鮒}言於君,無不行,求赦吾子,吾子不許。\ProperName{祁大夫}所不能也,而曰必由之,何也?」\ProperName{叔向}曰:「\ProperName{樂王鮒},從君者也,何能行?\ProperName{祁大夫}外舉不棄讎,內舉不失親,其獨遺我乎?\BookTitle{詩}曰:『有覺德行,四國順之。』夫子覺者也。」

\ProperName{晉侯}問\ProperName{叔向}之罪於\ProperName{樂王鮒}。對曰:「不棄其親,其有焉。」於是\ProperName{祁奚}老矣,聞之,乘馹而見\ProperName{宣子},曰:「\BookTitle{詩}曰:『惠我無疆,子孫保之。』\BookTitle{書}曰:『聖有謩勳,明徵定保。』夫謀而鮮過、惠訓不倦者,\ProperName{叔向}有焉,社稷之固也,猶將十世宥之,以勸能者。今壹不免其身,以棄社稷,不亦惑乎?\ProperName{鯀}殛而\ProperName{禹}興;\ProperName{伊尹}放\ProperName{大甲}而相之,卒無怨色;\ProperName{管}、\ProperName{蔡}爲戮,\ProperName{周公}右王,若之何其以\ProperName{虎}也棄社稷?子爲善,誰敢不勉?多殺何爲?」\ProperName{宣子}說,與之乘,以言諸公而免之。不見\ProperName{叔向}而歸,\ProperName{叔向}亦不告免焉而朝。

\theendnotes

\section[子產告范宣子輕幣\quad{\small 左傳 襄公二十四年}]{{\normalsize 左傳\ 襄公二十四年}\quad \ProperName{子產}告\ProperName{范宣子}輕幣}
\ProperName{范宣子}爲政,諸侯之幣重,\ProperName{鄭}人病之。

二月,\ProperName{鄭伯}如\ProperName{晉},\ProperName{子產}寓書於\ProperName{子西},以告\ProperName{宣子},曰:「子爲\ProperName{晉國},四鄰諸侯不聞令德,而聞重幣,\ProperName{僑}也惑之。\ProperName{僑}聞君子長國家者,非無賄之患,而無令名之難。夫諸侯之賄聚於公室,則諸侯貳。若吾子賴之,則\ProperName{晉國}貳。諸侯貳,則\ProperName{晉國}壞;\ProperName{晉國}貳,則子之家壞,何沒沒也!將焉用賄?夫令名,德之輿也;德,國家之基也。有基無壞,無亦是務乎!有德則樂,樂則能久。\BookTitle{詩}云『樂只君子,邦家之基』,有令德也夫!『上帝臨女,無貳爾心』,有令名也夫!恕思以明德,則令名載而行之,是以遠至邇安。毋寧使人謂子,『子實生我』,而謂『子浚我以生』乎?象有齒以焚其身,賄也。」\ProperName{宣子}說,乃輕幣。

\section[晏子不死君難\quad{\small 左傳 襄公二十五年}]{{\normalsize 左傳\ 襄公二十五年}\quad \ProperName{晏子}不死君難}
\ProperName{崔武子}見\ProperName{棠姜}而美之,遂取之。\ProperName{莊公}通焉,\ProperName{崔子}弒之。\endnote{\ProperName{崔子}之弒\ProperName{莊公},\BookTitle{觀止}省文,原作:\begin{quotation}\ProperName{齊棠公}之妻,\ProperName{東郭偃}之姊也。\ProperName{東郭偃}臣\ProperName{崔武子}。\ProperName{棠公}死,\ProperName{偃}御\ProperName{武子}以弔焉,見\ProperName{棠姜}而美之,使\ProperName{偃}取之,\ProperName{偃}曰:「男女辨姓,今君出自\ProperName{丁},臣出自\ProperName{桓},不可。」\ProperName{武子}筮之,遇\BookTitle{困}{\fontfamily{songextg}\selectfont ䷮}之\BookTitle{大過}{\fontfamily{songextg}\selectfont ䷛},史皆曰「吉」。示\ProperName{陳文子},\ProperName{文子}曰:「夫從風,風隕妻,不可娶也。且其\BookTitle{繇}曰:『困于石,據于蒺梨,入于其宮,不見其妻,凶。』『困于石』,往不濟也。『據于蒺梨』,所恃傷也。『入于其宮,不見其妻,凶』,無所歸也。」\ProperName{崔子}曰:「嫠也,何害?先夫當之矣。」遂取之。\ProperName{莊公}通焉,驟如\ProperName{崔氏},以\ProperName{崔子}之冠賜人。侍者曰:「不可。」公曰:「不爲\ProperName{崔子},其無冠乎?」\ProperName{崔子}因是,又以其間伐\ProperName{晉}也,曰:「\ProperName{晉}必將報。」欲弒公以說于\ProperName{晉},而不獲間。公鞭侍人\ProperName{賈舉},而又近之,乃爲\ProperName{崔子}間公。

夏,五月,\ProperName{莒}爲\ProperName{且于}之役故,\ProperName{莒子}朝于\ProperName{齊}。甲戌,饗諸北郭,\ProperName{崔子}稱疾不視事。乙亥,公問\ProperName{崔子},遂從\ProperName{姜氏},\ProperName{姜}入于室,與\ProperName{崔子}自側戶出,公拊楹而歌,侍人\ProperName{賈舉}止眾從者而入,閉門。甲興,公登臺而請,弗許;請盟,弗許;請自刃於廟,弗許。皆曰:「君之臣\ProperName{杼}疾病,不能聽命。近於公宮,陪臣干掫有淫者,不知二命。」公踰牆,又射之,中股,反隊,遂弒之。\ProperName{賈舉}、\ProperName{州綽}、\ProperName{邴師}、\ProperName{公孫敖}、\ProperName{封具},\ProperName{鐸父},\ProperName{襄伊},\ProperName{僂堙}皆死。\ProperName{祝佗父}祭於\ProperName{高唐},至,復命,不說弁而死於\ProperName{崔氏}。\ProperName{申蒯},侍漁者,退,謂其宰曰:「爾以帑免,我將死。」其宰曰:「免,是反子之義也。」與之皆死。\ProperName{崔氏}殺\ProperName{鬷蔑}于\ProperName{平陰}。\end{quotation}}

\ProperName{晏子}立於\ProperName{崔氏}之門外,其人曰:「死乎?」曰:「獨吾君也乎哉,吾死也?」曰:「行乎?」曰:「吾罪也乎哉,吾亡也?」曰:「歸乎?」曰:「君死,安歸?君民者,豈以陵民?社稷是主。臣君者,豈爲其口實?社稷是養。故君爲社稷死,則死之;爲社稷亡,則亡之。若爲己死,而爲己亡,非其私暱,誰敢任之?且人有君而弒之,吾焉得死之?而焉得亡之?將庸何歸?」門啓而入,枕尸股而哭。興,三踊而出。人謂\ProperName{崔子}「必殺之!」\ProperName{崔子}曰:「民之望也,舍之,得民。」

\theendnotes

\section[季札觀周樂\quad{\small 左傳 襄公二十九年}]{{\normalsize 左傳\ 襄公二十九年}\quad \ProperName{季札}觀\ProperName{周}樂}
\ProperName{吳公子札}來聘\endnote{「吳公子札來聘」下原有:「見\ProperName{叔孫穆子},說之。謂\ProperName{穆子}曰:『子其不得死乎!好善而不能擇人。吾聞君子務在擇人。吾子爲\ProperName{魯}宗卿,而任其大政,不慎舉,何以堪之?禍必及子!』」},請觀於\ProperName{周}樂。使工爲之歌\BookTitle{周南}、\BookTitle{召南},曰:「美哉!始基之矣,猶未也,然勤而不怨矣。」爲之歌\BookTitle{邶}、\BookTitle{鄘}、\BookTitle{衛},曰:「美哉淵乎!憂而不困者也。吾聞\ProperName{衛康叔}、\ProperName{武公}之德如是,是其\BookTitle{衛風}乎?」爲之歌\BookTitle{王},曰:「美哉!思而不懼,其\BookTitle{周}之東乎?」爲之歌\BookTitle{鄭},曰:「美哉!其細已甚,民弗堪也。是其先亡乎!」爲之歌\BookTitle{齊},曰:「美哉,泱泱乎!大風也哉!表東海者,其\ProperName{大公}乎!國未可量也!」爲之歌\BookTitle{豳},曰:「美哉,蕩乎!樂而不淫,其\ProperName{周公}之東乎!」爲之歌\BookTitle{秦},曰:「此之謂夏聲。夫能夏則大,大之至也,其\ProperName{周}之舊乎!」爲之歌\BookTitle{魏},曰:「美哉,渢渢乎!大而婉,險而易行。以德輔此,則明主也。」爲之歌\BookTitle{唐},曰:「思深哉!其有\ProperName{陶唐氏}之遺民乎!不然,何憂之遠也?非令德之後,誰能若是?」爲之歌\BookTitle{陳},曰:「國無主,其能久乎?」自\BookTitle{鄶}以下無譏焉。

爲之歌\BookTitle{小雅},曰:「美哉!思而不貳,怨而不言,其\BookTitle{周}德之衰乎?猶有先王之遺民焉!」爲之歌\BookTitle{大雅},曰:「廣哉,熙熙乎!曲而有直體,其\ProperName{文王}之德乎!」

爲之歌\BookTitle{頌},曰:「至矣哉!直而不倨,曲而不屈,邇而不偪,遠而不攜,遷而不淫,復而不厭,哀而不愁,樂而不荒,用而不匱,廣而不宣,施而不費,取而不貪,處而不底,行而不流。五聲和,八風平,節有度,守有序。盛德之所同也!」

見舞\BookTitle{象箾}、\BookTitle{南籥}者,曰:「美哉!猶有憾。」見舞\BookTitle{大武}者,曰:「美哉!\ProperName{周}之盛也,其若此乎!」見舞\BookTitle{韶濩}者,曰:「聖人之弘也,而猶有慙德,聖人之難也!」見舞\BookTitle{大夏}者,曰:「美哉!勤而不德,非\ProperName{禹},其誰能脩之?」見舞\BookTitle{韶箾}者,曰:「德至矣哉,大矣!如天之無不幬也,如地之無不載也。雖甚盛德,其蔑以加於此矣,觀止矣。若有他樂,吾不敢請已。」

\theendnotes 

\section[子產壞盡館垣\quad{\small 左傳\ 襄公三十一年}]{{\normalsize 左傳\ 襄公三十一年}\quad \ProperName{子產}壞盡館垣}
\ProperName{子產}相\ProperName{鄭伯}以如\ProperName{晉},\ProperName{晉侯}以我喪故,未之見也。\ProperName{子產}使盡壞其館之垣而納車馬焉。\ProperName{士文伯}讓之,曰:「敝邑以政刑之不脩,寇盜充斥,無若諸侯之屬辱在寡君者何,是以令吏人完客所館,高其閈閎,厚其牆垣,以無憂客使。今吾子壞之,雖從者能戒,其若異客何?以敝邑之爲盟主,繕完葺牆,以待賓客。若皆毀之,其何以共命?寡君使\ProperName{匄}請命。」

對曰:「以敝邑褊小,介於大國,誅求無時,是以不敢寧居,悉索敝賦,以來會時事。逢執事之不閒,而未得見;又不獲聞命,未知見時。不敢輸幣,亦不敢暴露。其輸之,則君之府實也,非薦陳之,不敢輸也。其暴露之,則恐燥濕之不時而朽蠹,以重敝邑之罪。\ProperName{僑}聞\ProperName{文公}之爲盟主也,宮室卑庳,無觀臺榭,以崇大諸侯之館。館如公寢,庫廄繕脩,司空以時平易道路,圬人以時塓館宮室;諸侯賓至,甸設庭燎,僕人巡宮;車馬有所,賓從有代,巾車脂轄,隸人、牧、圉各瞻其事;百官之屬各展其物;公不留賓,而亦無廢事;憂樂同之,事則巡之;教其不知,而恤其不足。賓至如歸,無寧菑患;不畏寇盜,而亦不患燥濕。今\ProperName{銅鞮}之宮數里,而諸侯舍於隸人,門不容車,而不可踰越;盜賊公行,而夭厲不戒。賓見無時,命不可知。若又勿壞,是無所藏幣以重罪也。敢請執事:將何所命之?雖君之有\ProperName{魯}喪,亦敝邑之憂也。若獲薦幣,脩垣而行,君之惠也,敢憚勤勞!」

\ProperName{文伯}復命。\ProperName{趙文子}曰:「信。我實不德,而以隸人之垣以贏諸侯,是吾罪也。」使\ProperName{士文伯}謝不敏焉。\ProperName{晉侯}見\ProperName{鄭伯},有加禮,厚其宴、好而歸之。乃築諸侯之館。

\ProperName{叔向}曰:「辭之不可以已也如是夫!\ProperName{子產}有辭,諸侯賴之,若之何其釋辭也?\BookTitle{詩}曰『辭之輯矣,民之協矣;辭之繹矣,民之莫矣』其知之矣!」

\section[子產論尹何爲邑\quad{\small 左傳\ 襄公三十一年}]{{\normalsize 左傳\ 襄公三十一年}\quad \ProperName{子產}論\ProperName{尹何}爲邑}
\ProperName{子皮}欲使\ProperName{尹何}爲邑。\ProperName{子產}曰:「少,未知可否。」\ProperName{子皮}曰:「愿,吾愛之,不吾叛也。使夫往而學焉,夫亦愈知治矣。」\ProperName{子產}曰:「不可。人之愛人,求利之也。今吾子愛人則以政,猶未能操刀而使割也,其傷實多。子之愛人,傷之而已,其誰敢求愛於子?子於\ProperName{鄭國},棟也。棟折榱崩,\ProperName{僑}將厭焉,敢不盡言?子有美錦,不使人學製焉。大官、大邑,身之所庇也,而使學者制焉,其爲美錦不亦多乎?\ProperName{僑}聞學而後入政,未聞以政學者也。若果行此,必有所害。譬如田獵,射御貫,則能獲禽,若未嘗登車射御,則敗績厭覆是懼,何暇思獲?」

\ProperName{子皮}曰:「善哉!\ProperName{虎}不敏。吾聞君子務知大者、遠者,小人務知小者、近者。我,小人也。衣服附在吾身,我知而慎之;大官、大邑,所以庇身也,我遠而慢之。\ProperName{微子}之言,吾不知也。他日我曰,子爲\ProperName{鄭國},我爲吾家,以庇焉,其可也。今而後知不足。自今請,雖吾家,聽子而行。」\ProperName{子產}曰:「人心之不同如其面焉,吾豈敢謂子面如吾面乎?抑心所謂危,亦以告也。」\ProperName{子皮}以爲忠,故委政焉。\ProperName{子產}是以能爲\ProperName{鄭國}。

\section[子產卻楚逆女以兵\quad{\small 左傳 昭公元年}]{{\normalsize 左傳\ 昭公元年}\quad \ProperName{子產}卻\ProperName{楚}逆女以兵}
% 元年春,
\ProperName{楚}\ProperName{公子圍}聘于\ProperName{鄭},且娶於\ProperName{公孫段}氏。\ProperName{伍舉}爲介。將入館。\ProperName{鄭}人惡之,使行人\ProperName{子羽}與之言,乃館於外。既聘,將以眾逆。\ProperName{子產}患之,使\ProperName{子羽}辭曰:「以敝邑褊小,不足以容從者,請墠聽命。」令尹命{大}宰\ProperName{伯州犁}對曰:「君辱貺寡大夫\ProperName{圍},謂\ProperName{圍}將使\ProperName{豐氏}撫有而室。\ProperName{圍}布几筵,告於\ProperName{莊}、\ProperName{共}之廟而來。若野賜之,是委君貺於草莽也,是寡大夫不得列於諸卿也。不寧唯是,又使\ProperName{圍}蒙其先君,將不得爲寡君老,其蔑以復矣。唯大夫圖之!」\ProperName{子羽}曰:「小國無罪,恃實其罪。將恃大國之安靖己,而無乃包藏禍心以圖之?小國失恃,而懲諸侯,使莫不憾者,距違君命,而有所壅塞不行是懼。不然,敝邑,館人之屬也,其敢愛\ProperName{豐氏}之祧?」\ProperName{伍舉}知其有備也,請垂櫜而入。許之。

\section[子革對靈王\quad{\small 左傳 昭公十二年}]{{\normalsize 左傳\ 昭公十二年}\quad \ProperName{子革}對\ProperName{靈王}}
\ProperName{楚子}狩于\ProperName{州來},次于\ProperName{潁尾}。使\ProperName{蕩侯}、\ProperName{潘子}、\ProperName{司馬督}、囂尹\ProperName{午}、陵尹\ProperName{喜},帥師圍\ProperName{徐}以懼\ProperName{吳}。\ProperName{楚子}次于\ProperName{乾谿},以爲之援。雨雪,王皮冠,\ProperName{秦}復陶,翠被,豹舄,執鞭以出。僕\ProperName{析父}從。

右尹\ProperName{子革}夕,王見之。去冠、被,舍鞭,與之語,曰:「昔我先王\ProperName{熊繹}與\ProperName{呂伋}、\ProperName{王孫牟}、\ProperName{燮父}、\ProperName{禽父}並事\ProperName{康王},四國皆有分,我獨無有。今吾使人於\ProperName{周},求鼎以爲分,王其與我乎?」對曰:「與君王哉!昔我先王\ProperName{熊繹}辟在\ProperName{荊山},篳路藍縷以處草莽,跋涉山林以事天子,唯是桃弧、棘矢以共禦王事。\ProperName{齊},王舅也;\ProperName{晉}及\ProperName{魯}、\ProperName{衛},王母弟也。\ProperName{楚}是以無分,而彼皆有。今\ProperName{周}與四國服事君王,將唯命是從,豈其愛鼎?」王曰:「昔我皇祖伯父\ProperName{昆吾},\ProperName{舊許}是宅。今\ProperName{鄭}人貪賴其田,而不我與。我若求之,其與我乎?」對曰:「與君王哉!\ProperName{周}不愛鼎,\ProperName{鄭}敢愛田?」王曰:「昔諸侯遠我而畏\ProperName{晉},今我大城\ProperName{陳}、\ProperName{蔡}、\ProperName{不羹},賦皆千乘,子與有勞焉,諸侯其畏我乎?」對曰:「畏君王哉!是四國者,專足畏也。又加之以\ProperName{楚},敢不畏君王哉?」

工尹\ProperName{路}請曰:「君王命剝圭以爲鏚柲,敢請命。」王入視之。

\ProperName{析父}謂\ProperName{子革}:「吾子,\ProperName{楚國}之望也。今與王言如響,國其若之何?」\ProperName{子革}曰:「摩厲以須,王出,吾刃將斬矣。」

王出,復語。左史\ProperName{倚相}趨過,王曰:「是良史也,子善視之!是能讀\BookTitle{三墳}、\BookTitle{五典}、\BookTitle{八索}、\BookTitle{九丘}。」對曰:「臣嘗問焉,昔\ProperName{穆王}欲肆其心,周行天下,將皆必有車轍馬跡焉。\ProperName{祭公謀父}作\BookTitle{祈招}之詩以止王心,王是以獲沒於\ProperName{祗宮}。臣問其詩而不知也。若問遠焉,其焉能知之?」王曰:「子能乎?」對曰:「能。其詩曰:『祈招之愔愔,式昭德音。思我王度,式如玉,式如金。形民之力,而無醉飽之心。』」

王揖而入,饋不食,寢不寐。數日,不能自克,以及於難。

\ProperName{仲尼}曰:「古也有志:『克己復禮,仁也。』信善哉!\ProperName{楚靈王}若能如是,豈其辱於\ProperName{乾谿}?」

\section[子產論政寬猛\quad{\small 左傳 昭公二十年}]{{\normalsize 左傳\ 昭公二十年}\quad \ProperName{子產}論政寬猛}
\ProperName{鄭}\ProperName{子產}有疾,謂\ProperName{子大叔}曰:「我死,子必爲政。唯有德者能以寬服民,其次莫如猛。夫火烈,民望而畏之,故鮮死焉;水懦弱,民狎而翫之,則多死焉。故寬難。」疾數月而卒。

\ProperName{大叔}爲政,不忍猛而寬。\ProperName{鄭國}多盜,取人於萑苻之澤。\ProperName{大叔}悔之,曰:「吾早從夫子,不及此。」興徒兵以攻萑苻之盜,盡殺之,盜少止。

\ProperName{仲尼}曰:「善哉!政寬則民慢,慢則糾之以猛。猛則民殘,殘則施之以寬。寬以濟猛,猛以濟寬,政是以和。」\BookTitle{詩}曰:『民亦勞止,汔可小康,惠此中國,以綏四方』,施之以寬也。『毋從詭隨,以謹無良;式遏寇虐,慘不畏明』,糾之以猛也。『柔遠能邇,以定我王』,平之以和也。又曰『不競不絿,不剛不柔,布政優優,百祿是遒』,和之至也!」

及\ProperName{子產}卒,\ProperName{仲尼}聞之,出涕曰:「古之遺愛也。」

\section[吳許越成\quad{\small 左傳 哀公元年}]{{\normalsize 左傳\ 哀公元年}\quad \ProperName{吳}許\ProperName{越}成}
\ProperName{吳王}\ProperName{夫差}敗\ProperName{越}于\ProperName{夫椒},報\ProperName{檇李}也。遂入\ProperName{越}。\ProperName{越子}以甲楯五千保于\ProperName{會稽},使大夫\ProperName{種}因\ProperName{吳}\ProperName{{大}宰嚭}以行成。

\ProperName{吳子}將許之。\ProperName{伍員}曰:「不可。臣聞之:『樹德莫如滋,去疾莫如盡。』昔\ProperName{有過}\ProperName{澆}殺\ProperName{斟灌}以伐\ProperName{斟鄩},滅\ProperName{夏后相},\ProperName{后緡}方娠,逃出自竇,歸于\ProperName{有仍},生\ProperName{少康}焉。爲\ProperName{仍}牧正,惎\ProperName{澆}能戒之。\ProperName{澆}使\ProperName{椒}求之,逃奔\ProperName{有虞},爲之庖正,以除其害。\ProperName{虞思}於是妻之以二\ProperName{姚},而邑諸\ProperName{綸},有田一成,有眾一旅。能布其德,而兆其謀,以收\ProperName{夏}眾,撫其官職;使\ProperName{女艾}諜\ProperName{澆},使\ProperName{季杼}誘\ProperName{豷},遂滅\ProperName{過}、\ProperName{戈},復\ProperName{禹}之績,祀\ProperName{夏}配天,不失舊物。今\ProperName{吳}不如\ProperName{過},而\ProperName{越}大於\ProperName{少康},或將豐之,不亦難乎!\ProperName{句踐}能親而務施,施不失人,親不棄勞。與我同壤,而世爲仇讎。於是乎克而弗取,將又存之,違天而長寇讎,後雖悔之,不可食已。\ProperName{姬}之衰也,日可俟也。介在蠻夷,而長寇讎,以是求伯,必不行矣!」弗聽。退而告人曰:「\ProperName{越}十年生聚,而十年教訓,二十年之外,\ProperName{吳}其爲沼乎!」

% Proofed 6 July 2022
% Ref.
% 楊伯峻, 春秋左傳註
% 十三经注疏整理本·春秋左傳正義
