\begin{enumerate}
    \item[一、] \ProperName{山陰}\ProperName{吳楚材}、\ProperName{吳調侯}叔侄編\BookTitle{古文觀止}爲古文選本中流傳最廣者。今取其選文二百二十二篇,校勘訂正,重排刊印。
    \item[二、] 本次校訂,以\ProperName{中華}書局\ 1959\ 年重印文學古籍刊行社重排\ProperName{映雪堂}本\BookTitle{古文觀止}爲底本,僅取其選文,而未錄其註評。
    \item[三、] \BookTitle{觀止}所選\ProperName{周}、\ProperName{秦}、\ProperName{漢}文殆出自經、史;\ProperName{六朝}、\ProperName{唐}、\ProperName{宋}、\ProperName{明}文多錄於集部。故校訂時均與經、史通行之校本、集部各通行之總集及作者之別集互校。
    \item[四、] 互校參考今人校本
    \begin{itemize}
        \item \BookTitle{十三經注疏}整理本:\BookTitle{春秋左傳正義}、\BookTitle{春秋公羊傳注疏}、\BookTitle{春秋\allowbreak 榖梁\allowbreak 傳\allowbreak 注疏}、\BookTitle{禮\allowbreak 記\allowbreak 正\allowbreak 義}\ {\small \ProperName{北京}大學出版社\ 2000, \ProperName{上海}古籍\ 2008 ---}
        % \item \ProperName{呂友仁}整理\BookTitle{禮記正義}\ {\small \ProperName{上海}古籍\ 2008}
        \item \ProperName{上海}師範大學古籍整理組校點\BookTitle{國語}\ {\small \ProperName{上海}古籍\ 1978}
        \item \BookTitle{戰國策}\ {\small \ProperName{上海}古籍\ 1985}
        \item \ProperName{中華}書局點校本\BookTitle{二十四史}:\BookTitle{史記}、\BookTitle{漢書}、\BookTitle{後漢書}、\BookTitle{三國志}、\BookTitle{新五代史}
        \item \ProperName{李慶甲}校點\BookTitle{楚辭集注}\ {\small \ProperName{上海}古籍\ 1979}
        \item \ProperName{李培南}等校點\BookTitle{文選}\ {\small \ProperName{上海}古籍\ 1986}
        \item \ProperName{龔斌}校箋\BookTitle{陶淵明集校箋}\ {\small \ProperName{上海}古籍\ 1996}
        \item \ProperName{瞿蜕園}、\ProperName{朱金城}校注\BookTitle{李白集校注}\ {\small \ProperName{上海}古籍\ 1980}
        \item \ProperName{陳允吉}校點\BookTitle{樊川文集}\ {\small \ProperName{上海}古籍\ 1978}
        \item \ProperName{馬其昶}校點\ProperName{馬茂元}整理\BookTitle{韓昌黎文集校注}\ {\small \ProperName{上海}古籍\ 1986}
        \item \ProperName{吳文志}等校點\BookTitle{柳宗元集}\ {\small \ProperName{中華}書局\ 1978}
        \item \ProperName{李逸安}校點\BookTitle{歐陽脩全集}\ {\small \ProperName{中華}書局\ 2001}
        \item \ProperName{洪本健}校箋\BookTitle{歐陽脩詩文集校箋}\ {\small \ProperName{上海}古籍\ 2009}
        \item \ProperName{曾棗莊}、\ProperName{金成禮}箋注\BookTitle{嘉祐集箋注}\ {\small \ProperName{上海}古籍\ 1993}
        \item \ProperName{孔凡禮}校點\BookTitle{蘇軾文集}\ {\small \ProperName{中華}書局\ 1986}
        \item \ProperName{王松齡}點校\BookTitle{東坡志林}\ {\small \ProperName{中華}書局\ 1981}
        \item \ProperName{曾棗莊}、\ProperName{馬德富}校點\BookTitle{欒城集}\ {\small \ProperName{上海}古籍\ 1987}
        \item \ProperName{陳杏珍}、\ProperName{晁繼周}點校\BookTitle{曾鞏集}\ {\small \ProperName{中華}書局\ 1984}
        \item \ProperName{中華}書局\ProperName{上海}編輯所編輯\BookTitle{臨川先生文集}\ {\small \ProperName{中華}書局\ 1959}
        \item \ProperName{魏建猷}、\ProperName{蕭善薌}點校\BookTitle{郁離子}\ {\small \ProperName{上海}古籍\ 1981}
        \item \ProperName{徐光大}點校\BookTitle{方孝孺集}\ {\small \ProperName{浙江}古籍\ 2013}
        \item \ProperName{吳建華}點校\BookTitle{王鏊集}\ {\small \ProperName{上海}古籍\ 2013}
        \item \ProperName{吳光}等編校\BookTitle{王陽明全集}\ {\small \ProperName{上海}古籍\ 1992}
        \item \ProperName{錢伯城}箋校\BookTitle{袁宏道集箋校}\ {\small \ProperName{上海}古籍\ 1981}
        \item \ProperName{周本淳}校點\BookTitle{震川先生集}\ {\small \ProperName{上海}古籍\ 1981}
    \end{itemize}
    \BookTitle{四部叢刊}本
    \begin{itemize}
        \item \BookTitle{六臣注文選}
        \item \BookTitle{范文正公集}
        \item \BookTitle{溫國文正司馬公文集}
        \item \BookTitle{直講李先生文集}
        \item \BookTitle{宋學士全集}
        \item \BookTitle{誠意伯文集}
        \item \BookTitle{遜志齋集}
    \end{itemize}
    \BookTitle{四庫全書}本
    \begin{itemize}
        \item \BookTitle{宋文鑑}
        \item \BookTitle{明文海}
        \item \BookTitle{震澤集}
        \item \BookTitle{青霞集}
        \item \BookTitle{弇州四部稿}
    \end{itemize}
    \ProperName{中華}書局影印本\BookTitle{全上古三代秦漢六朝文}、\BookTitle{全唐文}及\BookTitle{基本古籍庫}\ v7.0 所收各本,於校記具體註明。
    \item[五、] 選文出於經、史者與通行校本互校,異文凡有見於經、史之各本者,酌改,並出校勘記說明。
    \item[六、] 選文錄於集部者與通行集本、今校本互校,異文凡有見於各通行之總集,作者之別集者,均保留\BookTitle{觀止}原文,校\BookTitle{觀止}如\BookTitle{觀止},不出校勘記。
    \item[七、] 於\ProperName{清}始見之異文,予以改正,並出校勘記說明。如:\begin{itemize}
        \item 獨見於\BookTitle{古文觀止}之異文,
        \item 亦見於其他\ProperName{清}人結集之異文而於經、史各本、諸通行總集、作者別集無據者。
    \end{itemize}
    \item[八、] 異文已見於\ProperName{清}以前類書或不經見古文選本者,而於經、史之各本、諸通行總集、作者之別集無據者,出校勘記說明,酌改。
    \item[九、] \BookTitle{古文觀止}避諱處改回,並出校勘記說明。
    \item[十、] \BookTitle{觀止}或有節略省文之處,皆作保留,並以校記說明。如:\BookTitle{左傳}諸篇、\BookTitle{漢書}\BookTitle{鄒陽\allowbreak 獄\allowbreak 中\allowbreak 上\allowbreak 梁王書}等。
    \item[十一、] 選文首末之取捨,書信套語之有無,皆從\BookTitle{觀止},不出校勘記。
    \item[十二、] 常見通假字、異體字據校本徑改,不出校勘記。
    \item[十三、] 標點分段參前揭今人諸點校本。
    \item[十四、] \BookTitle{觀止}選文或有僞託存疑者註明。 
\end{enumerate}
  