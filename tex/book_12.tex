
\section[送天台陳庭學序\quad{\small 宋濂}]{{\normalsize 宋濂}\quad 送\ProperName{天台}\ProperName{陳庭學}序}
西南山水,惟\ProperName{川}\ProperName{蜀}最奇。然去\ProperName{中州}萬里,陸有\ProperName{劍閣}棧道之險,水有\ProperName{瞿唐}、\ProperName{灩澦}之虞。跨馬行{篁}\endnote{\BookTitle{觀止}作「則」,據各本改。}竹間,山高者,累旬日不見其巓際;臨上而俯視,絕壑萬仞,杳莫測其所窮,肝膽爲之掉栗。水行,則江石悍利,波惡渦詭,舟一失{勢}\endnote{「失勢」:\BookTitle{文章辨體彙選}同,\BookTitle{宋學士全集}、\BookTitle{明文海}「失」下無「勢」字。}尺寸,輒糜碎土沉,下飽魚鼈。其難至如此,故非仕有力者,不可以遊;非材有文者,縱遊無所得;非壯彊者,多老死於其地,嗜奇之士恨焉。

\ProperName{天台}\ProperName{陳}君\ProperName{庭學},能爲詩,由中書左司掾,屢從大將北征,有勞,擢\ProperName{四川}都指揮司照磨,由水道至\ProperName{成都}。\ProperName{成都},\ProperName{川}、\ProperName{蜀}之要地,\ProperName{揚子雲}、\ProperName{司馬相如}、\ProperName{諸葛武侯}之所居。英雄俊傑戰攻駐守之跡,詩人文士遊眺飲射、賦詠歌呼所,\ProperName{庭學}無不歷覽。既覽必發爲詩,以紀其景物時世之變,於是其詩益工。越三年,以例自免歸,會予於京師。其氣愈充,其語愈壯,其志意愈高,蓋得於山水之助者侈矣。

予甚自愧,方予少時,嘗有志於出遊天下,顧以學未成而不暇。及年壯可出,而四方兵起,無所投足。逮今聖主興而宇內定,極海之際,合爲一家,而予齒益加耄矣。欲如\ProperName{庭學}之遊,尚可得乎?然吾聞古之賢士,若\ProperName{顏回}、\ProperName{原憲},皆坐守陋室,蓬蒿沒戶,而志意常充然,有若囊括於天地者。此其故何也?得無有出於山水之外者乎?\ProperName{庭學}其試歸而求焉。茍有所得,則以告予,予將不一愧而已也。 

\theendnotes 

\section[閲江樓記\quad{\small 宋濂}]{{\normalsize 宋濂}\quad \ProperName{閲江樓}記}
\ProperName{金陵}爲帝王之州,自\ProperName{六朝}迄於\ProperName{南唐},類皆偏據一方,無以應山川之王氣。逮我皇帝定鼎於茲,始足以當之。由是聲教所暨,罔間朔南;存神穆清,與天同體。雖一豫一游,亦可爲天下後世法。京城之西北有\ProperName{獅子山},自\ProperName{盧龍}蜿蜒而來,\ProperName{長江}如虹貫,蟠繞其下。上以其地雄勝,詔建樓於巔。與民同遊觀之樂,遂錫嘉名爲「閱江」云。

登覽之頃,萬象森列,千載之祕,一旦軒露。豈非天造地設,以俟{大}\endnote{\BookTitle{觀止}作「夫」,據各本改。}一統之君,而開千萬世之偉觀者歟?當風日清美,法駕幸臨,升其崇椒,凭闌遙矚,必悠然而動遐思。見\ProperName{江}\ProperName{漢}之朝宗,諸侯之述職,城池之高深,關阨之嚴固,必曰:「此朕櫛風沐雨,戰勝攻取之所致也。」\ProperName{中夏}之廣,益思有以保之。見波濤之浩蕩,風帆之下上,番舶接跡而來庭,蠻琛聯肩而入貢,必曰:「此朕德綏威服,覃及內外之所及也。」四{夷}\endnote{\BookTitle{觀止}作「陲」,據各本改。}之遠,益思有以柔之。見兩岸之間,四郊之上,耕人有炙膚皸足之煩,農女有捋桑行饁之勤,必曰:「此朕拔諸水火,而登於袵席者也。」萬方之民,益思有以安之。觸類而{推}\endnote{\BookTitle{觀止}作「思」,據各本改。},不一而足。臣知斯樓之建,皇上所以發舒精神,因物興感,無不寓其致治之思。奚止閱夫\ProperName{長江}而已哉!% 以俟「大」一統之君:觀止作「夫」; 觀止作「陲」; 觀止作「思」

彼\ProperName{臨春}、\ProperName{結綺},非不華矣;\ProperName{齊雲}、\ProperName{落星},非不高矣。不過樂管弦之淫響,藏\ProperName{燕}、\ProperName{趙}之艷姬。{一}\endnote{\BookTitle{觀止}作「不」,據各本改。}旋踵間而感慨係之,臣不知其爲何說也。雖然,\ProperName{長江}發源\ProperName{岷山},委蛇七千餘里而始入海,白湧碧翻。\ProperName{六朝}之時,往往倚之爲天塹。今則南北一家,視爲安流,無所事乎戰爭矣。然則果誰之力歟?逢掖之士,有登斯樓而閱斯江者,當思聖德如天,蕩蕩難名,與\ProperName{神禹}疏鑿之功同一罔極。忠君報上之心,其有不油然而興耶?% 觀止作「不」; 觀止作「功」

臣不敏,奉旨撰記,故上推宵旰圖治之{切}\endnote{\BookTitle{觀止}作「功」,據各本改。}者,勒諸貞珉。他若留連光景之辭,皆略而不陳,懼褻也。% 

\theendnotes

\section[司馬季主論卜\quad{\small 劉基}]{{\normalsize 劉基}\quad \ProperName{司馬季主}論卜}
\ProperName{東陵侯}既廢,過\ProperName{司馬季主}而卜焉。\ProperName{季主}曰:「君侯何卜也?」\ProperName{東陵侯}曰:「久臥者思起,久蟄者思啟,久懣者思嚏。吾聞之:『蓄極則洩,閟極則達,熱極則風,壅極則通。一冬一春,靡屈不伸;一起一伏,無往不復。』僕竊有疑,願受教焉!」\ProperName{季主}曰:「若是,則君侯已喻之矣,又何卜爲?」\ProperName{東陵侯}曰:「僕未究其奧也,願先生卒教之」。\ProperName{季主}乃言曰:「嗚呼!天道何親?惟德之親。鬼神何靈?因人而靈。夫蓍枯草也,龜枯骨也,物也。人靈於物者也,何不自聽而聽於物乎?且君侯何不思昔者也?有昔者必有今日。是故碎瓦頹垣,昔日之歌樓舞館也;荒榛斷梗,昔日之瓊蕤玉樹也;露{蛬}\endnote{\BookTitle{觀止}作「蠶」,據\BookTitle{郁離子}改。}風蟬,昔日之鳳笙龍笛也;鬼燐螢火,昔日之金{釭}\endnote{\BookTitle{觀止}作「缸」,據\BookTitle{郁離子}改。}華燭也;秋荼春薺,昔日之象白駝峰也;丹楓白荻,昔日之\ProperName{蜀}錦\ProperName{齊}紈也。昔日之所無,今日有之不爲過;昔日之所有,今日無之不爲不足。是故一晝一夜,華開者謝;{一秋一春}\endnote{\BookTitle{觀止}作「一春一秋」,據\BookTitle{郁離子}改。},物故者新。激湍之下必有深潭,高丘之下必有浚谷,君侯亦知之矣,何以卜爲?」% 觀止作「蠶」;  觀止作「缸」 ; 觀止作「一春一秋」

\theendnotes

\section[賣柑者言\quad{\small 劉基}]{{\normalsize 劉基}\quad 賣柑者言}
\ProperName{杭}有賣果者,善藏柑,涉寒暑不潰。出之燁然,玉質而金色。{剖其中},乾若敗絮。\endnote{\BookTitle{誠意伯文集}原作:「置於市,賈十倍,人争鬻之。予貿得其一,剖之如有烟撲口鼻,視其中,則乾若敗絮。」}予怪而問之曰:「若所市於人者,將以實籩豆,奉祭祀,供賓客乎?將衒外以惑愚瞽{也}\endnote{\BookTitle{觀止}作「乎」,據各本改。}。甚矣哉,爲欺也。」

賣者笑曰:「吾業是有年矣,{吾頼是}以食吾軀\endnote{\BookTitle{觀止}「賴」上有「業」字,作「吾業頼是以食吾軀」,\BookTitle{古今寓言}、\BookTitle{明文英華}同,據\BookTitle{誠意伯文集}、\BookTitle{明文海}改。}。吾售之,人取之,未{嘗}\endnote{\BookTitle{觀止}作「未聞」,\BookTitle{明文英華}同,據各本改。}有言,而獨不足子所乎?世之爲欺者不寡矣,而獨我也乎?吾子未之思也。今夫佩虎符、坐皋比者,洸洸乎干城之具也,果能授\ProperName{孫}、\ProperName{吳}之略耶?峨大冠、拖長紳者,昂昂乎廟堂之器也,果能建\ProperName{伊}、\ProperName{皋}之業耶?盜起而不知禦,民困而不知救,吏奸而不知禁,法斁而不知理,坐縻廩粟而不知恥。觀其坐高堂,騎大馬,醉醇醲而飫肥鮮者,孰不巍巍乎可畏,赫赫乎可象也?又何往而不金玉其外、敗絮其中也哉?今子是之不察,而以察吾柑。」

予默{然}\endnote{\BookTitle{觀止}作「默默」,\BookTitle{明文英華}同,據各本改。}無以應,退而思其言,類\ProperName{東方生}滑稽之流,豈其憤世嫉邪者耶,而託於柑以諷耶?% 誠意伯文集作「置於市,賈十倍,人争鬻之。予貿得其一,剖之如有烟撲口鼻,視其中,則乾若敗絮。」 ; 誠意伯文集作「也」; 觀止作「吾業頼是以食吾軀」不通。; 誠意伯文集作「未嘗」 ;誠意伯文集作「默然」

\theendnotes

\section[深慮論\quad{\small 方孝孺}]{{\normalsize 方孝孺}\quad 深慮論}
慮天下者常圖其所難,而忽其所易;備其所可畏,而遺其所不疑。然而禍常發於所忽之中,而亂常起於不足疑之事。豈其慮之未周與?蓋慮之所能及者,人事之宜然,而出於智力之所不及者,天道也。當\ProperName{秦}之世而滅{六}\endnote{\BookTitle{觀止}脱「六」字,據各本補。}諸侯,一天下,而其心以爲\ProperName{周}之亡在乎諸侯之彊耳。變封建而爲郡縣,方以爲兵革可不復用,天子之位可以世守。而不知\ProperName{漢帝}起隴畆之{匹夫}\endnote{\BookTitle{觀止}作「起隴畆之中」,據各本改。},而卒亡\ProperName{秦}之社稷。\ProperName{漢}懲\ProperName{秦}之孤立,於是大建庶孽而爲諸侯。以爲同姓之親可以相繼而無變,而七國萌簒弑之謀。\ProperName{武}、\ProperName{宣}以後,稍剖析之而分其勢,以爲無事矣,而\ProperName{王莽}卒移\ProperName{漢}祚。\ProperName{光武}之懲\ProperName{哀}、\ProperName{平},\ProperName{魏}之懲\ProperName{漢},\ProperName{晉}之懲\ProperName{魏},各懲其所由亡而爲之備,而其亡也,{皆出其}\endnote{\BookTitle{觀止}作「蓋出於」,據各本改。}所備之外。\ProperName{唐太宗}聞\ProperName{武氏}之殺其子孫,求人於疑似之際而除之,而\ProperName{武氏}日侍其左右而不悟。\ProperName{宋太祖}見\ProperName{五代}方鎮之足以制其君,盡釋其兵權,使力弱而易制,而不知子孫卒困於{夷狄}\endnote{\BookTitle{觀止}避諱作「敵國」。}。此其人皆有出人之智,{負}\endnote{\BookTitle{觀止}脱「負」字,據各本補。}蓋世之才,其於治亂存亡之幾,思之詳而備之審矣。慮切於此,而禍興於彼,終至於亂亡者,何哉?蓋智可以謀人,而不可以謀天。良醫之子多死於病,良巫之子多死於鬼。{彼}\endnote{\BookTitle{觀止}脱「彼」字,據各本補。}豈工於活人,而拙於{活己之子}哉\endnote{\BookTitle{觀止}作「謀子也哉」,據各本改。}?乃工於謀人,而拙於謀天也。古之聖人知天下後世之變,非智慮之所能周,非法術之所能制。不敢肆其私謀詭計,而唯積至誠,用大德,以結乎天心。使天眷其德,若慈母之保赤子而不忍釋。故其子孫雖有至愚不肖者足以亡國,而天{下}\endnote{「下」\BookTitle{觀止}作「卒」,據各本改,\BookTitle{明文海}作「亦」。}不忍遽亡之,此慮之遠者也。夫茍不能自結於天,而欲以區區之智籠絡當世之務,而必後世之無危亡,此理之所必無者也,而豈天道哉!% 滅「六」諸侯:觀止脱; 起隴畆之「匹夫」:觀止作「中」; 「皆出其」所備之外:觀止作「蓋出於」; 「夷狄」: 觀止作「敵國」; 「負」蓋世之才:觀止脱; 「彼」豈工於活人:觀止脱; 「活己之子哉」:觀止作「謀子也哉」; 天「下」不忍:觀止作「卒」

\theendnotes

\section[豫讓論\quad{\small 方孝孺}]{{\normalsize 方孝孺}\quad \ProperName{豫讓}論}
士君子立身事主,既名知己,則當竭盡知謀,忠告善道,銷患於未形,保治於未然,俾身全而主安。生爲名臣,死爲上鬼,垂光百世,照耀簡策,斯爲美也。茍遇知己,不能扶危於未亂之先,而乃捐軀殞命於既敗之後,釣名沽譽,眩世{駭}俗\endnote{\BookTitle{觀止}作「眩世炫俗」,據各本改。}。由君子觀之,皆所不取也。% 眩世「駭俗」:觀止作「炫俗」

蓋嘗因而論之,\ProperName{豫讓}臣事\ProperName{智伯},及\ProperName{趙襄子}殺\ProperName{智伯},\ProperName{讓}爲之報仇,聲名烈烈,雖愚夫愚婦莫不知其爲忠臣義士也。嗚呼!\ProperName{讓}之死固忠矣!惜乎處死之道有未忠者存焉。何也?觀其漆身吞炭,謂其友曰:「凡吾所爲者極難,將以愧天下後世之爲人臣而懷二心者也。」謂非忠可乎?及觀斬{劍}三躍\endnote{\BookTitle{觀止}作「斬衣三躍」,\BookTitle{明文英華}同,據\BookTitle{遜志齋集}改。下同。},\ProperName{襄子}責以不死於\ProperName{中行氏},而獨死於\ProperName{智伯},\ProperName{讓}應曰:「\ProperName{中行氏}以眾人待我,我故以眾人報之。\ProperName{智伯}以國士待我,我故以國士報之。」即此而論,\ProperName{讓}有餘憾矣。% 斬「劍」三躍:觀止作「衣」, 下同。

\ProperName{段規}之事\ProperName{韓康},\ProperName{任章}之事\ProperName{魏獻},未聞以國士待之也。而\ProperName{規}也、\ProperName{章}也,力勸其主從\ProperName{智伯}之請,與之地,以驕其志而速其亡也。\ProperName{絺疵}之事\ProperName{智伯}亦未嘗以國士待之也,而\ProperName{疵}能察\ProperName{韓}、\ProperName{魏}之情以諫\ProperName{智伯},雖不用其言以至滅亡,而\ProperName{疵}之知謀忠告已無愧於心也。\ProperName{讓}既自謂\ProperName{智伯}待以國士矣,國士濟國之事也。當伯請地無厭之日,縱欲荒{棄}\endnote{\BookTitle{觀止}作「暴」,\BookTitle{明文英華}同,據\BookTitle{遜志齋集}改。}之時,爲\ProperName{讓}者正宜陳力就列,諄諄然而告之曰:「諸侯大夫各{受}\endnote{\BookTitle{觀止}作「安」,據各本改。}分地,無相侵奪,古之制也。今無故而取地於人,人不與而吾之忿心必生,與之則吾之驕心以起。忿必爭,爭必敗,驕必傲,傲必亡。」諄切懇告\endnote{\BookTitle{觀止}作「諄切懇至」,據各本改。},諫不從,再諫之,再諫不從,三諫之,三諫不從,移其伏劍之死。死於是日,伯雖頑冥不靈,感其至誠,庶幾復悟。和\ProperName{韓}\ProperName{魏},釋\ProperName{趙}圍,保全\ProperName{智}宗,守其祭祀。若然,則\ProperName{讓}雖死猶生也,豈不勝於斬{劍}而死乎?\ProperName{讓}於此時,曾無一語開悟主心,視伯之危亡,猶\ProperName{越人}視\ProperName{秦人}之肥瘠也。袖手旁觀,坐待成敗,國士之報曾若是乎?\ProperName{智伯}既死,而乃不勝血氣之悻悻,甘自附於刺客之流,何足道哉!何足道哉!% 絺疵:觀止作「郄疵」; 縱欲荒「棄」之時:觀止作荒「暴」; 各「受」分地:觀止作「安」; 諄切懇「告」:觀止作「至」

雖然,以國士而論,\ProperName{豫讓}固不足以當矣。彼朝爲仇敵,暮爲君臣,靦然而自得者,又\ProperName{讓}之罪人也。噫!

\theendnotes

\section[親政篇\quad{\small 王鏊}]{{\normalsize 王鏊}\quad 親政篇}
\BookTitle{易}之\BookTitle{泰}曰:「上下交,而其志同。」其\BookTitle{否}曰:「上下不交,而天下無邦。」蓋上之情達於下,下之情達於上,上下一體,所以爲泰。\endnote{今校本\BookTitle{王鏊集}及\BookTitle{震澤集}「所以爲泰」下原有「上之情壅閼而不得下達」,\BookTitle{觀止}無。\BookTitle{觀止}本篇從\BookTitle{文章辨體彙選}、\BookTitle{皇明疏钞}、\BookTitle{明文海}所本,下同,僅注\BookTitle{震澤集}文。}下之情壅閼而不得上聞,上下間隔雖有國{而}\endnote{\BookTitle{震澤集}作「如」。}無國矣,所以爲否也。交則泰,不交則否,自古皆然。而不交之弊,未有如近世之甚者。君臣相見,止於視朝數刻。上下之間,章奏批答相闗接,刑名法度相維持而已。非獨沿襲故事,亦其地勢使然,何也?國家常朝於\ProperName{奉天門},未嘗一日廢,可謂勤矣。然堂陛懸絶,威儀赫奕,御史糾儀,鴻臚舉不如法,通政司引奏,上特{視}\endnote{\BookTitle{震澤集}作「是」。}之。謝恩見辭,惴惴而退。上何嘗{治}\endnote{\BookTitle{震澤集}作「問」。}一事?下何嘗進一言哉?此無他,地勢懸絶。所謂堂上,遠於萬里。雖欲言,無由言也。% 震澤集「所以爲泰」下原有「上之情壅閼而不得下達」; 震澤集作「如」; 震澤集作「是」; 震澤集作「問」; 

愚以爲欲上下之交,莫若復古内朝之法。蓋\ProperName{周}之時,有三朝:庫門之外{爲正朝},詢謀大臣在焉\endnote{\BookTitle{震澤集}作「庫門之外爲外朝,詢大事在焉」};路門之外爲治朝,日視朝在焉;路門之内曰内朝,亦曰燕朝。\BookTitle{玉藻}云:「君日出而視朝,退適路寝聽政。」蓋視朝而見羣臣,所以正上下之分;聽政而適路寝,所以通遠近之情。\ProperName{漢}制大司馬,左、右、前、後将軍,{侍中}散騎諸吏\endnote{\BookTitle{震澤集}「侍中」下有「散騎常侍」,作「侍中散騎常侍散騎諸吏」。},爲中朝;丞相以下至六百石,爲外朝。\ProperName{唐}皇城之北,南三門曰\ProperName{承天},元正、冬至受萬國之朝貢,則御焉,蓋古之外朝也;其北曰\ProperName{太極門},其{內}\endnote{\BookTitle{觀止}作「西」,據各本改。}曰\ProperName{太極殿},朔望則坐而視朝,蓋古之正朝也;{又北曰}\endnote{\BookTitle{震澤集}「又北曰」下有「兩儀門其内曰」。}\ProperName{兩儀殿},常日聽朝而視事,蓋古之内朝也。\ProperName{宋}時,常朝則\ProperName{文德殿},五日一起居則\ProperName{垂拱殿},正旦、冬至、聖節稱賀則\ProperName{大慶殿},賜宴則\ProperName{紫宸殿}或\ProperName{集英殿},試進士則\ProperName{崇政殿}。侍從以下,五日一員,上殿謂之輪對,則必入{陳}\endnote{\BookTitle{震澤集}無「陳」字。}時政利害。内殿引見,亦或賜坐,或免穿靴。蓋亦三朝之遺意焉。蓋天有三垣,天子象之:正朝,象太微也;外朝,象天市也;内朝,象紫微也,自古然矣。% 震澤集作「庫門之外爲外朝,詢大事在焉」; 震澤集「侍中」下有「散騎常侍」; 觀止作「西」; 震澤集「又北曰」下有「兩儀門其内曰」; 震澤集無「陳」字

國朝聖節、正旦、冬至、大朝會則\ProperName{奉天殿},即古之正朝也;常日\endnote{\BookTitle{震澤集}作「朝」。}則\ProperName{奉天門},即古之外朝也;而内朝獨缺。然非缺也,\ProperName{華蓋}、\ProperName{謹身}、\ProperName{武英}等殿豈非内朝之遺制乎?\ProperName{洪武}中如\ProperName{宋濓}、\ProperName{劉基},\ProperName{永樂}以來如\ProperName{楊士奇}、\ProperName{楊榮}等日侍左右,大臣\ProperName{蹇義}、\ProperName{夏原吉}等常奏對便殿。于斯時也,豈有壅隔之患哉?今内朝未\endnote{\BookTitle{震澤集}作「罕」。}復,臨御常朝之後,人臣無復進見;三殿髙閟,鮮或窺焉,故上下之情壅而不通,天下之弊由是而積。\ProperName{孝宗}晩年深有慨於斯,屢召大臣於便殿,講論天下事,{方将}有爲\endnote{\BookTitle{震澤集}作「将大有爲」。},而民之無禄,不及覩至治之美,天下至今以爲恨矣。% 常「日」則奉天門:震澤集作「朝」; 今内朝「未」復:震澤集作「罕」; 震澤集作「将大有爲」

惟陛下遠法\ProperName{聖祖},近法\ProperName{孝宗},盡剗近世壅隔之弊。常朝之外,即\ProperName{文華}、\ProperName{武英}{二殿}\endnote{\BookTitle{震澤集}無「二殿」二字,\BookTitle{皇明疏钞}無「二」字。},倣古内朝之意,大臣三日或五日一次起居;侍從、臺諫各一員,上殿輪對;諸司有事咨决,上據所見決之;有難決者,與大臣面議之。不時引見羣臣。凡謝恩、辭見之類,皆得上殿陳奏。虚心而問之,和顔色而道之。如此,人人得以自盡。陛下雖身居九重,而天下之事燦然畢陳于前。外朝所以正上下之分,内朝所以通遠近之情。如此,豈有近世壅隔之弊哉?\ProperName{唐}、\ProperName{虞}之世,明目達聰,嘉言罔伏,野無遺賢,亦不過是而已。% 文華武英「二殿」:震澤集無「二殿」二字

\theendnotes

\section[尊經閣記\quad{\small 王守仁}]{{\normalsize 王守仁}\quad \ProperName{尊經閣}記}
經,常道也。其在於天謂之命,其賦於人謂之性,其主於身謂之心。心也,性也,命也,一也。通人物,達四海,塞天地,亙古今,無有乎弗具,無有乎弗同,無有乎或變者也。是常道也,其應乎感也,則爲惻隱,爲羞惡,爲辭讓,爲是非;其見於事也,則爲父子之親,爲君臣之義,爲夫婦之別,爲長幼之序,爲朋友之信。是惻隱也,羞惡也,辭讓也,是非也;是親也,義也,序也,別也,信也;{一也}\endnote{\BookTitle{觀止}脫「一也」,據各本補。}。皆所謂心也,性也,命也。通人物,達四海,塞天地,亙古今,無有乎弗具,無有乎弗同,無有乎或變者也,是常道也。{是常道也}\endnote{\BookTitle{觀止}脫下「是常道也」,據各本補。},以言其陰陽消息之行焉\endnote{\BookTitle{觀止}作「消長之行」,「焉」字脫,據各本改,下同。},則謂之\BookTitle{易};以言其紀綱政事之施焉,則謂之\BookTitle{書};以言其歌詠性情之發焉,則謂之\BookTitle{詩};以言其條理節文之著焉,則謂之\BookTitle{禮};以言其欣喜和平之生焉,則謂之\BookTitle{樂};以言其誠偽邪正之辯焉,則謂之\BookTitle{春秋}。是陰陽消息之行也,以至於誠偽邪正之辯也,一也。皆所謂心也,性也,命也。通人物,達四海,塞天地,亙古今,無有乎弗具,無有乎弗同,無有乎或變者也,夫是之謂\BookTitle{六經}。\BookTitle{六經}者非他,吾心之常道也。故\BookTitle{易}也者,志吾心之陰陽消息者也;\BookTitle{書}也者,志吾心之紀綱政事者也;\BookTitle{詩}也者,志吾心之歌詠性情者也;\BookTitle{禮}也者,志吾心之條理節文者也;\BookTitle{樂}也者,志吾心之欣喜和平者也;\BookTitle{春秋}也者,志吾心之誠偽邪正者也。君子之於\BookTitle{六經}也,求之吾心之陰陽消息而時行焉,所以尊\BookTitle{易}也;求之吾心之紀綱政事而時施焉,所以尊\BookTitle{書}也;求之吾心之歌詠性情而時發焉,所以尊\BookTitle{詩}也;求之吾心之條理節文而時著焉,所以尊\BookTitle{禮}也;求之吾心之欣喜和平而時生焉,所以尊\BookTitle{樂}也;求之吾心之誠偽邪正而時辯焉,所以尊\BookTitle{春秋}也。蓋昔{者}\endnote{\BookTitle{觀止}脫「者」字,據各本補。}聖人之扶人極,憂後世,而述\BookTitle{六經}也,猶之富家者之父祖慮其產業庫藏之積,其子孫者或至於遺{忘}\endnote{\BookTitle{觀止}作「亡」,據各本改,下同。}散失,卒困窮而無以自全也,而記籍其家之所有以貽之,使之世守其產業庫藏之積而享用焉,以免於困窮之患。故\BookTitle{六經}者,吾心之記籍也,而\BookTitle{六經}之實則具於吾心;猶之產業庫藏之實積,種種色色,具存於其家。其記籍者,特名狀數目而已。而世之學者,不知求\BookTitle{六經}之實於吾心,而徒考索於影響之間,牽制於文義之末,硜硜然以爲是\BookTitle{六經}矣。是猶富家之子孫不務守視享用其產業庫藏之實積,日遺{忘}散失,至爲窶人丐夫,而猶囂囂然指其記籍曰:「斯吾產業庫藏之積也」,何以異於是!嗚呼!\BookTitle{六經}之學,其不明於世,非一朝一夕之故矣。尚功利,崇邪說,是謂亂經;習訓詁,傳記誦,沒溺於淺聞小見以塗天下之耳目,是謂侮經;侈淫辭,競詭辯,飾奸心,盜行逐世,壟斷而自以爲通經,是謂賊經。若是者,是幷其所謂記籍者而割裂棄毀之矣,寧復知所以爲尊經也乎!% 別也信也「一也」:觀止脫; 是常道也,「是常道也」:觀止脫; 陰陽「消息之行焉」:觀止作「消長之行」,「焉」字脫,下同。; 蓋昔「者」聖人:觀止脱; 或至於遺「忘」:觀止作「亡」,下同。

\ProperName{越}城舊有\ProperName{稽山書院}在\ProperName{臥龍}西岡,荒廢久矣。郡守\ProperName{渭南}\ProperName{南}{君}\endnote{\BookTitle{觀止}脫「君」字,據各本補。}\ProperName{大吉}既敷政於民,則慨然悼末學之支離,將進之以聖賢之道。於是使\ProperName{山陰}令\ProperName{吳}君\ProperName{瀛}拓書院而一新之,又爲\ProperName{尊經}之閣於其後。曰:「經正,則庶民興;庶民興,斯無邪慝矣。」閣成,請予一言以諗多士。予既不獲辭,則爲記之若是。嗚呼!世之學者既得吾說而求諸其心焉,其\endnote{\BookTitle{觀止}作「則」,據各本改。}亦庶乎知所以爲尊經也矣。% 渭南南「君」大吉:觀止脱; 「其」亦庶乎:觀止作「則」

\theendnotes

\section[象祠記\quad{\small 王守仁}]{{\normalsize 王守仁}\quad \ProperName{象}祠記}
\ProperName{靈博}之山有\ProperName{象}祠焉,其下諸\ProperName{苗}夷之居者,咸神而{事}\endnote{\BookTitle{觀止}作「祠」,據各本改。}之。宣慰\ProperName{安}君因諸\ProperName{苗}夷之請,新其祠屋,而請記於予。予曰:「毁之乎?其新之也?」曰:「新之。」「新之也,何居乎?」曰:「斯祠之肇也,蓋莫知其原。然吾諸蠻夷之居是者,自吾父吾祖遡曾高而上,皆尊奉而禋祀焉,舉{之}而不敢廢也\endnote{\BookTitle{觀止}脫之字,據各本補。}。」予曰:「胡然乎?\ProperName{有庳}之{祠}\endnote{\BookTitle{觀止}作「有鼻之祀」,據各本改。},\ProperName{唐}之人蓋嘗毁之。\ProperName{象}之道,以爲子則不孝,以爲弟則傲。斥於\ProperName{唐}而猶存於今,毁於\ProperName{有庳}\endnote{\BookTitle{觀止}作「壞於有鼻」,據各本改。}而猶盛於兹土也,胡然乎?」% 咸神而「事」之:觀止作「祠」; 舉「之」而不敢廢:觀止脱; 有庳之祠:觀止「有鼻之祀」; 毁於有庳: 觀止作「壞於有鼻」

我知之矣,君子之愛若人也,推及於其屋之烏,而況於聖人之弟乎哉?然則{祀}\endnote{\BookTitle{觀止}作「祠」,\BookTitle{文章辨體彙選}同,據各本改。}者爲\ProperName{舜},非爲\ProperName{象}也。意\ProperName{象}之死,其在干羽既格之後乎?不然,古之驁桀者豈少哉?而\ProperName{象}之祠獨延於世,吾於是益\endnote{\BookTitle{觀止}作「蓋」,據各本改。}有以見\ProperName{舜}德之至,入人之深,而流澤之遠且久也。% 祀者爲舜:觀止作「祠」; 「益」有以見舜德之至:觀止作「蓋」

\ProperName{象}之不仁,蓋其始焉爾。又烏知其終之不見化於\ProperName{舜}也?\ProperName{書}不云乎:「克諧以孝,烝烝乂,不格姦,\ProperName{瞽瞍}亦允若」,則已化而爲慈父。\ProperName{象}猶不弟,不可以爲諧。進治於善,則不至於惡;不抵於姦,則必入於善。信乎,\ProperName{象}蓋已化於\ProperName{舜}矣!\ProperName{孟子}曰:「天子使吏治其國,\ProperName{象}不得以有爲也。」斯蓋\ProperName{舜}愛\ProperName{象}之深而慮之詳,所以扶持輔導之者之周也。不然,\ProperName{周公}之聖,而\ProperName{管}、\ProperName{蔡}不免焉。斯可以見\ProperName{象}之既\endnote{\BookTitle{觀止}作「見」,據各本改。}化於\ProperName{舜},故能任賢使能而安於其位,澤加於其民,既死而人懷之也。諸侯之卿,命於天子,蓋\ProperName{周}官之制。其殆倣於\ProperName{舜}之封\ProperName{象}歟?% 象之「既」化於舜:觀止作「見」

吾於是益有以信人性之善,天下無不可化之人也。然則\ProperName{唐}人之毁之也,據\ProperName{象}之始也,今之諸夷\endnote{\BookTitle{觀止}作「苗」,據各本改。}之奉之也,承\ProperName{象}之終也。斯義也,吾將以表於世,使知人之不善,雖若\ProperName{象}焉,猶可以改;而君子之修德,及其至也,雖若\ProperName{象}之不仁,而猶可以化之也。% 今之諸「夷」之奉之也:觀止作「苗」

\theendnotes

\section[瘞旅文\quad{\small 王守仁}]{{\normalsize 王守仁}\quad 瘞旅文}
維\ProperName{正德}四年秋月三日,有吏目云自京來者,不知其名氏;携一子一僕,将之任,過\ProperName{龍塲}投宿土\ProperName{苗}家。予從籬落間望見之,陰雨昏黑,欲就問訊北来事,不果。明早遣人覘之,已行矣。薄午有人自\ProperName{蜈蚣坡}来,云一老人死坡下,傍兩人哭之哀。予曰:「此必吏目死矣。傷哉!」薄暮復有人来云:「坡下死者二人,傍一人坐{歎}\endnote{\BookTitle{觀止}作「哭」,據各本改。}。」詢其状,則其子又死矣。明日復有人来云:「見坡下積尸三焉。」則其僕又死矣。嗚呼傷哉!念其暴骨無主,将二童子持畚鍤,往瘞之,二童子有難色然。予曰:「嘻\endnote{\BookTitle{觀止}作「噫」,據各本改。}!吾與爾猶彼也。」二童憫然涕下,請往;就其傍山麓爲三坎埋之,又以隻雞、飯三盂,嗟吁涕洟而告之。曰:% 傍一人坐「歎」:觀止作「哭」; 予曰「嘻」:觀止作「噫」

嗚呼傷哉!繄何人?繄何人?吾\ProperName{龍塲}驛丞\ProperName{餘姚}\ProperName{王守仁}也。吾與爾皆中土之產,吾不知爾郡邑,爾烏爲乎来爲兹山之鬼乎?古者重去其鄉,遊宦不踰千里。吾以竄逐而来此,宜也;爾亦何辜乎?聞爾官,吏目耳,俸不能五斗,爾率妻子躬耕,可有也,{烏}\endnote{\BookTitle{觀止}作「胡」,據各本改。下同。}爲乎以五斗而易爾七尺之軀?又不足,而益以爾子與僕乎?嗚呼傷哉!爾誠戀兹五斗而来,則宜欣然就道,{烏}爲乎吾昨望見爾容蹙然,盖不{任}\endnote{\BookTitle{觀止}作「勝」,據各本改。}其憂者?夫衝冒{霧}\endnote{\BookTitle{觀止}作「勝霜,據各本改。}露,扳援崖壁,行萬峰之頂,饑渴勞頓,筋骨疲憊,而又瘴厲侵其外,憂鬱攻其中,其能以無死乎?吾固知爾之必死,然不謂若是其速,又不謂爾子爾僕亦遽爾奄忽也。皆爾自取,謂之何哉!吾念爾三骨之無依而来瘞{爾},乃使吾有無窮之愴也,嗚呼傷哉!縱不爾瘞,幽崖之狐成羣,陰壑之虺如車輪,亦必能葬爾於腹,不致久暴{露}\endnote{\BookTitle{觀止}脫「露」字,據各本補。}爾。爾既已無知,然吾何能爲心乎?自吾去父母鄉國而来此,二年\endnote{觀止作「三年」,\BookTitle{陽明先生道學鈔}同,今校本\BookTitle{王陽明全集}及各本作「二年」,據改。}矣,歷瘴毒而茍能自全,以吾未嘗一日之戚戚也。今悲傷若此,是吾爲爾者重而自爲者輕也。吾不宜復爲爾悲矣。吾爲爾歌,爾聽之。歌曰:% 「烏」爲乎以:觀止作「胡」,下同。; 盖不「任」其憂者:觀止作「勝」; 衝冒「霧」露:觀止作「霜」; 而来瘞「爾」:觀止作「耳」; 不致久暴「露」爾:觀止脱; 三年矣:王文成公全書作「二年矣」,陽明先生道學鈔作「三年矣」

\begin{quote}
連峰際天兮,飛鳥不通;遊子懷鄉兮,莫知西東。莫知西東兮,維天則同。異域殊方兮,環海之中;達觀隨寓兮,{奚}\endnote{\BookTitle{觀止}作「莫」,據各本改。}必予宫?魂兮魂兮,無悲以恫!
\end{quote}% 「奚」必予宫:觀止作「莫」

又歌以慰之,曰:

\begin{quote}
與爾皆鄉土之離兮,蠻之人言語不相知兮。性命不可期,吾茍死於兹兮,率爾子僕来從予兮。吾與爾遨以嬉兮,驂紫彪而乘文螭兮,登望故鄉而嘘唏兮。吾茍獲生歸兮,爾子爾僕尚爾隨兮,{無以無侣悲兮}\endnote{「無以無侣悲兮」:\BookTitle{觀止}脫,據各本補。}。道傍之塚累累兮,多中土之流離兮,相與呼嘯而徘徊兮。飱\endnote{\BookTitle{觀止}作「餐」,據各本改。}風飲露,無爾饑兮;朝友麋鹿,暮猿與栖兮,爾安爾居兮,無爲厲於兹墟兮!
\end{quote} % 無以無侣悲兮:觀止脱 ; 「飱」風飲露:觀止作「餐」

\theendnotes

\section[信陵君救趙論\quad{\small 唐順之}]{{\normalsize 唐順之\endnote{此篇僅見於\BookTitle{古文觀止},未見於\BookTitle{荊川先生文集}各本及\BookTitle{四庫全書}各集,存疑。}}\quad \ProperName{信陵君}救\ProperName{趙}論}
論者以竊符爲\ProperName{信陵君}之罪,余以爲此未足以罪\ProperName{信陵}也。夫彊\ProperName{秦}之暴亟矣,今悉兵以臨\ProperName{趙},\ProperName{趙}必亡。\ProperName{趙},\ProperName{魏}之障也,\ProperName{趙}亡,則\ProperName{魏}且爲之後。\ProperName{趙}、\ProperName{魏},又\ProperName{楚}、\ProperName{燕}、\ProperName{齊}諸國之障也,\ProperName{趙}、\ProperName{魏}亡,則\ProperName{楚}、\ProperName{燕}、\ProperName{齊}諸國爲之後。天下之勢,未有岌岌於此者也。故救\ProperName{趙}者,亦以救\ProperName{魏};救一國者,亦以救六國也。竊\ProperName{魏}之符以紓\ProperName{魏}之患,借一國之師以分六國之災,夫奚不可者?然則\ProperName{信陵}果無罪乎?曰:又不然也。余所誅者,\ProperName{信陵君}之心也。

\ProperName{信陵}一公子耳,\ProperName{魏}固有王也。\ProperName{趙}不請救於王,而諄諄焉請救於\ProperName{信陵},是\ProperName{趙}知有\ProperName{信陵},不知有王也。\ProperName{平原君}以婚姻激\ProperName{信陵},而\ProperName{信陵}亦自以婚姻之故,欲急救\ProperName{趙},是\ProperName{信陵}知有婚姻,不知有王也。其竊符也,非爲\ProperName{魏}也,非爲六國也,爲\ProperName{趙}焉耳;非爲\ProperName{趙}也,爲一\ProperName{平原君}耳。使禍不在\ProperName{趙},而在他國,則雖撤\ProperName{魏}之障,雖撤六國之障,\ProperName{信陵}亦必不救。使\ProperName{趙}無\ProperName{平原},或\ProperName{平原}而非\ProperName{信陵}之姻戚,雖\ProperName{趙}亡,\ProperName{信陵}亦必不救。則是\ProperName{趙王}與社稷之輕重,不能當一\ProperName{平原公子},而\ProperName{魏}之兵甲,所恃以固其社稷者,祇以供\ProperName{信陵君}一姻戚之用。幸而戰勝,可也;不幸戰不勝,爲虜於\ProperName{秦},是傾\ProperName{魏國}數百年社稷以殉姻戚,吾不知信陵何以謝\ProperName{魏王}也?夫竊符之計,蓋出於\ProperName{侯生},而\ProperName{如姬}成之也。\ProperName{侯生}教公子以竊符,\ProperName{如姬}爲公子竊符於王之臥內,是二人亦知有\ProperName{信陵},不知有王也。

余以爲\ProperName{信陵}之自爲計,曷若以脣齒之勢激諫於王,不聽,則以其欲死\ProperName{秦}師者,而死於\ProperName{魏王}之前,王必悟矣。\ProperName{侯生}爲\ProperName{信陵}計,曷若見\ProperName{魏王}而說之救\ProperName{趙},不聽,則以其欲死\ProperName{信陵君}者,而死於\ProperName{魏王}之前,王亦必悟矣。\ProperName{如姬}有意於報\ProperName{信陵},曷若乘王之隙而日夜勸之救,不聽,則以其欲爲公子死者而死於\ProperName{魏王}之前,王亦必悟矣。如此,則\ProperName{信陵君}不負\ProperName{魏},亦不負\ProperName{趙};二人不負王,亦不負於\ProperName{信陵君}。何爲計不出此?\ProperName{信陵}知有婚姻之\ProperName{趙},不知有王。內則幸姬,外則鄰國,賤則\ProperName{夷門}野人,又皆知有公子,不知有王。則是\ProperName{魏}僅有一孤王耳。

嗚呼!自世之衰,人皆習於背公死黨之行,而忘守節奉公之道。有重相而無威君,有私讎而無義憤。如\ProperName{秦}人知有\ProperName{穰侯},不知有\ProperName{秦王},\ProperName{虞卿}知有布衣之交,不知有\ProperName{趙王},蓋君若贅旒久矣。由此言之,\ProperName{信陵}之罪,固不專係乎符之竊不竊也。其爲\ProperName{魏}也,爲六國也,縱竊符猶可;其爲\ProperName{趙}也,爲一親戚也,縱求符於王,而公然得之,亦罪也。

雖然,\ProperName{魏王}亦不得爲無罪也,兵符藏於臥內,\ProperName{信陵}亦安得竊之?\ProperName{信陵}不忌\ProperName{魏王},而徑請之\ProperName{如姬},其素窺\ProperName{魏王}之疎也;\ProperName{如姬}不忌\ProperName{魏王},而敢於竊符,其素恃\ProperName{魏王}之寵也。木朽而蛀生之矣。古者人君持權於上,而內外莫敢不肅。則\ProperName{信陵}安得樹私交於\ProperName{趙}?\ProperName{趙}安得私請救於\ProperName{信陵}?\ProperName{如姬}安得銜\ProperName{信陵}之恩?\ProperName{信陵}安得賣恩於\ProperName{如姬}?履霜之漸,豈一朝一夕也哉?由此言之,不特衆人不知有王,王亦自爲贅旒也。

故\ProperName{信陵君}可以爲人臣植黨之戒,\ProperName{魏王}可以爲人君失權之戒。\BookTitle{春秋}書葬\ProperName{原仲}、\ProperName{翬}帥師。嗟乎!聖人之爲慮深矣!

\theendnotes

\section[報劉一丈書\quad{\small 宗臣}]{{\normalsize 宗臣}\quad 報\ProperName{劉一丈}書} 
數千里外,得長者時賜一書,以慰長想,即亦甚幸矣。何至更辱饋遺,則不才益將何以報焉?書中情意甚殷,即長者之不忘老父,知老父之念長者深也。

至以「上下相孚,才德稱位」語不才,則不才有深感焉。夫才德不稱,固自知之矣;至于不孚之病,則尤不才爲甚。且今世之所謂孚者何哉?日夕策馬候權者之門,門者故不入,則甘言媚詞作婦人狀,袖金以私之。即門者持刺入,而主者又不即出見,立厩中僕馬之間,惡氣襲衣裾,即饑寒毒熱不可忍,不去也。抵暮,則前所受贈金者出,報客曰:「相公倦,謝客矣,客請明日來。」即明日又不敢不來。夜披衣坐,聞雞鳴,即起盥櫛,走馬抵門,門者怒曰:「爲誰?」則曰:「昨日之客來。」則又怒曰:「何客之勤也!豈有相公此時出見客乎?」客心恥之,强忍而與言曰:「亡奈何矣,姑容我入,」門者又得所贈金,則起而入之。又立向所立厩中。幸主者出,南面召見,則驚走匍匐階下。主者曰:「進!」則再拜,故遲不起,起則上所上夀金。主者故不受,則固請;主者故固不受,則又固請。然後命吏{內}之。則又再拜,又故遲不起,起則五六揖始出。出,揖門者曰:「官人幸顧我,他日來,幸{亡}阻我也!」門者答揖。大喜,奔出。馬上遇所交識,即揚鞭語曰:「適自相公家來,相公厚我!厚我!」且虚言狀。即所交識亦心畏相公厚之矣。相公又稍稍語人曰:「某也賢,某也賢。」聞者亦心計交贊之。此世所謂上下相孚也。長者謂僕能之乎?% 諸集作「內之」; 觀止作「無」

前所謂權門者,自嵗時伏臘一刺之外,即經年不往也。間道經其門,則亦掩耳閉目,躍馬疾走過之。若有所追逐者。斯則僕之褊{哉}\endnote{\BookTitle{觀止}作「衷」,據各本改。}。以此{常}不見悦于長吏,僕則愈益不顧也。每大言曰:「人生有命,吾惟守分爾矣。」長者聞{此}\endnote{\BookTitle{觀止}作「之」,據各本改。},得無厭其爲迂乎?\endnote{各集下有:「鄉園多故不能不動客子之愁,至于長者之抱才而困,則又令我愴然,有感天之與先生者甚厚,亡論長者不欲輕棄之,即天意亦不欲長者之輕棄之也,幸寧心哉。」}% 僕之褊「哉」:觀止作「衷」; 長者聞「此」得「毋」厭其爲迂乎:觀止作「之」;各集下有「鄉園多故不能不動客子之愁」云云。。。

% 鄉園多故不能不動客子之愁,至于長者之抱才而困,則又令我愴然,有感天之與先生者甚厚,亡論長者不欲輕棄之,即天意亦不欲長者之輕棄之也,幸寧心哉。

\theendnotes

\section[吳山圖記\quad{\small 歸有光}]{{\normalsize 歸有光}\quad \BookTitle{吳山圖}記}
\ProperName{吳}、\ProperName{長洲}二縣在郡治所,分境而治。而郡西諸山皆在\ProperName{吳縣}。其最高者\ProperName{穹窿}、\ProperName{陽山}、\ProperName{鄧尉}、\ProperName{西脊}、\ProperName{銅井},而\ProperName{靈巖},\ProperName{吳}之故宮在焉,尚有\ProperName{西子}之遺跡。若\ProperName{虎丘}、\ProperName{劍池},及\ProperName{天平}、\ProperName{尚方}、\ProperName{支硎},皆勝地也。而\ProperName{太湖}汪洋三萬六千頃,七十二峰沉浸其間,則海內之奇觀矣。

余同年友\ProperName{魏}君\ProperName{用晦}爲\ProperName{吳縣},未及三年,以高第召入爲給事中,君之爲縣有惠愛,百姓扳留之,不能得,而君亦不忍於其民;由是好事者繪\BookTitle{吳山圖}以爲贈。

夫令之於民,誠重矣。令誠賢也,其地之山川草木,亦被其澤而有榮也;令誠不賢也,其地之山川草木,亦被其殃而有辱也。君於\ProperName{吳}之山川,蓋增重矣。異時吾民將擇勝於巖巒之間,尸祝於浮屠\ProperName{老子}之宮也,固宜。而君則亦既去矣,何復惓惓於此山哉?昔\ProperName{蘇子瞻}稱\ProperName{韓魏公}去\ProperName{黃州}四十餘年,而思之不忘。至以爲\BookTitle{思黃州詩},\ProperName{子瞻}爲\ProperName{黃}人刻之於石。然後知賢者於其所至,不獨使其人之不忍忘而已,亦不能自忘於其人也。

君今去縣已三年矣。一日,與余同在內庭,出示此圖,展玩太息,因命余記之。噫,君之於吾\ProperName{吳},有情如此,如之何而使吾民能忘之也!

\section[滄浪亭記\quad{\small 歸有光}]{{\normalsize 歸有光}\quad \ProperName{滄浪亭}記}
浮圖\ProperName{文瑛},居\ProperName{大雲庵},環水,即\ProperName{蘇子美}\ProperName{滄浪亭}之\endnote{\BookTitle{觀止}脱「之」字,據各本補。}地也。亟求余作\BookTitle{滄浪亭記},曰:「昔\ProperName{子美}之記,記亭之勝也。請子記吾所以爲亭者。」% 滄浪亭「之」地也:觀止脱。

余曰:昔\ProperName{吳}、\ProperName{越}有國時,\ProperName{廣陵王}鎮\ProperName{吳}中,治\ProperName{{南}\endnote{\BookTitle{觀止}脱「南」字,據各本補。}園}於子城之西南。其外戚\ProperName{孫承佑},亦治園於其偏。迨\ProperName{淮}、{海}\endnote{\BookTitle{觀止}作「南」,據各本改。}納土,此園不廢。\ProperName{蘇子美}始建\ProperName{滄浪亭},最後禪者居之。此\ProperName{滄浪亭}爲\ProperName{大雲庵}也。有庵以來二百年,\ProperName{文瑛}尋古遺事,復\ProperName{子美}之構於荒殘滅沒之餘。此\ProperName{大雲庵}爲\ProperName{滄浪亭}也。夫古今之變,朝市改易。嘗登\ProperName{姑蘇}之臺,望五湖之渺茫,羣山之蒼翠,\ProperName{太伯}、\ProperName{虞仲}之所建,\ProperName{闔閭}、\ProperName{夫差}之所爭,\ProperName{子胥}、\ProperName{種}、\ProperName{蠡}之所經營,今皆無有矣。庵與亭何爲者哉?雖然,\ProperName{錢鏐}因亂攘竊,保有\ProperName{吳}、\ProperName{越},國富兵強,垂及四世。諸子姻戚,乘時奢僭,宮館苑囿,極一時之盛。而\ProperName{子美}之亭,乃爲釋子所欽重如此。可以見士之欲垂名於千載之後,不與其澌然而俱盡者,則有在矣。% 治「南」園:觀止脱; 淮「海」納土:觀止作「南」

\ProperName{文瑛}讀書喜詩,與吾徒遊,呼之爲\ProperName{滄浪僧}云。

\theendnotes

\section[青霞先生文集序\quad{\small 茅坤}]{{\normalsize 茅坤}\quad \BookTitle{青霞先生文集}序}
\ProperName{青霞}\ProperName{沈}君,繇錦衣經歷上書詆宰執,宰執深嫉之,方力搆其罪,頼明天子仁聖,特薄其譴,徙之塞上。當是時,君之直諫之名滿天下。已而,君累然携妻子出家塞上,會北{虜}\endnote{\BookTitle{觀止}作「敵」,據各本改。下同。}數內犯,而帥府以下束手閉壘,以恣虜之出沒,不及飛一鏃以相抗。甚且及虜之退,則割中土之戰沒者與野行者之馘以爲功。而父之哭其子,妻之哭其夫,兄之哭其弟者,往往而是,無所控籲。君既上憤疆塲之日弛,而下痛諸將士之日菅刈我人民以蒙國家也,數嗚咽欷歔。而以其所{飲}\endnote{\BookTitle{觀止}作「憂」,據各本改。}鬱發之於詩歌文章,以泄其懷,即集中所載諸什是也。% 會北「虜」數內犯:觀止作「敵」,下同。; 其所「飲」鬱:觀止作「憂」

君故以直諫爲重於時,而其所著爲詩歌文章又多所譏刺,稍稍傳播,上下震恐,始出死力相煽構,而君之禍作矣。君既沒,而一時閫寄所相與讒君者,尋且坐罪罷去。又未幾,故宰執之仇君者亦報罷。而君之門人給諫\ProperName{俞君},於是裒輯其平生所著若干卷,刻而傳之。而其子\ProperName{以敬},來請余予之首簡。

\ProperName{茅子}受讀而題之曰:若君者,非古之志士之遺乎哉?\ProperName{孔子}刪\BookTitle{詩},而\BookTitle{小弁}之怨親,\BookTitle{巷伯}之刺讒而下,其間忠臣、寡婦、幽人、懟士之什並列之爲\BookTitle{風},疏之爲\BookTitle{雅},不可勝數。豈皆古之中聲也哉?然\BookTitle{孔子}不遽遺之者,特憫其人,矜其志,猶曰「發乎情,止乎禮義」,「言之者無罪,聞之者足以爲戒」焉耳。予嘗按次\ProperName{春秋}以來\ProperName{屈原}之\BookTitle{騷}疑於怨,\ProperName{伍胥}之諫疑於脅,\ProperName{賈誼}之疏疑於激,\ProperName{叔夜}之詩疑於憤,\ProperName{劉蕡}之對疑於亢。然推\ProperName{孔子}刪\BookTitle{詩}之旨而裒次之,當亦未必無錄之者。君既沒,而海內之薦紳大夫至今言及君,無不酸鼻而流涕。嗚呼!集中所載\BookTitle{鳴劍}、\BookTitle{籌邊}諸什,試令後之人讀之,其足以寒賊臣之膽,而躍塞垣戰士之馬,而作之愾也,固矣。他日國家采風者之使出而覽觀焉,其能遺之也乎?予謹識之。

至於文詞之工不工,及當古作者之旨與否,非所以論君之大者也,予故不著。% 嘉靖癸亥孟春望日歸安茅坤拜手序

\theendnotes

\section[藺相如完璧歸趙論\quad{\small 王世貞}]{{\normalsize 王世貞}\quad \ProperName{藺相如}完璧歸\ProperName{趙}論}
\ProperName{藺相如}之完璧,人人\endnote{\BookTitle{觀止}脱一「人」字,據\BookTitle{弇州四部稿}集本補。}皆稱之,予未敢以爲信也。% 「人人」皆稱之:觀止脱一「人」字

夫\ProperName{秦}以十五城之空名,而\endnote{\BookTitle{觀止}脱「而」字,據集本補。}詐\ProperName{趙}而脅其璧。是時言取璧者情也,非欲以窺\ProperName{趙}也。\ProperName{趙}得其情則弗予,不得其情則予;得其情而畏之則予,得其情而弗畏之則弗予。此兩言決耳,奈之何既畏而復挑其怒也!% 「而」詐趙:觀止脱

且夫\ProperName{秦}欲璧,\ProperName{趙}弗予璧,兩無所曲直也。入璧而\ProperName{秦}弗與城,曲在\ProperName{秦};\ProperName{秦}城出\endnote{\BookTitle{觀止}倒作「出城」,據改。}而璧歸,曲在\ProperName{趙}。欲使曲在\ProperName{秦},則莫如棄璧;畏棄璧,則莫如弗予。夫\ProperName{秦王}既按圖以予城,又設九賓,齋而受璧,其勢不得不予城。璧入而城弗予,\ProperName{相如}則前請曰:「臣固知大王之弗予城也。夫璧非\ProperName{趙}寳也\endnote{「寶也」\BookTitle{觀止}作「璧乎」,據集本改。},而十五城\ProperName{秦}寳也。今使大王以璧故,而亡其十五城,十五城之子弟皆厚怨大王以棄我如草芥也。大王弗予城而紿\ProperName{趙}璧,以一璧故,而失信於天下,臣請辭\endnote{\BookTitle{觀止}脫「辭」字,據集本補。}就死於國,以明大王之失信。」\ProperName{秦王}未必不予\endnote{\BookTitle{觀止}作「返」,據集本改。}璧也。今奈何使舎人懷而逃之,而歸直於\ProperName{秦}?是時\ProperName{秦}意未欲與\ProperName{趙}絶耳。令\ProperName{秦王}怒而僇相如於市,\ProperName{武安君}十萬衆壓\ProperName{邯鄲},而責璧與信,一勝而\ProperName{相如}族,再勝而璧終入\ProperName{秦}矣。 % 秦「城出」:觀止作「出城」; 璧非趙「寳也」:觀止作「璧乎」; 臣請「辭」:觀止脱; 秦王未必不「予」璧:觀止作「返」

吾故曰:\ProperName{藺相如}之獲全於璧也,天也。若其勁\ProperName{澠池},柔\ProperName{信}\ProperName{平}\endnote{「信平」\BookTitle{觀止}作「廉頗」,據集本改。},則愈出而愈妙於用。所以能存\endnote{\BookTitle{觀止}作「完」,據集本改。}\ProperName{趙}者,天固曲成之哉。% 柔「信平」:觀止作「廉頗」; 所以能「存」趙者:觀止作「完」

\theendnotes

\section[徐文長傳\quad{\small 袁宏道}]{{\normalsize 袁宏道}\quad \ProperName{徐文長}傳}
\ProperName{徐渭}字\ProperName{文長},爲\ProperName{山陰}諸生,聲名藉甚。\ProperName{薛}公\ProperName{蕙}校\ProperName{越}時,奇其才,有國士之目。然數奇,屢試輒蹶。中丞\ProperName{胡}公\ProperName{宗憲}聞之,客諸幕。\ProperName{文長}每見,則葛衣烏巾,縱譚天下事。\ProperName{胡公}大喜。是時公督數邊兵,威{振}\endnote{\BookTitle{觀止}作「鎭」,據各本改。}東南,介胄之士,膝語蛇行,不敢舉頭,而\ProperName{文長}以部下一諸生傲之,議者方之\ProperName{劉真長}、\ProperName{杜少陵}云。會得白鹿,屬\ProperName{文長}作表,表上,\ProperName{永陵}喜。公以是益奇之,一切疏記\endnote{\BookTitle{觀止}作「計」,據各本改。},皆出其手。% 威「振」東南:觀止作「鎭」; 一切疏「記」:觀止作「計」

\ProperName{文長}自負才略,好奇計,談兵多中,視一世{士}\endnote{\BookTitle{觀止}作「事」,據各本改。}無可當意者,然竟不偶。\ProperName{文長}既已不得志於有司,遂乃放浪麯糵,恣情山水,走\ProperName{齊}、\ProperName{魯}、\ProperName{燕}、\ProperName{趙}之地,窮覽朔漠,其所見山奔海立,沙起{雲}\endnote{\BookTitle{觀止}作「雷」,據各本改。}行,{風}\endnote{\BookTitle{觀止}作「雨」,據各本改。}鳴樹偃,幽谷大都,人物魚鳥,一切可驚可愕之狀,一一皆達之于詩。其胸中又有勃然不可磨滅之氣,英雄失路托足無門之悲,故其爲詩,如嗔如笑,如水鳴峽,如種出土,如寡婦之夜哭,羈人之寒起,雖其體格時有卑者,然匠心獨出,有王者氣,非彼巾幗而事人者所敢望也。文有卓識,氣沉而法嚴,不以模擬損才,不以議論傷格,\ProperName{韓}、\ProperName{曾}之流亞也。\ProperName{文長}既雅不與時調合,當時所謂騷壇主盟者,\ProperName{文長}皆叱而{奴}\endnote{\BookTitle{觀止}作「怒」,據各本改。}之,故其名不出於\ProperName{越},悲夫!喜作書,筆意奔放如其詩,蒼勁中姿媚躍出,\ProperName{歐陽公}所謂「妖韶女老,自有餘態」者也。間以其餘,旁溢爲花鳥,皆超逸有致。卒以疑殺其繼室,下獄論死,\ProperName{張}太史\ProperName{元汴}力解乃得出。 % 視一世「士」:觀止作「事」; 沙起「雲」行:觀止作「雷」; 「風」鳴樹偃:觀止作「雨」; 皆叱而「奴」之:觀止作「怒」

晚年憤益深,佯狂益甚,顯者至門,或拒不納。時攜錢至酒肆,呼下隸與飲。或自持斧擊破其頭,血流被面,頭骨皆折,揉之有聲。或以利錐錐其兩耳,深入寸餘,竟不得死。\ProperName{周望}言:「晚歲詩文益奇,無刻本,集藏於家。」余同年有官\ProperName{越}者,托以抄錄,今未至。余所見者,\BookTitle{徐文長集}、\BookTitle{闕編}二種而已。然\ProperName{文長}竟以不得志於時,抱憤而卒。

\ProperName{石公}曰:「先生數奇不已,遂爲狂疾;狂疾不已,遂爲囹圄。古今文人牢騷困苦,未有若先生者也。雖然,\ProperName{胡公}間世豪傑,\ProperName{永陵}英主,幕中禮數異等,是\ProperName{胡公}知有先生矣;表上,人主悅,是人主知有先生矣。獨身未貴耳。先生詩文崛起,一掃近代蕪穢之習,百世而下,自有定論,胡爲不遇哉?\ProperName{梅客生}嘗寄余書曰:『\ProperName{文長}吾老友,病奇於人,人奇於詩。』余謂\ProperName{文長}無之而不奇者也。無之而不奇,斯無之而不奇也,悲夫!」

\theendnotes

\section[五人墓碑記 \quad{\small 張溥}]{{\normalsize 張溥}\quad 五人墓碑記}
五人者,蓋當\ProperName{蓼洲}\ProperName{周公}之被逮,{急}\endnote{\BookTitle{觀止}作「激」,\BookTitle{明詩紀事}同,據各本改。}於義而死焉者也。至於今,郡之賢士大夫請於當道,即除\ProperName{魏阉}廢祠之址以葬之,且立石于其墓之門,以旌其所爲。嗚呼!亦盛矣哉!% 「急」於義:觀止作「激」

夫五人之死,去今之墓而葬焉,其爲時止十有一月爾。夫十有一月之中,凡富貴之子,慷慨得志之徒,其疾病而死,死而湮沒不足道者,亦已眾矣。況草野之無聞者歟!獨五人之皦皦,何也?

予猶記\ProperName{周公}之被逮,在丁卯三月之望,吾社之行爲士先者,爲之聲義,歛貲財以送其行,哭聲震動天地。緹騎按劍而前,問:「誰爲哀者?」眾不能堪,抶而仆之。是時以大中丞撫\ProperName{吳}者,爲\ProperName{魏}之私人,\ProperName{周公}之逮所繇使也,\ProperName{吳}之民方痛心焉。於是乘其厲聲以呵,則譟而相逐,中丞匿於溷藩以免。既而以\ProperName{吳}民之亂請于朝,按誅五人,曰:\ProperName{顏佩韋}、\ProperName{楊念如}、\ProperName{馬杰}、\ProperName{沈揚}、\ProperName{周文元},即今之傫然在墓者也。

然五人之當刑也,意氣揚揚,呼中丞之名而詈之。譚笑以死。斷頭置城上,顏色不少變,有賢士大夫發五十金,買五人之脰而函之,卒與屍合。故今之墓中,全乎爲五人也。

嗟乎!大閹之亂,縉紳而能不易其志者,四海之大有幾人歟?而五人生於編伍之間,素不聞\BookTitle{詩}\BookTitle{書}之訓,激昂大義,蹈死不顧,亦曷故哉?且矯詔紛出,鉤黨之捕。遍於天下,卒以吾郡之發憤一擊,不敢復有株治。大閹亦逡巡畏義,非常之謀,難於猝發待聖人之出,而投繯道路,不可謂非五人之力也。

繇是觀之,則今之高爵顯位,一旦抵罪,或脫身以逃,不能容於遠近,而又有剪髮杜門,佯狂不知所之者。其辱人賤行,視五人之死,輕重固何如哉?是以\ProperName{蓼洲}\ProperName{周公}忠義暴于朝廷,贈謚美顯,榮於身後;而五人亦得以加其土封,列其姓名於大堤之上。凡四方之士,無不有過而拜且泣者,斯固百世之遇也!不然,令五人者保其首領,以老于戶牖之下,則盡其天年,人皆得以隸使之,安能屈豪傑之流,扼腕墓道,發其志士之悲哉?故余與同社諸君子,哀斯墓之徒有其石也,而爲之記,亦以明死生之大,匹夫之有重于社稷也。

賢士大夫者冏卿\ProperName{因之}\ProperName{吳公}、太史\ProperName{文起}\ProperName{文公}、\ProperName{孟長}\ProperName{姚公}也。

\theendnotes

% Proofed 3 July 2022
% Ref. 
% - 四部叢刊·宋學士文集
% - 郁離子, 上海古籍
% - 四部叢刊·遜志齋集
% - 方孝孺集, 浙江古籍
% - 四庫全書·震泽集·卷20
% - 王鏊集, 上海古籍
% - 王陽明全集, 上海古籍 pp. 253, 893, 951 
% - 四庫全書·明文海·卷185: 宗臣報劉一丈書
% - 四庫全書·青霞集
% - 四庫全書·弇州四部稿·卷110: 王世貞藺相如 
% - 袁宏道集箋校, 上海古籍(1981)
% - 四庫全書·明文海·卷四百五十七: 張溥五人墓碑記 
% - 震川先生集, 上海古籍